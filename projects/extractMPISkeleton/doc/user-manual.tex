\section{MPI Skeleton Generator}

In order to support separate compilation of programs, we have separated the
analysis into three parts:

\begin{itemize}
\item Generating summary information for each function in a given compilation unit.
\item Combining individual summary files into a larger summary of the program.
\item Extraction of the MPI Skeleton using the combined summary information.
\end{itemize}

\subsection{Invoking the tools}

\subsubsection{Generating summary information}

The first stage of the analysis is to provide summary information for each
function in the compilation unit. Currently, the signature of each function is
the call graph of the function, including information about how each argument
of the function is passed to called functions.

Given this program:
\begin{verbatim}
#include <stdio.h>
#include "mpi.h"

int main( argc, argv )
int argc;
char **argv;
{
    int rank, value;
    MPI_Init( &argc, &argv );

    MPI_Comm_rank( MPI_COMM_WORLD, &rank );
    do {
        if (rank == 0)
            scanf( "%d", &value );

        MPI_Bcast( &value, 1, MPI_INT, 0, MPI_COMM_WORLD );

        printf( "Process %d got %d\n", rank, value );
    } while (value >= 0);

    MPI_Finalize( );
    return 0;
}
\end{verbatim}

We would generate this signature:
\begin{verbatim}
("function-signature"
  ("function-name" "main")
    ("arg-count" "2")
    ("calls" "MPI_Bcast" "MPI_Comm_rank" "MPI_Finalize" "MPI_Init" "printf" "scanf")
    ("arg-mapping" ("0" ("MPI_Init" "0")) ("1" ("MPI_Init" "1"))))
\end{verbatim}

The line \verb|arg-mapping| records the information that \verb|argc| is passed
to \verb|MPI_Init| and \verb|argv| is passed to \verb|MPI_Init|.

The tool is invoked as follows:
\begin{verbatim}
  generateSignatures [options] file...
\end{verbatim}

The command line options are these:

\begin{verbatim}
 -signature:(o|output) filename - Use 'filename' as the output of the signature.
 -signature:(d|debug)           - Print debugging messages.
\end{verbatim}

%\end  of subsubsection on for generateSignatures

\subsubsection{Combining summaries}

Given signature information from one or more compilation units, we need to
compute the transitive closure of both the function call graph and also of the
parameter passing graph. With the transitive closure of both of these graphs it
is straightforward to detect dependencies between both function calls and data
flow with respcet to API calls. For example, given the transitive call graph,
then during the skeletonization pass we can perform a simple lookup to
determine if an arbitrary function would eventually result in a call to an API
function. Similarly, for we can consult the transitive parameter passing graph
for determining when parameters will get passed to the API.

The tool is invoked as follows:
\begin{verbatim}
  summarizeSignatures [options] <signature files>
\end{verbatim}

The command line options are these:

\begin{verbatim}
 -summarize:(o|output) filename - Use 'filename' as the output of the summarized signature.
 -summarize:(s|spec) filename   - Use 'filename' as the specification of the API.
                                - See section 2.
 -summarize:(d|debug)           - Print debugging messages.
\end{verbatim}

The last two are only needed for debugging.

You should specify all the signature files from the previous step of the
analysis at once. While the generation of the summary is additive, the overall
analysis should be more efficient if done at once.

%\end  of subsubsection on for summarizeSignatures

\subsubsection{Extracting Skeleton}

The tool is invoked as follows:

\begin{verbatim}
  extractMPISkeleton [options] file...
\end{verbatim}

The filenames are skeletonized and the output is written to files with `rose\_'
prepended to the filenames.

The command line options are these:

% TODO: are these section numbers correct?
\begin{verbatim}
 -skel:(o|outline)       - Outline everything not in the skeleton.
                         - See section 3.
 -skel:(s|spec) filename - Use 'filename' as the specification of the API.
                         - See section 2.
 -skel:(g|sig)  filename - Use 'filename' as the summary information.
 -skel:(d|debug)         - Print debugging messages.
 -skel:(p|pdf)           - Generate PDF of the generated AST.
\end{verbatim}

The last two are only needed for debugging. In order for the analysis to be
complete, a signature file generated from the \verb|summarizeSignatures| tool
must be provide.

Current limitations:

\begin{itemize}

\item The skeleton generator only supports C and C++.
    Fortran is not supported at the current time.

\end{itemize}

\subsection{API Specification}

The skeleton generator skeletonizes programs relative to one or more
API specifications.  This means that functions that are part of the
API are required to be preserved in the skeleton, and further code is
preserved based on their dependencies.  Each function that we would
like to treat as part of the API is specified by the function name,
number of parameters, and a list of dependency types for each
parameter.  The dependency types allow us to categorize code based on
how it impacts the API functions.  For example, in the case of MPI, we
will have code related to the data that is sent (the "payload"), and
code that is related to the destination and topology of the program.
These are often very different, and the API specification allows the
tool to tag program elements based on their role.  This information is
not currently used in any detail, but will be in future versions of
the skeleton generator.

APIs are specified in a configuration file that uses an s-expression
format.  For example, here is a subset of the MPI API:

\begin{verbatim}
(api-spec MPI
  ( dep-types payload topology tag other )
  ( default_deptype other )

  (
    (MPI_Init              2 )
    (MPI_Finalize          0 )
    (MPI_Abort             2 )

    (MPI_Comm_rank         2  (topology 1) )
    (MPI_Comm_size         2  (topology 1) )
    (MPI_Comm_split        4 )


    (MPI_Send              6  (payload 0 1 2) (topology 3) (tag 4) )
  )
)
\end{verbatim}

First, we define the API name via the api-spec tag.  This is followed
by an s-expression that contains a list of dependency types, prefaced
with the tag "dep-types".  The names of the dependency types are
user-specified.  Next, we indicate which dependency type should be
associated with arguments that are not explicitly specified for an API
function.  This allows the user to group uninteresting arguments in a
catch-all class, making the API specification a bit briefer.  Finally,
we have an expression that contains the API calls.  Each API call is a
sub-expression with the format:

\begin{verbatim}
(API_FUNCTION_NAME ARGUMENT_COUNT (deptype argA ..) (deptype argB ..) ...)
\end{verbatim}

For example, on MPI_send above, we see that it has 6 arguments, the
first three of which are related to payload, the fourth is topology,
and the fifth is tag.  Note that argument numbering is zero-based.
The sixth argument is not specified, and therefore takes on the
default dependency type "other".

Given a set of API specifications, we then provide an API collection
file that is used by the skeleton generator to know what set of APIs
to skeletonize relative to, and how to do so.  For example:

\begin{verbatim}
(api-spec-collection
  ( include-api "mpi_api.spec"     (omit-deps payload) )
  ( include-api "stdio_input.spec" (omit-deps buffer) )
)
\end{verbatim}

This tells the tool to include API specifications for MPI and a subset
of C STDIO functions.  In each case, it also specifies a dependency
type to use for selecting code for elimination.  In the MPI case, code
that relates to computaton of payload data is eliminated, and in the
STDIO case, code related to filling IO buffers is eliminated.  This
allows a user to have relatively fine control over what is and is not
removed.  The API collection file is specified using the -skel:s
command line option:

\begin{verbatim}
% extractMPISkeleton -skel:s /where/is/the/collection/file
\end{verbatim}

\subsection{Outlining}

The skeleton generator can work in an alternate mode, Outlining.  This mode is
specified with the following command line option:

\begin{verbatim}
  -skel:o
  -skel:outline
\end{verbatim}

In this mode, rather than removing the non-skeleton code, the tool will
"outline" (move into separate functions, the converse of inlining) all
non-skeleton code.  I.e., the non-skeleton code is moved out of the skeleton
code into separate functions.

\subsection{Code annotations}

Annotations used by this tool are specified in the following format:

\begin{verbatim}
  #pragma skel [specific pragma text here]
\end{verbatim}

For example,

\begin{verbatim}
  #pragma skel loop iterate atmost(10)
\end{verbatim}

\subsubsection{Loop annotations}

It is not uncommon for skeletonized code to no longer have the looping
behavior of the original.  For example, say we have an iterative
solver with the following stucture:

\begin{verbatim}
do {
  // some numerical computation
  for (i=0;i<n;i++) { current[i] = a_big_computation(); }

  // compute delta based on computation
  for (i=0;i<n;i++) { delta += fabs(current[i] - prev[i]); }

  // do some message passing
  MPI_Send(stuff);
} while (delta > eps);
\end{verbatim}

If the numerical computation is removed, we can be in a situation
where delta will not ever change and will never drop below the
threshold for termination - so the loop will iterate forever.  Or, a
similar situation, we may find that through some choice of initial
values, the skeleton may cause delta to be zero, so to loop will
iterate only once (or never, depending on the type of loop used).  In
both cases, we would see behavior that isn't representative of the
real program.  An annotation can be added to force the skeleton to
contain a loop that iterates a certain number of times  (by having
the skeleton generator introduce counters and code to increment and
test their values).

Three loop annotations are available:

\begin{verbatim}
  #pragma skel loop iterate exactly(n)
  #pragma skel loop iterate atmost(n)
  #pragma skel loop iterate atleast(n)
\end{verbatim}

These correspond to forcing an exact, upper, and lower bound on the
iteration count.  The pragma must be placed immediately preceeding the
loop of interest.  Loops constructed with 'for', 'while', or 'do while'
are all supported as well as loops containing break and continue statements.

Note however, how we handle breaks in the new loops: we treat the break as an
exception that overrides the 'exactly' or 'atleast' annotations.  So, if a break
is reached in a loop before the n-th iteration, the loop is exited but a warning
is printed to stderr.

In the above annotations, 'n' is a C expression that is interpreted in the
current scope of the program.
Note that ROSE does not currently support floating point constants in
such expressions in pragmas.
So, one must encode non-integer values as rational numbers.
For example, instead of 0.5, one would need to write 1/2.  More
complicated real numbers, like 0.21376, are easily specified using
division by an appropriate power of ten (e.g., 21376/100000).

For example, the following will produce a skeleton
in which the loop iterates exactly 14 times.

\begin{verbatim}
#pragma skel loop iterate exactly(14)
do {
  // some numerical computation
  for (i=0;i<n;i++) { current[i] = a_big_computation(); }

  // compute delta based on computation
  for (i=0;i<n;i++) { delta += fabs(current[i] - prev[i]); }

  // do some message passing
  MPI_Send(stuff);
} while (delta > eps);
\end{verbatim}

\subsubsection{Data declaration annotations}

If a program contains an array that should be preserved in the
skeleton, it is useful to have control over how it is initialized
since often the skeleton will NOT contain the computational code that
populates the array elements.  The initializer pragma allows these
element values to be specified.

\begin{verbatim}
#pragma skel initializer repeat(x)
int myArray[14];
\end{verbatim}

(Where 'x' is a C expression interpreted in the current scope of the program.)

This will result in code being generated that iterates over the array
elements assigning the value 'x' to them.  The variable initialization
annotation supports arrays in the "auto" storage class (but not yet
arrays in the "static" storage class).
Future versions of the skeletonizer will support
initialization of static arrays, and dynamically
allocated arrays.
Future versions will also support initializing non-array variables,
in which case we will have a pragma something like:

\begin{verbatim}
  #pragma skel initializer value(NULL)
  int *myArray;
\end{verbatim}

\subsubsection{Conditional statement annotations}

Like the loop termination criteria, loop tests will likely have
different behavior in a skeleton than in real circumstances.  A common
behavior in a skeleton after computations are removed is to have the
loop test evaluate always to either true or false.  The conditional
annotation currently allows programmers to experiment with skeletons
that will randomly branch one way or the other with a specified
probability.  This probability could easily be measured in the
original program.  For example, if we want a conditional where the
then-clause executes 60% of the time we would say:

\begin{verbatim}
#pragma skel condition prob(6/10)  // make true 60% of time.
  if (x % 2) {
    x += 5;
  } else {
    x = 0;
  }
\end{verbatim}

The general case is

\begin{verbatim}
  #pragma skel condition prob(p)
\end{verbatim}

(Where 'p' is a C expression interpreted in the current scope of the program
which should be a floating point number between 0 and 1.0.  As noted above,
floating point constants are not allowed due to current limitations of
ROSE.)

The drawback of this approach is that real code very likely will not
branch randomly, but will exhibit a pattern.  For example: T, T, T, F,
F, T, T, T, F, F.  Future versions of the skeleton generator will
provide facilities for supporting conditional branches that preserve
non-random patterns seen in practice.
