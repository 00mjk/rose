\section{Implementation}

\subsection{Input For Example Showing use of {\em Seeding}}

   Figure~\ref{Tutorial:exampleInputCode_bufferOverflow}
shows the example input used for demonstration of {\em seeding} 
a security flaw.

\begin{figure}[h!]
{\indent
{\mySmallFontSize

% Do this when processing latex to generate non-html (not using latex2html)
\begin{latexonly}
   \lstinputlisting{\SeedingExampleDirectory/inputCode_bufferOverflow_arrayIndexing.C}
\end{latexonly}

% Do this when processing latex to build html (using latex2html)
\begin{htmlonly}
   \verbatiminput{\SeedingExampleDirectory/inputCode_bufferOverflow_arrayIndexing.C}
\end{htmlonly}

% end of scope in font size
}
% End of scope in indentation
}
\caption{Example source code used as input to program in codes used in this chapter.}
\label{Tutorial:exampleInputCode_bufferOverflow}
\end{figure}


% \includepdf[pages={1}]{inputCode_bufferOverflow_arrayIndexing.C_before.pdf}
% \includepdf[pages={1}]{\SeedingExampleBuildDirectory/inputCode_bufferOverflow_arrayIndexing.C_before.pdf}
% \includepdf[pages={1}]{inputCode_bufferOverflow_arrayIndexing_C_before.pdf}

\subsection{Initial AST}

Figure~\ref{arrayIndexing_C_before} shows the AST before any cloning or security flaw
seeding.

\begin{figure}[h!]
%\vspace{1.45in}
\hspace{-0.35in}
%\centering
%\includegraphics[width=1.0\textwidth]{inputCode_bufferOverflow_arrayIndexing_C_before.pdf}
\includegraphics[height=6.5in,width=1.0\textwidth]{inputCode_bufferOverflow_arrayIndexing_C_before.pdf}
\caption{The AST from the original input code.}
\label {arrayIndexing_C_before}
\end{figure}

\subsection{Vulnerability Recognition on AST}

Figure~\ref{arrayIndexing_C_afterIdentificationOfVulnerabilities} shows the AST after
recognition of specific vulnerabilities (prior to cloning and seeding).

\begin{figure}[h!]
%\vspace{1.45in}
\hspace{-0.35in}
%\centering
%\includegraphics[width=1.0\textwidth]{inputCode_bufferOverflow_arrayIndexing_C_before.pdf}
\includegraphics[height=6.5in,width=1.0\textwidth]{inputCode_bufferOverflow_arrayIndexing_C_afterIdentificationOfVulnerabilities.pdf}
\caption{The AST after recognition of two types of buffer overflow vulnerabilities.}
\label {arrayIndexing_C_afterIdentificationOfVulnerabilities}
\end{figure}

\subsection{AST after cloning based on vulnerabilities}

Figure~\ref{arrayIndexing_C_afterCloneGeneration} shows the AST after any cloning of
the function(s) containing each of two different types of security flaws.  AST
shown prior to seeding.

\begin{figure}[h!]
%\vspace{1.45in}
\hspace{-0.35in}
%\centering
%\includegraphics[width=1.0\textwidth]{inputCode_bufferOverflow_arrayIndexing_C_before.pdf}
%\includegraphics[height=6.5in,width=1.3\textwidth]{inputCode_bufferOverflow_arrayIndexing_C_afterCloneGeneration.pdf}
\includegraphics[width=1.3\textwidth]{inputCode_bufferOverflow_arrayIndexing_C_afterCloneGeneration.pdf}
\caption{The AST after cloning each vulnerability in preparation for seeding security flaw.}
\label {arrayIndexing_C_afterCloneGeneration}
\end{figure}


\subsection{AST After Seeding}

Figure~\ref{arrayIndexing_C_afterSeedingOfSecurityFlaws} shows the AST after being seeded.
each of the two clones (based on vulnerabilities) are seeded with the associated seeding
mechanism for that type of vulnerability.

\begin{figure}[h!]
%\vspace{1.45in}
\hspace{-0.35in}
%\centering
%\includegraphics[width=1.0\textwidth]{inputCode_bufferOverflow_arrayIndexing_C_before.pdf}
%\includegraphics[height=6.5in,width=1.0\textwidth]{inputCode_bufferOverflow_arrayIndexing_C_afterSeedingOfSecurityFlaws.pdf}
\includegraphics[width=1.3\textwidth]{inputCode_bufferOverflow_arrayIndexing_C_afterSeedingOfSecurityFlaws.pdf}
\caption{The AST after seeding of different security flaws from different vulnerablities.}
\label {arrayIndexing_C_afterSeedingOfSecurityFlaws}
\end{figure}

\commentout{
\section{Generating the code representing the seeded bug}

    Figure~\ref{Tutorial:example_volatileTypeModifier}
shows a code that traverses each IR node and for and
SgInitializedName IR node checks it type.
The input code is shown in figure \ref{Tutorial:exampleInputCode_volatileTypeModifier},
the output of this code is shown in 
figure~\ref{Tutorial:exampleOutput_volatileTypeModifier}.


\begin{figure}[!h]
{\indent
{
%\mySmallFontSize
\mySmallestFontSize

% Do this when processing latex to generate non-html (not using latex2html)
\begin{latexonly}
   \lstinputlisting{\TutorialExampleDirectory/volatileTypeModifier.C}
\end{latexonly}

% Do this when processing latex to build html (using latex2html)
\begin{htmlonly}
   \verbatiminput{\SeedingExampleDirectory/volatileTypeModifier.C}
\end{htmlonly}

% end of scope in font size
}
% End of scope in indentation
}
\caption{Example source code showing how to detect {\em volatile} modifier. }
\label{Tutorial:example_volatileTypeModifier}
\end{figure}
}


\subsection{Final Seeded Source Code}

Figure~\ref{Tutorial:exampleOutput_bufferOverflow} shows the final
source code after being seeding two specific vulnerabilities, one
with a single seeded flaw (introduced via the array subscript), and
the other vulnerabily.
    

\begin{figure}[h!]
{\indent
{\mySmallestFontSize

% Do this when processing latex to generate non-html (not using latex2html)
\begin{latexonly}
   \lstinputlisting{\SeedingExampleBuildDirectory/rose_inputCode_bufferOverflow_arrayIndexing.C}
\end{latexonly}

% Do this when processing latex to build html (using latex2html)
\begin{htmlonly}
   \verbatiminput{\SeedingExampleBuildDirectory/rose_inputCode_bufferOverflow_arrayIndexing.C}
\end{htmlonly}

% end of scope in font size
}
% End of scope in indentation
}
\caption{Output of input code after seeding: rose\_inputCode\_bufferOverflow\_arrayIndexing.C}
\label{Tutorial:exampleOutput_bufferOverflow}
\end{figure}

