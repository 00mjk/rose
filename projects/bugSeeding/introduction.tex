\section{Introduction}

\fixme{Need to identify top 30-40 security flaws with NIST.}

   The purpose of security bug seeding is to automate the testing of static analysis
tools designed to report security flaws in software.  By seeding bugs into
real applications in many different ways, a tool's ability detect a high
percentage of known (seeded) bugs can be predictive of the tool's
ability to detect unknown bugs.  In principle, seeding of flaws could be applied 
to either source code or binary executables, but we restrict our attention to 
the seeding of security flaws in source code.  The seeding of security flaws 
could be applied more generally to arbitrary bug seeding, but we 
restrict our focus to only the seeding of security flaws into arbitrary applications.

   This initial document is to permit a focus on the design of
a translator built using ROSE for the source-to-source transformation
of arbitrary applications to include security flaws for the purpose
of testing arbitrary static analysis tools.  The issues are dominantly
language independent and it should be possible to generate source
code for input applications over a range of languages.

\subsection{Benifits}
   Outline benifits of automated generation of test codes.

1) Compare to automated approach to hand generated test codes (labor intensive)

2) Compare to test code to security flaw seeding of large scale applications
   a) Access to realiztic control flows and pointer analysis requirements
   b) Objuscation of tests based on what applications would be seeed.

