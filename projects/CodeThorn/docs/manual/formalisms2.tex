\usepackage{stmaryrd} %permits using bigsqcap

\newcommand{\outcomment}[1]{}

\newenvironment{formula}
{%
\begin{displaymath}%
\begin{array}{rcl}%
}%
{%
\end{array}%
\end{displaymath}%
}%

\outcomment{
\newenvironment{example}
{%
\textbf{Example}.%
}%
{%
%
}%

\newenvironment{remark}
{%
\textbf{Remark}:%
}%
{%
%
}%

\newenvironment{proof}
{%
\textbf{Proof}:%
}%
{%
$\Box$%
}%
}


\newcommand{\sourcefont}[0]{\ptsize{8}\ttfamily}
\newcommand{\sourcetext}[1]{{\sourcefont {#1}}}

\newenvironment{sourcecode}                          % Source Code Umgebung
{
\begin{minipage}[t]{40ex}%
%\prologcode\progfont%
\sourcefont
\begin{quote}%
\begin{tabbing}%
\hspace{3ex}\=\hspace{3ex}\=\hspace{3ex}\=\hspace{3ex}%
\=\hspace{3ex}\=\hspace{3ex}\=\hspace{3ex}\=\hspace{3ex}\=\hspace{3ex}\=\hspace{3ex}\=\hspace{3ex}\=\hspace{3ex}\=\hspace{3ex}\kill}%
{\end{tabbing}%
\end{quote}%
\end{minipage}%
}

\newenvironment{sourcecodenofontsize}                          % Source Code Umgebung fuer Doppel Lines
{
\begin{minipage}[t]{0.4\textwidth}%
%\prologcode\progfont%
%\sourcefont
%\begin{quote}%
\begin{tabbing}%
\hspace{3ex}\=\hspace{3ex}\=\hspace{3ex}\=\hspace{3ex}%
\=\hspace{3ex}\=\hspace{3ex}\=\hspace{3ex}\=\hspace{3ex}\=\hspace{3ex}\=\hspace{3ex}\=\hspace{3ex}\=\hspace{3ex}\=\hspace{3ex}\kill}%
{\end{tabbing}%
%\end{quote}%
\end{minipage}%
}


%logic
\newcommand{\logand}[0]{\land}
\newcommand{\logor}[0]{\lor}
\newcommand{\lognot}[0]{\lnot}
\newcommand{\impl}[0]{\to}
\newcommand{\dimpl}[0]{\impl} % double arrow
\newcommand{\existsone}[0]{\exists!}
\newcommand{\notexists}[0]{\not\!\!\:\exists}

\newcommand{\ordner}[1]{ [Ordner {#1}]}

\newcommand{\pe}[1]{[{#1}]}
%sets
\newcommand{\setunion}[0]{\cup}
\newcommand{\bigsetunion}[0]{\bigcup}
\newcommand{\setintersect}[0]{\cap}
\newcommand{\bigsetintersect}[0]{\bigcap}
\newcommand{\setdif}[0]{\backslash}
\newcommand{\tuple}[1]{({#1})}
\newcommand{\set}[1]{\{{#1}\}}
\newcommand{\pair}[2]{({#1},{#2})}
\newcommand{\tripel}[3]{({#1},{#2},{#3})}
\newcommand{\triple}[3]{({#1},{#2},{#3})}
\newcommand{\quadrupel}[4]{({#1},{#2},{#3},{#4})}

\newcommand{\cnodedef}[0]{\pair{ \set{ \pair{*}{\epsilon} } }{.} }
\newcommand{\cnodedia}[0]{\diamond}
\newcommand{\cnodeset}[0]{\set{\diamond}}

%\newcommand{\bottomelement}{\tuple{\emptyset,\emptyset}}
\newcommand{\topelement}{\ensuremath{\top}}
\newcommand{\bottomelement}{\ensuremath{\perp}}
\newcommand{\dach} {\symbol{94}}
\newcommand{\deref}[1]{{#1}\dach}
\newcommand{\sname}[1]{[{#1}]}
\newcommand{\lname}[2]{[{#1},{#2}]}
\newcommand{\MSG}[0]{FSG }
\newcommand{\MSGpoint}[0]{FSG}
\newcommand{\MSGs}[0]{FSGs }
\newcommand{\MSGspoint}[0]{FSGs}
\newcommand{\MSGClass}[0]{\mathcal{\MSG}}
\newcommand{\Fcal}[0]{\mathcal{F}} % set of Functions, written as mathcal{F}.
\newcommand{\ffix}[0]{f.} % notation for fixpoint iteration
\newcommand{\mc}[1]{\multicolumn{3}{l|}{{#1}}} % multicolumn macro used in tabs for graphs
\newcommand{\kstring}[1]{\lceil {#1}]_k} % formal notation for k-truncated call strings (#1)
\newcommand{\nl}[1]{\ensuremath{\#{#1}}}

\newcommand{\edge}[1]{[{#1}]}

%\newcommand{\PP}[1]{{\em {#1}}}
\newcommand{\defeq}[0]{\stackrel{\mathrm{def}}{=}}

\newcommand{\exclude}[1]{}
\newcommand{\nsubset}{\not\subset}         %\\nsubset
%\newcommand{\textflorin}{\textit{f}}       %\\textflorin
\newcommand{\setB}{{\mathord{\mathbb B}}}  %\\setB
\newcommand{\setC}{{\mathord{\mathbb C}}}  %\\setC
\newcommand{\setN}{{\mathord{\mathbb N}}}  %\\setN
\newcommand{\setQ}{{\mathord{\mathbb Q}}}  %\\setQ
\newcommand{\setR}{{\mathord{\mathbb R}}}  %\\setR
\newcommand{\setZ}{{\mathord{\mathbb Z}}}  %\\setZ
\newcommand{\coloncolon}{\mathrel{::}}     %\\coloncolon
\newcommand{\lsem}{\mathopen{\lbrack\mkern-3mu\lbrack}}  %\\lsemantics
\newcommand{\rsem}{\mathclose{\rbrack\mkern-3mu\rbrack}} %\\rsemantics   
\newcommand{\pow}[1]{\mathcal{P}({#1})} % powerset of #1

\newcommand{\fullcirc}[0]{\bullet}
\newcommand{\heap}[0]{\mathcal{H}} % H (used for elements of the heap)
\newcommand{\nullref}[0]{\diamond} % box used as symbol for the reference value null
\newcommand{\nullnode}[0]{\ensuremath{\diamond}}
\newcommand{\recvar}[1]{\ensuremath{\bar{{#1}}}}
\newcommand{\ip}[1]{\ensuremath{\widehat{{#1}}}}% Notation for interprocedural entities 

\newcommand{\meetoperator}[0]{join operator }  % intraprocedural meet operator
\newcommand{\meet}[0]{\ensuremath{\sqcup}}  % intraprocedural meet operator
\newcommand{\imeet}[0]{\ip{\meet}}      % interprocedural meet operator
\newcommand{\bigmeet}[0]{\bigsqcup}        % big intraprocedural meet operator
\newcommand{\ibigmeet}[0]{\ip{\bigmeet}}% big interprocedural meet operator
\newcommand{\biglub}[0]{\bigmeet} % big least upper bound operator
\newcommand{\bigglb}[0]{\bigsqcap} % big greatest lower bound operator
\newcommand{\glb}[0]{\sqcap} % greatest lower bound operator

\newcommand{\ifelse}[3]{                % if-else array for function definitions (3 parameters)
  \left\{\begin{array}{l@{\quad:\quad}l}%
      {#1} & {#2} \\%
      {#3} & \mbox{otherwise}%
    \end{array}%
  \right.}%

\newcommand{\longifelse}[4]{                % if-else array for function definitions (3 parameters)
  \left\{\begin{array}{l@{\quad:\quad}l}%
      {#1} & {#2} \\%
      {#3} & {#4} \\%
    \end{array}%
  \right.}%

\newcommand{\longswitch}[6]{                % if-else array for function definitions (3 parameters)
  \left\{\begin{array}{l@{\quad:\quad}l}%
      {#1} & {#2} \\%
      {#3} & {#4} \\%
      {#5} & {#6} \\%
    \end{array}%
  \right.}%

\newcommand{\progfont}[0]{\normalsize}     % Zeichensatz 2 der Prologbeispiele
\newcommand{\mystring}[1]{"{#1}"}
\newcommand{\PP}[1]{{\progfont#1}\normalsize}
\newcommand{\ez}[0]{in }  % Das Elementzeichen
\newcommand{\anypath}[2]{{#1} *\to {#2}}
\newcommand{\nonnullpath}[2]{{#1} +\to {#2}}

\newcommand{\pathCd}[2]{{#1}{#2}}
\newcommand{\pathCa}[2]{{#1}{#2}}
\newcommand{\pathJ}[2]{{#1}{#2}}
\newcommand{\pathM}[2]{{#1}{#2}}
\newcommand{\code}[1]{{\textup\texttt {#1}}}

\newcommand{\ro}[0]{\ensuremath{\{\}}} % root in a graph
