\documentclass[natbib]{article}
\usepackage{microtype}
\usepackage{lmodern}
\usepackage{url}
\usepackage{xspace}
\usepackage{calc}
\usepackage{enumerate}
\usepackage{listings}
\usepackage{amsmath,amssymb}
\usepackage{rotating}
\usepackage{colortbl}
\usepackage{pifont}
\usepackage{tikz}
%\usetikzlibrary{shapes,shadows,arrows,calc,positioning,fit,matrix,mindmap,trees}
%\usepackage{pgfplots}
%\usepackage{pgfplotstable}
\usepackage{booktabs}
\usepackage{natbib}
\usepackage{colortbl}
% pantone colors

% More sensible defaults akin to \sloppy
% \tolerance 1414
% \hbadness 1414
% \emergencystretch 1.5em
% \hfuzz 0.3pt
% \widowpenalty=10000
% \clubpenalty=10000
% \vfuzz
% \hfuzz
% \raggedbottom

\newcommand{\ignore}[1]{}
\newcommand{\st}{\textit{s.\,t.}\xspace}
\newcommand{\eg}{\textit{e.\,g.}\xspace}
\newcommand{\ie}{\textit{i.\,e.}\xspace}
\newcommand{\cf}{\textit{cf.}\xspace}

\newcommand{\blackarrow}{{\color{black} \Pisymbol{pzd}{217}}}
\newcommand{\redarrow}{{\color{DarkRed} \Pisymbol{pzd}{217}}}
\newcommand{\minibox}[2]{\begin{minipage}{#1}\raggedright #2\end{minipage}}

\newcommand{\enquote}[1]{``#1''}

%\newcommand{\fixme}[1]{\begin{tikzpicture}
%\node[bottom color=red!80!white, top color=red!70!black, rounded corners,
%      font=\bf\color{white}\footnotesize] {
%  \begin{minipage}{.75\columnwidth}
%    FIXME\\
%    #1
%  \end{minipage}
%};
%\end{tikzpicture}
%}

\lstset{
  language=C,
  basicstyle=\small,%\scriptsize, %\footnotesize\ttfamily,
  keywordstyle={\bf},
  keywordstyle={[2]\it},%\color{Blue!40!black}},
  breaklines=true,
  identifierstyle=,
  stringstyle=\bf,
  commentstyle=\it\color{black!80},
  captionpos=b,
  numbers=left,
  stepnumber=3,
  columns=fullflexible
}

\begin{document}
\title{CodeThorn}

\author{\small Markus Schordan, Marc Jasper, Joshua Asplund, Maximilian Fecke, Adrian Prantl}
%\end{tabular}
\date{September 20, 2017}

\maketitle

\begin{abstract}
\noindent CodeThorn is a tool for analyzing C/C++ programs by combining approaches
from data flow analysis, constraint-based analysis, and model
checking. The main focus in the development of CodeThorn is to explore program analysis algorithms while combining above
approaches and to investigate methods for combining static analysis with
methods for software verification.

The input language is currently restricted to a subset of C.

\end{abstract}

\tableofcontents

%-------------------------------------------------------------------------

\section{Introduction}
\label{sec:intro}

CodeThorn was initially developed as a tool for exploring approaches
for reachbility analysis and verification of linear temporal logic (LTL)
formulae based on finite state systems~\cite{schordan2014combining}. 
This was later extended to
also perform specialization of programs and program equivalence
checking~\cite{schordan2014verification}. CodeThorn is based on the 
ROSE compiler infrastructure\footnote{\url{http://www.rosecompiler.org/}} 
and uses the ROSE abstract syntax tree as basis for its input. 
A number of components have been moved
from CodeThorn to ROSE over time. What remains are command line
options that allow to access those features conveniently and also to
reproduce some published results.

\nocite{roseWWW}

\subsection{Use in Competitions}

Since 2012, CodeThorn has been successfully used to participate in the
international RERS Challenge\footnote{\url{http://www.rers-challenge.org}} 
on program analysis and verification~\cite{schordan2014combining}.
Among other accomplishments, the use of CodeThorn helped to become overall winner and obtain the method
combination award in RERS 2014 as well as to win the 1st place in the Sequential LTL track of the 
recent 2017 iteration of RERS\footnote{See \url{http://www.rers-challenge.org} for detailed competition results}.
Participating in the challenge led to many improvements in the tool such as an efficient
parallelization of the analysis~\cite{jasper2016multi} and to the development of new model checking 
approaches~\cite{jasper2014counterexample}.

Starting in 2016, the LTL model checking infrastructure of CodeThorn has been successfully
applied to generate parallel verification benchmarks. These benchmarks were used for 
a new parallel track of the RERS Challenges 2016 and 2017~\cite{jasper2017rers}.


\subsection{Benchmarks}

We have found that benchmarks of the RERS Challenge
serve as an excellent guidance in crafting this tool and investigating
the impact and performance of each of the approaches on the overall
results. For the RERS programs, LTL formulae are
provided. This allows to verify behavioral properties of these
programs. Reachability properties can be verified by checking the
reachability of failing assertions.

For program equivalence checking and data race detection the
Polybench/C 3.2 suite has provided a good basis for evaluation. By
generating various polyhedral variants of the 30+ benchmarks, CodeThorn
can be used to check the equivalence of two given programs and verify
whether the optimizations are semantics preserving. 
Furthermore,
parallel OpenMP for loops are recognized and can be checked not to
introduce data races. Currently these approaches are extended to
address other large scale applications.

\section{Installation}

No additional configuration is required because
CodeThorn is configured as part of ROSE. In order to use all features of CodeThorn however,
the SPOT LTL model checking library version 1.2.6\footnote{This version of SPOT can be downloaded here: 
\url{https://www.lrde.epita.fr/dload/spot/spot-1.2.6.tar.gz}} is required. In addition, some
experimental features require the Z3 SMT 
solver\footnote{\url{https://github.com/Z3Prover/z3}}. Please provide the options
\verb+--with-spot=<spot-install-dir>+ and \verb+--with-z3=<z3-install-dir>+ to ROSE's configure command.

Run 'make', 'make install', and optionally 'make check' in the
\verb+projects/+ \verb+CodeThorn+ directory to install CodeThorn. CodeThorn is
installed as 'codethorn' (at the same location as other ROSE tools, in
the 'bin' directory of the ROSE installation).

\section{Command Line Options}
The command line options of CodeThorn are parsed by Boost's program options 
library\footnote{\url{http://www.boost.org/doc/libs/1_63_0/doc/html/program_options.html}}.
The \texttt{(=0)} syntax found for example in the description \texttt{--status [=arg(=1)] (=0)} 
denotes that the default value of option \enquote{status} is 0 (or false).
In addition, the \texttt{[=arg(=1)]} indicates that option \enquote{status} is set to 1 (or true) 
if it is provided without an argument.

Command line options can alternatively be passed to CodeThorn by using a configuration file.
In order to use this feature, the name of the configuration file needs to be set using the 
command line option \texttt{--config} (or just \texttt{-c}). For example, we could create a file
\enquote{myConfiguration.config} with the following content:
\begin{verbatim}
max-time=60
max-memory=8000000000
resource-limit-diff=10000
status=yes
log-level=trace
\end{verbatim}

This configuration file sets thresholds for the exploration of reachable program states: 
The exploration will terminate either after 60 seconds have passed or when approximately 8GB of RAM have been used,
depending on what happens first. Option \enquote{resource-limit-diff} is used to specify that these
resource constraints should be checked after every 10,000 newly discovered program states. 
In addition, status messages and a trace of the
analysis will be displayed. Note that implicit values cannot be used in configuration files,
an explicit argument has to be provided for example to the option \enquote{status}.
A command line using our configuration file could look as follows:
\begin{verbatim}
codethorn programToBeAnalyzed.c --config=myConfiguration.config
\end{verbatim}

\subsection{Main options}
The following command line options are listed when running \verb+codethorn --help+.
These main options below comprise general analysis parameters such as the exploration mode or resource constraints.
More detailed options belonging to individual aspects of CodeThorn are listed in the following sections and can be
seen by running \verb+codethorn --help-<name-of-detailed-options>+. 
\begin{verbatim}
  -c [ --config ] arg                   Use the configuration specified in file
                                        <arg>.
  --colors [=arg(=1)] (=1)              Use colors in output.
  --csv-stats arg                       Output statistics into a CSV file 
                                        <arg>.
  --display-diff arg                    Print statistics every <arg> computed 
                                        estates.
  --exploration-mode arg                Set mode in which state space is 
                                        explored. ([breadth-first]|depth-first|
                                        loop-aware|loop-aware-sync)
  -h [ --help ]                         Produce this help message.
  --help-cegpra                         Show options for CEGRPA.
  --help-eq                             Show options for program equivalence 
                                        checking.
  --help-exp                            Show options for experimental features.
  --help-pat                            Show options for pattern search mode.
  --help-svcomp                         Show options for SV-Comp specific 
                                        features.
  --help-rers                           Show options for RERS specific features
  --help-ltl                            Show options for LTL verification.
  --help-par                            Show options for analyzing parallel 
                                        programs.
  --help-vis                            Show options for visualization output 
                                        files.
  --help-data-race                      Show options for data race detection.
  --help-info                           Show options for program info.
  --status [=arg(=1)] (=0)              Show status messages.
  --reduce-cfg [=arg(=1)] (=1)          Reduce CFG nodes that are irrelevant 
                                        for the analysis.
  --internal-checks                     Run internal consistency checks 
                                        (without input program).
  --cl-options arg                      Specify command line options for the 
                                        analyzed program (as one quoted 
                                        string).
  --input-values arg                    Specify a set of input values. (e.g. 
                                        "{1,2,3}")
  --input-values-as-constraints [=arg(=1)] (=0)
                                        Represent input var values as 
                                        constraints (otherwise as constants in 
                                        PState).
  --input-sequence arg                  Specify a sequence of input values. 
                                        (e.g. "[1,2,3]")
  --log-level arg (=none,>=warn)        Set the log level ("x,>=y" with x,y in:
                                        (none|info|warn|trace|debug)).
  --max-transitions arg                 Passes (possibly) incomplete STG to 
                                        verifier after <arg> transitions have 
                                        been computed.
  --max-iterations arg                  Passes (possibly) incomplete STG to 
                                        verifier after <arg> loop iterations 
                                        have been explored. Currently requires 
                                        --exploration-mode=loop-aware[-sync].
  --max-memory arg                      Stop computing the STG after a total 
                                        physical memory consumption of 
                                        approximately <arg> Bytes has been 
                                        reached.
  --max-time arg                        Stop computing the STG after an 
                                        analysis time of approximately <arg> 
                                        seconds has been reached.
  --max-transitions-forced-top arg      Performs approximation after <arg> 
                                        transitions.
  --max-iterations-forced-top arg       Performs approximation after <arg> loop
                                        iterations. Currently requires 
                                        --exploration-mode=loop-aware[-sync].
  --max-memory-forced-top arg           Performs approximation after <arg> 
                                        bytes of physical memory have been 
                                        used.
  --max-time-forced-top arg             Performs approximation after an 
                                        analysis time of approximately <arg> 
                                        seconds has been reached.
  --resource-limit-diff arg             Check if the resource limit is reached 
                                        every <arg> computed estates.
  --rewrite                             Rewrite AST applying all rewrite system
                                        rules.
  --run-rose-tests                      Run ROSE AST tests.
  --threads arg                         (experimental) Run analyzer in parallel
                                        using <arg> threads.
  -v [ --version ]                      Display the version of CodeThorn.
\end{verbatim}


\subsection{Counterexample-Guided Prefix Refinement Analysis}
The Counterexample-guided prefix refinement analysis (CEGPRA)~\cite{jasper2014counterexample} is a special instance 
of CEGAR for reactive, PLC-like systems.
Based on an over-approximating initial model of the analyzed program's behavior, model checking is performed. In an iterative process, spurious counterexamples are removed by guided unrolling of the actual program's reachable state space. 
\begin{verbatim}
  --csv-stats-cegpra arg           Output statistics regarding the 
                                   counterexample-guided prefix refinement 
                                   analysis (CEGPRA) into a CSV file <arg>.
  --cegpra-ltl arg                 Select the ID of an LTL property that should
                                   be checked using cegpra (between 0 and 99).
  --cegpra-ltl-all [=arg(=1)] (=0) Check all specified LTL properties using 
                                   CEGPRA.
  --cegpra-max-iterations arg      Select a maximum number of counterexamples 
                                   anaylzed by CEGPRA.
  --viz-cegpra-detailed arg        Generate visualization (.dot) output files 
                                   with prefix <arg> for different stages 
                                   within each loop of CEGPRA.
\end{verbatim}


\subsection{Program Equivalence Checking}
The following list of options is relevant to the program equivalence checking capabilities of CodeThorn.
\begin{verbatim}
  --dump-sorted arg                      (experimental) Generates sorted array 
                                        updates in file <file>.
  --dump-non-sorted arg                  (experimental) Generates non-sorted 
                                        array updates in file <file>.
  --rewrite-ssa [=arg(=1)] (=0)         Rewrite SSA form: Replace use of SSA 
                                        variable by rhs of its assignment (only
                                        applied outside loops or unrolled 
                                        loops).
  --print-rewrite-trace [=arg(=1)] (=0) Print trace of rewrite rules.
  --print-update-infos [=arg(=1)] (=0)  Print information about array updates 
                                        on stdout.
  --rule-const-subst [=arg(=1)] (=1)    Use const-expr substitution rule.
  --rule-commutative-sort [=arg(=1)] (=0)
                                        Apply rewrite rule for commutative sort
                                        of expression trees.
  --specialize-fun-name arg             Function of name <arg> to be 
                                        specialized.
  --specialize-fun-param arg            Function parameter number to be 
                                        specialized (starting at 0).
  --specialize-fun-const arg            Constant <arg>, the param is to be 
                                        specialized to.
  --specialize-fun-varinit arg          Variable name of which the 
                                        initialization is to be specialized 
                                        (overrides any initializer expression).
  --specialize-fun-varinit-const arg    Constant <arg>, the variable 
                                        initialization is to be specialized to.
\end{verbatim}



\subsection{Experimental}
Experimental features that are not (yet) fully integrated.
\begin{verbatim}
  --annotate-terms [=arg(=1)] (=0)      Annotate term representation of 
                                        expressions in unparsed program.
  --eliminate-stg-back-edges [=arg(=1)] (=0)
                                        Eliminate STG back-edges (STG becomes a
                                        tree).
  --generate-assertions [=arg(=1)] (=0) Generate assertions (pre-conditions) in
                                        program and output program (using ROSE 
                                        unparser).
  --precision-exact-constraints [=arg(=1)] (=0)
                                        Use precise constraint extraction.
  --trace-file arg                      Generate STG computation trace and 
                                        write to file <arg>.
  --explicit-arrays [=arg(=1)] (=1)     Represent all arrays explicitly in 
                                        every state.
  --z3                                  RERS specific reachability analysis 
                                        using z3.
  --rers-upper-input-bound arg          RERS specific parameter for z3.
  --rers-verifier-error-number arg      RERS specific parameter for z3.
  --ssa [=arg(=1)] (=0)                 Generate SSA form (only works for 
                                        programs without function calls, loops,
                                        jumps, pointers and returns).
\end{verbatim}


\subsection{Pattern Search}
These options correspond to the pattern search exploration mode. During state-space exploration, it systematically 
unrolls repeating input/output patterns in order to reach deep areas of the state space of reactive systems. It was 
used as a black-box analysis during participation in the RERS Challenge.
\begin{verbatim}
  --pattern-search-max-depth arg (=10)  Maximum input depth that is searched 
                                        for cyclic I/O patterns.
  --pattern-search-repetitions arg (=100)
                                        Number of unrolled iterations of cyclic
                                        I/O patterns.
  --pattern-search-max-suffix arg (=5)  Maximum input depth of the suffix that 
                                        is searched for failing assertions 
                                        after following an I/O-pattern.
  --pattern-search-exploration arg      Exploration mode for the pattern 
                                        search. Note: all suffixes will always 
                                        be checked using depth-first search. 
                                        ([depth-first]|breadth-first)
\end{verbatim}


\subsection{SV-COMP}
Options specific to analyzing programs of the SV-COMP competition (work in progress).
\begin{verbatim}
  --svcomp-mode [=arg(=1)] (=0) Sets default options for all following 
                                SVCOMP-specific options.
  --error-function arg          Detect a verifier error function with name 
                                <arg> (terminates verification).
  --witness-file arg            Write an SV-COMP witness (counterexample) to 
                                file <arg>.
\end{verbatim}


\subsection{RERS Challenge}
The following list contains options that are relevant when analyzing programs of the RERS Challenge.
\begin{verbatim}
  --csv-assert arg                      Output assert reachability results into
                                        a CSV file <arg>.
  --eliminate-arrays [=arg(=1)] (=0)    Transform all arrays into single 
                                        variables.
  --iseq-file arg                       Compute input sequence and generate 
                                        file <arg>.
  --iseq-length arg                     Set length <arg> of input sequence to 
                                        be computed.
  --iseq-random-num arg                 Select random search and number <arg> 
                                        of paths.
  --rers-binary [=arg(=1)] (=0)         Call RERS binary functions in analysis.
  --rers-numeric [=arg(=1)] (=0)        Print RERS I/O values as raw numeric 
                                        numbers.
  --rersmode [=arg(=1)] (=0)            Sets several options such that RERS 
                                        specifics are utilized and observed.
  --stderr-like-failed-assert [=arg(=1)] (=0)
                                        Treat output on stderr similar to a 
                                        failed assert.
\end{verbatim}


\subsection{Linear Temporal Logic (LTL)}
Options below allow to check whether an analyzed program satisfies Linear Temporal Logic (LTL) properties 
(currently restrcited to input/ouput traces). Option \enquote{--check-ltl} is used to specify an input LTL file
in the format of the RERS Challenge\footnote{The following link leads to an exemplary input file: 
\url{http://www.rers-challenge.org/2014Isola/problems/constraints-RERS14-5.txt}}.
\begin{verbatim}
  --csv-spot-ltl arg                    Output SPOT's LTL verification results 
                                        into a CSV file <arg>.
  --csv-stats-size-and-ltl arg          Output statistics regarding the final 
                                        model size and results for LTL 
                                        properties into a CSV file <arg>.
  --check-ltl arg                       Take a text file of LTL I/O formulae 
                                        <arg> and check whether or not the 
                                        analyzed program satisfies these 
                                        formulae. Formulae should start with 
                                        '('. Use "csv-spot-ltl" option to 
                                        specify an output csv file for the 
                                        results.
  --single-property arg                 Number (ID) of the property that is 
                                        supposed to be analyzed. All other LTL 
                                        properties will be ignored. ( Use 
                                        "check-ltl" option to specify an input 
                                        property file).
  --counterexamples-with-output [=arg(=1)] (=0)
                                        Reported counterexamples for LTL or 
                                        reachability properties also include 
                                        output values.
  --inf-paths-only [=arg(=1)] (=0)      Recursively prune the transition graph 
                                        so that only infinite paths remain when
                                        checking LTL properties.
  --io-reduction arg                    (work in progress) Reduce the 
                                        transition system to only 
                                        input/output/worklist states after 
                                        every <arg> computed EStates.
  --keep-error-states [=arg(=1)] (=0)   Do not reduce error states for the LTL 
                                        analysis.
  --ltl-in-alphabet arg                 Specify an input alphabet used by the 
                                        LTL formulae. (e.g. "{1,2,3}")
  --ltl-out-alphabet arg                Specify an output alphabet used by the 
                                        LTL formulae. (e.g. "{19,20,21,22,23,24
                                        ,25,26}")
  --ltl-driven [=arg(=1)] (=0)          Select mode to verify LTLs driven by 
                                        SPOT's access to the state transitions.
  --reset-analyzer [=arg(=1)] (=1)      Reset the analyzer and therefore the 
                                        state transition graph before checking 
                                        the next property. Only affects 
                                        ltl-driven mode.
  --no-input-input [=arg(=1)] (=0)      remove transitions where one input 
                                        states follows another without any 
                                        output in between. Removal occurs 
                                        before the LTL check. [yes|=no]
  --std-io-only [=arg(=1)] (=0)         Bypass and remove all states that are 
                                        not standard I/O.
  --with-counterexamples [=arg(=1)] (=0)
                                        Add counterexample I/O traces to the 
                                        analysis results. Applies to reachable 
                                        assertions and falsified LTL properties
                                        (uses RERS-specific alphabet).
  --with-assert-counterexamples [=arg(=1)] (=0)
                                        Report counterexamples leading to 
                                        failing assertion states.
  --with-ltl-counterexamples [=arg(=1)] (=0)
                                        Report counterexamples that violate LTL
                                        properties.
\end{verbatim}


\subsection{Parallel Process Graphs}
These options allow to generate random parallel process graphs in the form of synchronized labeled transition systems.
In addition, CodeThorn can be used to explore the state space of parallel interleavings of such process graphs. 
When selecting \enquote{--ltl=mode=mine}, random LTL properties are mined based on subsets of the analyzed process graphs. 
These features have been used to generate benchmarks for the Parallel LTL track of the RERS Challenge~\cite{jasper2017rers}.
\begin{verbatim}
  --seed arg                            Seed value for randomly selected 
                                        integers (concurrency-related 
                                        non-determinism might still affect 
                                        results).
  --generate-automata arg               Generate random control flow automata 
                                        (file <arg>) that can be interpreted 
                                        and analyzed as a parallel program.
  --num-automata arg                    Select the number of parallel automata 
                                        to generate.
  --num-syncs-range arg                 Select a range for the number of random
                                        synchronizations between the generated 
                                        automata (csv pair of integers).
  --num-circles-range arg               Select a range for the number of 
                                        circles that a randomly generated 
                                        automaton consists of (csv pair of 
                                        integers).
  --circle-length-range arg             Select a range for the length of 
                                        circles that are used to construct an 
                                        automaton (csv pair of integers).
  --num-intersections-range arg         Select a range for the number of 
                                        intersections of a newly added circle 
                                        with existing circles in the automaton 
                                        (csv pair of integers).
  --automata-dot-input arg              Reads in parallel automata with 
                                        synchronized transitions from a given 
                                        .dot file.
  --keep-systems [=arg(=1)] (=0)        Store computed parallel systems (over- 
                                        and under-approximated STGs) during 
                                        exploration  so that they do not need 
                                        to be recomputed.
  --use-components arg                  Selects which parallel components are 
                                        chosen for analyzing the (approximated)
                                        state space ([all] | subsets-fixed | 
                                        subsets-random).
  --fixed-subsets arg                   A list of sets of parallel component 
                                        IDs used for analysis (e.g. 
                                        "{1,2},{4,7}"). Use only with 
                                        "--use-components=subsets-fixed".
  --num-random-components arg           Number of different random components 
                                        used for the analysis. Use only with 
                                        "--use-components=subsets-random". 
                                        Default: min(3, <num-parallel-component
                                        s>)
  --parallel-composition-only [=arg(=1)] (=0)
                                        If set to "yes", then no approximation 
                                        will take place. Instead, the parallel 
                                        compositions of the respective 
                                        sub-systems will be expanded 
                                        (sequentialized). Skips any LTL 
                                        analysis. ([yes|no])
  --num-components-ltl arg              Number of different random components 
                                        used to generate a random LTL property.
                                        Default: value of option 
                                        --num-random-components (a.k.a. all 
                                        analyzed components)
  --minimum-components arg              Number of different parallel components
                                        that need to be explored together in 
                                        order to be able to analyze the mined 
                                        properties. (Default: 3).
  --different-component-subsets arg     Number of random component subsets. The
                                        solver will be run for each of the 
                                        random subsets. Use only with 
                                        "--use-components=subsets-random" 
                                        (Default: no termination).
  --ltl-mode arg                        "check" checks the properties passed to
                                        option "--check-ltl=<filename>". "mine"
                                        searches for automatically generated 
                                        properties that adhere to certain 
                                        criteria. "none" means no LTL analysis 
                                        (default).
  --mine-num-verifiable arg             Number of verifiable properties 
                                        satisfying given requirements that 
                                        should be collected (Default: 10).
  --mine-num-falsifiable arg            Number of falsifiable properties 
                                        satisfying given requirements that 
                                        should be collected (Default: 10).
  --minings-per-subsets arg             Number of randomly generated properties
                                        that are evaluated based on one subset 
                                        of parallel components (Default: 50).
  --ltl-properties-output arg           Writes the analyzed LTL properties to 
                                        file <arg>.
  --promela-output arg                  Writes a promela program reflecting the
                                        synchronized automata of option 
                                        "--automata-dot-input" to file <arg>. 
                                        Includes LTL properties if analyzed.
  --promela-output-only [=arg(=1)] (=0) Only generate Promela code, skip 
                                        analysis of the input .dot graphs.
  --output-with-results [=arg(=1)] (=0) Include results for the LTL properties 
                                        in generated promela code and LTL 
                                        property files .
  --output-with-annotations [=arg(=1)] (=0)
                                        Include annotations for the LTL 
                                        properties in generated promela code 
                                        and LTL property files .
  --verification-engine arg             Choose which backend verification 
                                        engine is used (ltsmin|[spot]).
\end{verbatim}


\subsection{Visualization}
Transition graphs and other data structures can be visualized using the follwing command line options.
The main option that activates most of the visualization features is \verb+--viz=yes+.
\begin{verbatim}
  --rw-clusters [=arg(=1)] (=0)         Draw boxes around data elements from 
                                        the same array (read/write-set graphs).
  --rw-data [=arg(=1)] (=0)             Display names of data elements 
                                        (read/write-set graphs).
  --rw-highlight-races [=arg(=1)] (=0)  Highlight data races as large red dots 
                                        (read/write-set graphs).
  --dot-io-stg arg                      Output STG with explicit I/O node 
                                        information in dot file <arg>.
  --dot-io-stg-forced-top arg           Output STG with explicit I/O node 
                                        information in dot file <arg>. Groups 
                                        abstract states together.
  --tg1-estate-address [=arg(=1)] (=0)  Transition graph 1: Visualize address.
  --tg1-estate-id [=arg(=1)] (=1)       Transition graph 1: Visualize 
                                        estate-id.
  --tg1-estate-properties [=arg(=1)] (=1)
                                        Transition graph 1: Visualize all 
                                        estate-properties.
  --tg1-estate-predicate [=arg(=1)] (=0)
                                        Transition graph 1: Show estate as 
                                        predicate.
  --tg2-estate-address [=arg(=1)] (=0)  Transition graph 2: Visualize address.
  --tg2-estate-id [=arg(=1)] (=0)       Transition graph 2: Visualize 
                                        estate-id.
  --tg2-estate-properties [=arg(=1)] (=0)
                                        Transition graph 2: Visualize all 
                                        estate-properties.
  --tg2-estate-predicate [=arg(=1)] (=0)
                                        Transition graph 2: Show estate as 
                                        predicate.
  --visualize-read-write-sets [=arg(=1)] (=0)
                                        Generate a read/write-set graph that 
                                        illustrates the read and write accesses
                                        of the involved threads.
  --viz [=arg(=1)] (=0)                 Generate visualizations (.dot) outputs.

\end{verbatim}


\subsection{Data Race Detection}
Options for data race detection.
\begin{verbatim}
  --data-race [=arg(=1)] (=0)      Perform data race detection.
  --data-race-csv arg              Write data race detection results in 
                                   specified csv file <arg>. Implicitly enables
                                   data race detection.
  --data-race-fail [=arg(=1)] (=0) Perform data race detection and fail on 
                                   error (codethorn exit status 1). For use in 
                                   regression verification. Implicitly enables 
                                   data race detection.
\end{verbatim}


\subsection{Information}
The following option allows to display additional information about the analysis.
\begin{verbatim}
  --print-varid-mapping [=arg(=1)] (=0) Print all information stored in var-id 
                                        mapping after analysis.

\end{verbatim}


\section{Analysis Overview}

The analysis is performed in five phases:

\begin{enumerate}
\item Syntactic and semantic analysis of the input program (ROSE). The program is analyzed and represented in memory as an annotated abstract syntax tree (AST).
\item Control flow analysis. We compute a control flow graph (CFG) for the AST. Transitions, as computed for the state transition system in the next phase, correspond to edges in the CFG.
\item Computation of the state transition system.
\item LTL checking. Input to the LTL checking phase are the state transition system and the LTL formulae.
\item Reporting of analysis results. Reachability of failing assertions or verification errors is computed based 
on the transition system. Results for LTL formulae are computed solely by the LTL checker.
\end{enumerate}

\noindent States of the analyzed program are represented as follows:

\subsection{Program State}

A program state consists of a label (lab), a property state (pstate),
a constraint set (cset), and an IO property (io). $PState = Var
\rightarrow Val$ where $Val$ is either a constant or $top$. $Val$ is a
lifted integer set. An execution state is defined as $EState = Lab \times
PState \times Constraints \times IO$ where $Constraints$ is a set of
constraints, and $IO$ determines whether one of the variables in $PState$ is
an input or output variable. More specifically, whether a variable is
read from stdin, or printed to stdout or stderr. Furthermore it
determines whether the state produces an output which is caused by a
failed assert.

\bibliographystyle{plain}
\bibliography{codethorn}

\end{document}
