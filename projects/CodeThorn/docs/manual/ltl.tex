% !TEX root = codethorn.tex
\section{Verifying LTL formul\ae}

\newcommand{\ffalse}{\ensuremath{\mathit{false}}}
\newcommand{\ttrue}{\ensuremath{\mathit{true}}}

The current LTL formula checker in CodeThorn is an experiment in how
far a simple dataflow-based approach implemented on the State
Transition Graph can go. The algorithm consists of two layers. At the
outer layer, the abstract syntax tree of the formula is traversed in
bottom-up order. Every sub-expression of the LTL formula found by the
traversal is reduced to a single value in the Boolean lattice (\cf
Fig.~\ref{fig:bool_lattice}) for each state in the State Transition
Graph. This evaluation is performed using a dataflow analysis, which
represents the inner layer of the algorithm.

\begin{figure}
  \centering
   \begin{tikzpicture}[scale=.9]
    \small
    \node (top) {$\top$} 
    child {node (-1)    {\ffalse}}
    child {node (0)     {}
      edge from parent[draw=none]
      child {node (bot) {$\bot$} 
      edge from parent[draw=none]
      }
    }
    child {node (1)    {\ttrue}} 
    ; 
    \draw (-1)    -- (bot);
    \draw (1)     -- (bot);
  \end{tikzpicture} 
  \caption{Boolean lattice the LTL formul\ae\ are reduced to}
  \label{fig:bool_lattice}
\end{figure}

\newcommand{\lub}{\ensuremath{\sqcup}\xspace}
\newcommand{\Lub}{\ensuremath{\bigsqcup}\xspace}
\newcommand{\state}{\ensuremath{\mathit{s}}\xspace}
\newcommand{\STG}{\ensuremath{\mathrm{STG}}\xspace}
\newcommand{\States}{\ensuremath{\mathit{States}}\xspace}
\newcommand{\prop}[1]{\ensuremath{p_{\state,#1}}\xspace} 
\newcommand{\propp}[1]{\ensuremath{p_{\state',#1}}\xspace} 
\newcommand{\G}{\ensuremath{\mathrm{G}}\xspace}
\newcommand{\F}{\ensuremath{\mathrm{F}}\xspace}
\newcommand{\X}{\ensuremath{\mathrm{X}}\xspace}
\newcommand{\R}{\ensuremath{\mathrm{R}}\xspace}
\newcommand{\U}{\ensuremath{\mathrm{U}}\xspace}
\newcommand{\WU}{\ensuremath{\mathrm{WU}\xspace}}
\newcommand{\comb}{\ensuremath{\mathit{succs}}\xspace}

\subsection{A minimum-fixpoint approximation for all-quantified LTL expressions}
The approach we are using is extremely fast, but it computes only an
approximation (a safe upper bound in the Boolean lattice) of the
precise verification result. The approximation happens whenever
multiple paths join at a state, in which case the least upper bound
\lub\ of the results from the individual paths is propagated.

The remainder of this section will discuss how the transfer functions
for each element of the LTL grammar are defined. We are using the
following notational conventions. The
$\STG={\mathit{States},\mathit{Transitions},\mathit{Start}}$ is the
(reduced) State Transition Graph (or Kripke Model
\citep[pg. 27ff]{Clarke1999}). For each state \state\ we have an array of $n$
properties \prop{i}, $i \in [0\dots n]$, where $n+1$ is the number of
sub-expressions in the LTL formula.

\subsubsection{Boolean Operators: \texttt{!}, \texttt{\&}, \texttt{|}}
The Boolean operators over LTL expressions have the least
computationally complex transfer functions as their effect is only
local to each state.

\begin{align*}
e &= !a  & \forall\state\in\States\colon\prop{e} &:= \neg\prop{a}\\
e &= a \& b & \forall\state\in\States\colon\prop{e} &:= \prop{a} \cap \prop{b}\\
e &= a | b & \forall\state\in\States\colon\prop{e} &:= \prop{a} \cup \prop{b}\\
\end{align*}

\subsubsection{The G operator}
The global operator $\G a$ yields true iff the sub-expression $a$ is
true at all states. The transfer function is defined as follows:

\[ e = \G a\colon\qquad\forall\state\in\States\colon\prop{e}
:=\prop{a}\cap\Lub_{\state\prime\in succ(\state)}{\propp{e}} \]

Since we only want to handle all-quantified LTL expressions, it is
safe to join the information from multiple paths using the \lub
operator: If all successors of a node share the yield the same result,
by induction, all paths starting at \state have identical results at
every node, thus we can merge their heads at \state. If the results of
the successors diverge, \state will go to $\top$, (the safe
approximation) and the result at \state is unknown.

Due to the way the logical operations on Boolean lattice are defined,
not all is lost for these states. For example, $\top \vee \ttrue
\equiv \ttrue$ and $\top \wedge \ffalse \equiv \ffalse$, so it is
possible for these unknown states to still be overruled by other
states for which we do have a precise result.

To calculate the solution of the transfer function over the entire
\STG, we perform a fixpoint search over the entire graph; this is
implemented as a backwards-directed dataflow analysis. Fist,
every state is initialized to $\bot$. The join operator is $\lub$. The
transfer function uses the Boolean lattice $\cap$ operation to combine
$a$ with the results of the joined successors. The fixpoint working
set is initialized to all final states $\in\STG$ plus all states where
$e\neq\bot$, because the \STG may contain infinite loops that are
otherwise not reachable from a final state. 	

\subsubsection{The X operator}
The next operator $X a$ yields true iff $a$ is true in the next state.
\[ e = \X a\colon\qquad\forall\state\in\States\colon\prop{e}
:=\Lub_{\state\prime\in succ(\state)}{\propp{a}} \]

\subsubsection{The F, R, and U operators}
The following table shows the backwards-directed dataflow
specification for the \F, \R, and \U operators.

\begin{tabular}{lllll}
\toprule
Operator & $\mathit{init}$ & $\mathit{start?}$ & $\mathit{join}$ & $\mathit{calc}$ \\\midrule

$e = \F a$    & $\bot$ & $(\prop{a}\neq\bot)$ & $\lub$ & $\prop{a}\cup\comb$ \\
$e = a\ \R b$ & $\bot$ & $(\prop{b}\neq\bot)$ & $\lub$ & $\prop{b}\cap(\prop{a}\cup\comb)$\\
$e = a\ \U b$ & $\bot$ & $(\prop{b}\neq\bot)$ & $\lub$ & $\prop{b}\cup(\prop{a}\cap\comb)$\\
\bottomrule
\end{tabular}

\subsubsection{The WU operator}
The WU operator is syntactically rewritten into the equivalent
\[ e = a\ \WU b\quad\equiv\quad \G a \vee (a\ \U b) \]
expression.

\subsubsection{Handling Output states}
\begin{itemize}
\item are propagated to the next I/O state of the same type
\end{itemize}

\subsubsection{Handling Output states}
We use the IO attribute in the \STG to determine whether a state is an
output state. If the state has a \verb|STDOUT_CONST| or
\verb|STDOUT_VAR| attribute, and, in the latter case, the variable
being output is constrained to an integer value, the property \verb|O|
$\mathit{value}$ is \ttrue. If the state performs I/O, but the
constraints are inconclusive, all \verb|O| properties are $\top$,
otherwise, $\bot$. The \STG treats assertions as I/O, which implies
that dead ends in the input program (paths without traditional I/O
leading to an exit) will be reported as performing no I/O at all.

Negated I/O
\subsection{Reducing the State Transition Graph}

\subsection{Debugging: Finding counterexamples}