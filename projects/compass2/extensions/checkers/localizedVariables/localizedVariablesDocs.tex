% 9.5.07
% This is a sample documentation for Compass in the tex format.
% We restrict the use of tex to the following subset of commands:
%
% \section, \subsection, \subsubsection, \paragraph
% \begin{enumerate} (no-nesting), \begin{quote}, \item
% {\tt ... }, {\bf ...}, {\it ... }
% \htmladdnormallink{}{}
% \begin{verbatim}...\end{verbatim} is reserved for code segments
% ...''
%

\section{Localized Variables}
\label{LocalizedVariables::overview}

This checker looks for variable declarations and try to determine if
the first use is far far the declaration. There can be two ways where
the use is far. Either it is used only in an inner scope. Then the
declaration can be moved in that scope. Or the it is used not directly
after the block of declaration the variables is from. Then the
declaration should be moved down.

This checker try to check for item 18 of \emph{C++ Coding Standards}
(Sutter and al., 2005).

\subsection{Parameter Requirements}

   No parameter is needed.

\subsection{Non-Compliant Code Example}

\begin{verbatim}
void print(int);

void f()
{
  int i; //i should be declared at the for loop.
  int sum = 0; //sum is OK.

  //i is only used in the for scope.
  for (i = 0; i < 10; ++i) {
    //sum is used right after the block of declaration it belongs.
    sum += i;
  }

  //sum is used in the scope of definition.
  print(sum);
}
\end{verbatim}

\subsection{Compliant Solution}

\begin{verbatim}
void print(int);

void f()
{
  int sum = 0;

  for (int i = 0; i < 10; ++i) {
    sum += i;
  }

  print(sum);
}
\end{verbatim}

\subsection{Mitigation Strategies}
\subsubsection{Static Analysis}

The checker uses a scoped symbol table to track some properties about
variable: Was the variable used? Was the variable in the same scope as
its declaration? Was the variable used right after the the
declaration? Is the declaration right before the current point of the
traversal? The traversal updates these flags. Once a scope finished to
be traversed, since the variables declared in the scope cannot be used
any further, they are checked for the flags: used, used in the same scope
and used right after its declaration.

There is another flag saying if the variable is a constant. In this
case only check that the variable have been used, whereever it is.

This implementation does not care about aliasing. For more
informations see subsection
\ref{section:LocalizedVariablesChecker::limitations}.

\subsection{References}

Alexandrescu A. and Sutter H. {\it C++ Coding Standards 101 Rules, Guidelines, and Best Practices}. Addison-Wesley 2005.

\subsection{Limitations}
\label{section:LocalizedVariablesChecker::limitations}

This checker does not do alias analyzing. In the case you use a reference
of a variable in an inner scope and re-use in the scope of declaration,
the checker will not see the use and propose to move the variable in the
inner scope. The effort for supporting the problem is very big, and the
result is to handle programs with weird behavior that should have been
written differently.

