%General Suggestion: Ideally, the documentation of a style checker should 
%be around one page.
\section{Paper: Cyclomatic Complexity}

This is a checker to detect functions with high complexity.
High complexity is defined by Mc Cabe's cyclomatic complexity metric.
This metric measures the amount of branches in a function, i.e.
through if and switch conditions.

\subsection{Non-Compliant Code Examples}
\begin{verbatim}
void fail() {
  int x;
  x=5;
  if (x>3) {
  }
  if (x>3) {
  }
  if (x>3) {
  }
}
\end{verbatim}

\subsection{Compliant Solution}
\begin{verbatim}
void pass() {
  int x;
  x=5;
  if (x>3) {
  }
}
\end{verbatim}



\subsection{Parameter Requirements}

CyclomaticComplexity.maxComplexity defines the max value for the complexity analysis.

\subsection{Implementation}

The algorithm searches for each function the number of occurances of:

\begin{verbatim}
  if (isSgIfStmt(node) || isSgCaseOptionStmt(node) || isSgForStatement(node) || isSgDoWhileStmt(node) || isSgWhileStmt(node)) {
    complexity++;
  }
\end{verbatim}

\subsection{References}

\htmladdnormallink{McCabe}{} , ``Thomas McCabe, A Complexity Measure - IEEE Transactions on Software Engineering, Vol SE-2, No.4, December 1976.''


