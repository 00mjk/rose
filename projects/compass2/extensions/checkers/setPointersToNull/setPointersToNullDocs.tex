% 9.5.07
% This is a sample documentation for Compass in the tex format.
% We restrict the use of tex to the following subset of commands:
%
% \section, \subsection, \subsubsection, \paragraph
% \begin{enumerate} (no-nesting), \begin{quote}, \item
% {\tt ... }, {\bf ...}, {\it ... }
% \htmladdnormallink{}{}
% \begin{verbatim}...\end{verbatim} is reserved for code segments
% ...''
%

\section{Set Pointers To Null}
\label{SetPointersToNull::overview}

% write your introduction
Dangling pointers can lead to exploitable double-free and access-freed-memory
vulnerabilities. A simple yet effective way to eliminate dangling pointers and
avoid many memory related vulnerabilities is to set pointers to NULL after they
have been freed. Calling free() on a NULL pointer results in no action being
taken by free().

\subsection{Parameter Requirements}

%Write the Parameter specification here.
   No Parameter specifications.

\subsection{Implementation}

%Details of the implementation go here.

\subsection{Non-Compliant Code Example}

% write your non-compliant code subsection
In this example, the type of a message is used to determine how to process the
message itself. It is assumed that message\_type is an integer and message is a
pointer to an array of characters that were allocated dynamically. If
message\_type equals value\_1, the message is processed accordingly. A similar
operation occurs when message\_type equals value\_2. However, if message\_type
== value\_1 evaluates to true and message\_type == value\_2 also evaluates to
true, then message will be freed twice, resulting in an error.

\begin{verbatim}

if (message\_type == value\_1) {
  /* Process message type 1 */
  free(message);
}
/* ...*/
if (message\_type == value\_2) {
   /* Process message type 2 */
  free(message);
}

\end{verbatim}

\subsection{Compliant Solution}

% write your compliant code subsection
As stated above, calling free() on a NULL pointer results in no action being
taken by free(). By setting message equal to NULL after it has been freed, the
double-free vulnerability has been eliminated.

\begin{verbatim}

if (message_type == value_1) {
  /* Process message type 1 */
  free(message);
  message = NULL;
}
/* ...*/
if (message_type == value_2) {
  /* Process message type 2 */
  free(message);
  message = NULL;
}

\end{verbatim}

\subsection{Mitigation Strategies}
\subsubsection{Static Analysis} 

Compliance with this rule can be checked using structural static analysis checkers using the following algorithm:

\begin{enumerate}
\item Write your checker algorithm
\end{enumerate}

\subsection{References}

% Write some references
% ex. \htmladdnormallink{ISO/IEC 9899-1999:TC2}{https://www.securecoding.cert.org/confluence/display/seccode/AA.+C+References} Forward, Section 6.9.1, Function definitions''

\htmladdnormallink{ISO/IEC 9899-1999}{https://www.securecoding.cert.org/confluence/display/seccode/MEM01-A.+Eliminate+dangling+pointers} MEM01-A. Eliminate dangling pointers
