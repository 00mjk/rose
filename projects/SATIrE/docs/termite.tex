\documentclass[10pt,twoside]{scrreprt}
\usepackage{pldoc}

% Dummy condition for HTML vs. PDF
\def \xMode \xPDF

\usepackage{pl}
\usepackage{logo}
\usepackage{html}
\usepackage{plpage}

\makeindex
\ifx \xMode \xHTML

% Configure latex2html
\htmloutput{termite-html} % Output directory
\htmlmainfile{index} % Main document file
\bodycolor{white} % Page colour
\linkimage{home}{home.gif} % Images for links
\linkimage{index}{yellow_pages.gif}
\linkimage{contents}{index.gif}
\linkimage{up}{up.gif}
\linkimage{previous}{prev.gif}
\linkimage{next}{next.gif}
\linkimage{summary}{info.gif}

\fi

\title{The TERMITE library}

\author{Adrian Prantl\\
  Institut f\"{u}r Computersprachen\\
  Technische Universit\"{a}t Wien\\
  E-Mail: \email{adrian@complang.tuwien.ac.at}
}

\date{\today}

\begin{document}

\maketitle

\tableofcontents

\chapter{Introduction}

The TERM Iteration and Transformation Environment (Termite) is a
Prolog library that allows easy manipulation and analysis of C++
programs. It is particularly well suited to specify source-to-source
program transformations, static program analyses and program
visualizations. Termite builds upon the intermediate representation of
SATIrE. More information on SATIrE can be found at
\url{http://www.complang.tuwien.ac.at/satire}.

\section{The SATIrE term representation}

SATIrE can export an external term representation of the abstract
syntax tree (AST) of a C++ program. This term representation contains
all information that is necessary to correctly unparse the program,
including line and column information of every expression. The terms
are also annotated with the results of any preceding PAG analysis. The
syntax of the term representation was designed to match the syntax of
Prolog terms. This allows it to be manipulated by Prolog programs very
naturally.

\subsection{Grammar of TERMITE terms}

% DO NOT REMOVE THE FOLLOWING LINE:
%% GRAMMAR:
% THANK YOU
\chapter{Library Reference}

% DO NOT REMOVE THE FOLLOWING LINE:
%%INPUTS%%
% THANK YOU

\printindex

\end{document}
