\documentclass[10pt,twoside]{scrreprt}
\usepackage{pldoc}

% Dummy condition for HTML vs. PDF
\def \xMode \xPDF
\usepackage{logo}
\usepackage{pl}
\usepackage{html}
\usepackage{plpage}

\makeindex
\ifx \xMode \xHTML

% Configure latex2html
\htmloutput{termite-html} % Output directory
\htmlmainfile{index} % Main document file
\bodycolor{white} % Page colour
\linkimage{home}{home.gif} % Images for links
\linkimage{index}{yellow_pages.gif}
\linkimage{contents}{index.gif}
\linkimage{up}{up.gif}
\linkimage{previous}{prev.gif}
\linkimage{next}{next.gif}
\linkimage{summary}{info.gif}

\fi

\title{The TERMITE library}

\author{Adrian Prantl\\
  Institut f\"{u}r Computersprachen\\
  Technische Universit\"{a}t Wien\\
  E-Mail: \email{adrian@complang.tuwien.ac.at}
}

\date{\today}

\begin{document}

\maketitle

\tableofcontents

\chapter{Introduction}

The TERM Iteration and Transformation Environment (Termite) is a
Prolog library that allows easy manipulation and analysis of C++
programs. It is particularly well suited to specify source-to-source
program transformations, static program analyses and program
visualizations. Termite builds upon the intermediate representation of
SATIrE. More information on SATIrE can be found at
\url{http://www.complang.tuwien.ac.at/satire}.

\section{Using Termite}

Depending on the desired architecture there are several ways to
integrate Termite into the work flow of a larger tool. For a flexible
recombination of several analyses and/or transformations it is best to
treat Termite programs as interpreted scripts that read/write AST
terms from the standard input an output. If performance and stability
are sought for, it is also possible to call Termite programs
transparently from a SATIrE analyzer.

\subsection{Using Termite for a standalone process}
The most flexible and convenient way to work with the Termite library
is by using it to define filter operations on streams of source
code. This way one can follow the UNIX tradition of having a
collection of small self-contained programs that can be combined to
create larger work flows. Depending on the expected input and
generated output, several types of Termite programs can be
distinguished. Typical examples are:
\begin{description}
\item[A source-to-source transformer] is a program that reads in an
  AST, then performs some transformation and outputs the transformed
  AST. (Example: loop unrolling)
\item[An analyzer] is a program that reads in an AST, performs some
  analysis and outputs the analysis result as attributes of the
  AST. (Example: loop bound analysis)
\item[A visualization] is a program that reads in an AST and
  outputs a visualization, e.\,g., in a GUI window or a PostScript
  file. (Example: Call-graph $\rightarrow$ Graphviz (DOT))
\item[A source generator] is a program that reads in a specification
  and outputs an AST in termite format.
\item[A compiler] is a program that translates an AST into a different
  language,                                                     e.\,g.,
  melmac\footnote{\url{http://www.complang.tuwien.ac.at/gergo/melmac/}}
  or wcetcc.
\end{description}
In order to generate a Termite term from one or more source files a
compiler front end must be invoked. Two possibilities are supported
and available in the SATIrE distribution:

\subsubsection{EDG C/C++ front end from the ROSE compiler}
If SATIrE was configured with the ROSE connection
enabled\footnote{ROSE must be installed separately beforehand and is
  available at \url{http://www.rosecompiler.org}}, conversion tools
are available to translate source code to term files and vice versa.
To translate source code into a Termite term the program
\texttt{c2term} is available:

\paragraph{$>$ c2term}
% DO NOT REMOVE THE FOLLOWING LINE:
%% C2TERM:
% THANK YOU
The \texttt{c2term} program invokes the commercial EDG C++ front end
embedded into the ROSE compiler to parse one or more source files. The
abstract syntax tree (AST) is then translated into the ROSE immediate
representation which in turn is converted into the textual term
serialization. The program passes additional options to the EDG front
end.

The opposite direction is managed by the \texttt{term2c} conversion
utility. It works by reading in a term file and then rebuilding the
ROSE intermediate representation. Finally, this data structure is
passed to the ROSE unparser. The EDG front end is not involved in this
step any more.

\paragraph{$>$ term2c}
% DO NOT REMOVE THE FOLLOWING LINE:
%% TERM2C:
% THANK YOU

\bigskip

Since both converters use standard input and output per default it is
possible to concatenate multiple Termite programs with the help of
UNIX pipes. This way it is possible to build new chains of program
transformations or analyzers on the fly without having to recompile
the whole project.

\paragraph{Example:}
\begin{verbatim}
c2term a.c b.c | ./transform1.pl | term2c -s '.transformed'
\end{verbatim}
In this example pipeline, two C source file are joined into one
project which is dumped to a stream in the Termite format. The stream
is then transformed by a Prolog program. Finally the two source files
are unparsed by the \verb|term2c| converter with the new suffix
``.transformed'' attached to the file names.

\subsubsection{Using the Clang C/Objective C front end}
While the commercial EDG front end offers a high-quality C++ parser,
license restrictions encumber its free distribution together with
other tools. Most notably, the ROSE compiler redistributes only a
32-bit precompiled binary version of the EDG front end. It is,
however, possible to buy other licenses from the Edison Design Group.

If C++ support is not needed, there is a free alternative available
from the LLVM compiler project. Designed especially for use with LLVM
a front end for C-like languages called \texttt{clang} is published
under a BSD-style license. The \texttt{clang} front end can be
downloaded at \url{http://clang.llvm.org/}. The front end is written
in C++ and creates an intermediate representation very similar to that
of ROSE and therefore makes a good candidate to replace the EDG front
end in SATIrE. The C99 and Objective C languages are supported very
well by \texttt{clang}, whereas C++ support is still under
development.

In order to connect SATIrE with the \texttt{clang} front end, we
decided to take the route via the Termite representation. This way,
the front end is cleanly decoupled from the rest of the system and
uses the Termite terms as a stable interface. The Termite term
generator is implemented as a pass over the \texttt{clang}
intermediate representation and is available via the
\verb|-emit-term| command line option. The term generator is not
integrated with upstream \texttt{clang}, but distributed as a patch
against a current SVN version together with SATIrE. 

To build the \texttt{clang} front end for use with SATIrE a special
\verb|make clang| target is available at the toplevel which fetches
the needed version of \texttt{clang} from the subversion repository,
applies the patch, and compiles and installs the patched front end to
\verb|$prefix/bin|.

\subsubsection{Uparsing Termite terms without SATIrE}
Invoking the \texttt{term2c} program is sometimes too cumbersome, for
example, when only a few expressions should be unparsed for debugging
purposes. For these occasions an independent term$\rightarrow$C
converter is implemented in pure Prolog and available both in the
Termite library and as a stand-alone script. The predicate is called
\verb|unparse/1| and expects a Termite term as argument.

\subsection{As part of a SATIrE analyzer}

If execution speed is an issue, the steps of writing the Termite
representation to disk (or a pipe) and parsing the terms (which, when
output as a text, are significantly larger than the original source
files) can be optimized away. If SATIrE was configured with SWI-Prolog
support enabled, the term representation will be built in memory using
the external interface of an embedded SWI-Prolog interpreter. Using
this in-memory term, a Termite program can be executed without leaving
the current process. The resulting term can again be translated to the
ROSE intermediate representation directly from memory using the
SWI-Prolog interface.

Using this work flow, the whole analyzer (or transformer, ...) can be
distributed as a single self-contained executable.

\chapter{The Termite  term representation}

SATIrE can export an external term representation of the abstract
syntax tree (AST) of a C++ program. This term representation contains
all information that is necessary to correctly unparse the program,
including line and column information of every expression. The terms
are also annotated with the results of any preceding PAG analysis. The
syntax of the term representation was designed to match the syntax of
Prolog terms. This allows it to be manipulated by Prolog programs very
naturally.

\section{Grammar of TERMITE terms}

The following section gives a formal definition of the grammar of
Termite terms as it is used by the program \verb|termite_lint|
which can be used to verify the validity of arbitrary terms.

% DO NOT REMOVE THE FOLLOWING LINE:
%% GRAMMAR:
% THANK YOU

\chapter{Library Reference}

% DO NOT REMOVE THE FOLLOWING LINE:
%%INPUTS%%
% THANK YOU

\printindex

\end{document}
