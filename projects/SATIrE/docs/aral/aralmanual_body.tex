% paper structure
% introduction     :OK
% architecture     :missing
% Example          :in prep, SATIrE specific
% Related work     :OK
% Conlusion        :OK

% C++ Analysis Tools: http://decomp.ulb.ac.be/roelwuyts/playground/canalysistools/


\outcomment{
\begin{abstract} 

The analysis results annotation language (ARAL) is designed to be a general format that allows
the exchange of analysis information between program analysis tools either in a separate or
through program annotations. It is suitable for representing flow-sensitive and context-sensitive
information and aims at supporting a wide range of analyses with focus on data-flow analysis and
abstract interpretation in general. For example, the language is general enough to represent any
analysis information that is computed by AbsInt's Program Analyzer Generator (PAG). Beside
exchange of analysis information, the purpose of the language also is the support of testing
analyzers and allowing to manually add annotations, as is important for worst-case execution
time analysis. The development of ARAL is motivated by the need of making analysis results
persistent for enabling whole-program analysis of very-large software, and the goal of information
exchange between program analysis tools.

\end{abstract}
}

\chapter{Introduction}
\label{section:aral}
The analysis results annotation language is designed to be suitable
for annotating flow-sensitive and context-sensitive analysis
results. It allows to represent computed data of analysis tools, but
without putting restrictions on the semantics of an analysis. Clearly,
the annotations allow to represent analysis result data, but the semantics of
the data are defined for each analysis separately. For example, the
language is general enough to represent analysis information specified
in PAG's %\cite{Martin98} 
DATLA language. Since PAG is integrated into
SATIrE this is a requirement for reusing any analysis information
computed with a PAG generated analyzer. ARAL consists of sets, lists,
tuples, maps, and some primitive data types. Additionally each
analysis information has an Analysis Identifier. This identifier
allows to associate semantics with the analysis information.

The analysis results annotation language (ARAL) is designed to be a
general format that allows to exchange analysis information between
\satire{} and other tools. 

%Therefore it allows to exchange WCET
%analysis specific information, but also information for supporting
%analyses (e.g. interval analysis is an supporting analysis for loop
%bound analysis). This capability is necessary for use case R.

ARAL is suitable for annotating flow-insensitive as well as
flow-sensitive and context-insensitive as well as context-sensitive
analysis results. It allows to represent computed data of analysis
tools, but without putting restrictions on the semantics of an
analysis. Clearly, the annotations allow to represent analysis result
data, but the semantics of the data are defined for each analysis
separately. For example, the language is general enough to represent
analysis information specified in AbsInt's PAG DATLA language. Since
PAG is integrated into \satire{} this is a requirement for reusing any
analysis information computed with a PAG generated analyzer. ARAL
consists of sets, lists, tuples, maps, and some primitive data
types. Additionally each analysis information has an Analysis
Identifier. This identifier allows to associate semantics with the
analysis information.

%To associate the analysis information with a program fragment
%a label and a mapping exists to associate the analysis information with
%a program fragment.

%\outcomment{

\chapter{Language}

TODO: the language shall be described by discussion of type-system
and data-representation only. The mapping and file structure shall be
explained in the file format chapter. ARAL is for administration
of analysis results (combination of files etc.). A COMPUTATION section will allow compact representation of analysis results.\\

ARAL is a strictly typed language. A data element can only have
exactly one type and there is no void type. For a proper use of the
language it is important to be aware of the concept of the mapping
section and the result section(s). In a result section analysis data
is reqpresented independent from the intermediate representation on
which is was computed. The mapping section defines how the analysis
information of a result section is associated with locations in a
source code of a program and similarly, how it is associated with
locations in an intermediate representation. The two different kinds
of sectionos support a separation of concerns: the analysis data
itself and how it is associated with locations in a program. The
locations are labeled, and those labels are used in the data
section. For expressions (and its subexpressions) IDs are computed
that those IDs are used in the DATA section. A mapping section can
provide information on how to map the IDs to source code constructs.

\section{ARAL Grammar}

{
\newcommand{\regexstar}[1]{\{~#1~\}$^*$}
\newcommand{\regexplus}[1]{\{~#1~\}$^+$}
\newcommand{\regexgroup}[1]{\{~#1~\}}
\newcommand{\regexopt}[1]{[~#1~]}
\newcommand{\lexrange}[2]{\underline{#1}-\underline{#2}}
\newcommand{\lb}[0]{\underline{(}~}
\newcommand{\rb}[0]{\underline{)}}
\newcommand{\comma}[0]{\underline{,}~}
\newcommand{\garrow}[0]{$\rightarrow$}

\newenvironment{GRAMMAR}                          % Prolog Umgebung
{
\begin{minipage}[t]{40ex}%
\prologcode\progfont%
\begin{quote}%
\begin{tabbing}%
\hspace{3ex}\=\hspace{3ex}\=\hspace{3ex}\=\hspace{3ex}%
\=\hspace{3ex}\=\hspace{4ex}\=\hspace{3ex}\=\hspace{3ex}\=\hspace{3ex}\=\hspace{3ex}\=\hspace{3ex}\=\hspace{3ex}\=\hspace{3ex}\kill}%
{\end{tabbing}%
\end{quote}%
\end{minipage}%
}

This subsection gives a precise definition of the ARAL syntax. The
grammar is specified in the usual notation for context-free grammars,
plus some convenient abbreviations:

\begin{itemize}
\item Non terminals start with a capital letter.
\item Terminals are underlined.
\item Optional constructs are enclosed in square brackets: [ Optional ]
\item Alternatives are separated by $\mid$: A $\mid$ B
\item Zero or more occurrences are enclosed in curly braces with a star: \regexstar{zero or more}
\item One or more occurrences take a '+': \regexplus{ one or more}
\item Sequences of constructs are separated by spaces.
\item Ranges are written with a dash, e.g. \underline{0}-\underline{9}.
\item Braces indicate grouping, e.g. \{ A $\mid$ B \} C
\item Some productions are explained with comments: \% comment.
\end{itemize}

Primitive tokens are 'Number' (\regexplus{\underline{0}-\underline{9}}) and identifiers, 'Id' (\regexgroup{\lexrange{A}{Z}$\mid$\lexrange{a}{z}}\regexstar{\underline{\_}$\mid$\lexrange{A}{Z}$\mid$\lexrange{a}{z}$\mid$\lexrange{0}{9}}).
ARAL comments start with a double slash and end at the end of the current line. Whitespace may be inserted anywhere between tokens.

The grammar productions can be found in Fig.~\ref{fig:annotationfileformat},\ref{fig:aralone}, \ref{fig:araltwo}.

\begin{figure}[!ht]
\begin{GRAMMAR}
{AralFile}      \>\>\>\>\>\garrow\> \underline{ANALYSIS} 
                 \>\>\>\>\> \> \regexopt{ConfigSection}
                 \>\>\>\>\> \> \regexopt{MappingSection} 
                 \>\>\>\>\> \> \regexopt{ResultSection} 
                 \>\>\>\>\> \>       \underline{END}

{ResultSection}    \>\>\>\>\>\garrow\> \underline{RESULT}
                 \>\>\>\>\> \> NameSection
                 \>\>\>\>\> \> TypeSection 
                 \>\>\>\>\> \> DataSection 
                 \>\>\>\>\> \> \underline{END}

{NameSection}    \>\>\>\>\>\garrow\> \underline{NAME} AnalaysisIdent

{ConfigSection}  \>\>\>\>\>\garrow\> \underline{CONFIG} ConfigType \underline{:} ConfigData \underline{;}

{MappingSection}  \>\>\>\>\>\>\garrow\> \underline{MAPPING} \regexstar{\underline{map}\lb~FromType\comma~Type \rb ~\underline{:} Map \underline{;}} 
                  \>\>\>\>\>\>\% keyword IDMAP is deprecated

{FromType}       \>\>\>\>\>\garrow\> Label
                 \>\>\>\>\> \> $\mid$ ProgId

{TypeSection}    \>\>\>\>\>\garrow\> \underline{TYPE} \regexstar{TypeDef} \regexopt{\underline{$\langle$} ContextType \underline{$\rangle$}} AnnotType 

{DataSection}    \>\>\>\>\>\garrow\> \underline{DATA} \regexstar{AnnotData}

{AnalysisIdent}  \>\>\>\>\>\garrow\> Id
{ConfigType}     \>\>\>\>\>\garrow\> Type
{ConfigData}     \>\>\>\>\>\garrow\> Data
{AnnotType}      \>\>\>\>\>\garrow\> Type
{ContextType}    \>\>\>\>\>\garrow\> Type

{TypeDef}        \>\>\>\>\>\garrow\> Id \underline{=} Type \underline{;}

{Type}           \>\>\>\>\>\garrow\> \underline{set}\lb Type \rb
                 \>\>\>\>\> \> $\mid$ \underline{list}\lb Type \rb
                 \>\>\>\>\> \> $\mid$ \underline{tuple}\lb Type$_1$,...,Type$_n$ \rb
                 \>\>\>\>\> \> $\mid$ \underline{flat} \lb Type \rb
                 \>\>\>\>\> \> $\mid$ \underline{lift} \lb Type \rb
                 \>\>\>\>\> \> $\mid$ \underline{map} \lb Type \comma Type \rb
%                 \>\>\>\>\> \> $\mid$ \underline{enum} \lb Id \regexstar{\comma Id} \underline{$\mid$} \regexopt{Edge \regexstar{\comma Edge}} \underline{$\mid$} Id \underline{$\mid$} Id \rb
                 \>\>\>\>\> \> $\mid$ \underline{label}
                 \>\>\>\>\> \> $\mid$ \underline{constraint}
                \>\>\>\>\> \> $\mid$ \underline{string}
                \>\>\>\>\> \> $\mid$ \underline{int(} Number \underline{)}
                 \>\>\>\>\> \> $\mid$ IdNumber
                 \>\>\>\>\> \> $\mid$ Basic
                 \>\>\>\>\> \> $\mid$ ProgId

{Label}           \>\>\>\>\>\garrow\> $\texttt{\underline{@}}$Number

{IdNumber}       \>\>\>\>\>\garrow\> $\underline{\texttt{\#}}$Number

%Edge             \>\>\>\>\>\garrow\> Id $\leq$ Id
\end{GRAMMAR}
\caption{Grammar for the ARAL Annotation File Format (Part 1, see Fig. \ref{fig:aralone} for productions of AnnotData, Basic, ProgId, and Map)}
\label{fig:annotationfileformat}
\end{figure}

\begin{figure}[!ht]
\begin{GRAMMAR}
{AnnotData}      \>\>\>\>\>\garrow\> LocationSpec FlowSpec InfoElement \underline{;}

{LocationSpec}   \>\>\>\>\>\garrow\> [ ExpPath ] LocSpecTarget

{LocSpecTarget}  \>\>\>\>\>\garrow\>Label
                 \>\>\>\>\> \>$\mid$ \underline{(} Label \underline{,} Label \underline{)} \hspace{2cm}\% not implemented 
                 \>\>\>\>\> \>$\mid$ \underline{program}
                 \>\>\>\>\> \>$\mid$ \underline{file} \underline{(}  String \underline{)} 
                 \>\>\>\>\> \>$\mid$ \underline{function} \underline{(}  String \underline{)} 

{ExpPath}        \>\>\>\>\>\garrow\> \underline{call} \underline{(} Number \underline{)} \underline{:} \regexopt{\underline{param} \underline{(} Number \underline{)}\underline{:}} \% not implemented

{FlowSpec}       \>\>\>\>\>\garrow\> \underline{pre} $\mid$ \underline{post} $\mid$ \underline{noflow}

{InfoElement}    \>\>\>\>\>\garrow\> \regexopt{\underline{$<$}Context\underline{$>$}} Data \regexstar{\underline{,} \underline{$<$}Context\underline{$>$} Data}

{Basic}          \>\>\>\>\>\garrow\> \underline{bool} $\mid$ \underline{string} $\mid$ \underline{real} $\mid$ NumRange

{ProgId}         \>\>\>\>\>\garrow\> \underline{funcid} $\mid$ \underline{declid} $\mid$ \underline{stmtid} $\mid$ \underline{exprid} $\mid$ \underline{varid}


{Context}        \>\>\>\>\>\garrow\> Data

{NumRange}       \>\>\>\>\>\garrow\> \underline{[} NumExp \underline{..} NumExp \underline {]} \% not implemented

{NumExp}         \>\>\>\>\>\garrow\> [-] Number
                 \>\>\>\>\> \>$\mid$ NumExp NumOperator NumExp
                 \>\>\>\>\> \>$\mid$ [-] \lb~NumExp \rb

{NumOperator}    \>\>\>\>\>\garrow\> \underline{+} $\mid$ \underline{-} $\mid$ \underline{*} $\mid$ \underline{/} $\mid$ \underline{\%} $\mid$ \underline{\symbol{"5E}}

\end{GRAMMAR}
\caption{ARAL Grammar (Part 2)}
\label{fig:aralone}
\end{figure}

\begin{figure}[!ht]
\begin{GRAMMAR}
{Data}    \>\>\>\>\>\garrow\> Set
                 \>\>\>\>\> \>$\mid$ List
                 \>\>\>\>\> \>$\mid$ Tuple
                 \>\>\>\>\> \>$\mid$ Map
                 \>\>\>\>\> \>$\mid$ $\underline{\$}$ Constraint $\underline{\$}$
                 \>\>\>\>\> \>$\mid$ Primitive

{Set}            \>\>\>\>\>\garrow\> \underline{\{} ElementSeq \underline{\}}

{List}           \>\>\>\>\>\garrow\> \underline{[} ElementSeq \underline{]}

{Tuple}          \>\>\>\>\>\garrow\> \lb ElementSeq \rb

{Map}            \>\>\>\>\>\garrow\> \underline{\{} $\underline{default} $ $\underline{\rightarrow}$ DefaultElem \regexopt{\underline{$\backslash$} MapElemSeq} \underline{\}}

{DefaultElem} \>\>\>\>\>\garrow\> Data

{MapElemSeq}     \>\>\>\>\>\garrow\> Data $\underline{\rightarrow}$ Data \regexstar{\comma Data $\underline{\rightarrow}$ Data}

{ElementSeq}     \>\>\>\>\>\garrow\> \regexopt{Data \regexstar{\underline{,} Data}}

{Constraint}     \>\>\>\>\>\garrow\> Number
                 \>\>\>\>\> \> $\mid$ Label
%                 \>\>\>\>\> \> $\mid$ \lb Label\comma Label\rb
                 \>\>\>\>\> \> $\mid$ Constraint LogOperator Constraint
                 \>\>\>\>\> \> $\mid$ Constraint RelOperator Constraint
                 \>\>\>\>\> \> $\mid$ Constraint NumOperator Constraint 
                 \>\>\>\>\> \> $\mid$ $\underline{(}$ Constraint $\underline{)}$ 


{Primitive}      \>\>\>\>\>\garrow\> Number
                 \>\>\>\>\> \>$\mid$ String
%                 \>\>\>\>\> \>$\mid$ Char
                 \>\>\>\>\> \>$\mid$ \underline{\texttt{true}} \% not implemented
                 \>\>\>\>\> \>$\mid$ \underline{\texttt{false}} \% not implemented
                 \>\>\>\>\> \>$\mid$ \underline{\texttt{top}}
                 \>\>\>\>\> \>$\mid$ \underline{\texttt{bot}}

{RelOperator}   \>\>\>\>\>\garrow\> \underline{$\texttt{<}$} $\mid$ \underline{$\texttt{<=}$} $\mid$ \underline{$\texttt{=}$} $\mid$ \underline{$\texttt{<>}$} $\mid$ \underline{$\texttt{>=}$} $\mid$ \underline{$\texttt{>}$}

{LogOperator}   \>\>\>\>\>\garrow\> \underline{$\texttt{and}$} $\mid$ \underline{$\texttt{or}$} $\mid$ \underline{$\texttt{not}$}

\end{GRAMMAR}
\caption{ARAL Grammar (Part 3)}
\label{fig:araltwo}
\end{figure}
}

%\begin{figure}[h!]
%\begin{GRAMMAR}
%{SourceAnnot}\>\>\>\>\>\garrow\> \underline{\#pragma} \underline{ARAL} {AnalysisIdent} AnnotData
%\end{GRAMMAR}
%\caption{ARAL Source Code Annotation Format}
%\label{fig:sourceannotationformat}
%\end{figure}
%}

\section{Operator Precedence}

Logical, relational, and numerical operators ordered from lower to higher precedence:

\vspace{0.3cm}
\begin{tabular}{|l|r|}\hline
Precedence & Operator\\\hline\hline
right & \texttt{not}\\\hline
left & \texttt{or}\\\hline
left & \texttt{and}\\\hline
left & \texttt{= <>}\\\hline
left & \texttt{< > >= <=} \\\hline
left & \texttt{+ -}\\\hline
left & \texttt{* / \%}\\\hline
left & \texttt{\^}\\\hline
\end{tabular}

The operators (except for \texttt{\%} and \texttt{\^} ) conform to the UML2 object constraint language.

\outcomment{
{\footnotesize
\vspace{0.2cm} \textbf{Acknowledgements.} 
This work has been partially supported by the ARTIST2 Network of
Excellence 
%(\href{http://www.artist-embedded.org/}
({http://www.artist-embedded.org/}), and the research project
``Integrating European Timing Analysis Technology'' (ALL-TIMES) under
contract No 215068 funded by the 7th EU R\&D Framework Programme.
}
}

\section{Type System}

ARAL provides (from version 0.9.16 upwards) a type system. The type
system ensures that the mappings provided in the mapping section
conform to the specified type and that the data represented in the
DATA subsection of a result section conforms to the type(s) defined in
the respective TYPE section of that result section.

The type system is based on name equivalence of types, and two types are equivalent if one of the following conditions holds

\begin{itemize}
\item They are the same basic type.
\item They are formed by applying the same constructor on to structurally equivalent types (this only applies to pre-defined types).
\item They have the same name.
\end{itemize}

Predefined types are set, list, map, tuple, and basic types. New types can be introduced by combining other types and associating names with those types.
For example, in a TYPE section for the results of a reaching definition analysis we can define the type as
\begin{verbatim}
TYPE
T = tuple(label,varid); 
RD = lift(set(T));
\end{verbatim}

This defines a type RD to be a lifted type of T which is a tuple of a label and a variable-idnumber. In the corresponding DATA section information such as 
\begin{verbatim}
DATA
{(@2,#5),(@2,#6)}
\end{verbatim}

can be represented. This DATA section represents the fact that at label 2, two variables with id-numbers 5 and 6 may be defined. 

A mapping section may define a type that is also defined in a DATA section. The type checker ensures that if the same name is used for a type in the mapping section and in a data section, or in multiple data sections, that the types are indeed equivalent. This allows to include mapping sections in different ARAL files without the need to have a DATA section around that produces DATA for which this mapping is indeed required. It also permits that different DATA sections can define the same types of the same name. This allows to easily merge different ARAL files.

The type of an analysis result can be defined by introducing new type names or without doing so. Here are examples of 2 different TYPE sections that define types for the same type of analysis results:

\begin{verbatim}
TYPE
MyLiftedSet = lift(set(tuple(label,varid)));

TYPE
MyLab = Label;
MyTuple = tuple(MyLab,varid);
MySet = set(MyTuple);
MyLiftedSet = lift(MySet);
\end{verbatim}

The type for this analysis can also be defined as 

\begin{verbatim}
TYPE
lift(set(tuple(label,varid)));
\end{verbatim}

This type is not equivalent with the type MyLiftedSet. It is structurally equivalent but not name equivalent - combined types that are not associated with a name are anonymous types. It is ensured that structurally equivalent anonymous types in different TYPE sections are equivalent.


\chapter{File Format}

An ARAL file consists of three sections where some have several subsections. The grammar for the ARAL file
is shown in Fig.~\ref{fig:annotationfileformat}. In the file
information is provided about the configuration of the analyzer and
the system level parameters that were taken into account by the
analysis, the type of the analysis data, and the data that was
computed for each program point. The three sections are:

\begin{description}
\item [Configuration Section.] In this section the values of system
level parameters and configuration options of the analyzer are
provided.
\item [Mapping Section.] If an analyzer uses unique identifiers for
program constructs such as functions, statements, expressions,
variables a mapping to some readable representation can be provided in
this section. For example, the unique identifier for an expression can
be E22, but the better readable representation is the expression as
used in the source code, ``$a+b$''. The mapping section allows to
provide such information. Alternatively the readable information can
be represented in the analysis information itself, but usually the
unique identifiers are better suitable for tools, but the readable
form is only used in visualization or output that is presented to the
user. Therefore this important indirection is supported in ARAL.
\item [Result Section.] An arbitrary number of results sections are allowed.
Each result section consists of three sub sections.

\begin{description}
\item [Name Section.] An identifier is provided for the name of the analysis that was performed.
\item [Type Section.] An analysis Data Type Definition must be
provided.  The Type Definition allows to define exactly what kind of
analysis information is used in the annotations, such as whether a set
is lifted or not (i.e. whether it may contain a special element for
the top and bottom element of a lattice), value ranges of numbers, or
whether constraints are reported as annotations. This section allows
to specify the type of the data that is provided in the DATA section.
\item [Data Section.]
In the data section the actual analysis data is provided. The grammar
for the annotation data is shown in Fig. \ref{fig:aralone} and
Fig. \ref{fig:araltwo}. The DATA section may hold arbitrary many
elements of annotation data. An element of annotation data consists of the following three entities:

\begin{description}
\item [Location Specifier.] Each analysis information is associated with a
program location and its corresponding program fragment. The program
location is represented by an optional Location Reference and a unique
label. The label is represented by a unique number.
The Location Reference allows to specify with which
part of a program fragment an annotation is associated with. This allows to
annotate multiple subexpressions of an expression without breaking up
the expression. An example is 'param($N$):12' for specifying that an annotation
of the $N$th formal parameter of a function is associated with label 12.

\item [Flow Specifier.] A Flow Specifier allows to define whether an analysis information is a pre or post information in a flow-sensitive information. If the information is flow-insensitive it is denoted as 'noflow'.
\item [List of Data Elements.] A data element consists of
\begin{description}
\item [Context.] For a context-sensitive analysis each context is represented in the annotations by an identifier which allows to keep context sensitive information separate. For a context-insensitive analysis the context identifier is the same for all analysis results computed for a program or function.
\item [Data.] The analysis data represents the analysis
information computed at a location in a program. A location can be
associated with a function, a statement, an expression, or with a scope.
\end{description}
\end{description}
\end{description}

Analysis data can consist of a set, list, tuple, map, string, int, float,
constraint, or specific program information: numeric VariableIds,
ExpressionIds, StatementIds, FunctionIds and Labels. Two special
symbols exist for the top and bottom element of lattices, 'top' and 'bot'. Sets, lists,
and maps can only contain data elements of the same type, whereas
tuples can contain elements of different type. In an constraint
(in)equations on program locations can be provided
(e.g. flow-constraints). In a data element the types of data can be
arbitrarily nested and combined.
\end{description}

\outcomment{
An ARAL analysis information annotation consists of:

\begin{description}
\item [Analysis Identifier.] Allows to identify the analysis information and associate the data with semantics.
\item [Location.] Each analysis information is associated with a
program location and its corresponding program fragment. The program
location is represented by an optional Location Reference and a unique
label. The label is represented by a unique number and a mapping of
labels is maintained by the IR-Mapper (see Section
\ref{section:architecture}) between the AST and the ICFG
representation. The Location Reference allows to specify with which
part of a program fragment an annotation is associated. This allows to
annotate multiple subexpressions of an expression without breaking up
the expression. An example is param($N$) for specifying an annotation
of the $N$th formal parameter of a function, or 'entry' and 'exit' for
denoting that an annotation is the entry or exit information of a
function (see Fig \ref{example:sourcelevel2}).
\item [Flow-Specifier.] A Flow-Specifier allows to define whether an analysis information is a pre or post information in a flow-sensitive information. If the information is flow-insensitive it is denoted as '-'.
\item [List of Data Elements.] A data element consists of
\begin{description}
\item [Context Identifier.] For a context-sensitive analysis each context is represented in the annotations by an identifier which allows to keep context sensitive information separate. For a context-insensitive analysis the context identifier is the same for all analysis results computed for a program or function.
\item [Analysis Data.] The analysis data represents the analysis
information computed at a location in a program. A location can be
associated with a function, a statement, an expression, or with a scope.
\end{description}
\end{description}

Analysis data can consist of a set, list, tuple, map, string, number,
constraints, or specific program information: numeric VariableIds,
ExpressionIds, StatementIds, FunctionIds and Labels. Two special
symbols exist for the top and bottom element of lattices. Sets, lists,
and maps can only contain data elements of the same type, whereas
tuples can contain elements of different type. In an constraint
(in)equations on program locations can be provided
(e.g. flow-constraints). In a data element the types of data can be
arbitrarily nested and combined.
}



\chapter{API}

The ARAL API allows to read, access, manipulate, and write ARAL files.

\section{Front End}

To create the ARAL-IR from an ARAL file the parser must be invoked
with \verb+Aral::Translator::frontEnd();+
If no parse-error occurred, the Front End returns a pointer to the
root of the ARAL-IR of type \verb+Aral::File*+.
The ARAL-IR, once available in memory, can be deleted by calling
the destructor of the root object. The destructors are implemented to
perform a deep-destruct.

\section{ARAL-IR}
Every IR node has a method \verb+accept(AbstractVisitor& v)+ that can be used
to invoke a traversal on the ARAL-IR. Three pre-defined Visitors exist.
\begin{itemize}
\item AbstractDataVisitor (only pure abstract methods)
\item EmptyDataVisitor
\item DataToStringVisitor;
\end{itemize}

\subsection{AbstractDataVisitor}

Each method has 2-3 visit methods.

\begin{itemize}
\item preVisitX
\item inVisitX
\item postVisitX
\end{itemize}

where X is the name of the resepective ARAL class.

\begin{itemize}
\item If a node is a leaf node or has only one child it has a pre and post
visit methods.
\item If a node is a container node or has at least two
children it also has an inVisit method.
\end{itemize}

The Visitor performs a combined pre/in/post order traversal in one pass.

\subsection{EmptyDataVisitor}
Is a visitor that has only empty methods. If you are writing your own
Visitor, you have to implemented a number of visit-methods. In this
case it is best to start with EmptyVisitor and override those methods
where certain actions are to be defined to certain ARAL-IR nodes. An example is the
generation of the DATA section in some specific custom format. The DataToStringVisitor
is an example, where the entire ARAL-IR is tranlated to the ARAL syntax representation. It overrides
several methods of the EmptyDataVisitor.

\subsection{DataToStringVisitor}
This is the pre-defined Visitor that generates ARAL, i.e. the ARAL
Back End implementation. It also servers as an example to see how the
Visitor can be used to translate the ARAL-IR to some other format. For
example, instead of generating a string you could map ARAL to your own
internal representation.

All Aral nodes (of type Aral::Data) have the following methods:

\begin{tabular}{p{6cm}cp{7.2cm}}
std::string toString()& :& translates the ARAL-IR subtree to Aral-Syntax and
                         returns it as a std::string.\\

Aral::Data* deepCopy()& : &clones the ARAL-IR subtree and returns a
                         pointer to the cloned sub tree.\\
void\newline accept(Aral::AbstractDataVisitor\&)& :& accepts any Visitor class that inherits
                         from AbstractDataVisitor, or EmptyDataVisitor, 
                         or DataToStringVisitor. See above for details.\\
Data* getParent()   &  :& returns the parent of an ARAL-IR node. The parents are always
                         automatically set internally by operations on the ARAL-IR.\\

void autoLinkParent(Data*)& : &for internal use of maintaining parent pointers.\\
virtual bool isEqual(Data*)&: &for internal use (not finished yet)\\
bool isLessThan(Data*)     &:& for internal use (not finished yet)\\
\end{tabular}

\section{ARAL Back End}

The Back End can be invoked by calling Aral::Translator::backEnd(root)
where root is the root node of the ARAL-IR. The Back End is
implemented by calling the toString function on the root object. The
toString function is implemented by using the DataToStringVisitor.
The Back End generates the ARAL file in a fixed layout and returns a
std::string\footnote{Currently generating the std::string representation and a file
are synonymous. This will change in the next version.}
