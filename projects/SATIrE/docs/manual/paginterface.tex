\chapter{Interfacing Compilers with PAG}

Author: Gerg\H{o} B\'ar\'any

%\topmargin      -16.0mm
%%\oddsidemargin  -11.0mm
%%\evensidemargin -11.0mm
%\oddsidemargin  -1.0mm
%\evensidemargin -1.0mm
%\textheight     243.5mm
%%\textwidth      183.0mm
%\textwidth      164.0mm

\section{Introduction}
\subsection{Overview}

PAG is a tool for generating program analyzers that can be used with
existing compilers or built into new ones. The analyses themselves
are specified in the high-level functional language FULA, which is
compiled into a C library performing the analysis.

Since the analyzer needs access to the program's control flow graph
(CFG) and the abstract syntax tree (AST) for each statement, some
sort of interface between the compiler and the analyzer must be
implemented. A tool called GON can automatically generate this
interface, but only for compilers that are written from scratch.

The purpose of this document is to describe the glue code that must be
written in order to connect an existing compiler with PAG. It also
investigates which parts of the interface can be generated
automatically. These are exactly those parts which are generated
automatically by SATIrE's internal tool PIG (PAG Interface Generator).

\subsection{Required files}

The interface must provide the following files, each of which is
explained in more detail below:
\begin{itemize}
\item \verb|edges|: defines the edge types that may occur in the CFG
\item \verb|syn|: a tree grammar describing the abstract syntax of
the language
\item \verb|pagoptions|: describes which access functions for
syntactic lists are implemented
\item \verb|syntree.h|: defines all types for the abstract syntax
tree
\item \verb|iface.h|: defines all types that must be provided by the
interface
\item \verb|syntree.c|: contains the code for access functions (or
macros) to the CFG and the AST
\end{itemize}

The rest of the document gives detailed information on what each of
these files should contain. It considers the CFG part first, then
the AST.

\section{The CFG Interface}

This section describes the files, types and functions for CFG access
that the interface must provide.

The CFG consists of a set of nodes, each numbered with a unique id
starting from 0. Every node represents a basic block, possibly
containing several instructions. Interprocedural edges are only
allowed from \verb|Call| to \verb|Start| and back from \verb|End| to
\verb|Return| nodes.

\subsection{The {\tt edges} file}

The \verb|edges| file lists all edge types occurring in the CFG.
Each type name must be listed on a separate line, line comments
beginning with \verb|//| are allowed.

The first edge type must be the type for local edges, called for
instance \verb|local_edge|; these are the edges from a function call
nodes to the corresponding return nodes. The second type is the type
\verb|bb_intern| of edges connecting statements inside a basic block.

PAG turns this specification into an \verb|enum| called
\verb|o_edges|, defining the enumeration constant
\verb|o_|\(n_i\)\verb|=|\(i-\)1 for the \(i\)-th edge type (\(i \ge 1\)).

\subsection{Required types}

The following types must be defined in \verb|iface.h|:

\begin{center}
\begin{tabular}{|l|l|l|}
\hline
Name                 & Description & Type Restrictions \\
\hline
\hline
\verb|KFG|           & The CFG itself
    & must be a pointer \\
\hline
\verb|KFG_NODE_TYPE| & Type of node classes
    & enumeration type or \verb|int| \\
\hline
\verb|KFG_NODE_ID|   & Type of node identifiers
    & must be \verb|int| \\
\hline
\verb|KFG_NODE|      & Type of CFG nodes
    & must be a pointer type \\
\hline
\verb|KFG_NODE_LIST| & Type of node lists
    & must be a pointer type \\
\hline
\verb|KFG_EDGE_TYPE| & Type of edge classes
    & enumeration type or \verb|int| \\
\hline
\end{tabular}
\end{center}

The type \verb|KFG_NODE_TYPE| must at least contain the enumeration
constants \verb|RETURN|, \verb|CALL|, \verb|START|, \verb|END|, and
\verb|INNER| with values \(0\ldots4\). It may support further
constants with larger values.

Further, \verb|KFG_EDGE_TYPE| must contain constants for local and
basic-block-internal edges with values 0 and 1; the rest of the
contants should also be analogous to those declared in the
\verb|edges| file.

\subsection{Required functions}

PAG requires the front end to implement a number of functions for
accessing, and traversing the CFG in numerous ways.

\begin{longtable}{|p{.55\textwidth} | p{.38\textwidth}|}
\hline
Prototype & Description \\
\hline
\hline \endhead
\verb|KFG kfg_create(KFG)| & initialize the CFG \\
\hline
\verb|int kfg_num_nodes(KFG)| & number of nodes in the CFG \\
\hline
\verb|KFG_NODE_TYPE kfg_node_type(| \verb|KFG, KFG_NODE)|
    & type of the node \\
\hline
\verb|KFG_NODE_ID kfg_get_id(KFG,| \verb|KFG_NODE)|
    & identifier of the node \\
\hline
\verb|KFG_NODE kfg_get_node(KFG,| \verb|KFG_NODE_ID)|
    & node with the given identitier \\
\hline
\verb|int kfg_get_bbsize(KFG,| \verb|KFG_NODE)|
    & number of instructions in the node \\
\hline
\(t\) \verb|kfg_get_bbelem(KFG,| \verb|KFG_NODE, int)|
    & the \(n\)-th instruction of the node, starting with 0; \(t\) is
      the AST type \\
\hline
\raggedright \verb|void kfg_node_infolabel_print_fp|
    \verb|(FILE *, KFG, KFG_NODE, int)|
    & write a textual description of the \(n\)-th instruction of the node
      to the file (used for visualization) \\
\hline
\verb|KFG_NODE_LIST| \verb|kfg_predecessors(KFG,| \verb|KFG_NODE)|
    & list of predecessors of the node \\
\hline
\verb|KFG_NODE_LIST| \verb|kfg_successors(KFG,| \verb|KFG_NODE)|
    & list of successors of the node \\
\hline
\verb|KFG_NODE kfg_get_call(KFG,| \verb|KFG_NODE)|
    & the call node belonging to the given return node \\
\hline
\verb|KFG_NODE kfg_get_return(KFG,| \verb|KFG_NODE)|
    & the return node belonging to the given call node \\
\hline
\verb|KFG_NODE kfg_get_start(KFG,| \verb|KFG_NODE)|
    & the start node belonging to the gi\-ven end node \\
\hline
\verb|KFG_NODE kfg_get_end(KFG, KFG_NODE)|
    & the end node belonging to the given start node \\
\hline
\verb|const int *kfg_get_beginnings(KFG)|
    & Returns a pointer to an array of procedure numbers, terminated by
      \(-1\), to start the analysis with. If the function returns an
      empty list (contains only \(-1\)) then the analyzer selects entry
      point automatically. \\
\hline
\verb|int kfg_replace_beginnings(KFG,| \verb|const int *)|
    & replaces the beginnings list of the front end, can be called after
      initialization of the CFG before the analysis; returns 1 for
      success, 0 if the feature is not supported, \(-1\) for an error \\
\hline
\end{longtable}

% MS: changed column width for all tables to fit new format
\begin{longtable}{|p{.55\textwidth} | p{.38\textwidth}|}
\hline
Prototype & Description \\
\hline
\hline \endhead
\verb|KFG_NODE kfg_node_list_head| \verb|(KFG_NODE_LIST)|
    & head of list \\
\hline
\verb|KFG_NODE_LIST kfg_node_list_tail| \verb|(KFG_NODE_LIST)|
    & list without the first element \\
\hline
\verb|int kfg_node_list_is_empty| \verb|(KFG_NODE_LIST)|
    & 1 if the list is empty, 0 otherwise \\
\hline
\verb|int kfg_node_list_length| \verb|(KFG_NODE_LIST)|
    & length of node list \\
\hline
\end{longtable}

\begin{longtable}{|p{.55\textwidth} | p{.38\textwidth}|}
\hline
Prototype & Description \\
\hline
\hline \endhead
\verb|unsigned int kfg_edge_type_max(KFG)|
    & number of different edge types \\
\hline
\verb|KFG_EDGE_TYPE kfg_edge_type| \verb|(KFG_NODE, KFG_NODE)|
    & type of the edge from the first node to the second; runtime error
      if there is no such edge \\
\hline
\verb|int kfg_which_in_edges(KFG_NODE)|
    & returns a bitmask with a bit corresponding to an edge type set
      if there is an incoming edge of that type \\
\hline
\verb|int kfg_which_out_edges(KFG_NODE)|
    & as \verb|kfg_which_in_edges| but for outgoing edges \\
\hline
\end{longtable}

\begin{longtable}{|p{.55\textwidth} | p{.38\textwidth}|}
\hline
Prototype & Description \\
\hline
\hline \endhead
\verb|int kfg_num_procs(KFG)|
    & number of procedures in the CFG \\
\hline
\verb|char *kfg_proc_name(KFG, int)|
    & static pointer to the name of a procedure \\
\hline
\verb|KFG_NODE kfg_numproc(KFG, int)|
    & entry node of a procedure \\
\hline
\verb|int kfg_procnumnode(KFG, KFG_NODE)|
    & number of the procedure the node belongs to \\
\hline
\verb|int kfg_procnum(KFG, KFG_NODE_ID)|
    & number of the procedure the node with the given id belongs to \\
\hline
\end{longtable}

\begin{longtable}{|p{.55\textwidth} | p{.38\textwidth}|}
\hline
Prototype & Description \\
\hline
\hline \endhead
\verb|KFG_NODE_LIST kfg_all_nodes(KFG)|
    & list of all nodes \\
\hline
\verb|KFG_NODE_LIST kfg_entrys(KFG)|
    & list of all entry nodes \\
\hline
\verb|KFG_NODE_LIST kfg_calls(KFG)|
    & list of all call nodes \\
\hline
\verb|KFG_NODE_LIST kfg_returns(KFG)|
    & list of all return nodes \\
\hline
\verb|KFG_NODE_LIST kfg_exits(KFG)|
    & list of all exit nodes \\
\hline
\end{longtable}

All of these functions may optionally be implemented as C macros.
They must be declared in \verb|iface.h|.

\subsection{Data structures for the CFG}

Given the specifications of the access functions, designing an
appropriate data structure is easy: \verb|KFG_NODE| can be
implemented as a pointer to a structure containing an id, a type, a
list of statements (the basic block represented by this node) and
the size of this list, lists of predecessors and successors
(possibly with the appropriate edge types), the number of the
procedure the node belongs to, and precomputed bitmasks for the
\verb|kfg_which_in_edges| and \verb|kfg_which_out_edges| functions.

Depending on the underlying language, it might not be necessary to
explicitly store the types of the edges connecting two nodes, since
this can often be determined from the types of the nodes alone. For
instance, the edge from a `normal' node will be a normal edge, the
edge from a return node will be a return edge; for an if node, the
edge will be a true or false edge depending on whether the successor
is stored as the `true' or `false' successor of this node. Also, the
list of successors will usually have at most two elements (one for
the true case, one for false).

\verb|KFG_NODE_LIST| can be defined as \verb|KFG_NODE *|. Lists of
nodes are then implemented as null-terminated arrays of
\verb|KFG_NODE|, ensuring fast direct access and traversal. Since
the CFG is required to be constant once it has been created, one
need not worry about the possible costs of modifying such arrays at
runtime.

The \verb|KFG| itself can just be a pointer to a structure
containing a list of all nodes, lists of special nodes (procedure
entry nodes, calls, returns and exits), a list of (procedure number,
procedure name) pairs and a list of procedure numbers to start the
analysis with. In the list of all nodes, each node can be stored at
the index corresponding to its id, ensuring fast lookups and
technically eliminating the need for storing the id explicitly.

These types should be made available to PAG through the
\verb|iface.h| file.

\subsection{Implementation of the CFG functions}

For a data structure as described above, implementing the CFG access
functions required by PAG is rather straightforward; most functions
just return a certain structure field or array element. All operations
except \verb|kfg_create| and the procedure name or number lookups
can be implemented to run in constant time.

Automatic conversion of an existing CFG or other intermediate
representation from a compiler front end might be possible in
principle, especially if the data structures are similar enough to
the ones described above. Conversely, it should be possible to
generate the required access functions and leave the existing CFG
completely unchanged. The problem with these approaches is the
difficulty of specifying just what should be converted in which way;
the specification for the conversion tool would in general have to
be very complicated. Therefore it appears more reasonable to write
the necessary code by hand.

The main part of this work consists of collecting all statements to
basic blocks, linking these with each other and computing the
auxiliary information. Note that the requirement that a node
represent a whole basic block is not enforced by PAG, it is merely
strongly suggested for efficiency reasons. It is possible to store
exactly one statement per node, making creation of the CFG somewhat
simpler. The example C compiler front end that is shipped with PAG
uses this approach.

The code described in this section should reside in \verb|syntree.c|
(possibly \verb|#include|d from other files).

\section{The AST Interface}

The AST interface consists of a tree grammar describing the
structure of the tree, the corresponding C type declarations, and C
functions implementing syntax tree access, type tests and type
conversions, and syntactic list traversal.

\subsection{The {\tt syn} file}

The \verb|syn| file describes the abstract syntax of the language
under consideration by a tree grammar. Figure~\ref{syn}, taken from
\cite{pag}, shows a brief excerpt of an example \verb|syn| file.

\begin{figure}
    \centering
    \begin{boxedverbatim}
SYNTAX
START: mirStmt

mirStmt: CFGCall(exp: mirExpr)
       | CFGEndCall(exp: mirExpr, sym: mirSymbol*)
         ...
       ;

mirExpr: mirChar(str: CHAR, type: mirType)
         ...
       ;

CHAR == chr;
INT  == snum;\end{boxedverbatim}
    \caption{An example {\tt syn} file}
    \label{syn}
\end{figure}

\verb|START| specifies the start symbol of the grammar, every
instruction inside a CFG node must be associated with an AST of this
type. Alternatives for a production are grouped together. For
instance, a \verb|mirStmt| is either a node of type \verb|CFGCall|
with a child node \verb|exp| of type \verb|mirExpr| or a node of
type \verb|CFGEndCall| with two child nodes \verb|exp| as above and
\verb|sym| of type \verb|mirSymbol*|. The \verb|*| indicates that
this type is a syntactic list of \verb|mirSymbol| terms. This
abbreviation is provided by PAG since lists are so common.

The production for \verb|mirExpr| contains a reference to the type
\verb|CHAR|. There is no grammar rule for this type; rather, it is
an alias type defined by the equivalence \verb|CHAR == chr|. This
means that it corresponds to the built-in FULA type \verb|chr|. The
interface needs to provide conversion functions for each such
alias type to enable the analysis to use values of these types.

\subsection{Required types}

For each type (including alias types) defined in the \verb|syn|
file, a C type with the same name must be declared. For each
syntactic list over a type \(T\) used in the tree grammar, a type
named \verb|LIST_|\(T\) must be defined.

Further, it is possible to define cursors for syntactic lists. These
are abstract data types for traversing syntactic lists, enabling
them to be used in ZF expressions in the FULA language. For
each list over a type \(T\), the types \verb|_LIST_|\(T\)\verb|_cur|
and \verb|LIST_|\(T\)\verb|_cur| must be defined, where the latter
is a pointer to the former.

The types described in this section must be declared in
\verb|syntree.h|.

\subsection{Required functions}

For every type constructor \(c\)\verb|(|\(n_1\)\verb|:|\(t_1\)\verb|,|
\ldots\verb|,|\(n_k\)\verb|:|\(t_k\)\verb|)| of type \(t\) in the
\verb|syn| file, PAG requires functions for element access and for
type testing.

The access functions are
\(t_i\ t\)\verb|_|\(c\)\verb|_get_|\(n_i\)\verb|(|\(t\)\verb|)| for
accessing the child named \(n_i\) of type constructor \(c\) for type
\(t\). The functions take a node of type \(t\) as their sole
argument and return a node of type \(t_i\).

The test functions are
\verb|int is_op_|\(t\)\verb|_|\(c\)\verb|(|\(t\)\verb|)|. These
return 1 if the tree node of type \(t\) passed in is labeled with
the constructor \(c\), 0 otherwise.

Alias types, declared as \(t\) \verb|==| \(p\) in the \verb|syn|
file, need special treatment. They are internally represented as C
types but need the ability to be converted to FULA types. This must
be realized by functions of the form
\verb|char *|\(t\)\verb|_get_value(|\(t\)\verb|)| returning a string
representation of the value of \(t\). PAG can then convert this
string to the primitive FULA type \(p\).

For each syntactic list over a type \(T\) the following functions
must be defined:

\begin{longtable}{|p{.47\textwidth} | p{.47\textwidth}|}
\hline
Prototype & Description \\
\hline
\hline \endhead
\verb|int LIST_|\(T\)\verb|_empty(LIST_|\(T\)\verb|)|
    & 1 if the list is empty, 0 otherwise \\
\hline
\(T\) \verb|LIST_|\(T\)\verb|_hd(LIST_|\(T\)\verb|)|
    & head of the list \\
\hline
\verb|LIST_|\(T\) \verb|LIST_|\(T\)\verb|_tl(LIST_|\(T\)\verb|)|
    & tail of the list \\
\hline
\end{longtable}

As explained above, it is possible to define syntactic list cursors.
The cursor functions for lists over a type \(T\) are given in the
following table:

\begin{longtable}{|p{.47\textwidth} | p{.47\textwidth}|}
\hline
Prototype & Description \\
\hline
\hline
\verb|void LIST_|\(T\)\verb|_cur_reset(LIST_|\(T\)\verb|_cur,|
    \verb|LIST_|\(T\)\verb|)|
    & initialize the cursor \\
\hline
\raggedright
    \verb|void LIST_|\(T\)\verb|_cur_is_empty|
    \verb|(LIST_|\(T\)\verb|_cur)|
    & 1 if the list is empty, 0 otherwise \\
\hline
\(T\) \verb|LIST_|\(T\)\verb|_cur_get(LIST_|\(T\)\verb|_cur)|
    & current element of the list \\
\hline
\verb|void LIST_|\(T\)\verb|_cur_next(LIST_|\(T\)\verb|_cur)|
    & advance the cursor by one element \\
\hline
\verb|void LIST_|\(T\)\verb|_cur_destroy(LIST_|\(T\)\verb|_cur)|
    & destructor of the cursor (optional) \\
\hline
\end{longtable}

All of the AST functions must be defined via \verb|syntree.c|.

\subsection{The {\tt pagoptions} file}

The \verb|pagoptions| file tells PAG which features respecting
syntactic lists are supported by the front end. The contents of this
file should almost always be the following:

\begin{quote}
\begin{verbatim}
LIST_is_empty       : 1
LIST_hd             : 1
LIST_tl             : 1
LIST_cursor         : 1 
LIST_cursor_destroy : 1
\end{verbatim}
\end{quote}

The first three lines indicate that the basic syntactic list
features are supported. Since these are always required if syntactic
lists are used, there is not much choice in whether to define them.

The last two lines indicate that support for list cursors and for
list cursor destructors is present. Implementing cursors efficiently
should not be very difficult either once lists are supported at all,
so these can be defined as well.

If the implementor should decide not to support a certain feature,
the indicator in the corresponding line can be set to \verb|0| to
reflect this.

\subsection{Additional requirements}

The PAG manual lists a few other functions that must be present if a
front end is written from scratch. See page 89 of \cite{pag} for
details.

\subsection{Automatic AST interface generation}

The \verb|syn| file alone is enough to generate an implementation of
almost all of the abstract syntax tree data types and functions
automatically.

This is not possible for alias types, however: The converter cannot
know which C type should be used as the internal representation for
the type. Thus, the user must provide an appropriate type definition
and an implementation of the corresponding \verb|get_value| function
for each alias type.

% TODO: rephrase me and make me coherent
The problem with this approach is that it would create a completely
new AST which will in general not be identical to the one already
implemented in the compiler front end. The obvious solution to this
is just using the existing AST and only generating appropriate
access and test functions.

For this idea to work, it is possible to write a simple tool that
parses the \verb|syn| file and another file provided by the user,
describing the functions to be generated with a simple macro
language.

Consider a production \(c_j\)\verb|(|\(n_1\)\verb|:|\(t_1\)\verb|,|
\ldots\verb|,|\(n_k\)\verb|:|\(t_k\)\verb|)| for a type \(t\) in the
\verb|syn| file. It is reasonable to assume that most AST nodes
share a basic structure, making accesses to their fields all very
similar, depending only on some of the following:

\begin{itemize}
\item the type name \(t\) of the node itself
\item an expression denoting the particular node object
\item the name \(c_j\) of the type constructor
\item the index \(j\) of the constructor, denoting that this is the
    \(j\)-th alternative for type \(t\) in the \verb|syn| file
\item the name \(n_i\) of the requested field
\item the index \(i\) of the requested field
\item the type name \(t_i\) of the requested field (this might be
    useful for type casts)
\end{itemize}

If the code is the same up to these details, it is rather simple to
create it as an instance of a macro description with these
parameters. These macros would be provided by the user, an interface
generation tool would then turn them into code for all AST functions
required by PAG. The accesses are expected to be uniform in most,
but not in all cases, so special cases (for certain types) must be
handled as well.

Here is a fictional example of the possible syntax of such a
specification:

\begin{quote}
\begin{verbatim}
get("mirStmt", NODE, CONSTR, CONSTR_IDX, 
               FIELD, FIELD_IDX, FIELD_TYPE)
{
    return NODE->children[CONSTR_IDX][FIELD_IDX];
}

get(TYPE, NODE, CONSTR, CONSTR_IDX, 
                FIELD, FIELD_IDX, FIELD_TYPE)
{
    return NODE->children.f_##CONSTR.FIELD;
}
\end{verbatim}
\end{quote}

The indended meaning of this snippet is the following: Functions for
type \verb|mirStmt| are created by the first rule because the head
matches just this type. All functions for the other types are
matched by the second rule.

Children of a node of type \verb|mirStmt| are accessed by indexing a
two-dimensional array of child nodes. Thus, to access the first
child (\verb|expr|) of the second production (\verb|CFGEndCall|),
the function
\begin{quote}
\begin{verbatim} 
mirExpr mirStmt_CFGEndCall_get_exp(mirStmt node)
{
    return node->children[1][0];
}
\end{verbatim}
\end{quote}
would be generated.

For all other types the second rule would apply, producing for
instance the code
\begin{quote}
\begin{verbatim}
mirType mirExpr_mirChar_get_type(mirExpr node)
{
    return node->children.f_mirChar.type;
}
\end{verbatim}
\end{quote}
for accessing the second child node in the first production for the
\verb|mirExpr| type.
Notice the use of the C preprocessor's token pasting operator
\verb|##| to construct the field name \verb|f_mirChar| from the
constant prefix \verb|f_| and the constructor's name.

Things are a bit more complicated for the test functions: The
AST presumably already has some sort of type test using integer
constants (or even strings). However, the numbering or naming for
these might be different from the one that the conversion tool would
create by itself. In this case, the user would have to specify some
sort of mapping between the constructor names in the \verb|syn| file
and the actual constants used in the AST.

For the alias types defined in the \verb|syn| file, an interface
generation tool cannot know which C type was intended to implement
this alias type. Thus the tool cannot generate code for alias types,
the user must provide this code himself.

\subsection{Comparison to the CFG interface}

The difference to the CFG interface is the following: While the CFG
uses comparatively few different functions, the AST calls for a
quite large number of functions, all of which are instances of just
one of two patterns (assuming that all AST nodes have the same basic
structure).

Thus generating the AST functions from a simple description should
be rather easy, while in the CFG case much more detailed
specifications would be needed. This would in many cases lead to
descriptions that are just as complex as a manual implementation
of the function, thus losing the advantages of automatic code
generation.

\section{Summary}

It is possible to write a tool which generates large parts of the
interface between an existing compiler front end and PAG
automatically. The input to this tool would consist of a \verb|syn|
file describing the AST structure in the compiler, and a second file
of macros describing the way fields of the AST are accessed.

The tool creates from these the complete AST interface
implementation and stubs for the alias type conversion functions as
well as the CFG access functions. The programmer must then fill in
the function definitions and provide the necessary type definitions.

The automatically generated AST saves the programmer time if the
specification is significantly shorter than the resulting program,
i.e. if writing the specification is less tedious than writing the
access code by hand. This should be the case if access to the AST
nodes is similar in most cases, subsuming many functions under one
macro specification.

\end{document}
