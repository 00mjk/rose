% This is a first shot at a somewhat comprehensive SATIrE manual. It is all
% in one file, which is not very nice. This file should be broken up into
% separate files for the chapters once the structure of the document has
% stabilized.

\documentclass[a4paper,12pt]{report}

\input{manual_vars}

\title{SATIrE Manual}
\author{}

\pagestyle{headings}

\begin{document}
\maketitle

\chapter*{About this document}

This is the manual for SATIrE version \version.

This manual is not done yet. Parts are missing, other parts are probably
wrong, and the overall structure is not fixed. In short, this manual is the
best documentation SATIrE has ever had :-)

\tableofcontents

\chapter{Introduction}
\label{chap:introduction}

SATIrE (\emph{Static Analysis Tool Integration Engine}) is a framework for
combining various tools for static analysis of computer programs. Its aim is
to support a wide range of source-level analyses and transformations
(including annotations and instrumentation) for C and C++ programs.

SATIrE is being developed at Vienna University of
Technology\footnote{\texttt{http://www.tuwien.ac.at}} and University of
Applied Sciences Technikum
Wien\footnote{\texttt{http://www.technikum-wien.at}}. It lives at:
\begin{center}
\verb|http://www.complang.tuwien.ac.at/satire/|
\end{center}

Major software products integrated in SATIrE are:
\begin{itemize}
\item the Program Analyzer Generator (PAG)
\item relevant parts of the ROSE source-to-source infrastructure and its
binding to the EDG C and C++ frontend
\item Termite
\end{itemize}

Work on using clang as an alternative C frontend is in progress.

Additionally, SATIrE comes with a number of standard analyses that can be
used as building blocks when implementing custom program analyzers and
transformers.

Development of SATIrE has been funded within the ARTIST2 Network of
Excellence on Embedded Systems
Design\footnote{\texttt{http://www.artist-embedded.org}} and the ALL-TIMES
project\footnote{\texttt{http://www.all-times.org}}.

\chapter{SATIrE Analyzer Architecture}
\label{chap:analyzer}

This chapter describes the general structure of analyzers and their
interactions with the infrastructure provided by the SATIrE framework.

Note: Most of the identifiers described in this chapter live in the
\texttt{SATIrE} C++ namespace. They can be accessed by including the
\verb|<satire.h>| header.

\section{Command Line Flags}
\label{sec:command_line}

SATIrE contains a command line parser that reads options and input file
names and encapsulates them in an instance of the \texttt{AnalyzerOptions}
class.

The command line can be parsed using the
\begin{verbatim}
    AnalyzerOptions *extractOptions(int argc, char **argv);
\end{verbatim}
function. A list of all the flags understood by the command line parser, and
information on how to access this information from SATIrE analyzers, is
given in Appendix~\ref{appendix:command_line}.

\section{Program Input and Output}
\label{sec:program_io}

For each of the different program representations supported by SATIrE
(see Chapter~\ref{chap:program_representation}), there is a corresponding
function to build that representation from appropriate inputs. Each of these
functions takes an \texttt{AnalyzerOptions} object; besides specifying input
file names, this object contains additional options such as user requests
for sanity checks on the program representation, or for visualizations (of
the ICFG).

The ROSE AST (Section~\ref{sec:rose_ast}) can be built by calling:
\begin{verbatim}
    SgProject *createRoseAst(AnalyzerOptions *options);
\end{verbatim}
Note that this either reads source code, or a binary representation of a
previously constructed ROSE AST, depending on flags passed on the command
line.

The SATIrE ICFG (Section~\ref{sec:satire_icfg}) is built from a
ROSE AST by calling:
\begin{verbatim}
    CFG *createICFG(SgProject *astRoot, AnalyzerOptions *opts);
\end{verbatim}

Despite there being special functions for explicit creation of various
specialized program representations, it is often best to use SATIrE's
general high-level \texttt{Program} (Section~\ref{sec:satire_program})
class. Building a \texttt{Program} is performed by calling its
\begin{verbatim}
    Program(AnalyzerOptions *o);
\end{verbatim}
constructor.

Annotated or transformed programs can be output by calling the
\begin{verbatim}
    void outputProgramRepresentation(Program *program,
                                     AnalyzerOptions *options);
\end{verbatim}
function. This function will output the program as source code, as a binary
AST representation, or as a Termite term, depending on the command line
options. It will also output a visualization of the ICFG if requested.

Using these parts, one can build a very small SATIrE program that already
allows various forms of program visualization and transformation:
\begin{verbatim}
    #include <satire.h>

    using namespace SATIrE;

    int main(int argc, char **argv)
    {
        AnalyzerOptions *options = extractOptions(argc, argv);
        Program *program = new Program(options);
        outputProgramRepresentation(program, options);
    }
\end{verbatim}

The tasks this program performs depend on the command line flags passed to
it by the user. Despite its power, this program is probably shorter than the
compiler command line you need to build and run it correctly :-)

\section{Analyzer Interface}
\label{sec:analyzer_interface}

This section describes the general interface SATIrE analyzers are expected
to implement. This interface is the public interface of the abstract
\texttt{Analysis} class, from which all analyzers should be derived. There
is a more elaborate \texttt{DataFlowAnalysis} subclass for data-flow
analyzers generated using SATIrE and PAG; this class is described in more
detail in Chapter~\ref{chap:data_flow}.

The analyzer interface consists of several parts; first, there is a part for
meta information about the analyzer:
\begin{verbatim}
    virtual std::string identifier() const = 0;
    virtual std::string description() const = 0;
\end{verbatim}
The identifier is expected to be a single word naming the analyzer, while
the description is a brief human-readable summary of what the analyzer does.

The second part of the interface comprises methods for running the analysis
itself, and for performing actions depending on the results the analysis
computed:
\begin{verbatim}
    virtual void run(Program *program) = 0;
    virtual void processResults(Program *program) = 0;
\end{verbatim}
Note that these methods take as argument an instance of the general
\texttt{Program} class. This enables SATIrE to use a single analyzer
interface, while each analyzer can still decide which part of this program
representation (AST, ICFG, etc.) to run on. See
Chapter~\ref{chap:program_representation} for details on this issue.

Conceptually, the \texttt{run} method is meant to be a read-only analyzer
run that only collects information about the program and leaves the program
representation unchanged; the \texttt{processResults} method can then
annotate or transform the program as appropriate given those results. This
structure encourages modular analysis and transformation, but the separation
is not enforced by SATIrE.

The third part of the analyzer interface concerns dependencies between
analyzers. One of SATIrE's goals is modular construction of program
analyzers by combining, and building on, results from supporting analyses.
Where an analysis relies on results from other analyses, it must be ensured
that the supporting analyses are run before the client analysis. This is the
aim of the methods related to analyzer dependencies:
\begin{verbatim}
    void dependsOnAnalysis(Analysis *analysis);
    std::vector<Analysis *> &dependencies() const;
    void clearDependencies();
\end{verbatim}
The \texttt{dependsOnAnalysis} method is used to declare that the receiver
of the method call depends on results from the analyzer passed in the method
argument; the other two methods can be used to query the dependencies
declared so far, or to remove all of these dependencies.

\section{Running an Analyzer}
\label{sec:run_analyzer}

To run an analyzer with (or without) dependencies, rather than calling
its~\texttt{run} method directly, SATIrE's analysis scheduler should be
used. As far as users are concerned, this amounts to calling a single method
on the global scheduler object provided by SATIrE:
\begin{verbatim}
    analysisScheduler.runAnalysisWithDependencies(analysis,
                                                  program);
\end{verbatim}
This call is like calling
\begin{verbatim}
    analysis->run(program);
\end{verbatim}
except that the scheduler is aware of the dependencies between analyzers,
and ensures that they are satisfied.

If analyzer~\(A\) depends on analyzers~\(B\) and~\(C\), SATIrE's analysis
scheduler will thus ensure that~\(B\) and~\(C\) will be run---in some
unspecified order that is consistent with all the dependencies in the
system---before \(A\)'s \texttt{run} method is finally called by the
scheduler.

\section{ARAL: Representing Analysis Results}
\label{sec:aral}

This section is not here yet.

\chapter{Program Representation in SATIrE}
\label{chap:program_representation}

This chapter describes the various ways SATIrE can represent programs under
analysis. The major representations are the ROSE abstract syntax tree~(AST),
SATIrE's interprocedural control flow graph~(ICFG), and the Termite term
representation of the AST. The \texttt{Program} class provides a unified
container for these representations.

\section{The ROSE AST}
\label{sec:rose_ast}

SATIrE uses the EDG C and C++ frontend provided with ROSE for parsing
programs. Thus, the AST used by ROSE is an important form of program
representation in SATIrE.\footnote{When support for clang is nearing
completion, SATIrE will offer the possibility to use clang instead of EDG as
the frontend. However, the representation built using this alternative
frontend will still use the ROSE AST classes.}

The ROSE AST is an object-oriented AST modeling the abstract syntax of a
given C or C++ program in great detail. ROSE programs can be traversed and
transformed in various ways; the reader is referred to the ROSE
documentation for details\footnote{\texttt{http://www.rosecompiler.org}}.
ROSE's program transformation capabilities and its unparser are used by
SATIrE to annotate ASTs and unparse them to annotated source code.

\section{The SATIrE ICFG}
\label{sec:satire_icfg}

SATIrE provides a representation of the program under analysis as an
inter-procedural control flow graph~(ICFG). The ICFG is designed to support
data-flow analysis using PAG, but it can also be used for other types of
analysis. The traversal mechanism provided by SATIrE (described below) makes
it possible to use the ICFG for flow-insensitive analysis in a way that is
similar to ROSE's AST traversals.

\subsection{Structure of the ICFG}
\label{sec:icfg_structure}

The ICFG is a directed graph consisting of nodes connected by labeled edges.
Each node contains a single statement, which can be a statement from the
original program, a transformed version of some original program statement,
or a new statement that does not directly correspond to a program statement,
but rather to a program point. Statements are represented using ROSE classes
as far as possible; statement types not occurring in ROSE are implemented by
subclassing ROSE's \texttt{SgStatement} class (via SATIrE's
\texttt{IcfgStmt} class).

Edges model (all) possible control flow, with labels providing information
on the type of flow: \verb|normal_edge| for regular flow; \verb|jump_edge|
for unconditional jumps; \verb|true_edge| and \verb|false_edge| for branches
(including loops); \verb|call_edge|, \verb|return_edge|, and
\verb|local_edge| for function calls, returns, and for connecting call and
return site in the caller, respectively. There are no interprocedural edges
except for calls and returns.

The following paragraphs list SATIrE's ICFG statements grouped by topic.

\paragraph{Variable scopes} The ICFG represents not only variable
declarations, but also `undeclarations' where variables go out of scope. The
corresponding statements are \verb|DeclareStmt| and \verb|UndeclareStmt|.
Compared to the ROSE AST, variable declarations are normalized such that
each \verb|DeclareStmt| declares exactly one variable; any initializers are
modeled using subsequent assignments.

\paragraph{Function boundaries} Functions (also called `procedures') in the
ICFG have explicit \verb|FunctionEntry| and \verb|FunctionExit| statements.
All calls and returns pass through these statements.

\paragraph{Function arguments} The ICFG makes argument passing explicit. It
introduces special global variables, collectively referred to as `tmpvars',
which behave similarly to argument registers in machine code. In particular,
before each function call node (see below), the ICFG contains a series of
nodes with \verb|ArgumentAssignment| statements. Each of these assigns the
value of some argument expression to a tmpvar. Within each function, the
entry node is followed by a sequence of \verb|ParamAssignment| statements
that assign each argument tmpvar to the corresponding parameter variable
inside the function.

\paragraph{Function return values} Returning values from functions is
similar to the handling of function arguments. Each \verb|return| statement
in the original program introduces an assignment to the global return
tmpvar. After the return from a function call, the ICFG introduces a
\verb|ReturnAssignment| statement that assigns the tmpvar's value to another
tmpvar specific to this call site. The original expression containing the
function call is rewritten in the ICFG to refer to the function's return
value through this tmpvar.

\paragraph{Function calls} Calls to functions that can be resolved
statically by the ICFG builder are represented by \verb|FunctionCall| and
\verb|FunctionReturn| ICFG statements. From the call node, there is a
\verb|call_edge| to the called function's entry node; that function's exit
node is connected to the caller's return node by a \verb|return_edge|. The
call and return nodes in the caller are also connected by a
\verb|local_edge|. Calls that the ICFG builder cannot resolve by itself are
modeled similarly, but using \verb|ExternalCall| and \verb|ExternalReturn|
statements; some analyzers may be able to discover call targets and add
appropriate edges to the ICFG (SATIrE's points-to analysis does this, see
Section~\ref{sec:analysis_pointsto}). Constructor and destructor calls are
modeled like normal function calls if they can be resolved; otherwise, they
are modeled using \verb|ConstructorCall| and \verb|DestructorCall|
statements, respectively.

\paragraph{Loops and branches} All \verb|for| loops are normalized to
\verb|while| loops with the same semantics. The point just after the loop
exit is marked with a \verb|WhileJoin| statement; the point where the paths
from an \verb|if| statement converge is marked with \verb|IfJoin|.

\paragraph{Short-circuiting operators} The control flow related to operators
that do not necessarily evaluate all of their arguments---the \verb|&&|,
\verb&||&, \verb|?:| operators---must be modeled explicitly in the ICFG.
This is done using the \verb|LogicalIf| statement. This statement is like a
regular \verb|if| statement; the node containing it has outgoing
\verb|true_edge| and \verb|false_edge| edges to nodes modeling evaluation of
subexpressions as appropriate. Intermediate results are stored in tmpvars.

\subsection{Accessing ICFG Information}
\label{sec:icfg_information}

The SATIrE ICFG is represented by an instance of SATIrE's \texttt{CFG}
class. The members of this class provide a lot of information about the
ICFG, but there is no nice public interface yet. The header file
\verb|cfg_support.h| provides access to the definition of this class and its
helpers.

As the ICFG was implemented to support data-flow analysis using PAG, SATIrE
provides a complete implementation of the ICFG query functions required by
PAG and documented in Chapter~9 of the PAG manual\footnote{PAG is not
available to the general public. You might be able to get a copy of PAG,
or at least its manual, by asking AbsInt nicely. PAG's homepage is at
\texttt{http://www.absint.de/pag/}.}. The \texttt{KFG} type
required by PAG is defined to be the \verb|CFG *| pointer type.

\subsection{Traversing the ICFG}
\label{sec:icfg_traversal}

SATIrE provides an ICFG traversal mechanism similar in spirit to ROSE's AST
traversals. This traversal is an efficient way to visit all statements in
the ICFG to perform flow-insensitive analysis. Note that statements are
visited in no particular order; in fact, even visits to statements in
different functions may be intermingled.

The traversal mechanism is implemented in the abstract
\texttt{IcfgTraversal} class in \verb|CFGTraversal.h|. Users must define a
derived class and provide a definition for at least this pure virtual
method:
\begin{verbatim}
    virtual void icfgVisit(SgNode *node) = 0;
\end{verbatim}
This method will be invoked by the traversal mechanism on each statement of
the ICFG. Additionally, it is also invoked for each initializer expression
provided for a global variable in the program. The ICFG traversal only
touches the roots of initializer expressions and ICFG statements; if you
wish to descend deeper into expressions and statements, you will need to use
some additional mechanism (e.\,g., ROSE's AST traversals).

In addition to the visit method, the following methods may also be
overridden:
\begin{verbatim}
    virtual void atIcfgTraversalStart();
    virtual void atIcfgTraversalEnd();
\end{verbatim}
These methods are called before the first time the visit method is called,
and after the last time the visit method has been called, respectively. The
default implementations do nothing.

As noted above, the order in which parts of the ICFG are visited may appear
chaotic. The traversal mechanism provides a number of member functions to
help in finding out what is being visited. The
\begin{verbatim}
    bool is_icfg_statement() const;
\end{verbatim}
function can be called from within the visit method to determine whether the
node being visited is a global initializer expression or a statement in the
ICFG. When visiting ICFG statements, the following methods may also be
called:
\begin{verbatim}
    int get_node_id() const;
    int get_node_procnum() const;
\end{verbatim}
These provide unique numeric identifiers for the current ICFG node, and for
the procedure the current node is part of.

The traversal itself is started by calling the following method of the
\texttt{IcfgTraversal} class on an ICFG:
\begin{verbatim}
    void traverse(CFG *icfg);
\end{verbatim}

\section{Termite Terms}
\label{sec:representation_termite}

This section is not here yet. But a reference to Chapter~\ref{chap:termite}
is here.

\section{SATIrE's \texttt{Program} class}
\label{sec:satire_program}

The \texttt{Program} class defined by SATIrE encapsulates the various forms
of program representation mentioned above. \texttt{Program} enables SATIrE
to use a single interface for all analyzers, even though internally they may
prefer different program representations. It has public members
corresponding to the options associated with the program, and the
representations that have been computed for it:
\begin{verbatim}
    AnalyzerOptions *options;
    SgProject *astRoot;
    CFG *icfg;
    PrologTerm *prologTerm;
\end{verbatim}

The \texttt{options} member is initialized by \texttt{Program}'s
constructor, which takes a mandatory \verb|AnalyzerOptions *| argument. This
constructor typically also ensures that the ROSE AST member is initialized.
However, the ICFG member will typically be \texttt{NULL} until it is needed;
analyzers that run on the ICFG should check this pointer and initialize it
using the \texttt{createICFG} function (Section~\ref{sec:program_io}) if
necessary.

The \texttt{prologTerm} member can point to a Termite term representing the
program. It is initialized by the \texttt{outputProgramRepresentation}
function if the option to output a Termite term is set.

\chapter{Writing Data-Flow Analyzers}
\label{chap:data_flow}

This chapter describes how to implement data-flow analyzers using PAG and
SATIrE. In this connection, PAG is used to generate a data-flow analyzer
from a functional specification, while SATIrE provides the program
representation (the ICFG) the analyzer runs on, as well as a number of
useful support functions, supporting types, and supporting analyses.

\section{Implementing an Analyzer}
\label{sec:pag_analyzer}

Implementation of a new data-flow analyzer starts with initializing an
analyzer directory using SATIrE's \texttt{newanalysis} script.
\texttt{newanalysis} can be called with an analyzer name provided on the
command line, in which case it will create a new directory with that name;
otherwise, it will use the current directory as the analyzer directory. In
any case, \texttt{newanalysis} creates a number of C++ code and header files
as well as some additional files.

\texttt{newanalysis} is also meant to create a Makefile for the analyzer.
However, this can only be done if it knows the name of the carrier type of
the analysis. This name can either be passed on the command line, or
\texttt{newanalysis} attempts to find it in the \texttt{.optla} file of the
analysis. If no \texttt{.optla} file has been written yet, the script will
not create a Makefile; you can invoke it again later, once the
\texttt{.optla} file is there.

How to implement the analysis specification itself is covered in detail in
the PAG manual. SATIrE supports both separate \texttt{.set} and
\texttt{.optla} files, as well as a single \texttt{.optla} file that also
contains all type information, as described in the PAG manual. Once (a
rudimentary version of) the analysis specification exists, and a Makefile
has been generated (possibly by a repeated invocation of
\texttt{newanalysis}), simply type \verb|make|. This will call PAG to
generate C code from the specification, compile that code and SATIrE's
support code, and build an executable with the same name as the analyzer
directory.

This executable is your analyzer program: It takes file names and additional
options on the command line and performs the analysis on the input programs.
The options determine how analysis results are visualized or made permanent.
See Appendix~\ref{appendix:command_line} on the command line flags you can
pass to the analyzer.

\section{SATIrE Support Features}
\label{sec:support_stuff}

This section lists all CFG attributes, types, and auxiliary functions
accessible from PAG analysis specifications when using the SATIrE system.
Declarations for all of these features are provided by SATIrE; the user need
not repeat these declarations.

\newcommand{\ttitemX}[1]{\item[\texttt{#1}]}
\newcommand{\ttitem}[1]{\ttitemX{#1} \mbox{} \\}

\subsection{Attributes}
\label{sec:support_attributes}

A CFG attribute is accessible from the analysis specification by name,
essentially like a variable that may have different values in different
places. See Chapter 12 of the PAG manual for details on the general concept.
The attributes are presented in a way that is similar to the syntax used in
the \texttt{.optla} file.


\paragraph{\texttt{GLOBAL} Attributes}
\begin{description}
\ttitem{globals: *VariableSymbolNT \#}
    A list of the global variables in the program. Types and initializers of
    global variables can be accessed via auxiliary functions, see below.

\ttitem{numtypes: unum \#}
    The number of different data types in the program. Types are associated
    with unique numbers that can be manipulated using auxiliary functions,
    see below.

\ttitem{numexprs: unum \#}
    The number of lexically different expressions in the program. Each
    expression is associated with a unique integer that can be manipulated
    using auxiliary functions, see below.
\end{description}

\paragraph{\texttt{ROUTINE} Attributes}
\begin{description}
\ttitem{procnum: snum \#}
    A unique numerical identifier for the current procedure. Procedure
    numbers are never negative.
\end{description}

\paragraph{\texttt{BLOCK} Attributes}
\begin{description}
\ttitem{label: snum \#}
    A unique numerical label for the block; as each block in SATIrE consists
    of exactly one statement, this can also be considered a unique label for
    each statement. Labels are never negative.
\end{description}

\paragraph{\texttt{POSITION} Attributes}
\begin{description}
\ttitem{position: snum \#}
    A numerical label for what PAG calls the `position' of the analysis
    information: The pair (procnum, position) is a unique (opaque)
    identifier for the interprocedural analysis context at the current point
    in the analysis.

\ttitem{context: ContextInfo \#}
    A ContextInfo object identifying the current interprocedural analysis
    context.
\end{description}

\subsection{Types}
\label{sec:support_types}

SATIrE defines a number of data types that can be used in PAG analysis
specifications. In contrast to the AST types defined in the syn file, the
types listed here behave more like built-in PAG types. In particular, they
can be used in the analyzer's carrier type.

The following types are provided as opaque identifiers for important data:

\begin{description}
\ttitem{VariableId}
    A unique identifier for each variable in the program. Two variables have
    the same identifier iff they are in fact the same variable. That is,
    variables with the same name, but in different scopes, have different
    \texttt{VariableId}s. There are conversion functions described below
    that convert \texttt{VariableSymbolNT}s or \texttt{VarRefExp}s to the
    variable's \texttt{VariableId}.

\ttitem{ExpressionId}
    A unique identifier for each expression in the program. Expressions are
    identified iff they are structurally equivalent, that is, they consist
    of identical operators applied to identical operands. Leaf variables are
    compared as \texttt{VariableId}s are, so two occurrences of \verb|a + b|
    in the program text will get the same \texttt{ExpressionId} iff they
    refer to the same variables \texttt{a} and \texttt{b}.

\ttitem{TypeId}
    A unique identifier for each type in the program. Types are identical if
    they are the same basic type or the same class type (i.\,e., have the
    same definition, structural equivalence does not suffice) or are derived
    from the same basic/class type using exactly the same specifiers and
    modifiers (pointer, \texttt{const}, etc.).

\ttitem{Location}
    An abstract `memory region' computed by points-to analysis for every
    program variable and other expression that refers to an object in
    memory, such as a pointer dereference, structure field access, etc. Two
    expressions may be aliases iff they correspond to the same
    \texttt{Location}; conversely, expressions that have different
    \texttt{Location}s are definitely not aliases.

    See Section~\ref{sec:analysis_pointsto} for more information on SATIrE's
    points-to analysis.

\ttitem{ContextInfo}
    An abstract object representing the current interprocedural analysis
    context. This is meant mainly to be passed to support functions that can
    provide context-sensitive information.
\end{description}

Comparisons for equality and total ordering relations for these types are
provided by SATIrE.

\subsection{Functions}
\label{sec:support_functions}

Auxiliary functions are functions provided by the SATIrE library to support
some common operations that would be impossible or very complicated to
implement in FULA. The definitions for these functions are automatically
linked against the generated analyzer. The declarations below are
automatically included in any analyzer created with SATIrE.

\begin{description}
\ttitem{is\_unary :: Expression -> bool;}
    test whether an Expression is a unary expression

\ttitem{is\_binary :: Expression -> bool;}
    test whether an Expression is a binary expression

\ttitem{is\_value :: Expression -> bool;}
    test whether an Expression is a value expression (a constant appearing
    in the source code)

\ttitem{unary\_get\_child :: Expression -> Expression;}
    get the operand expression of a unary expression

\ttitem{binary\_get\_left\_child :: Expression -> Expression;}
    get left child of a binary expression

\ttitem{binary\_get\_right\_child :: Expression -> Expression;}
    get right child of a binary expression

\ttitem{is\_subtype\_of :: Type, Type -> bool;}
    test whether type1 is a subtype of type2 (in the object oriented class
    hierarchy)

\ttitem{expr\_type :: Expression -> Type;}
    get the type of an expression

\ttitem{global\_get\_type :: VariableSymbolNT -> Type;}
    get the type of the global variable

\ttitem{global\_has\_initializer :: VariableSymbolNT -> bool;}
    test whether the global variable has an initializer expression

\ttitem{global\_get\_initializer :: VariableSymbolNT -> Expression;}
    get the initializer expression of the global variable

\ttitem{varsym\_varid :: VariableSymbolNT -> VariableId;}
    maps a variable symbol to its VariableId

\ttitem{varref\_varid :: Expression -> VariableId;}
    maps a VarRefExp to its VariableId; it is an error to call this with any
    other subtype of Expression!

\ttitem{expr\_exprid :: Expression -> ExpressionId;}
    maps the expression to its ExpressionId

\ttitem{exprid\_expr :: ExpressionId -> Expression;}
    maps the expression identifier to the actual expression it represents

\ttitem{is\_tmpvarid :: VariableId -> bool;}
    determines whether the variable identifier refers to a temporary
    variable introduced by SATIrE (for logical values, function return
    values, etc.)

\ttitem{varid\_str :: VariableId -> str;}
    gives the name of the variable with the given identifier

\ttitem{exprid\_str :: ExpressionId -> str;}
    gives the string representation of the expression with the given
    identifier

\ttitem{varid\_exprid :: VariableId -> ExpressionId;}
    maps a variable identifier to an expression identifier which denotes a
    VarRefExp for that variable

\ttitem{type\_typeid :: Type -> TypeId;}
    convert a type to its corresponding TypeId

\ttitem{typeid\_type :: TypeId -> Type;}
    convert a type identifier to the actual type it represents

\ttitem{typeid\_str :: TypeId -> str;}
    convert a type identifier to a string representation of the type

\ttitem{exprid\_typeid :: ExpressionId -> TypeId;}
    get the type identifier for a given expression identifier

\ttitem{add\_tmpvarid :: TypeId -> VariableId;}
    creates a new, unique temporary variable of the given type; this
    function returns different values for each call

\ttitem{stmt\_asttext :: Statement -> str;}
    returns a string representing the structure of the given statement in a
    format very similar to PAG's pattern syntax

\ttitem{expr\_asttext :: Expression -> str;}
    returns a string representing the structure of the given expression in a
    format very similar to PAG's pattern syntax

\ttitemX{varid\_has\_location :: VariableId -> bool;}
\ttitem{varid\_location :: VariableId -> Location;}
    determine the abstract memory location corresponding to a variable;
    program variables have locations, SATIrE's temporary variables do not

\ttitemX{exprid\_has\_location :: ExpressionId -> bool;}
\ttitem{exprid\_location :: ExpressionId -> Location;}
    determine the abstract memory location corresponding to an expression,
    this can be a simple variable reference, but also a pointer dereference
    or other more complex expression; expressions that denote one of
    SATIrE's temporary variables, or expressions that do not denote an
    object in memory (an "lvalue") do not have locations

\ttitem{location\_varsyms :: Location -> *VariableSymbolNT;}
    returns the list of program variables stored in the given location

\ttitem{may\_be\_aliased :: Location -> bool;}
    determines whether the given location may be "aliased", i.e., whether
    some other location may hold a pointer to it

\ttitem{is\_ptr\_location :: Location -> bool;}
    determines whether the given location contains a pointer to some other
    location

\ttitem{dereference :: Location -> Location;}
    returns the pointed-to location of a given location that holds a pointer
\end{description}

\section{Abstract Syntax of SATIrE ICFG Statements}

The tree grammar describing the abstract syntax of the statements in the
ICFG is defined in the \texttt{syn} file in the SATIrE distribution. This is
the file PAG uses to generate its pattern matching code.

\section{Access to Call Strings}
\label{sec:call_strings}

SATIrE provides access to the call strings computed by PAG during
context-sensitive interprocedural analysis. After a PAG analysis has
completed, context information is added to the ICFG in its \texttt{contexts}
member. This is a container of \texttt{Context} objects as defined in
\verb|Context.h|. Each \texttt{Context} contains the corresponding procedure
and position identifiers as well as the call string itself: a sequence of
calls.

This feature is work in progress. In the future, it will be more powerful;
in particular, it will provide a way to access the caller's \texttt{Context}
from a given \texttt{Context}.

The \verb|--output-call-strings| command line flag instructs SATIrE
data-flow analyzers to print call string information. This call string
information looks something like this (the format is not set in stone):
\begin{verbatim}
    my_abs/0/0: main/613  ->  my_abs
    my_abs/0/1: main/613  ->  encode/233  ->  my_abs
    main/16/0: <spontaneous>
\end{verbatim}

Each line represents a context and the associated call string. The line
starts with the identifier of the context: the name of the current function,
its number, and the number of the context within this function. The call
string is a sequence of call sites leading to this function. Each call site
is identified by the name of the function containing the call, and the
identifier of the ICFG node containing the call statement. This ensures that
call sites can be uniquely identified even in functions that contain more
than one call to the same callee. Some contexts are identified as
`spontaneous'; this is typically the case for functions that are not called
from other functions in the program.

\chapter{Termite: Symbolic Program Analysis and Transformation}
\label{chap:termite}

This chapter is not here yet.

\appendix

\chapter{Installing SATIrE}
\label{appendix:installing}

See the \texttt{README} file in the SATIrE distribution. Use the
\texttt{--without-pag} configure flag if you wish to install SATIrE without
PAG support.

\chapter{Analyzers Provided by SATIrE}
\label{appendix:analyzers}

This appendix lists the analyzers that come with SATIrE to provide support
for other analyzers you wish to build.

\section{Data-Flow Analyzers}
\label{sec:analysis_dataflow}

SATIrE comes with a number of data-flow analyzers implemented using PAG.
Currently, these are:

\begin{center}
\begin{tabular}{l l p{.5\textwidth}}
Name & Carrier & Description \\
\hline
\texttt{constprop} & \verb|cp_LiftedState| &
    sketch of a constant folding/constant propagation analysis for integers
\end{tabular}
\end{center}

\subsection{Using Provided Data-Flow Analyzers}
\label{sec:analysis_dataflow_use}

The predefined data-flow analyzers are included in the \texttt{libsatiredfa}
library, and the declarations of the corresponding C++ classes can be
included with the \texttt{satiredfa.h} header file. SATIrE's
\texttt{newanalysis} script will generate Makefiles and code skeletons that
use these automatically.

To run some predefined analyzer before your analysis, use SATIrE's analysis
scheduler (see Section~\ref{sec:run_analyzer}). At some point before the
\texttt{run} method of your analysis is called, you can declare a dependency
on the predefined analyzer using the \texttt{dependsOnAnalysis} method.

For instance, to use SATIrE's \texttt{constprop} analyzer before another
data-flow analyzer implemented using SATIrE, add the following statement
before the call to \texttt{run} in your analyzer's \texttt{main.C} file:

\begin{verbatim}
    analysis->dependsOnAnalysis(new DataFlowAnalysis(
                new SATIrE::constprop::Implementation()));
\end{verbatim}

The analysis scheduler will take care of running all analyzers in an order
that satisfies their dependencies.

You can access another analyzer's analysis data as described in the PAG
manual in the chapter entitled `Advanced Usage: Multiple Analyses'.
Essentially, each analysis has certain associated \texttt{NODE} and
\texttt{POSITION} attributes that another analysis can access from its
transfer functions.

Note that the exchange of analysis data is only guaranteed to work if all
involved analyzers use the same fixed-point iteration scheme. By default,
all analyzers generated with SATIrE use the same iterator
(\texttt{iterate1.t}).

\subsection{SATIrE Developers: Adding Analyzers}
\label{sec:analysis_dataflow_add}

To add a data-flow analyzer to the SATIrE library, follow these steps:

\begin{enumerate}
\item Add the analyzer specification in its own subdirectory under SATIrE's
\texttt{examples} directory. The analyzer \emph{must have an analysis
prefix} declared using \texttt{prefix:} in the problem description.
\item Add the type declarations for the analyzer's carrier type, and all of
the types it is constructed from, to the \texttt{pagoptions.set} file in
SATIrE's \texttt{src/analyzer/astaccess/satire} directory. To avoid name
clashes, you should add the analyzer's prefix to these type names.
\item Add the analyzer name to the definition of the \texttt{ANALYZERS}
variable in \texttt{src/analyzer/provided/dataflow/Makefile.am}. You can
ignore the rest o the Makefile; it will take care of building the analyzer,
including it in \texttt{libsatiredfa}, and installing the result.
\item Add a brief summary describing the analysis in the table above.
\item Do a \emph{very thorough} cleaning and rebuild of SATIrE, and test
whether other analyzers can access your analysis data.
\end{enumerate}

If all you are adding is a data-flow analyzer without any external support
functions (except those provided by SATIrE), this should work.

\section{Points-to Analysis}
\label{sec:analysis_pointsto}

%// The basic abstraction computed by this analysis is the Location, an
%// abstract memory region. Variables and functions (identified by their
%// respective symbols) live in Locations. Pointers are modeled by points-to
%// relations between Locations: each Location may have at most one "base
%// location", which is what it may point to. In this unification-based
%// analysis, Locations are merged when the same pointer may point to each of
%// them; that is, if the program contains the assignment sequence
%//      p = &a; p = &b;
%// then p's Location will point to a Location that contains both variables a
%// and b.
%// Each array is treated as a single object, i.e., all members live in the
%// same Location. It is assumed that array indexing and pointer arithmetic
%// always stay in the same object (as required by the Standard), so these
%// are safely ignored.
%// In contrast, structs are treated field-sensitively: Each struct instance
%// corresponds to a location, and each of its fields has its own location as
%// well. Such structs are collapsed when needed (if pointer arithmetic is
%// performed on a pointer to the structure or one of its members).
%
%// In principle, the heap can be modeled using the same abstractions.
%// However, this requires function summaries for allocation functions, which
%// are not here yet.
%
%    Points-to analysis must be activated using the
%    \texttt{--run-pointsto-analysis} command line flag; otherwise, analyzers
%    trying to access \texttt{Location}s will segfault. The analysis is a
%    variant of Steensgaard's flow-insensitive, context-insensitive analysis.
%    It works on C programs that are complete, i.\,e., do not call any
%    external functions. Casting and structures are supported. Pointer
%    arithmetic is supported as long as each pointer stays within one memory
%    region (array).


SATIrE comes with a flow-insensitive context-insensitive unification-based
points-to analysis in the spirit of Steensgaard~[cite\dots]. The analysis
computes points-to information for complete C programs, i.\,e., programs
that do not call any external functions. The analysis supports all C
language features, including typecasts, structures, arrays, pointer
arithmetic, and function pointers. It does not compute alias pairs; however,
a client analysis could in principle compute may-alias pairs from the
points-to representation.

The basic abstraction computed by the points-to analysis is the
\texttt{Location}, an abstract memory region. Variables and functions
(identified by their respective symbols) live in \texttt{Location}s.
Pointers are modeled by points-to relations between \texttt{Location}s: Each
\texttt{Location} may have at most one `base location', which is what it may
point to. In this unification-based analysis, \texttt{Location}s are merged
when the same pointer may point to each of them; that is, if the program
contains the assignments
\begin{verbatim}
    p = &a;
    p = &b;
\end{verbatim}
then \verb|p|'s \texttt{Location} will point to a \texttt{Location} that
contains both variables \verb|a| and \verb|b|.

Each array is treated as a single object, i.\,e., all members live in the
same \texttt{Location}. It is assumed that array indexing and pointer
arithmetic always stay in the same object (as required by the Standard), so
these are safely ignored. In contrast, structures are treated
field-sensitively: Each \texttt{struct} instance corresponds to a
\texttt{Location}, and each of its fields has its own \texttt{Location} as
well. Such structures are collapsed when needed (if pointer arithmetic is
performed on a pointer to the structure or one of its members).

Steensgaard's basic analysis is almost linear in the program size, taking
\(O(N \alpha(N))\), where \(N\) is some reasonable measure of the program
size and \(\alpha\) is an inverse of Ackermann's function. The SATIrE
implementation is a little more complex because of its more sophisticated
handling of structures. In practice, it is still very fast, however.

The points-to analysis runs on the ICFG, using the traversal mechanism
described in Section~\ref{sec:icfg_traversal}. It is most easily activated
by invoking the analyzer with the \texttt{--run-pointsto-analysis} command
line flag.

The \texttt{CFG} class contains a member that may be instantiated to a
points-to analyzer object:
\begin{verbatim}
    SATIrE::Analyses::PointsToAnalysis *pointsToAnalysis;
\end{verbatim}

The \texttt{PointsToAnalysis} class is defined in the \verb|pointsto.h|
header file. It implements the general analyzer interface described in
Section~\ref{sec:analyzer_interface}. The most important methods for
accessing points-to analysis results are:
\begin{verbatim}
    Location *expressionLocation(SgExpression *expr);
    Location *symbol_location(SgSymbol *sym);
\end{verbatim}
for access to \texttt{Location}s for expressions or variable/function
symbols, where two expressions/symbols may be aliases iff they are
associated with the same \texttt{Location} pointer;
\begin{verbatim}
    const std::list<SgSymbol *> &
    location_symbols(Location *loc) const;
\end{verbatim}
for access to all the symbols that are associated with a given
\texttt{Location};
\begin{verbatim}
    bool mayBeAliased(Location *loc) const;
\end{verbatim}
to find out whether there might be a pointer pointing to a given
\texttt{Location}; and
\begin{verbatim}
    Location *base_location(Location *loc);
    bool valid_location(Location *loc) const;
\end{verbatim}
to check what (if anything) a given \texttt{Location} may point to.

While the points-to analysis works fine, there is work in progress on making
it more precise and applicable to a wider range of programs. Precision will
be improved by adding context-sensitive elements; it will be possible to use
the analysis even for incomplete programs by providing summaries for
external functions. Using such summaries for allocation functions, the
latter will also allow analysis of points-to relationships on the heap.

\section{Loop Bounds Analysis}
\label{sec:analysis_loopbounds}

\section{Coherence Analysis}
\label{sec:analysis_coherence}

\chapter{SATIrE Analyzer Command Line Flags}
\label{appendix:command_line}

SATIrE analyzers take a number of command line flags, arbitrarily
intermingled with input file names. This appendix lists these flags grouped
by topic.

\section{General Flags}
\label{sec:general_flags}

The flags listed in this section are available for use with all analyzers
that use the \texttt{AnalyzerOptions} object (see
Section~\ref{sec:command_line}).

\subsection{Front End Options}

The following options control various issues related to reading input
programs:
{\footnotesize
\begin{verbatim}
--language=c++|c99|c89   select input language [default=c++]
--frontend-warnings      show Front End warnings when parsing file(s)
--no-frontend-warnings   do not show Front End warnings when parsing
                         file(s) [default]
-I<path>                 specify path for include files
--input-binary-ast=<FILENAME> read AST from binary file instead of a
                         source file
\end{verbatim}
}

\subsection{General Analysis Options}

These options mostly deal with what sanity checks and provided analysis
steps should be performed by SATIrE.

{\footnotesize
\begin{verbatim}
--check-ast              run all ROSE tests for checking
                         whether ROSE-AST is correct
--no-check-ast           do not run ROSE AST tests [default]
--check-icfg             run PAG's ICFG consistency checks
--no-check-icfg          do not run ICFG checks [default]
--analysis-files=all|cl  analyse all source files or only those
                         specified on the command line [default=cl]
--analysis-annotation    annotate analysis results in AST and output
                         [default]
--no-analysis-annotation do not annotate analysis results in AST
--number-expressions     number expressions and types in the ICFG [default]
--no-number-expressions  do not number expressions and types in the ICFG
--run-pointsto-analysis  run a points-to analysis on the ICFG
--resolve-funcptr-calls  resolve indirect calls using pointer analysis
\end{verbatim}
}

The default setting is not to run points-to analysis
(Section~\ref{sec:analysis_pointsto}), and thus also not to attempt to
resolve function pointer calls.

\subsection{Output Options}

These flags control the amount and type of output---informational messages,
annotated programs, analysis results---from the analyzer.

{\footnotesize
\begin{verbatim}
--statistics             output analyzer statistics on stdout
--no-statistics          do not show analyzer statistics on stdout
--verbose                output analyzer info on stdout
--no-verbose             do not print analyzer info on stdout
--output-text            print analysis results for each statement
--output-collectedfuncs  print all functions that are collected for
                         the icfg generation
--output-source=<FILENAME> generate source file with annotated 
                         analysis results for each statement
--output-term=<FILENAME> generate Prolog term representation of input
                         program AST
--output-icfg=<FILENAME> output icfg of input program
--output-binary-ast=<FILENAME> write AST to binary file
--warn-deprecated        warn about the use of deprecated features
--no-warn-deprecated     do not warn about the use of deprecated features
--help                   print this help message on stdout
--help-rose              print the ROSE help message on stdout
\end{verbatim}
}

The default output settings are as if the \texttt{--no-statistics
--verbose --warn-deprecated} flags had been specified.

\subsection{Multiple Input/Output Files Options}

This option can be used to set a file name prefix for output of several source
files for several input files.

{\footnotesize
\begin{verbatim}
--output-sourceprefix=<PREFIX> generate for each input file one output file
                         with prefixed name
\end{verbatim}
}

\section{PAG-Specific Flags}
\label{sec:pag_flags}

The flags listed in this section are only available if SATIrE has been built
with PAG support (see Appendix~\ref{appendix:installing}).

\subsection{PAG-Specific Analysis Options}

These flags control options mostly related to PAG's fixed point search: The
maximal call string length to use for context-sensitive interprocedural
analysis, whether to compute SATIrE's context information (see
Section~\ref{sec:call_strings}) from PAG's representation, the CFG ordering
to use, and issues related to trading memory against execution time.

{\footnotesize
\begin{verbatim}
--callstringlength=<num> set callstring length to <num> [default=0]
--callstringinfinite     select infinite callstring (for non-recursive
                         programs only)
--compute-call-strings   compute representation of call strings [default]
--no-compute-call-strings do not compute call strings
--output-call-strings    experimental: print call strings used by PAG
--no-output-call-strings do not attempt to output call strings [default]
                         programs only)
--output-context-graph=<FILENAME> output DOT graph of calling contexts
--cfgordering=<num>      set ordering that is used by the iteration
                         algorithm where
                               <num> = 1 : dfs preorder [default]
                                       2 : bfs preorder
                                       3 : reversed dfs postorder
                                       4 : bfs postorder
                                       5 : topsort scc dfs preorder
                                       6 : topsort scc bfs preorder
                                       7 : topsort scc reversed bfs
                                           dfs postorder
                                       8 : topsort scc bfs postorder
--pag-memsize-mb=<num>   allocate <num> MB of memory for PAG analysis
--pag-memsize-perc=<num> allocate <num>% of system memory (autodetected)
--pag-memsize-grow=<num> grow memory if less than <num>% are free after GC
\end{verbatim}
}

The default settings in addition to the defaults indicated above are:
\texttt{--pag-memsize-mb=5 --pag-memsize-grow=30}.

\subsection{GDL Output Options}

These options control the GDL visualization that PAG can generate for use
with the aiSee program. The most interesting flag is
\texttt{--gdl-nodeformat}, which can be used multiple times to specify a
combination of formatting options. These formats control whether SATIrE's
supporting \texttt{Id} types (see Section~\ref{sec:support_types}) are to be
printed as numeric identifiers, as variable names/source code, or as both,
in the analysis information. The \texttt{asttext} setting controls whether
ICFG statements are to be shown as source code or as a structural
representation of the AST (similar to PAG's patterns).

{\footnotesize
\begin{verbatim}
--gdl-preinfo            output analysis info before cfg nodes
--no-gdl-preinfo         do not output analysis info before cfg nodes
--gdl-postinfo           output analysis info after cfg nodes
--no-gdl-postinfo        do not output analysis info after cfg nodes
--gdl-nodeformat=FORMAT  where FORMAT=varid|varname|exprid|exprsource
                                     |asttext|no-varid|no-varname
                                     |no-exprid|no-exprsource|no-asttext
                         the format can be specified multiple times to
                         have different formats printed at the same node
                         The output is only affected if VariableId
                         and/or ExpressionId is used in the carrier type
--output-gdl=<FILENAME>  output program as gdl graph
--output-gdlanim=<DIRNAME> output animation gdl files in 
                         directory <dirname>
\end{verbatim}
}

The default settings are as if the following flags were specified on the
command line:
{\footnotesize
\begin{verbatim}
    --no-gdl-preinfo --gdl-postinfo
    --gdl-nodeformat=no-asttext
    --gdl-nodeformat=no-varid --gdl-nodeformat=varname
    --gdl-nodeformat=no-exprid --gdl-nodeformat=exprsource
\end{verbatim}
}

That is, statements, variables, and expressions are printed only as they
would be represented in source code. Analysis information is associated with
the outgoing edges of ICFG nodes.

\end{document}
