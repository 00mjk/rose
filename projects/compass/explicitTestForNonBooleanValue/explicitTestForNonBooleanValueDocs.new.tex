% 9.5.07
% This is a sample documentation for Compass in the tex format.
% We restrict the use of tex to the following subset of commands:
%
% \section, \subsection, \subsubsection, \paragraph
% \begin{enumerate} (no-nesting), \begin{quote}, \item
% {\tt ... }, {\bf ...}, {\it ... }
% \htmladdnormallink{}{}
% \begin{verbatim}...\end{verbatim} is reserved for code segments
% ...''
%

\section{Explicit Test For Non Boolean Value}
\label{ExplicitTestForNonBooleanValue::overview}

   This test examines all the test statements whether there is a statement that calls a function call returning a non-boolean value and that does not compare the return value to an explicit value.
   For example, if a function \texttt{foo()} returns an integer value and the function is used in a conditional statement, such as \texttt{"if", "while", "do-while", "for"}, or the first operand of "?" operator, the boolean expressions in the conditional statement should always use an explicit test of equality or non-equality. Therefore the following code can pass this checker:

\begin{verbatim}
	if(foo()!=0)
	{// do something}
\end{verbatim}

whereas, 
\begin{verbatim}
	if(foo())
	{// do something}
\end{verbatim}

will be caught by this checker because \texttt{foo()} returns an integer, non-boolean value.

\subsection{Parameter Requirements}

%Write the Parameter specification here.
None   

\subsection{Implementation}

%Details of the implementation go here.
This pattern is detected using a simple traversal. It traverses AST to search conditional statements and if an implicit expression is used in the test, AST contains a casting expression node underneath the conditional statement to convert from a non-boolean values to a boolean value. The checker captures this structure.


\subsection{Non-Compliant Code Example}

% write your non-compliant code subsection

\begin{verbatim}
int bar();

void foo()
{
  int i;
  if(bar())
    i = 2;

  while(bar())
    i = 3;

  do {
    i = 4;
  } while(bar());

  for(i=0; bar(); i++)
    i =5;

  i = (bar() ? 6 : 7);

  for(i = (bar() ? 8 : 9); bar(); i++)
    i = 10;
}
\end{verbatim}

\subsection{Compliant Solution}

% write your compliant code subsection

\begin{verbatim}
	if(foo()!=0)
	{// do something}
\end{verbatim}

\subsection{Mitigation Strategies}
\subsubsection{Static Analysis} 

Compliance with this rule can be checked using structural static analysis checkers using the following algorithm:

\begin{enumerate}
\item Check if a node is a conditional statement
\item Check further if the conditional statement contains an implicit expression.
\end{enumerate}

\subsection{References}

% Write some references
% ex. \htmladdnormallink{ISO/IEC 9899-1999:TC2}{https://www.securecoding.cert.org/confluence/display/seccode/AA.+C+References} Forward, Section 6.9.1, Function definitions''
The Programming Research Group, High-Integrity C++ Coding Standard Manual, Item 5.2: ``For boolean epressions('if', 'for', 'while', 'do' and the first operand of the ternary operator '?:') involving non-boolean values, always use an explicit test of equality or non-equality."
