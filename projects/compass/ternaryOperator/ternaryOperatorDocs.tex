% 9.5.07
% This is a sample documentation for Compass in the tex format.
% We restrict the use of tex to the following subset of commands:
%
% \section, \subsection, \subsubsection, \paragraph
% \begin{enumerate} (no-nesting), \begin{quote}, \item
% {\tt ... }, {\bf ...}, {\it ... }
% \htmladdnormallink{}{}
% \begin{verbatim}...\end{verbatim} is reserved for code segments
% ...''
%

\section{Ternary Operator}
\label{TernaryOperator::overview}

% write your introduction
This checker detects a expression that uses the ternary operator. The rationale for this checker is, according to ``High-Integrity C++ Coding Standard Manual", because evaluation of a complex condition is best achieved through explicit conditional statements.

\subsection{Parameter Requirements}

%Write the Parameter specification here.
None.

\subsection{Implementation}

%Details of the implementation go here.
This checker is implemented with a simple traversal. It traverses AST, checks if a statement uses a ternary operator, and reports it if yes.

\subsection{Non-Compliant Code Example}

% write your non-compliant code subsection

\begin{verbatim}

void foo()
{
	int i = 0;
	int j;

	(i==3) ? (j=1) : (j=2);
}
\end{verbatim}

\subsection{Compliant Solution}

% write your compliant code subsection

\begin{verbatim}
void foo()
{
	int i = 0;
	int j;

	if(i == 4)
		j = 1;
	else
		j = 2;
}
\end{verbatim}

\subsection{Mitigation Strategies}
\subsubsection{Static Analysis} 

Compliance with this rule can be checked using structural static analysis checkers using the following algorithm:

\begin{enumerate}
\item For each node, check if a node represents a ternary operator.
\item If yes, report it
\end{enumerate}

\subsection{References}

% Write some references
% ex. \htmladdnormallink{ISO/IEC 9899-1999:TC2}{https://www.securecoding.cert.org/confluence/display/seccode/AA.+C+References} Forward, Section 6.9.1, Function definitions''

The Programming Research Group, High-Integrity C++ Coding Standard Manual, Item 10.20: ``Do not use the thernary operator(?:) in expressions."
