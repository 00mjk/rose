% 9.5.07
% This is a sample documentation for Compass in the tex format.
% We restrict the use of tex to the following subset of commands:
%
% \section, \subsection, \subsubsection, \paragraph
% \begin{enumerate} (no-nesting), \begin{quote}, \item
% {\tt ... }, {\bf ...}, {\it ... }
% \htmladdnormallink{}{}
% \begin{verbatim}...\end{verbatim} is reserved for code segments
% ...''
%

\section{Other Argument}

\label{OtherArgument::overview}

This checker enforce the name convention of the first argument in copy
constructors and copy operators. This is taken from rule 23 from
\emph{the Elements of C++ Style} (Misfeldt and al., 2004). The
parameter should be called \texttt{other}. This checker also accept
two other naming conventions: \texttt{that} and the class name in
lower camel case.

\subsection{Parameter Requirements}

There is no parameter requirement.

\subsection{Non-Compliant Code Example}

\begin{verbatim}
A::A(const A& foo)
{
  //...
}
\end{verbatim}

\subsection{Compliant Solution}

\begin{verbatim}
A::A(const A& other)
{
  //...
}
\end{verbatim}

\subsection{Mitigation Strategies}
\subsubsection{Static Analysis}

Compliance with this rule can be checked using structural static
analysis checkers using the following algorithm:

\begin{enumerate}
\item  For all constructors and operator fulfilling the
copy requirement of the C++ standard, check that the first parameter is
of the three possible name.
\end{enumerate}

\subsection{References}

Bumgardner G., Gray A., and Misfeldt T. {\it The Elements of C++
Style}. Cambridge University Press 2004.
