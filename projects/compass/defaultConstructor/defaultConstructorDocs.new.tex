% 9.5.07
% This is a sample documentation for Compass in the tex format.
% We restrict the use of tex to the following subset of commands:
%
% \section, \subsection, \subsubsection, \paragraph
% \begin{enumerate} (no-nesting), \begin{quote}, \item
% {\tt ... }, {\bf ...}, {\it ... }
% \htmladdnormallink{}{}
% \begin{verbatim}...\end{verbatim} is reserved for code segments
% ...''
%

\section{Default Constructor}
\label{DefaultConstructor::overview}

{\it The Elements of C++ Style} item \#97 states that
\begin{quote}
Declare a default constructor for every class you create. Although some compilers may be able to automatically generate a more efficient implementation in some situations, choose an explicit default constructor for added clarity.
\end{quote}

\subsection{Parameter Requirements}
This checker takes no parameters and inputs source file.

\subsection{Implementation}
This pattern is checked using a simple AST traversal that seeks out instances of SgClassDefinition. The children of these instances are stored in a successor container and looped over to find a default constructor. If no such default constructor exists then a violation is flagged.

\subsection{Non-Compliant Code Example}
The following trivial example does not declare a default constructor.

\begin{verbatim}
class Class
{
  public:
    ~Class(){}
}; //class Class
\end{verbatim}

\subsection{Compliant Solution}
The compliant solution simply adds a default constructor to the class definition.

\begin{verbatim}
class Class
{
  public:
    Class(){}
    ~Class(){}
}; //class bad
\end{verbatim}

\subsection{Mitigation Strategies}
\subsubsection{Static Analysis} 

Compliance with this rule can be checked using structural static analysis checkers using the following algorithm:

\begin{enumerate}
\item Perform AST traversal visiting the member functions of class definitions.
\item If no constructor is found for a single class definition then flag violation.
\item Report any violations.
\end{enumerate}

\subsection{References}

Bumgardner G., Gray A., and Misfeldt T. {\it The Elements of C++ Style}. Cambridge University Press 2004
