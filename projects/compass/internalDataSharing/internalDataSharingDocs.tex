% 9.5.07
% This is a sample documentation for Compass in the tex format.
% We restrict the use of tex to the following subset of commands:
%
% \section, \subsection, \subsubsection, \paragraph
% \begin{enumerate} (no-nesting), \begin{quote}, \item
% {\tt ... }, {\bf ...}, {\it ... }
% \htmladdnormallink{}{}
% \begin{verbatim}...\end{verbatim} is reserved for code segments
% ...''
%

\section{Internal Data Sharing}
\label{InternalDataSharing::overview}

Classes should usually not return handles to internal data from methods. A
`handle' in this sense is a non-const reference to a member or copy of a
pointer member, as the caller could change internal state through such an
object. This checker reports such cases. One possible exception to this rule
are overloaded operators, which often return such handles (so they can be
combined with other operators to bigger expressions), and this checker
provides a parameter to define whether operators should be allowed to return
internals.

\subsection{Parameter Requirements}

The bool flag {\bf InternalDataSharing.operatorsExcepted} states whether
overloaded operators are excepted from this checker's rules, i.e. whether
they are allowed to return internal data.

\subsection{Non-Compliant Code Example}

\begin{verbatim}
class A
{
public:
    // not OK: returning non-const pointer member
    int *retptr() { return p; }
    // not OK: returning non-const reference to data pointed to by member
    int &retref() { return *p; }
    // not OK: returning non-const reference to member
    int *&retptrref() { return p; }

private:
    int *p;
};
\end{verbatim}

\subsection{Compliant Solution}

\begin{verbatim}
class B
{
public:
    // OK: returning copy of pointed-to data
    int retint() { return *p; }
    // OK: const ptr
    const int *retcptr() { return p; }
    // OK: const ref
    const int &retcref() { return *p; }

    // maybe OK, depending on "operatorsExcepted" parameter
    int &operator*() { return *p; }

private:
    int *p;
};
\end{verbatim}

\subsection{Mitigation Strategies}
\subsubsection{Static Analysis} 

Compliance with this rule can be checked using structural static analysis checkers using the following algorithm:

\begin{enumerate}
\item While traversing the program representation, set a flag upon entering a
member function definition that has a non-const pointer or reference return
type.
\item Report any return statement within such a flagged function that returns
a (possibly dereferenced) member variable.
\end{enumerate}

\subsection{References}

% Write some references
% ex. \htmladdnormallink{ISO/IEC 9899-1999:TC2}{https://www.securecoding.cert.org/confluence/display/seccode/AA.+C+References} Forward, Section 6.9.1, Function definitions''
A reference in the literature is: H.~Sutter, A.~Alexandrescu: ``C++ Coding
Standards'', Item~42: ``Don't give away your internals''. Their notion of a
handle is more general, however, as it also includes basic types that are used
as handles, such as ints that are used as file descriptors.
