\documentclass[english,12pt, titlepage]{article}      % Specifies the document class
\usepackage{babel}
\usepackage{moreverb}
\usepackage{alltt}
\usepackage{mathptm}
\usepackage{epsf}

%\usepackage{latexsym}
%\usepackage{rotating}
%\usepackage{IRA,protitle}
%\usepackage{german,amsfonts,amssymb}                                           
\usepackage{amssymb}

\newcommand{\outcomment}[1]{}
\newcommand{\logand}{~\land~}
\newcommand{\logor}{~\lor~}
\newcommand{\lognot}{~\lnot}
\newcommand{\impl}{\to}
\newcommand{\existsone}[0]{~\exists!}

\newcommand{\setunion}{\cup}
\newcommand{\setintersect}{\cap}
\newcommand{\setdif}{\backslash}

\newcommand{\set}[1]{\{{#1}\}}
\newcommand{\pair}[2]{({#1},{#2})}
\newcommand{\tripel}[3]{({#1},{#2},{#3})}
\newcommand{\quadrupel}[4]{({#1},{#2},{#3},{#4})}


\newcommand{\cnodedef}[0]{\pair{ \set{ \pair{*}{\epsilon} } }{.} }
\newcommand{\cnode}[0]{\diamond}
\newcommand{\cnodeset}[0]{\set{\diamond}}

\newcommand{\dach} {\symbol{94}}
\newcommand{\deref}[1]{{#1}\dach}

\newcommand{\nsubset}{\not\subset}         %\\nsubset
\newcommand{\textflorin}{\textit{f}}       %\\textflorin
\newcommand{\setB}{{\mathord{\mathbb B}}}  %\\setB
\newcommand{\setC}{{\mathord{\mathbb C}}}  %\\setC
\newcommand{\setN}{{\mathord{\mathbb N}}}  %\\setN
\newcommand{\setQ}{{\mathord{\mathbb Q}}}  %\\setQ
\newcommand{\setR}{{\mathord{\mathbb R}}}  %\\setR
\newcommand{\setZ}{{\mathord{\mathbb Z}}}  %\\setZ
\newcommand{\coloncolon}{\mathrel{::}}     %\\coloncolon
\newcommand{\lsemantics}{\mathopen{\lbrack\mkern-3mu\lbrack}}  %\\lsemantics
\newcommand{\rsemantics}{\mathclose{\rbrack\mkern-3mu\rbrack}} %\\rsemantics

\newcommand{\forwardfunctions}{forward-functions}
\newcommand{\reversefunctions}{reverse-functions}
\newcommand{\commitfunctions}{commit-functions}

\newcommand{\bsshortversion}{2.0}
\newcommand{\bslongversion}{2.0.0}



\title{Backstroke \bsshortversion\\ User Manual}  % Declares the document's title.
\author{Markus Schordan}      % Declares the author's name.
%\date{}      % Deleting this command produces today's date.

\begin{document}             % End of preamble and beginning of text.

\maketitle                   % Produces the title.

\tableofcontents        
\clearpage

\section{Backstroke}

\subsection{Introduction}
Backstroke is a tool for generating code required to run a computation
in reverse (backwards). It is used in the context of parallel descrete
event simulation to generate code that allows to restore a previous
state in execution of a simulation. The advantage of reverse code is
that it stores less information that would be reqired if the entire
state would be stored.

Executing the Backstroke generated code instead of the original code, will
store some additional information that is required to perform the
computation in reverse (in comparison to the original program). The
implemented approach is a variant of incremental state saving. The
particular instance of incremental state saving that is implemented in
Backstroke, restores memory in reverse. It addresses dynamic memory
allocation and can be applied to C/C++.

The generated Code of Backstroke is C++ code (also when the input code
is pure C). The backstroke library is implemented in C++.

\subsection{Conventions}

Backstroke generates reverse code, separated in \forwardfunctions,
\reversefunctions, and \commitfunctions. The functionality of reversability is
implemented by all three functions and the Backstroke library. Backstroke currently provides exactly one mode for incremental statesaving for which Backstroke technically only
{\em generates} the \forwardfunctions. For the \reversefunctions~ and \commitfunctions~
generic variants exist in the backstroke library. However, those
generic functions do require the \forwardfunctions~ to be executed in order
to operate correctly. We will use the term ``reverse code generation'' to refer to all code that is generated and required to perform a reverse computation (including \forwardfunctions, \reversefunctions, \commitfunctions, Backstroke library), but refer specifically to \forwardfunctions~ generation, if this is technically relevant. 

\section{Installation}

Backstroke requires the installation of ROSE (\verb+http://www.rosecompiler.org+).
After the installation of ROSE make the installed ROSE libraries available in \verb+LD_LIBRARY_PATH+.
The compilation of Backstroke requires the rose library to be available. There are no other dependencies.

\section{Using Backstroke}

\subsection{What Backstroke Does}
% list the exact tasks 

Backstroke generates \forwardfunctions~ (from the providid original
code). The \forwardfunctions~ execute the statements in the same order as
the original program, but store additional information for all memory
modiyfing operations. Memory modifying operations are all forms of
assignment, memory allocation, and memory deallocation.

The reverse-functions and commit-functions are implemented in the Backstroke
library and only depend on the recorded information of the
\forwardfunctions. Therefore it is sufficient to link the generated
\forwardfunctions~ with the backstroke library. Backstroke generates the
\forwardfunctions~ and the library is provided as part of the backstroke
distribution.

For each input file Backstroke generates one output file. The output
file contains \forwardfunctions~ for all functions that are present in the
input file.

\subsection{What Backstroke Does Not Do}

Backstroke does not perform the integration of the generated code into
the original application. The original application must be modified by
the user to take the generated \forwardfunctions~ into account and link with
the backstroke library.

\subsection{Invoking Backstroke}

\verb+backstroke [Options] input-file output-file+

\begin{description}
\item [--ignore-external-functions]: ignore all external functions not present in the translation unit. By default backstroke will require implementations of all functions to be able to generate reverse code. For large projects, involving multiple translation units, this is not always possible. Therefore this check can be turned off.
\item [--custom-namespace-name=$<$Name$>$]: provide customized name of namespace for generated code. Default is ``backstroke''.
\end{description}

\subsection{Integration of Backstroke Generated Code}

We will show the use of Backstroke with a ROSS model. A ROSS model
consists of code implementing an event function, some data structures,
and calls to ROSS functions. To simplify the use of Backstroke, it is
best to have the event function and all functions that are called by
the event function in files separate from the simulation code.

Backstroke generates all code in a separate namespace ``reverse''. For
example, if the original function was called foo, the generated
reverse code is in ``reverse::foo''.

\section{Examples}

\begin{figure}[h!]
\begin{boxedverbatim}
        void event(State* state) {           
          if(state->x > 20) {
            state -> x = 0;
          } else {
            state->++x;
          }                  
\end{boxedverbatim}
\caption{Example of updates on a state variable.}\label{exampleprogram1}
\end{figure}

%\bibliography{backstroke}
%\bibliographystyle{unsrt} 
%\hyphenation{}
\end{document}

