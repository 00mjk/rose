\chapter{Installing and Building POP}\label{ch:install-build}

Now that you've presumably perused the first chapter
describing all the great features of POP, you're
understandably anxious to leap in and run some
simulations yourself.
The general procedure for unpacking and building a POP executable
will be described here. Exceptions to this procedure for some
architectures will be noted.

\section{Supported architectures}\label{sec:architectures}

POP uses standard Fortran and will work on any machine
with a compliant Fortran compiler (Fortran here denotes
Fortran 90 or later -- FORTRAN 77, FORTRAN 66, FORTRAN IV
or other antiquated dialects are only supported to the
extent that F90 is backward compatible).
Two levels of parallelism are supported and combinations
of these two levels can be used on architectures which
support them.  For shared-memory parallelism, OpenMP
can be used.  For distributed-memory parallelism, MPI
or SHMEM can be used.  In clustered SMP architectures,
OpenMP can be used for multiple processors on a node while
MPI can be used between nodes.

\section{Obtaining the POP code}\label{sec:obtain}

If you have not done so already, you must first actually
download a version of the POP code, maybe even the latest
version.  The preferred place to get the POP code is our
main web site at Los Alamos:

\noindent
\htmladdnormallink{http://climate.acl.lanl.gov/models/pop}
                  {http://climate.acl.lanl.gov/models/pop}

\noindent
from which you probably downloaded or are viewing
this manual.

If you are running POP as part of the NCAR Climate
System Model (CCSM), you should download POP as
part of the CCSM distribution.

\section{Creating a run directory}\label{sec:create-directory}

The POP distribution you obtained is probably in the
form of a compressed tar file.  You must first uncompress
the tar file using

{\em uncompress popXXXX.tar}

\noindent
where XXXX refers to the version number which appears
in the file name.  Then the tar archive must be unpacked
using

{\em tar -xvf popXXXX.tar}.

\noindent
This process will result in a directory named pop with
several subdirectories containing source code, templates
for various input files, utilities and test codes.

In order to build a version of POP to run, a directory
with the appropriate makefiles and input files must be
created.  In the main POP directory, a script has been
provided to create all the necessary structure. Typing

{\em setup\_run\_dir dirname [model]}

\noindent
will create a directory called {\em dirname} with all
the necessary makefiles and input files.  A sub-directory
called {\em compile} will also be created to provide a
work area for the compilation process.  The optional
argument {\em model} will copy files that are specific
to a standard resolution or model setup.  One example
of such a setup is the
\hyperref{{\em test} case}
         {{\em test} case (see Sec. }{ }
         {sec:testPOP}
which should be used to test the code for the first
time.

\section{Building POP}\label{sec:build}

\subsection{Make Procedure}\label{sec:make}

The POP {\em make} procedure consists of several steps that are
governed by options in an architecture-specific {\em archdir.gnu}
file. The correct {\em .gnu} file is chosen based on an environment
variable called ARCHDIR, which is
a combination of vendor name and communication paradigm. For example,
if you are compiling for an SGI Origin you will have the choice
of {\em sgi\_serial}, {\em sgi\_mpi}, {\em sgi\_shmem}, {\em sgi\_omp},
or {\em sgi\_mpiomp} depending on whether you want to run serially,
in parallel using message-passing (MPI or SHMEM), in parallel using
thread-based parallelism (OpenMP) or in parallel using a hybrid of
threads and message-passing.  The {\em archdir.gnu} file contains
the proper paths and compiler options for the particular architecture
chosen. A variety of these {\em .gnu} files have been provided
with the standard distribution in the {\em input\_templates}
directory.  If a {\em archdir.gnu} file does not
exist for your choice, a file {\em generic.gnu} exists with
comments on how to configure it for your particular machine
and environment.  Also, if
your site has an unusual setup for locations of compilers and various
other tools like netCDF, you may have to edit the {\em archdir.gnu}
file to reflect the different setup.

POP version 2.0 or later requires version 3.5 or later of the 
netCDF library.  If not already installed on your system, the 
netCDF package can be obtained free of charge from

\noindent
\htmladdnormallink{
 http://my.unidata.ucar.edu/content/software/netcdf/index.html}
{http://my.unidata.ucar.edu/content/software/netcdf/index.html}
\noindent

Once the ARCHDIR environment variable has been set, typing {\em gmake}
should start the {\em make} process. The {\em make} process includes a
step which runs scripts to generate dependencies for the {\em makefile}.
After the dependencies are generated, the source code passes through a
preprocessor and the resulting routines are finally compiled into an
executable named {\bf pop}.

If you wish to compile a version of POP suitable for
debugging or wish to turn all optimizations off, typing

{\em gmake OPTIMIZE=no}

\noindent
will create an exectutable named {\bf pop\_db} for this
purpose.

\section{Domain decomposition}\label{sec:domain-decomp}

In order to understand some aspects of compiling and running
POP, a few words must be said here about how POP breaks up
a problem to run on different threads and processors.  Note
that even the serial versions decompose the domain in order
to achieve better performance on cache-based microprocessors.

In POP, the full horizontal domain size 
{\em (nx\_global,ny\_global)} is broken up into domains 
or blocks. The size of these blocks can be chosen to 
achieve better performance as described below.  Any block 
size can be chosen, but to avoid padding the domain with 
extra points, the block size in each direction should be 
chosen such that it divides the global domain size in that 
direction evenly.

Once the domain has been decomposed into blocks, the
blocks are distributed among the processors or nodes, ignoring
blocks that only contain land points.  The distribution
of blocks across processors or nodes can be performed using 
either a load-balanced distribution to try to give all processors
an equal amount of work or a Cartesian distribution
which ensures that the block's north, south, east and
west neighbors remain nearest neighbors.  A
load-balanced distribution is generally better for
the baroclinic section of the code; a Cartesian
distribution is better for the barotropic solver.
Different distributions can be specified for the baroclinic
and barotropic parts of the code.

Such a domain decomposition allows some flexibility
in tuning the model for the best performance.  Generally,
a smaller block size will improve processor performance
on cache-based microprocessors and a smaller block
size should ensure a better load balance and better
land point elimination.  However,
smaller block sizes add complexity to the communication
routines (boundary updates, global reductions) and
will result in a performance penalty for the barotropic
solver.  The user will need to experiment
with a few combinations to find the best configuration
for the simulation being run.

\section{Compile-time options}\label{sec:compile-opts}

There are a few options for POP that must be determined at
compile time. Some of these options are set by editing
modules; one option requires a preprocessor
directive which is handled by a
C-language preprocessor ({\em cpp}) or equivalent (if the
Fortran compiler understands such directives).  The
options below are the {\em only} options that need to
be decided at compile time; all other options are set
at run time through a namelist input file.

\subsection{Domain}\label{sec:compile-domain}

The full model size must be set in the domain\_size module
(in the file {\em domain\_size.F90} located in the run directory).
The horizontal extents of the grid {\em nx\_global} and 
{\em ny\_global} are defined here as well as the number of 
vertical levels {\em km}.  The number of tracers
{\em nt} also must be defined here and must be at least two
to handle potential temperature and salinity which are always
assumed to be tracer number one and two, respectively.

\subsection{Blocks}\label{sec:compile-blocks}

The size of the blocks for which the domain is
\hyperref{decomposed}
         {decomposed (see Sec. }{)}
         {sec:domain-decomp}
is set by two parameters in the domain\_size module
(contained in the file {\em domain\_size.F90} in the run
directory).  The parameter
{\em block\_size\_x} determines the number of physical
grid points in the first horizontal dimension;  the
parameter {\em block\_size\_y} determines the number
of physical grid points in the (wait for it...) second
horizontal dimension.  The two parameters 
{\em max\_blocks\_clinic} and
{\em max\_blocks\_tropic} determine the maximum number
of blocks that can be distributed onto a processor or node.
An initial guess for these two parameters can be made
by dividing the total number of blocks by the number
of processors you plan to run on
\begin{equation}
{{(nx\_global/block\_size\_x)
  (ny\_global/block\_size\_y)}\over{nprocs}}.
\end{equation}
Only an initial guess is required; when running the code,
the actual numbers required will be output so that the
user can set these parameters correctly.  Also, note that
these parameters can be set higher than required with
no penalty other than memory use.  For example, the
parameters can be set for the lowest processor count
you plan to use and then the same executable can be used
to run at higher processor counts with no change other
than the namelist inputs for number of processors.

As mentioned in previous sections, finding the optimal
block size and distribution can require experimentation.
However, a starting point for users who are familiar
with previous versions of POP is to set the block size such
that only one block is distributed on each processor.
In this case, {\em block\_size\_x = nx\_global/NPROCX}
where {\em NPROCX} is the value found in the old POP
{\em .gnu} files; values for the y direction are computed
analogously.  The {\em max\_blocks} parameters
are then set to one.  After using this configuration,
users can experiment with reducing the block size to
improve performance. 

\subsection{Coupled model runs}\label{sec:compile-coupled}

If POP is run in coupled mode, the default interface
communicates with the NCAR CCSM Flux Coupler.  This requires
message-passing calls and linking with additional libraries
(including MPI whether MPI is used internally or not).
To enable this capability, the coupled option must be
requested on the makefile command line:

{\em gmake COUPLED=yes}.

\noindent
This option turns on the COUPLED ifdef flag for the 
preprocessor so that the code necessary for model
coupling is included during the preprocessing phase.

\subsection{Debugging}\label{sec:compile-debug}

Similar to the coupling option, if you wish to create
a non-optimized pop executable for use with a debugging
tool, you must specify this on the make command line:

{\em gmake OPTIMIZE=no}.

\noindent
By default, this will create an executable named
{\bf pop\_db} rather than the usual {\bf pop}.

\section{Testing POP}\label{sec:testPOP}

The POP distribution includes a simple test case that can be used
for a variety of purposes, including validation, performance 
tuning and benchmarking.  The key point is that there are no input 
fields: the model grid, topography, initial state, equation of 
state coefficients and wind stress (there is no other forcing 
enabled) are all generated internally.  The only file that is 
read is {\em pop\_in}.

To run the test problem, type 

{\em ./setup\_run\_dir test test} 

\noindent 
and a
directory called {\em test} will be created that contains all of the
appropriate files.  The default model size is 192x128x20 grid points,
though this can be changed arbitrarily by the user.  The grid
generated internally is an equally-spaced latitude-longitude global
grid with idealized landmasses. Make the executable as described
\hyperref{earlier.}
         {in Sec. }{.}
         {sec:build}
The {\em pop\_in} file defaults to running 20 steps with
full diagnostics output every step.  Note that for performance
benchmarks, the diagnostic frequency should be set to `never' 
as the global diagnostics are expensive and typically 
requested relatively infrequently (e.g. every 10 days) in a 
production simulation.

To use this test case for validation, the user can compare their ouput
with the file {\em pop/input\_templates/pop\_sgi.log.test} which
contains the output of a 20 step calculation run on 4 processors of
an SGI Origin3000.  The results from other computer platforms should
agree reasonably well -- within roundoff for the first step for
most of the larger fields (the nearly-zero fields are very sensitive
and my not agree to that level of precision).

Once the answers have been validated using the 192x128x20 grid,
performance and scaling can be investigated by varying the grid
size (in {\em domain\_size.F90}) and the number of processors (in the
pop\_in file).
