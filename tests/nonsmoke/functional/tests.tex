\chapter{ROSE Tests}

\label{tests:introduction}

\section{How We Test}

ROSE includes a number of test codes. These test codes test:
\begin{enumerate}
   \item Robustness of translators built using ROSE. \\
         A test translator ({\tt testTranslator}) is built and it is used to process a number of
         test codes (both the compilation of the test code and the compilation of
         the generated source code it tested to make sure that they both compile
         properly). No execution of the generated code is attempted after compilation.
         These tests are used to verify the proper operation of ROSE as part of
         the standard SVN check-in process for all developers.

   \item Execution of the code generated by the translator built using ROSE. \\
         Here tests are done to verify that the translator generated correct code
         that resulted in the same result as the original code.  

   \item Robustness of the internal mechanisms within ROSE. \\
         Here tests are done on separately developed features within the ROSE
         infrastructure (e.g. the AST Rewrite Mechanism, Loop Optimizations, etc.).
\end{enumerate}

Specific directories of tests include:
\begin{itemize}

   \item CompileTests \\
   This directory contains code fragments that test the internal compiler mechanisms.
Many code fragments or whole codes are present either have previously or continue to present
problems in the compilation (demonstrate bugs).  The CompileTests directory consists of several directories.
The {\tt README} file in the CompileTests directory gives more specific information.

The test codes developed here are intended to be a small test of ROSE (a much larger
regression test suit will be available separately; and is used separately).
These tests are divided into caegories:
\begin{enumerate}
   \item C\_tests \\
      These are tests of the differences between the C subset of C++ and C. Specifically
      these are typically C codes that will not compile with a C++ compiler, even under the 
      subset of C language rules used to invoke the subset of C (-rose:C or -rose:C\_only)
      UNLESS the source files have the ".c" suffix, as opposed to any other suffix
      (e.g. ".C"). These are all specific to the C89 standard, which is what is typically 
      assumed when refering to the C language (C99 is covered separately).
   \item C99\_tests \\
      These are tests specific to C99 (new features not in C).
   \item UPC\_tests \\
      These are tests that are specific to UPC modifiers, recognized by EDG and handled
      in the Sage III AST.  This support for UPC does not constitute a UPC compiler, a 
      UPC specific runtime system would be required for that.
   \item Cxx\_tests \\
      These are the C++ test files (there are more tests here than elsewhere).
   \item C\_subset\_of\_Cxx\_tests \\
      This is the subset of C++ represented by the C language rules (it is not all of C).
      There are test codes here which contain {\tt \#if \_\_cplusplus} to represent some
      differences between the syntax of C and C++ (typically {\tt enum} and {\tt struct}
      specifiers are requird for C where they are not required for C++.
   \item RoseExample\_tests \\
      These are examples of ROSE project source code, and testing using ROSE to compile
    examples of ROSE source code.
   \item PythonExample\_tests \\
      These tests use the {\tt Python.h} header file and are part of tests of code
    generated by SWIG.
   \item ExpressionTemplateExample\_tests \\
      These are a number of tests demonstrating the use of expression templates.
      They are separated out because they take a long time to compile using ROSE.
      This is part of work to understand why expression templates take so long 
      to compile generally.
\end{enumerate}

\commentout{
   \item RunTests \\
   This directory contains subdirectories representing code that uses libraries that
ROSE can optimize.  Its purpose is to test the optimization mechanisms within ROSE.
% A testing harness (used in A++/P++ and developed by Brian Miller) will be use to report
% the performance of the optimizations at a later stage.
}

   \item roseTests \\
   This directory tests the internal ROSE infrastructure. It contains separate 
subdirectories for individual parts of ROSE. See ROSE/tests/roseTests/README 
for details.

\end{itemize}

\fixme{Complete the list of subdirectories that hold tests (in the ROSE/tests directory).}

