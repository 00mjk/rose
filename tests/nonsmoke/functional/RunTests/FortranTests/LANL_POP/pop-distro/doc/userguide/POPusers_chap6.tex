\chapter{Trouble-shooting}\label{ch:trouble}

If you encounter problems getting�POP to run, a list of
known bugs is maintained at the POP website and the user
should check there before contacting the developers.
In addition, there are other common sources of error
listed below with suggestions for avoiding such problems.

\subsection{Timestep stability criteria}
\label{sec:trouble-timestep}

As part of the POP diagnostics, the Courant number for
various processes can be output.  The Courant number
should always be less than one and often should be
kept under 0.5 for stability.  If such a CFL condition
is exceeded, reducing the timestep may be a solution.
However, it should be noted that often other
underlying problems may be causing these limits to
be exceeded (e.g. unrealistically high velocities)
and reducing the timestep will only prolong the agony.
In such cases, the particular CFL condition that is
violated can help pinpoint the source of the problem.

\subsection{Checking the global energy and work balances}
\label{sec:trouble-balance}

\subsection{Getting the right combination of mixing options}
\label{sec:trouble-mixing}

\subsection{Getting the right combination of forcing options}
\label{sec:trouble-forcing}

\subsection{Properly normalizing input salinity values}
\label{sec:trouble-salinity}

The units for salinity in the model are g/g.� Care must be
taken to renormalize salinity data, which is often in g/kg.

\section{Running POP on PCs}
\label{sec:PCPOP}

POP has been successfully run on a single-processor Intel PC
with Windows NT. POP was built with Digital Visual
Fortran 5.0 as a Win32 console application (meaning you run
it with a command line in a DOS window).  The make system
used for other platforms was not used on the PC. Instead, a
standard Visual Studio project was created.

Source files added to the project were
\begin{itemize}
\item All .F90 and .C files in source subdirectory
\item All .F90 files in serial subdirectory
\end{itemize}

In addition to the defaults, these Fortran preprocessor
options `/fpp' was needed.
To enable POP to run the test problem without stack overflow,
reserve stack memory was set to 64 MB. A lower
value may be possible, but 48 MB was not enough.
The debug version required the link option: /nodefaultlib:libc.
The release version (optimized) ran the test problem in about
four minutes (262 seconds) on a 450 MHz, 256 MB
Pentium II PC.
