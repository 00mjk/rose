\chapter{POP tools}\label{ch:POPtools}

This chapter describes various tools available for
creating POP input or analyzing POP output.

\section{Visualizing output}\label{sec:viz}

There is currently no standard tool for visualizing POP
output.  There are two freely-available software packages
that can be used.

Ferret is a visualization tool developed by Steve Hankin at
NOAA�s Pacific Marine Environmental Laboratory (PMEL) in Seattle.
It is designed specifically for visualizing ocean model output
and data. Ferret is constantly being improved and extended, and
there is a very active email-based user group.  For more
information about Ferret, check the Ferret website

\htmladdnormallink{http://ferret.wrc.noaa.gov/Ferret/}
                  {http://ferret.wrc.noaa.gov/Ferret/}.

Other free visualization and analysis packages are available
from the DOE-sponsored Program for Climate Model Diagnosis 
and Intercomparison (PCMDI) at Lawrence Livermore National 
Laboratory (LLNL).  Although originally designed for 
visualizing atmospheric model output from the Atmospheric 
Model Intercomparison Project (AMIP), many are applicable 
to ocean model output and future ocean analysis tools are 
being added.�  For more information, check the website

\htmladdnormallink{http://www-pcmdi.llnl.gov/software/}
                  {http://www-pcmdi.llnl.gov/software/}.

\section{Transformations from general grids}\label{sec:transgrid}

All POP output is on the computational grid.  For
general grids that are not based on latitude and longitude
(e.g. the displaced-pole or tripole grids), analyzing or
visualizing data leads to a distorted view of the world
and colleagues may begin to question your geography.
Transformation to latitude-longitude grids can be performed
using the Spherical Remapping and Interpolation Package (SCRIP),
available from

\htmladdnormallink{http://www.acl.lanl.gov/lanlclimate/SCRIP/}
                  {http://www.acl.lanl.gov/lanlclimate/SCRIP/}.

\section{File format conversion}\label{sec:formatconvert}

POP now supports direct netCDF output and such output is
recommended unless it creates a performance bottleneck
for large grids.  In cases where binary output is required,
conversion to netCDF can be achieved off-line through a
utility called bin2nc.  Unfortunately, this utility is
being re-worked to handle the new binary output format
so is unavailable at this time.  It would be relatively
easy for a user to create such a code however.

\section{Generating grid and bottom topography files}
\label{sec:gridgen}

\subsection{Horizontal grid and topography}\label{sec:gridgen-horiz}

A graphical tool for generating horizontal grids and creating 
bottom topography has been developed.  A working version is 
currently in the process of being released and will support 
almost-global Mercator grids, global displaced-pole grids,
global tripole grids and regional grids.  Some standard grids 
and topography used in current production runs are available 
from the POP website.

\subsection{Vertical grid}\label{sec:gridgen-vert}

There is a vertical grid generating routine in the tools/grids
subdirectory.  This code is essentially the same code used to
generate vertical grids internally in POP, but can be used to
generate a vertical grid file off-line which can be edited to
suit your simulation.  Note that when changing the vertical
grid, you will need to re-generate the bottom topography.

