\chapter{UPC Support}
\label{chap::upcsupport}

%-------------------------------------------------------------
%-------------------------------------------------------------
\section{Introduction}
ROSE supports Unified Parallel C (UPC) programs.
UPC~\cite{UPC:Web} is a famous extension of the C99 programming language to support high performance computing using a partitioned global address space (PGAS) memory model.
ROSE leverages the EDG frontend to parse input UPC programs and generate EDG IR with UPC extensions.
It then converts the EDG IR into ROSE's internal AST and provides unparsing support for the AST.
An example UPC-to-C translator is also provided to demonstrate how one can use ROSE to 
translate UPC programs into C programs with runtime calls to the Berkeley UPC (BUPC) runtime system V. 2.6.0~\cite{BUPC:Web}.
More information on UPC is availalbe from the language specification at: 
\htmladdnormallink{UPC Language Specifications (pdf)}{http://www.gwu.edu/~upc/publications/LBNL-59208.pdf}.

%-------------------------------------------------------------
%-------------------------------------------------------------
\section{Supported UPC Constructs}
% ROSE currently supports all UPC constructs as defined in UPC 1.1.1.
ROSE currently supports all UPC constructs as defined in UPC 1.2, using
all the standard header files: {\em upc.h}, {\em upc\_relaxed.h}, {\em upc\_strict.h}, 
{\em upc\_collective.h}, {\em upc\_io.h}. Additional Non-standard Berkeley 
UPC extensions are supported using the UCB {\em bupc\_extensions.h} header file.

\footnote{
The supported version is limited by the EDG
frontend, which only supports UPC 1.1.1 (\lstinline{__UPC_VERSION__} string
is defined as 200310L).
ROSE uses EDG 3.3 currently and it originally only supported UPC 1.0.
We merged the UPC 1.1.1 support from EDG 3.10 into our EDG 3.3 frontend. 
%It seems like that the latest EDG 4.0 still only supports UPC 1.1.1 but we
%can easily extend it to support UPC 1.2 given the minor language changes from 1.1.1 to 1.2.
We have also added the required wotk to support UPC 1.2. Please let us know if anything is
left out and we will fix it.}.
A list of those UPC constructs and their corresponding ROSE AST representations are given below:
\begin{verbatim}
MYTHREAD                SgUpcMythread
THREADS                 SgUpcThreads
upc_barrier             SgUpcBarrierStatement
upc_blocksizeof         SgUpcBlocksizeofExpression
upc_elemsizeof          SgUpcElemsizeofExpression
upc_fence               SgUpcFenceStatement
upc_forall              SgUpcForAllStatement
upc_localsizeof         SgUpcLocalsizeofExpression
upc_notify              SgUpcNotifyStatement
upc_wait                SgUpcWaitStatement
strict/relaxed/shared   SgUPC_AccessModifier
UPC_MAX_BLOCKSIZE       1073741823 (~1GB)
\end{verbatim}
As we can see, most UPC constructs are represented by their corresponding dedicated ROSE AST nodes. 
A few others, such as \lstinline{strict}, \lstinline{relaxed} and
\lstinline{shared}, are represented as instances of \lstinline{SgUPC_AccessModifier}.
\lstinline{UPC_MAX_BLOCKSIZE} is treated as a macro and is expanded to a predefined integer constant value.

Much of the UPC language is supported via UPC specific language runtime
libraries.  These calls are simply normal functions and not represented 
as special UPC specific IR nodes.  
Such runtime library function calls are, in general, 
not a property of the UPC grammar and thus not represented in the AST 
as specialized IR nodes.
The function names are those from the 
UPC Langauge Specifications and so must be detected based on the function
name.  No specialized support in currently present in ROSE to classify 
these function or identify them as UPC secific, but such functionality
could be easily written. Such support could be added to the SageInteface.
If users write such specific support they are welcme to contribute it 
back to ROSE.

Alternatively ROSE provides function identification based on header files
where the declarations occured and this would be used to automate the recognition of UPC 
specific runtime support functions.  See the use of the {\em classifyFileName()}
function in {\em tests/roseTests/fileLocation\_tests}
directory for examples of how file names can be classified.  Alternatively,
just the infomation in the source position reports the file name of the
original declaration and this can be used to identify specific functions as 
defined in there associated header files, e.g. {\em upc\_collective.h}, 
{\em upc\_io.h}, or {\em bupc\_extensions.h}.

%-------------------------------------------------------------
%-------------------------------------------------------------
\section{Command Line Options}
ROSE can automatically recognize a source file with a suffix of \textit{.upc} as a UPC input and turn on its UPC support.
For other UPC files without the \textit{.upc} extension, a command line
option (\textit{-rose:UPC\_only or -rose:UPC}) is available to turn on the
UPC support explicitly.
In addition, \textit{-rose:upc\_threads n} can be used to enable ROSE's
support for UPC static threads compilation with n threads.

%-------------------------------------------------------------
%-------------------------------------------------------------
\section{Example UPC Code Acceptable for ROSE}
We give some output after ROSE's source-to-source translation of some example UPC input. 
These UPC input are actually some of ROSE's daily regression test input available from \textit{ROSE/tests/CompileTests/UPC\_tests}.

Figure~\ref{Manual:UPC:hello} shows the output of the ROSE
identityTranslator handling a hello program in UPC.
\begin{figure}[!h]
{\indent
  {\mySmallFontSize
    \begin{latexonly}
    \lstinputlisting{\TopBuildDirectory/tests/CompileTests/UPC_tests/rose_hello.upc}
    \end{latexonly}
    \begin{htmlonly}
    \verbatiminput{\TopBuildDirectory/tests/tests/CompileTests/UPC_tests/rose_hello.upc}
    \end{htmlonly}
  }
}
\caption{Output of an UPC hello program}
\label{Manual:UPC:hello}
\end{figure}

Figure~\ref{Manual:UPC:strict} shows the handling of UPC language
constructs related to memory consistency. 
\begin{figure}[!h]
{\indent
  {\mySmallFontSize
    \begin{latexonly}
    \lstinputlisting{\TopBuildDirectory/tests/CompileTests/UPC_tests/rose_consistency.upc}
    \end{latexonly}
    \begin{htmlonly}
    \verbatiminput{\TopBuildDirectory/tests/tests/CompileTests/UPC_tests/rose_consistency.upc}
    \end{htmlonly}
  }
}
\caption{Output for UPC strict}
\label{Manual:UPC:strict}
\end{figure}


Figure~\ref{Manual:UPC:forall1} shows the use of \lstinline{upc_forall}
with \lstinline{continue} and Figure~\ref{Manual:UPC:forall2} shows the use
of \lstinline{upc_forall} with \lstinline{affinity}.

\begin{figure}[!h]
{\indent
  {\mySmallFontSize
    \begin{latexonly}
    \lstinputlisting{\TopBuildDirectory/tests/CompileTests/UPC_tests/rose_forall_continue.upc}
    \end{latexonly}
    \begin{htmlonly}
    \verbatiminput{\TopBuildDirectory/tests/tests/CompileTests/UPC_tests/rose_forall_continue.upc}
    \end{htmlonly}
  }
}
\caption{Output for upc\_forall with continue}
\label{Manual:UPC:forall1}
\end{figure}

\begin{figure}[!h]
{\indent
  {\mySmallFontSize
    \begin{latexonly}
    \lstinputlisting{\TopBuildDirectory/tests/CompileTests/UPC_tests/rose_forall_affinity.upc}
    \end{latexonly}
    \begin{htmlonly}
    \verbatiminput{\TopBuildDirectory/tests/tests/CompileTests/UPC_tests/rose_forall_affinity.upc}
    \end{htmlonly}
  }
}
\caption{Output for upc\_forall with affinity }
\label{Manual:UPC:forall2}
\end{figure}

\clearpage
ROSE's support for various uses of \lstinline{shared} and unshared UPC
variables is given in
Figure~\ref{Manual:UPC:sharedaa} and Figure~\ref{Manual:UPC:sharedab}. 
All kinds of shared, shared to shared, shared to private, and private to
shared variables can be correctly parsed and unparsed.
\begin{figure}[!h]
{\indent
  {\mySmallFontSize
    \begin{latexonly}
    \lstinputlisting{\TopBuildDirectory/tests/CompileTests/UPC_tests/rose_shared.upc.aa}
    \end{latexonly}
    \begin{htmlonly}
    \verbatiminput{\TopBuildDirectory/tests/tests/CompileTests/UPC_tests/rose_shared.upc.aa}
    \end{htmlonly}
  }
}
\caption{Output for UPC shared: part A}
\label{Manual:UPC:sharedaa}
\end{figure}

\begin{figure}[!h]
{\indent
  {\mySmallFontSize
    \begin{latexonly}
    \lstinputlisting{\TopBuildDirectory/tests/CompileTests/UPC_tests/rose_shared.upc.ab}
    \end{latexonly}
    \begin{htmlonly}
    \verbatiminput{\TopBuildDirectory/tests/tests/CompileTests/UPC_tests/rose_shared.upc.ab}
    \end{htmlonly}
  }
}
\caption{Output for UPC shared: part B}
\label{Manual:UPC:sharedab}
\end{figure}

\clearpage
Support for UPC \lstinline{lock}s is demonstrated in
Figure~\ref{Manual:UPC:lock}. 
\begin{figure}[!h]
{\indent
  {\mySmallFontSize
    \begin{latexonly}
    \lstinputlisting{\TopBuildDirectory/tests/CompileTests/UPC_tests/rose_lock.upc}
    \end{latexonly}
    \begin{htmlonly}
    \verbatiminput{\TopBuildDirectory/tests/tests/CompileTests/UPC_tests/rose_lock.upc}
    \end{htmlonly}
  }
}
\caption{Output for UPC Locks}
\label{Manual:UPC:lock}
\end{figure}

%-------------------------------------------------------------
%-------------------------------------------------------------
\clearpage
\section{An Example UPC-to-C Translator Using ROSE}
An example UPC-to-C translator, namely roseupcc, is provided to 
demonstrate how one can use ROSE to build a translator translating UPC
programs into C programs with runtime calls to the Berkeley UPC (BUPC)
runtime system V. 2.6.0. 
The source files of roseupcc are located in
\textit{ROSE/projects/UpcTranslation}.
%roseupcc follows the BUPC runtime interface specification
%3.10~\cite{BUPC3.10}.
%we don't reiterate details about why the translation is done in a certain way here. 
Please be advised that only a subset of UPC (including variable handling) is
translated currently since the translator is meant to be a starting example.
Also variable handling is arguably the most difficult part of a UPC
implementation.

Translation result for the UPC hello program (shown in
Figure~\ref{Manual:UPC:hello}) is given in Figure~\ref{Manual:UPC:hello-trans}. 
Mostly, high level \lstinline{SageInterface} functions are used to
easily translate the ROSE AST. 
BUPC-specified preprocessing directives, such as \lstinline{#include "upcr.h"} 
and \lstinline{#define UPCR_WANT_MAJOR 3}, are inserted.
The original user main function is rewritten to \lstinline{user_main} with
runtime initialization (\lstinline{UPCR_BEGIN_FUNCTION}) and termination 
(\lstinline{UPCR_EXIT_FUNCTION}) functions. 
\lstinline{upc_barrier} is simply replaced with a call to its corresponding
runtime function \lstinline{upcr_barrier()}.
\lstinline{UPCRI_ALLOC_filename_xxx()} handles per file UPCRI allocation of
possible shared variables (none here).  
\lstinline{UPCRI_INIT_filename_xxx()} function is used for per-file initialization of
shared and unshared (thread local) data.

\begin{figure}[!h]
{\indent
  {\mySmallFontSize
    \begin{latexonly}
    \lstinputlisting{\TopBuildDirectory/projects/UpcTranslation/tests/rose_hello.c}
    \end{latexonly}
    \begin{htmlonly}
    \verbatiminput{\TopBuildDirectory/projects/UpcTranslation/tests/rose_hello.c}
    \end{htmlonly}
  }
}
\caption{Translation of upc hello}
\label{Manual:UPC:hello-trans}
\end{figure}


%\clearpage
Implementing various UPC variable accesses is complex due to the two level
memory scheme used by the partitioned global memory address space.
The translation has to handle variable declaration, memory allocation,
value initialization, variable access, and so on. 

UPC variables can be roughly categorized as shared and unshared (or thread
local) variables. 
For shared variables, they can be divided into two categories:
statically allocated shared variables (including global shared variables
and local static shared variables) and dynamically allocated shared
variables. 
%Runtime library routines are sufficient to support dynamically allocated
%shared variables. 
We focus on statically allocated shared variables here as
an example.
%But extra compiler translation is needed for
%implementing statically allocated shared variables. 
Translation of statically allocated \lstinline{shared} variables is demonstrated in
Figure~\ref{Manual:UPC:shared1-transaa} and
Figure~\ref{Manual:UPC:shared1-transab} for an input code shown in
Figure~\ref{Manual:UPC:shared1}.
\begin{figure}[!h]
{\indent
  {\mySmallFontSize
    \begin{latexonly}
    \lstinputlisting{\TopSourceDirectory/projects/UpcTranslation/tests/shared_1.upc}
    \end{latexonly}
    \begin{htmlonly}
    \verbatiminput{\TopSourceDirectory/projects/UpcTranslation/tests/shared_1.upc}
    \end{htmlonly}
  }
}
\caption{Example input for shared variables}
\label{Manual:UPC:shared1}
\end{figure}

\begin{figure}[!h]
{\indent
  {\mySmallFontSize
    \begin{latexonly}
    \lstinputlisting{\TopBuildDirectory/projects/UpcTranslation/tests/rose_shared_1.c.aaf}
    \end{latexonly}
    \begin{htmlonly}
    \verbatiminput{\TopBuildDirectory/projects/UpcTranslation/tests/rose_shared_1.c.aaf}
    \end{htmlonly}
  }
}
\caption{Translation of UPC shared variables, part A}
\label{Manual:UPC:shared1-transaa}
\end{figure}

The BUPC runtime distinguishes between phaseless (blocksize ==1 or 0 or
unspecified) and phased (all other cases) shared
pointers for better performance. 
So two types of global scoped proxy pointers (\lstinline{upcr_pshared_ptr_t}
and \lstinline{upcr_shared_ptr_t}) are used to represent static shared variables. 
Global static shared variables directly use their names as the proxy
pointer names (such as \lstinline{global_counter}).
Local static variables use their mangled name (e.g. \lstinline{_lsscounter3769188422_}) for those pointers to
avoid name collision.
Accesses to shared variables are implemented by a set of runtime library
calls, such as \lstinline{UPCR_GET_SHARED()} and
\lstinline{UPCR_PUT_SHARED_VAL}.
Again, as shown in Figure~\ref{Manual:UPC:shared1-transab}, \lstinline{UPCRI_ALLOC_filename_xxx()} handles per file UPCRI allocation of shared variables and 
\lstinline{UPCRI_INIT_filename_xxx()} function is used for per-file
initialization of those data.

ROSE provides a set of AST interface functions to help developers handle
UPC-specific types and facilitate translation. Those functions include 
\lstinline{bool isUpcSharedType()}, 
\lstinline{bool isUpcSharedArrayType()}, 
\lstinline{bool isUpcPhaseLessSharedType()}, 
\lstinline{bool isUpcPrivateToSharedType()}, etc. 
A type mangling function (\lstinline{mangleType()}) is also provided
to implement the Itanium C++ ABI specification~\cite{ItaniumCppABI}. 

Unshared variables in UPC (also called Thread local data, or TLD) consists
of local auto variables and global ( and static local) variables.
Local auto variables do not need special translation. 
But global and static local variables do. 
Implementation details for them vary depending on their scope, internal or
external linkage, if they are scalar or array types, if they point to
shared data, if they have initializers, and so on.
But the basic scheme for variable declaration, allocation/initialization, and
accesses is similar to the handling of shared UPC variables.
Please refer to the BUPC runtime interface specification
3.10~\cite{BUPC3.10} for details.
We only provide a translation example in
Figure~\ref{Manual:UPC:unshared1-transaa} and
Figure~\ref{Manual:UPC:unshared1-transab} for an input code shown in
Figure~\ref{Manual:UPC:unshared1}.

%No translations are needed for extern global unshared variables unless they
%are pointers to shared data. 
%Local static unshared variables are promoted to non-static type at global
%scope using lstinline{UPCR_TLD_DEFINE or UPCR_TLD_DEFINE_TENTATIVE})
%Unshared pointer to shared data 


\begin{figure}[!h]
{\indent
  {\mySmallFontSize
    \begin{latexonly}
    \lstinputlisting{\TopBuildDirectory/projects/UpcTranslation/tests/rose_shared_1.c.abf}
    \end{latexonly}
    \begin{htmlonly}
    \verbatiminput{\TopBuildDirectory/projects/UpcTranslation/tests/rose_shared_1.c.abf}
    \end{htmlonly}
  }
}
\caption{Translation of UPC shared variables, part B}
\label{Manual:UPC:shared1-transab}
\end{figure}

\begin{figure}[!h]
{\indent
  {\mySmallFontSize
    \begin{latexonly}
    \lstinputlisting{\TopSourceDirectory/projects/UpcTranslation/tests/unshared_1.upc}
    \end{latexonly}
    \begin{htmlonly}
    \verbatiminput{\TopSourceDirectory/projects/UpcTranslation/tests/unshared_1.upc}
    \end{htmlonly}
  }
}
\caption{Example input for non-shared variables}
\label{Manual:UPC:unshared1}
\end{figure}

\begin{figure}[!h]
{\indent
  {\mySmallFontSize
    \begin{latexonly}
    \lstinputlisting{\TopBuildDirectory/projects/UpcTranslation/tests/rose_unshared_1.c.aaf}
    \end{latexonly}
    \begin{htmlonly}
    \verbatiminput{\TopBuildDirectory/projects/UpcTranslation/tests/rose_unshared_1.c.aaf}
    \end{htmlonly}
  }
}
\caption{Translation of UPC unshared variables, part A}
\label{Manual:UPC:unshared1-transaa}
\end{figure}

\begin{figure}[!h]
{\indent
  {\mySmallFontSize
    \begin{latexonly}
    \lstinputlisting{\TopBuildDirectory/projects/UpcTranslation/tests/rose_unshared_1.c.abf}
    \end{latexonly}
    \begin{htmlonly}
    \verbatiminput{\TopBuildDirectory/projects/UpcTranslation/tests/rose_unshared_1.c.abf}
    \end{htmlonly}
  }
}
\caption{Translation of UPC unshared variables, part B}
\label{Manual:UPC:unshared1-transab}
\end{figure}


%-------------------------------------------------------------
%-------------------------------------------------------------
%\section{Summary}

