%%%%%%%%%%%%%%%%%%%%%%%%%%%%%%%%%%%%%%%%%%%%%%%%%%%%%%%%%%%%%%%%%%%%%%%%%%%%%%%
%
% Author: Gary Yuan
% Date: 8/7/06
% File: qm.tex
% Purpose: qm.sh documentation
% Updated 12/5/2009
%
%%%%%%%%%%%%%%%%%%%%%%%%%%%%%%%%%%%%%%%%%%%%%%%%%%%%%%%%%%%%%%%%%%%%%%%%%%%%%%%

\chapter{Testing Within ROSE}
\label{testing::overviewOfTesting}

\section{Introduction}
   Testing is a particularly important part of the ROSE project.
We credit our particularly high level of automated testing for
why we are able to maintain a compiler project such as ROSE
with so few people.  All testing is automated and the results
of the testing drives higher levesl of testing (from
developer testing, to automated internal daily testing, to
external nightly testing).  Different levels of testing
with different types of feedback provide a basis for the health
of the project to be visible to either the individual developer,
the rest of the team, or to an external audience.

   Out testing exists on three levels, and uses external
tools (Hudson: http://wiki.hudson-ci.org/display/HUDSON/Meet+Hudson).
% We have added a new level of testing using Hudson
% (http://wiki.hudson-ci.org/display/HUDSON/Meet+Hudson).
% It will make things more robust shortly, I think, but in the mean time
% we have managed to make the external SVN repository less
% robust in the process. The process now is that we have
\begin{enumerate}
   \item Hudson test each "git" branch before it gets checked into the internal
     SVN repository. For this we use an internal compile farm of
     about 7 machines and test different versions of configure options
     (including different versions of compilers).
   \item Test the internal SVN repository on different architectures, and if it passes
     push it out to the external repository.
   \item  Checkout the external SVN repository and have it be tested on the
     external compile farm of about 25 machines (with different OS's) at
     NMI (University of Wisc, Madison).
\end{enumerate}

   We test so that we can detect errors as quickly as possible and
provide as much transparency as possible to our external users.
Our goal is to communicate honestly so that we can avoid any 
misunderstandings, it is in our best interest to have this
policy.

   ROSE uses {\tt autoconf}, {\tt automake}, and {\tt libtool}.
As a result our generated {\em Makefiles} follow the GNU
standards for makefiles and contain the associated makefile rules:
\begin{itemize}
   \item {\tt make }: used to compile all of ROSE.
   \item {\tt make check}: used to run tests in each directory.
   \item {\tt make distcheck}: used to build a GNU standard distribution and run {\tt check} on
    on that distribution.
\end{itemize}
Additional stepts such a {\tt build} and {\tt configure} are covered in the
ROSE Installation guide.

This chapter is organized by laying out the different
ways in which ROSE is tests.  All testing scripts and the information
required for our approach to be used with other projects is made available
in the ROSE distribution. 
Section \ref{testing::twoRepositories} explains why we maintain two
different SVN repositories internally and externally.
Section \ref{testing::developerTests} explains what tests are done by developers, before checkin to SVN.
Section \ref{testing::dailyInternalTests} explains what daily internal tests are done.
Section \ref{testing::dailyExternalTests} explains what daily external tests are done.

\section{Tail of Two SVN Repositories}
\label{testing::twoRepositories}
   We maintain two SVN repositories, one internal and the other external.
The internal SVN repository is only available to the ROSE core
development team.  After passing specific tests, parts of the internal
SVN repository as submitted to the external SVN repository (hosted by SciDAC at 
Lawrence Berkley Laboratory).  The external SVN repository has public
visability.  It is always a snapshot of the internal SVN repository 
(for the revision that passed the required tests to be made externally available).
Importantly, the SVN repository is always trying to be a working copy of
ROSE. Non-compilable code should neve be checked in, all of ROSE as it
is represented in the SVN repository should pass {\tt make distcheck}.
Atuomated tests verify this or identify developers who are at fault.
Peer pressure is generally enough so that everyone fixes their code 
in the morning as a result of results of nightly tests.

   The external SVN repository also has a number of branches, these support
external collaborations and represent a community of people who contribute
work to the ROSE project (which is a community open source project).


\section{Developer Tests}
\label{testing::developerTests}
   Developer tests are required to run and pass {\tt make distcheck}
before checking in work to the internal SVN repository.  It is at
their discression if they want to only run {\tt make check}, which is
faster and less through. In general all developers know who is careful and
who is not and interally we can deal with people who break the ROSE 
builds too often.  If the developers test pass then the work
can be checked into the internal SVN repository.

\section{Daily Internal Tests}
\label{testing::dailyInternalTests}
   There are daily internal tests at LLNL (twice per day)
that use the internal SVN repository and checkout versions
of automated tests on three different platforms:
\begin{itemize}
   \item 32-bit Linux (Red Hat system 5)
   \item 64-bit Linux (Red Hat system 4)
   \item Max OSX version 10.5 (x86)
\end{itemize}
There three system are used to evaluate and report the status of ROSE.  Crontabs are used
to initiate the tests.  The crontabs call scripts in the ROSE/scripts
\fixme{We should provide an example of the crontabs used (output of 'crontab -l', for example)}.


\section{Daily External Tests}
\label{testing::dailyExternalTests}
   There are nightly external tests on the NSF Middleware Initiative compile farm
at the University of Wisconson. The results of these tests are made public
via link on the main ROSE web page.  This provides transparency to our external
collaborators and that is is public help keep us at our best.




{\scriptsize \newcommand{\hlstd}[1]{\textcolor[rgb]{0,0,0}{#1}}
\newcommand{\hlkey}[1]{\textcolor[rgb]{0.7,0.41,0.09}{#1}}
\newcommand{\hlnum}[1]{\textcolor[rgb]{1,0,0}{#1}}
\newcommand{\hltyp}[1]{\textcolor[rgb]{0,1,0}{#1}}
\newcommand{\hlesc}[1]{\textcolor[rgb]{1,0.13,1}{#1}}
\newcommand{\hlstr}[1]{\textcolor[rgb]{1,0,0}{#1}}
\newcommand{\hldstr}[1]{\textcolor[rgb]{1,0,0}{#1}}
\newcommand{\hlcom}[1]{\textcolor[rgb]{0,0,1}{#1}}
\newcommand{\hldir}[1]{\textcolor[rgb]{1,0.13,1}{#1}}
\newcommand{\hlsym}[1]{\textcolor[rgb]{0,0,0}{#1}}
\newcommand{\hlline}[1]{\textcolor[rgb]{0,0,1}{#1}}

\title{/home/yuan5/qmsh/qmalt.sh}
\oddsidemargin -3mm
\textwidth 165,2truemm
\topmargin 0truept
\headheight 0truept
\headsep 0truept
\textheight 230truemm

\definecolor{bgcolor}{rgb}{1,1,1}
\noindent
\ttfamily
\hlstd{}\hlline{\ \ \ \ 1\ }\hlstd{}\hlcom{\#\ Usage\\
}\hlline{\ \ \ \ 2\ }\hlcom{}\hlstd{\\
}\hlline{\ \ \ \ 3\ }\hlstd{}\hlkey{if\ }\hlstd{((\ }\hltyp{\$}\hlstd{}\hlcom{\#\ $<$\ 4\ ));\ then\\
}\hlline{\ \ \ \ 4\ }\hlcom{}\hlstd{\hlstd{\ \ }}\hltyp{echo\ }\hlstd{}\hlstr{"Usage:\ qm.sh\ $<$f\textbar o$>$\ $<$QMTest\ test\ class$>$\ $<$ROSE\ translator$>$\ $<$Backend\ Compiler\textbar NULL$>$\ \{compiler\ arguments...\}\ \{Test\ arguments\ (-testopt:$<$$>$)...\}"}\hlstd{\\
}\hlline{\ \ \ \ 5\ }\hlstd{\hlstd{\ \ }}\hltyp{exit\ }\hlstd{}\hlnum{1\\
}\hlline{\ \ \ \ 6\ }\hlnum{}\hlstd{}\hlkey{fi\\
}\hlline{\ \ \ \ 7\ }\hlkey{\\
}\hlline{\ \ \ \ 8\ }\hlkey{}\hlstd{}\hlcom{\#\ Functions\\
}\hlline{\ \ \ \ 9\ }\hlcom{}\hlstd{\\
}\hlline{\ \ \ 10\ }\hlstd{includeFullPath\ ()\ \{\\
}\hlline{\ \ \ 11\ }\hlstd{\hlstd{\ \ }}\hltyp{local\ }\hlstd{BACK=`}\hltyp{pwd}\hlstd{`\\
}\hlline{\ \ \ 12\ }\hlstd{\\
}\hlline{\ \ \ 13\ }\hlstd{\hlstd{\ \ }ARG=`}\hltyp{echo\ \$ARG\ }\hlstd{\textbar \ sed\ -e\ }\hlstr{"s/-I//g"}\hlstd{`\\
}\hlline{\ \ \ 14\ }\hlstd{\hlstd{\ \ }}\hltyp{cd\ \$ARG\\
}\hlline{\ \ \ 15\ }\hltyp{\hlstd{\ \ }}\hlstd{ARG=-I`}\hltyp{pwd}\hlstd{`\\
}\hlline{\ \ \ 16\ }\hlstd{\\
}\hlline{\ \ \ 17\ }\hlstd{\hlstd{\ \ }}\hltyp{cd\ \$BACK\\
}\hlline{\ \ \ 18\ }\hltyp{\hlstd{\ \ }return\ }\hlstd{}\hlnum{0\\
}\hlline{\ \ \ 19\ }\hlnum{}\hlstd{\}\ }\hlcom{\#\ get\ the\ absolute\ path\ of\ all\ include\ directories\\
}\hlline{\ \ \ 20\ }\hlcom{}\hlstd{\\
}\hlline{\ \ \ 21\ }\hlstd{}\hlcom{\#\#\#\#\#\#\#\#\#\#\#\#\#\#\#\#\#\#\#\#\#\#\#\#\#\#\#\#\#\#\#\#\#\#\#\#\#\#\#\#\#\#\#\#\#\#\#\#\#\#\#\#\#\#\#\#\#\#\#\#\#\#\#\#\#\#\#\#\#\#\#\#\#\#\#\#\#\#\#\#\\
}\hlline{\ \ \ 22\ }\hlcom{}\hlstd{\\
}\hlline{\ \ \ 23\ }\hlstd{}\hlcom{\#\ Globals\\
}\hlline{\ \ \ 24\ }\hlcom{}\hlstd{\\
}\hlline{\ \ \ 25\ }\hlstd{}\hltyp{declare\ }\hlstd{-i\ COUNT=}\hlnum{0\\
}\hlline{\ \ \ 26\ }\hlnum{}\hlstd{}\hltyp{declare\ }\hlstd{-i\ FLAG=}\hlnum{0\\
}\hlline{\ \ \ 27\ }\hlnum{\\
}\hlline{\ \ \ 28\ }\hlnum{}\hlstd{TEST=BADTEST.qmt}\hlcom{\#\ error\ in\ test\ creation\\
}\hlline{\ \ \ 29\ }\hlcom{}\hlstd{MODE=}\hltyp{\$1}\hlstd{}\hlcom{\#\ The\ naming\ mode\ of\ the\ script\\
}\hlline{\ \ \ 30\ }\hlcom{}\hlstd{TEST\textunderscore CLASS=}\hltyp{\$2}\hlstd{}\hlcom{\#\ QMTest\ class\\
}\hlline{\ \ \ 31\ }\hlcom{}\hlstd{PROGRAM=}\hltyp{\$3}\hlstd{}\hlcom{\#\ executable\ name\\
}\hlline{\ \ \ 32\ }\hlcom{}\hlstd{BACKEND=}\hltyp{\$4}\hlstd{}\hlcom{\#\ The\ execution\ string\ with\ backend\ compiler\\
}\hlline{\ \ \ 33\ }\hlcom{}\hlstd{ARGUMENTS=}\hlstr{"['-I\$PWD',"}\hlstd{}\hlcom{\#\ argument\ stub\ general\\
}\hlline{\ \ \ 34\ }\hlcom{}\hlstd{OFILE=}\hlstr{""}\hlstd{}\hlcom{\#\ The\ original\ object\ file\\
}\hlline{\ \ \ 35\ }\hlcom{}\hlstd{\\
}\hlline{\ \ \ 36\ }\hlstd{}\hlcom{\#\#\#\#\#\#\#\#\#\#\#\#\#\#\#\#\#\#\#\#\#\#\#\#\#\#\#\#\#\#\#\#\#\#\#\#\#\#\#\#\#\#\#\#\#\#\#\#\#\#\#\#\#\#\#\#\#\#\#\#\#\#\#\#\#\#\#\#\#\#\#\#\#\#\#\#\#\#\#\\
}\hlline{\ \ \ 37\ }\hlcom{}\hlstd{\\
}\hlline{\ \ \ 38\ }\hlstd{}\hlkey{for\ }\hlstd{ARG\ }\hlkey{in\ }\hlstd{}\hltyp{\$}\hlstd{@\\
}\hlline{\ \ \ 39\ }\hlstd{}\hlkey{do\\
}\hlline{\ \ \ 40\ }\hlkey{\hlstd{\ \ }}\hlstd{((COUNT++))\\
}\hlline{\ \ \ 41\ }\hlstd{\\
}\hlline{\ \ \ 42\ }\hlstd{\hlstd{\ \ }}\hlkey{if\ }\hlstd{((COUNT\ $>$\ }\hlnum{4}\hlstd{));\ }\hlkey{then\\
}\hlline{\ \ \ 43\ }\hlkey{\\
}\hlline{\ \ \ 44\ }\hlkey{\hlstd{\ \ \ \ }if\ }\hlstd{[[\ }\hltyp{\$\{ARG:0:9\}}\hlstd{\ ==\ }\hlstr{"-testopt:"}\hlstd{\ ]];\ }\hlkey{then\\
}\hlline{\ \ \ 45\ }\hlkey{\hlstd{\ \ \ \ \ \ }}\hlstd{ARGUMENTS=}\hlstr{"\$\{ARGUMENTS\}\ '`echo\ \$ARG\ \textbar \ sed\ -e\ 's/-testopt://g'`',\ "}\hlstd{\\
}\hlline{\ \ \ 46\ }\hlstd{\hlstd{\ \ \ \ \ \ }}\hltyp{continue\\
}\hlline{\ \ \ 47\ }\hltyp{\hlstd{\ \ \ \ }}\hlstd{}\hlkey{fi\ }\hlstd{}\hlcom{\#\ parse\ out\ specific\ options\ to\ test\ only\ and\ not\ to\ backend\\
}\hlline{\ \ \ 48\ }\hlcom{}\hlstd{\\
}\hlline{\ \ \ 49\ }\hlstd{\hlstd{\ \ \ \ }BACKEND=}\hlstr{"\$\{BACKEND\}\ \$ARG"}\hlstd{\ }\hlcom{\#\ build\ original\ compile-line\\
}\hlline{\ \ \ 50\ }\hlcom{}\hlstd{\\
}\hlline{\ \ \ 51\ }\hlstd{\hlstd{\ \ \ \ }}\hlcom{\#case\#\#\#\#\#\#\#\#\#\#\#\#\#\#\#\#\#\#\#\#\#\#\#\#\#\#\#\#\#\#\#\#\#\#\#\#\#\#\#\#\#\#\#\#\#\#\#\#\#\#\#\#\#\#\#\#\#\#\#\#\#\#\#\#\#\#\#\#\#\#\#\\
}\hlline{\ \ \ 52\ }\hlcom{}\hlstd{\\
}\hlline{\ \ \ 53\ }\hlstd{\\
}\hlline{\ \ \ 54\ }\hlstd{\hlstd{\ \ \ \ }}\hlkey{case\ }\hlstd{}\hltyp{\$ARG\ }\hlstd{}\hlkey{in\\
}\hlline{\ \ \ 55\ }\hlkey{}\hlstd{-I*)\ includeFullPath;;\\
}\hlline{\ \ \ 56\ }\hlstd{\\
}\hlline{\ \ \ 57\ }\hlstd{*.c\ \textbar \ *.cpp\ \textbar \ *.C\ \textbar \ *.[cC]*\ )\\
}\hlline{\ \ \ 58\ }\hlstd{\hlstd{\ \ \ \ \ \ \ \ \ \ }}\hlkey{if\ }\hlstd{[[\ }\hltyp{\$\{ARG:0:1\}}\hlstd{\ !=\ }\hlstr{'/'}\hlstd{\ ]];\ }\hlkey{then\\
}\hlline{\ \ \ 59\ }\hlkey{\hlstd{\ \ \ \ \ \ \ \ \ \ \ \ }}\hlstd{ARG=}\hlstr{"`pwd`/\$ARG"}\hlstd{\\
}\hlline{\ \ \ 60\ }\hlstd{\hlstd{\ \ \ \ \ \ \ \ \ \ }}\hlkey{fi\ }\hlstd{}\hlcom{\#\ take\ care\ of\ absolute\ paths\\
}\hlline{\ \ \ 61\ }\hlcom{}\hlstd{\\
}\hlline{\ \ \ 62\ }\hlstd{\hlstd{\ \ }}\hlcom{\#\ rename\ the\ QMTest\ output\ test\ file.\ Replace\ space,\ period,\ and\ plus\\
}\hlline{\ \ \ 63\ }\hlcom{}\hlstd{\hlstd{\ \ }}\hlcom{\#\ with\ their\ equivalents\ and\ change\ all\ chars\ to\ lower\ case.\\
}\hlline{\ \ \ 64\ }\hlcom{}\hlstd{\hlstd{\ \ }}\hlkey{if\ }\hlstd{[[\ }\hltyp{\$MODE\ }\hlstd{=\ }\hlstr{'f'}\hlstd{\ ]];\ }\hlkey{then\\
}\hlline{\ \ \ 65\ }\hlkey{\hlstd{\ \ \ \ \ \ \ \ \ \ \ \ }}\hlstd{TEST=`}\hltyp{echo\ \$ARG\ }\hlstd{\textbar \ sed\ -e\ }\hlstr{'s/$\backslash$//\textunderscore /g'}\hlstd{\ \textbar \ sed\ -e\ }\hlstr{'s/$\backslash$./\textunderscore dot\textunderscore /g'}\hlstd{\ \textbar \ $\backslash$\\
}\hlline{\ \ \ 66\ }\hlstd{\hlstd{\ \ }sed\ -e\ }\hlstr{'s/+/plus/g'}\hlstd{\ \textbar \ gawk\ }\hlstr{'\{print\ tolower(\$0)\}'}\hlstd{`.qmt\\
}\hlline{\ \ \ 67\ }\hlstd{\hlstd{\ \ }}\hlkey{fi\\
}\hlline{\ \ \ 68\ }\hlkey{}\hlstd{;;\ }\hlcom{\#\ case\ C/C++\ source\ files\\
}\hlline{\ \ \ 69\ }\hlcom{}\hlstd{\\
}\hlline{\ \ \ 70\ }\hlstd{-o)\\
}\hlline{\ \ \ 71\ }\hlstd{\hlstd{\ \ }}\hlkey{if\ }\hlstd{[[\ }\hltyp{\$MODE\ }\hlstd{=\ }\hlstr{'o'}\hlstd{\ ]];\ }\hlkey{then\\
}\hlline{\ \ \ 72\ }\hlkey{\hlstd{\ \ \ \ }}\hlstd{FLAG=}\hlnum{1\\
}\hlline{\ \ \ 73\ }\hlnum{\hlstd{\ \ }}\hlstd{}\hlkey{elif\ }\hlstd{[[\ }\hltyp{\$MODE\ }\hlstd{=\ }\hlstr{'f'}\hlstd{\ ]];\ }\hlkey{then\\
}\hlline{\ \ \ 74\ }\hlkey{\hlstd{\ \ \ \ }}\hlstd{FLAG=}\hlnum{2\\
}\hlline{\ \ \ 75\ }\hlnum{\hlstd{\ \ }}\hlstd{}\hlkey{fi\ }\hlstd{}\hlcom{\#\ spike\ out\ the\ object\ flag\\
}\hlline{\ \ \ 76\ }\hlcom{}\hlstd{\\
}\hlline{\ \ \ 77\ }\hlstd{\hlstd{\ \ }}\hltyp{continue\\
}\hlline{\ \ \ 78\ }\hltyp{}\hlstd{;;\ }\hlcom{\#\ case\ -o\\
}\hlline{\ \ \ 79\ }\hlcom{}\hlstd{\\
}\hlline{\ \ \ 80\ }\hlstd{*)\ ;;\ }\hlcom{\#\ default\\
}\hlline{\ \ \ 81\ }\hlcom{}\hlstd{\hlstd{\ \ \ \ }}\hlkey{esac\\
}\hlline{\ \ \ 82\ }\hlkey{\\
}\hlline{\ \ \ 83\ }\hlkey{\hlstd{\ \ \ \ }}\hlstd{}\hlcom{\#esac\#fi\#\#\#\#\#\#\#\#\#\#\#\#\#\#\#\#\#\#\#\#\#\#\#\#\#\#\#\#\#\#\#\#\#\#\#\#\#\#\#\#\#\#\#\#\#\#\#\#\#\#\#\#\#\#\#\#\#\#\#\#\#\#\#\#\#\#\#\#\\
}\hlline{\ \ \ 84\ }\hlcom{}\hlstd{\\
}\hlline{\ \ \ 85\ }\hlstd{\\
}\hlline{\ \ \ 86\ }\hlstd{\hlstd{\ \ \ \ }}\hlkey{if\ }\hlstd{((FLAG\ $>$\ }\hlnum{0}\hlstd{));\ }\hlkey{then\\
}\hlline{\ \ \ 87\ }\hlkey{\hlstd{\ \ \ \ \ \ }}\hlstd{OFILE=}\hltyp{\$ARG\\
}\hlline{\ \ \ 88\ }\hltyp{\hlstd{\ \ \ \ }}\hlstd{}\hlkey{fi\ }\hlstd{}\hlcom{\#\ name\ the\ object\ file\ after\ -o\ declaration\\
}\hlline{\ \ \ 89\ }\hlcom{}\hlstd{\\
}\hlline{\ \ \ 90\ }\hlstd{\hlstd{\ \ \ \ }}\hlkey{if\ }\hlstd{((FLAG\ ==\ }\hlnum{1}\hlstd{));\ }\hlkey{then\\
}\hlline{\ \ \ 91\ }\hlkey{\hlstd{\ \ \ \ \ \ }if\ }\hlstd{[[\ }\hltyp{\$\{ARG:0:1\}}\hlstd{\ !=\ }\hlstr{'/'}\hlstd{\ ]];\ }\hlkey{then\\
}\hlline{\ \ \ 92\ }\hlkey{\hlstd{\ \ \ \ \ \ \ \ }}\hlstd{ARG=}\hlstr{"`pwd`/\$ARG"}\hlstd{\\
}\hlline{\ \ \ 93\ }\hlstd{\hlstd{\ \ \ \ \ \ }}\hlkey{fi\ }\hlstd{}\hlcom{\#\ if\ argument\ not\ specified\ with\ absolute\ path\ then\ append\ PWD\\
}\hlline{\ \ \ 94\ }\hlcom{}\hlstd{\\
}\hlline{\ \ \ 95\ }\hlstd{\hlstd{\ \ \ \ \ \ }}\hlcom{\#\ rename\ the\ QMTest\ output\ test\ file,\ replace\ space,\ period,\ and\ plus\\
}\hlline{\ \ \ 96\ }\hlcom{}\hlstd{\hlstd{\ \ \ \ \ \ }}\hlcom{\#\ with\ equivalent\ symbols\ and\ change\ all\ chars\ to\ lower\ case.\\
}\hlline{\ \ \ 97\ }\hlcom{}\hlstd{\hlstd{\ \ \ \ \ \ }TEST=`}\hltyp{echo\ \$ARG\ }\hlstd{\textbar \ sed\ -e\ }\hlstr{'s/$\backslash$//\textunderscore /g'}\hlstd{\ \textbar \ sed\ -e\ }\hlstr{'s/$\backslash$./\textunderscore dot\textunderscore /g'}\hlstd{\ \textbar \ $\backslash$\\
}\hlline{\ \ \ 98\ }\hlstd{\hlstd{\ \ \ \ }sed\ -e\ }\hlstr{'s/+/plus/g'}\hlstd{\ \textbar \ gawk\ }\hlstr{'\{print\ tolower(\$0)\}'}\hlstd{`.qmt\\
}\hlline{\ \ \ 99\ }\hlstd{\\
}\hlline{\ \ 100\ }\hlstd{\hlstd{\ \ \ \ \ \ }FLAG=}\hlnum{0}\hlstd{}\hlcom{\#\ reset\ FLAG\\
}\hlline{\ \ 101\ }\hlcom{}\hlstd{\hlstd{\ \ \ \ \ \ }}\hltyp{continue\\
}\hlline{\ \ 102\ }\hltyp{\hlstd{\ \ \ \ }}\hlstd{}\hlkey{elif\ }\hlstd{((FLAG\ ==\ }\hlnum{2}\hlstd{));\ }\hlkey{then\\
}\hlline{\ \ 103\ }\hlkey{\hlstd{\ \ \ \ \ \ }}\hlstd{FLAG=}\hlnum{0}\hlstd{}\hlcom{\#\ reset\ FLAG\\
}\hlline{\ \ 104\ }\hlcom{}\hlstd{\hlstd{\ \ \ \ \ \ }}\hltyp{continue\\
}\hlline{\ \ 105\ }\hltyp{\hlstd{\ \ \ \ }}\hlstd{}\hlkey{fi\ }\hlstd{}\hlcom{\#\ if\ the\ -o\ flag\ used;\ create\ the\ object\ name\ and\ TEST\ name\ from\ object\\
}\hlline{\ \ 106\ }\hlcom{}\hlstd{\\
}\hlline{\ \ 107\ }\hlstd{\hlstd{\ \ \ \ }}\hlcom{\#fi\#if\#\#\#\#\#\#\#\#\#\#\#\#\#\#\#\#\#\#\#\#\#\#\#\#\#\#\#\#\#\#\#\#\#\#\#\#\#\#\#\#\#\#\#\#\#\#\#\#\#\#\#\#\#\#\#\#\#\#\#\#\#\#\#\#\#\#\#\#\#\#\\
}\hlline{\ \ 108\ }\hlcom{}\hlstd{\\
}\hlline{\ \ 109\ }\hlstd{\\
}\hlline{\ \ 110\ }\hlstd{\hlstd{\ \ \ \ }ARGUMENTS=}\hlstr{"\$\{ARGUMENTS\}\ '\$ARG',\ "}\hlstd{\\
}\hlline{\ \ 111\ }\hlstd{\hlstd{\ \ }}\hlkey{fi\ }\hlstd{}\hlcom{\#\ if\ argument\ is\ not\ qm.sh\ argument\\
}\hlline{\ \ 112\ }\hlcom{}\hlstd{\\
}\hlline{\ \ 113\ }\hlstd{\hlstd{\ \ }}\hlcom{\#fi\#done\#\#\#\#\#\#\#\#\#\#\#\#\#\#\#\#\#\#\#\#\#\#\#\#\#\#\#\#\#\#\#\#\#\#\#\#\#\#\#\#\#\#\#\#\#\#\#\#\#\#\#\#\#\#\#\#\#\#\#\#\#\#\#\#\#\#\#\#\#\#\\
}\hlline{\ \ 114\ }\hlcom{}\hlstd{\\
}\hlline{\ \ 115\ }\hlstd{\\
}\hlline{\ \ 116\ }\hlstd{}\hlkey{done\ }\hlstd{}\hlcom{\#\ for\ all\ command-line\ arguments\ to\ qm.sh\\
}\hlline{\ \ 117\ }\hlcom{}\hlstd{\\
}\hlline{\ \ 118\ }\hlstd{OBJECT=}\hltyp{\$\{TEST\%\%.*\}}\hlstd{.o}\hlcom{\#\ name\ the\ object\ after\ the\ test\ file\ name\\
}\hlline{\ \ 119\ }\hlcom{}\hlstd{ARGUMENTS=}\hlstr{"\$\{ARGUMENTS\}\ '-o',\ '\$OBJECT']"}\hlstd{\\
}\hlline{\ \ 120\ }\hlstd{\\
}\hlline{\ \ 121\ }\hlstd{}\hlcom{\#done\#case\#\#\#\#\#\#\#\#\#\#\#\#\#\#\#\#\#\#\#\#\#\#\#\#\#\#\#\#\#\#\#\#\#\#\#\#\#\#\#\#\#\#\#\#\#\#\#\#\#\#\#\#\#\#\#\#\#\#\#\#\#\#\#\#\#\#\#\#\#\#\\
}\hlline{\ \ 122\ }\hlcom{}\hlstd{\\
}\hlline{\ \ 123\ }\hlstd{\\
}\hlline{\ \ 124\ }\hlstd{}\hlkey{case\ }\hlstd{}\hltyp{\$TEST\textunderscore CLASS\ }\hlstd{}\hlkey{in\\
}\hlline{\ \ 125\ }\hlkey{\hlstd{\ \ }}\hlstd{strings.SubStringTest)\ qmtest\ create\ -o\ }\hltyp{\$TEST\ }\hlstd{-a\ program=}\hlstr{"\$PROGRAM"}\hlstd{\ -a\ substring=}\hlstr{"ERROR\ SUMMARY:\ 0\ errors\ from\ 0\ contexts"}\hlstd{\ -a\ arguments=}\hlstr{"\$ARGUMENTS"}\hlstd{\ }\hltyp{test\ \$TEST\textunderscore CLASS}\hlstd{;;\\
}\hlline{\ \ 126\ }\hlstd{\\
}\hlline{\ \ 127\ }\hlstd{\hlstd{\ \ }*)\ qmtest\ create\ -o\ }\hlstr{"\$TEST"}\hlstd{\ -a\ program=}\hlstr{"\$PROGRAM"}\hlstd{\ -a\ arguments=}\hlstr{"\$ARGUMENTS"}\hlstd{\ }\hltyp{test\ \$TEST\textunderscore CLASS}\hlstd{;;\\
}\hlline{\ \ 128\ }\hlstd{}\hlkey{esac\ }\hlstd{}\hlcom{\#\ create\ qmtest\ test\ file\ with\ test\ class\\
}\hlline{\ \ 129\ }\hlcom{}\hlstd{\\
}\hlline{\ \ 130\ }\hlstd{}\hlcom{\#esac\#main\#\#\#\#\#\#\#\#\#\#\#\#\#\#\#\#\#\#\#\#\#\#\#\#\#\#\#\#\#\#\#\#\#\#\#\#\#\#\#\#\#\#\#\#\#\#\#\#\#\#\#\#\#\#\#\#\#\#\#\#\#\#\#\#\#\#\#\#\#\#\\
}\hlline{\ \ 131\ }\hlcom{}\hlstd{\\
}\hlline{\ \ 132\ }\hlstd{\\
}\hlline{\ \ 133\ }\hlstd{}\hlkey{if\ }\hlstd{[[\ }\hltyp{\$\{BACKEND:0:4\}}\hlstd{\ ==\ }\hlstr{"NULL"}\hlstd{\ ]];\ }\hlkey{then\\
}\hlline{\ \ 134\ }\hlkey{\hlstd{\ \ }}\hlstd{touch\ }\hltyp{\$OFILE\ }\hlstd{$>$\&\ /dev/null\ }\hlcom{\#\ create\ dummy\ file\ and\ pipe\ error\ to\ NULL\\
}\hlline{\ \ 135\ }\hlcom{}\hlstd{\hlstd{\ \ }}\hltyp{exit\ }\hlstd{}\hlnum{0\hlstd{\ \ \ \ }}\hlstd{}\hlcom{\#\ always\ exit\ 0\\
}\hlline{\ \ 136\ }\hlcom{}\hlstd{}\hlkey{fi\ }\hlstd{}\hlcom{\#\ skip\ backend\ compilation\\
}\hlline{\ \ 137\ }\hlcom{}\hlstd{\\
}\hlline{\ \ 138\ }\hlstd{}\hlcom{\#\ Execute\ backend\ compilation\ with\ original\ compile-line\\
}\hlline{\ \ 139\ }\hlcom{}\hlstd{}\hltyp{\$BACKEND\\
}\hlline{\ \ 140\ }\hltyp{exit\ \$}\hlstd{?}\\
\mbox{}\\
\normalfont
 }

\section{QMTest: Introduction}
{\tt qm.sh} is a wrapper for the compile-line in an arbitrary project build 
system that creates qmtest test files that test ROSE. The basic assumption
is that it is possible to isolate and modify the compile-line command in most
project build systems. For example, Makefile systems using {\tt make} specify
compile-line commands after labels delimited by a colon. One example of this
may be:

\begin{center}
  {\tt gcc -c -g -Wall hello.c}
\end{center}
From this line {\tt qm.sh} would create a qmtest test file that executes
a ROSE translator in the place of {\tt gcc} but with the exact same arguments
{\tt -c -g -Wall} and more if the user of {\tt qm.sh} should specify them.
{\tt qm.sh} also mimics the compile-line process of the project's build system
so that all dependencies are built as normal by the backend compiler.

\subsection{Usage}

{\tt qm.sh <f|o> <QMTest test class> <ROSE translator> <Backend compiler> 
\{compiler arguments\ldots\} \{ROSE arguments\ldots\}}

\begin{description}
  \item[{\large {\tt <f|o>}}] : The output file naming mode. Option ``f'' specifies
	{\tt qm.sh} to use source file names in naming output {\tt .qmt} files.
	Option ``o'' specifies {\tt qm.sh} to use object file names, as
	specified by the compile-line {\tt -o} flag to the backend compiler,
	for naming {\tt .qmt} output files.
  \item[$<$QMTest test class$>$] : The test class of the created test file,
	i.e rose.RoseTest or command.ExecTest.
  \item[$<$ROSE translator$>$] : The full path specifying a ROSE translator.
  \item[$<$Backend compiler$>$] : The name of the backend compiler used in
	the normal compilation of the project build system. Specify ``NULL'' as
	the $<$Backend compiler$>$ if you want to skip backend compilation.
  \item[\{compiler arguments\ldots\}] : The arguments specified on
	the command-line of the project build system.
  \item[\{ROSE arguments\ldots\}] : The arguments to the ROSE translator
	prefixed with -rose:$<$ROSE argument$>$, e.x. 
	{\tt -rose:--edg:no\_warnings}. Note, these may be placed anywhere
	after the $<$Backend Compiler$>$ argument.
\end{description}

\subsection{Variables}

\begin{description}
  \item[COUNT] : The for loop counter, keeps track of number of {\tt qm.sh}
	arguments.
  \item[FLAG] : Logical flag variable used in naming output {\tt .qmt} files.
  \item[TEST] : The name of the QMTest test file created.
  \item[TEST\_CLASS] : The QMTest class specified on command-line.
  \item[PROGRAM] : The ROSE translator specified on command-line.
  \item[BACKEND] : The original command-line of the project build system with
	the backend compiler.
  \item[ARGUMENTS] : The compile-line arguments specified on the command-line
	with any script user specified arguments for the ROSE translator such
	as {\tt --edg:no\_warnings} bound for the QMTest test file.
  \item[LAST\_ARG] : The closing stub to the QMTest arguments format along with
	the {\tt -o <object file name>} argument.
  \item[ARG] : The current compile-line argument place holder used in
	constructing the argument format to QMTest arguments
	{\tt ARGUMENTS="['arg1','arg2',...,'argN']"}. 

\end{description}

\subsection{Execution Walkthrough}
{\tt qm.sh} is broken into code blocks which each perform some procedure. These
blocks are delimited with a solid line of 80 {\tt \#} characters.

\subsection{Backend and ROSE arguments}

\begin{figure}[!ht]
{\scriptsize
\begin{verbatim}
1 for ARG in $@
2 do
3   ((COUNT++))
                                                                                
4   if ((COUNT > 3)); then
                                                                                
5     if [[ ${ARG:0:6} == "-rose:" ]]; then
6       ARGUMENTS="${ARGUMENTS} '`echo $ARG | sed -e 's/-rose://g'`', "
7       continue
8     fi
                                                                                
9     BACKEND="${BACKEND} $ARG" # build original compile-line
\end{verbatim}
}
\caption{Backend and ROSE argument construction block}
\end{figure}

This block of code builds the original compile line of the project's build
system along with the arguments passed specifically to the ROSE compiler. In
the {\tt for} loop all the arguments passed to {\tt qm.sh} are looped through,
however, the first three arguments are skipped due to the {\tt if} statement on
line 4. All other arguments after the third are considered arguments of either
ROSE or the original project's build system. ROSE arguments must be prefixed
with {\tt -rose:<ROSE argument>} when specified on the compile-line. Each
argument with this prefix is stripped of the prefix {\tt -rose:} and added to
the {\tt ARGUMENT} list of the QMTest test file. ROSE arguments are not
carried over to the {\tt BACKEND} compile-line variable but all other arguments
are appended without change with the exception of the 
{\tt -o <Object file Name>} flag.

\subsection{Relative Path Compile-line Arguments}

\begin{figure}[!ht]
{\scriptsize
\begin{verbatim}
1    case $ARG in
2        -I*) includeFullPath;;
                                                                                
3        [!/]*.c | [!/]*.cpp | [!/]*.C | [!/]*.[cC]* ) ARG="`pwd`/$ARG";;
                                                                                
4        -o) BOOL=1 ; continue;; # spike out -o outputfilename
                                                                                
5        *) ;;
6    esac
\end{verbatim}
}
\caption{Relative to Absolute Paths in Arguments}
\end{figure}

This block of code handles all compile-line arguments containing relative file
or include paths. The {\tt case...esac} switch statement compares against
patterns indicative of C/C++ source files or an include directive. All source
files without absolute paths stemming from root are simply appended with their
present working directory. Directories specified by the {\tt -I} include 
directive call the function {\tt includeFullPath} which changed relative paths
to absolute paths.

\subsection{Naming QMTest Files}

\begin{figure}[!ht]
{\scriptsize
\begin{verbatim}
1      if [[ ${ARG:0:1} != '/' ]]; then
2        ARG=`pwd`/$ARG
3      fi
                                                                                
4      TEST=`echo $ARG | sed -e 's/\//_/g' | sed -e 's/\./_/g' | gawk '{print tolower($0)}'`.qmt
                                                                                
5      OBJECT=${TEST%%.*}.o
\end{verbatim}
}
\caption{Naming procedure for QMTest Files}
\end{figure}

At this block of code it is assumed that {\tt ARG} contains name of either the 
source or object file specified by the command-line. This name is must first
contain its absolute path to prevent name collisions which is handled by the
{\tt if} construct on lines 1-3. The {\tt TEST} name is then created on line 4
by replacing any {\tt '/'}(forward slashes) or {\tt '.'}(dots) in {\tt ARG} 
with underscores. The {\tt OBJECT} name is simply the {\tt TEST} name value
with the {\tt.o} extension. The object file name argument held in {\tt OBJECT}
is appended to the end of the QMTest argument list along with the {\tt -o} 
flag. Note that QMTest does not allow capital alphabetic letters or periods
in the names of individual tests.

\subsection{Create QMTest test and Execute Backend}

\begin{figure}[!ht]
{\scriptsize
\begin{verbatim}
1 qmtest create -o "$TEST" -a program="$PROGRAM" -a arguments="$ARGUMENTS" test $TEST_CLASS;;
                                                                                
2 $BACKEND # Execute the old command-line to fake the makefile
3 exit $?
\end{verbatim}
}
\caption{Create .qmt and Execute Backend}
\end{figure}

Line 1 creates a {\tt .qmt} QMTest file with the name {\tt TEST} that executes
{\tt PROGRAM} with arguments {\tt ARGUMENTS} using the class {\tt TEST\_CLASS}.
The {\tt .qmt} test file is created in the present working directory of the
project's build system file structure under the ``make'' process.
Lines 2-3 execute the reconstructed original backend compile-line of project's
build system. The script {\tt qm.sh} exits with the same code as the exit
status of the backend process.

\subsection{Example}

The following example edits a trivial makefile and builds QMTest files with
{\tt qm.sh} by editing the makefile.

\begin{figure}[!ht]
{\scriptsize
\begin{verbatim}
CXX = g++
CFLAGS = -g -Wall

CPU.out : main.o registers.o reader.o decoder.o
        $(CXX) $(CFLAGS) -o CPU.out reader.o registers.o decoder.o main.o
                                                                                
main.o : main.c registers.h reader.h decoder.h instruction.h
        $(CXX) $(CFLAGS) -c main.c -o main.o
                                                                                
registers.o : registers.c registers.h main.h
        $(CXX) $(CFLAGS) -c registers.c -o registers.o
                                                                                
reader.o : reader.c reader.h instruction.h
        $(CXX) $(CFLAGS) -c reader.c -o reader.o
                                                                                
decoder.o : decoder.c decoder.h
        $(CXX) $(CFLAGS) -c decoder.c -o decoder.o
\end{verbatim}
}
\caption{makefile before editing}
\end{figure}
By inserting the {\tt qm.sh} wrapper before each instance of {\tt g++} in this
case it is possible to generate {\tt .qmt} test files. The modified makefile is
shown below:

\begin{figure}[!ht]
{\scriptsize
\begin{verbatim}
QM = /home/yuan5/RoseQMTest/scripts/qm.sh
ROSE = /home/yuan5/bin/identityTranslator
MYCC = $(QM) rose.RoseTest $(ROSE)
CXX = $(MYCC) g++
ROSEFLAGS = -rose:--edg:no_warnings
CFLAGS = $(ROSEFLAGS) -g -Wall
                                                                                
CPU.out : main.o registers.o reader.o decoder.o
        $(CXX) $(CFLAGS) -o CPU.out reader.o registers.o decoder.o main.o
                                                                                
main.o : main.c registers.h reader.h decoder.h instruction.h
        $(CXX) $(CFLAGS) -c main.c -o main.o
                                                                                
registers.o : registers.c registers.h main.h
        $(CXX) $(CFLAGS) -c registers.c -o registers.o
                                                                               
reader.o : reader.c reader.h instruction.h
        $(CXX) $(CFLAGS) -c reader.c -o reader.o
                                                                                
decoder.o : decoder.c decoder.h
        $(CXX) $(CFLAGS) -c decoder.c -o decoder.o
\end{verbatim}
}
\caption{makefile after editing}
\end{figure}

After the edits have taken place it is evident that {\tt qm.sh} wraps around
each compile-line of the makefile. The arguments to {\tt qm.sh} are themselves
encompassed by the variable {\tt MYCC} leaving minimal edits to the makefile
itself. The makefile may now be run with {\tt make} and the project will
be made along with all the QMTest {\tt .qmt} files.

\begin{figure}[!ht]
{\scriptsize
\begin{verbatim}
bash-2.05b$ make
/home/yuan5/RoseQMTest/scripts/qm_file.sh rose.RoseTest /home/yuan5/bin/identityTranslator 
g++ -rose:--edg:no_warnings -g -Wall -c main.c
/home/yuan5/RoseQMTest/scripts/qm_file.sh rose.RoseTest /home/yuan5/bin/identityTranslator 
g++ -rose:--edg:no_warnings -g -Wall -c registers.c
/home/yuan5/RoseQMTest/scripts/qm_file.sh rose.RoseTest /home/yuan5/bin/identityTranslator
g++ -rose:--edg:no_warnings -g -Wall -c reader.c
/home/yuan5/RoseQMTest/scripts/qm_file.sh rose.RoseTest /home/yuan5/bin/identityTranslator 
g++ -rose:--edg:no_warnings -g -Wall -c decoder.c
/home/yuan5/RoseQMTest/scripts/qm_file.sh rose.RoseTest /home/yuan5/bin/identityTranslator 
g++ -rose:--edg:no_warnings -g -Wall -o CPU.out reader.o registers.o decoder.o main.o
\end{verbatim}
}
\caption{{\tt make} output}
\end{figure}

\begin{figure}[!ht]
{\scriptsize
\begin{verbatim}
bash-2.05b$ find . -name "*.qmt" 
_home_yuan5_roseqmtest_project_p2_cpu_out.qmt
_home_yuan5_roseqmtest_project_p2_decoder_c.qmt
_home_yuan5_roseqmtest_project_p2_main_c.qmt
_home_yuan5_roseqmtest_project_p2_reader_c.qmt
_home_yuan5_roseqmtest_project_p2_registers_c.qmt
\end{verbatim}
}
\caption{{\tt find . -name "*.qmt"} output}
\end{figure}

\subsection{Running the Tests}

This section describes how to collect and run the test created by {\tt qm.sh}
after building the project with an edited build system. When the project has
completed building, the QMTest files will most likely be scattered across all
the local directories containing their object file counterparts. Thus it's
necessary to collect them all into one directory which will serve as a QMTest
database. From the directory where {\tt make} or the project's build system
was launched type the command:
\begin{center}
{\tt find . -name "*.qmt" -exec mv \{\} test\_database $\backslash$; }
\end{center}

This will recursively find all files with extensions {\tt .qmt} and move them
to the directory test\_database which was created by the user. Change directory
to test\_database and type the command:
\begin{center}
{\tt qmtest -D`pwd` create-tdb}
\end{center}
This command will allow QMTest to access the test files by creating a test
database. Once this test database has been created by QMTest it is possible to
run tests from the command-line or GUI with the respective commands:

\begin{verbatim}
qmtest run -o results.qmr 
# runs command-line and writes QMTest output to results.qmr 

qmtest gui 
# runs the QMTest GUI by which the user may read results stored in results.qmr 
# or run additional tests.
\end{verbatim}


