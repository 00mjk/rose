% vim: set fdm=marker

\chapter{RTED: Runtime Error Detection} %{{{

\label{RTED}


\section{Overview}%{{{

RTED\footnote{\url{http://rted.public.iastate.edu/}} is a test suite
developed by Iowa State University's High Performance Computing
Group\footnote{\url{http://www.it.iastate.edu/research/hpcg/}}.  It illustrates
a number of runtime errors, many of which result from undefined behavior in the
C and C++ specifications.

The RTED project in ROSE is a program that instruments its input to use
a supplied runtime system with the aim that when the instrumented program runs,
any undefined behavior is immediately caught, as well as certain other runtime
errors such as certain memory leaks.  For example, in the following code:

\begin{verbatim}
    int foo() {
        int a[3];
        int n;

        int* p = &a[ 2 ];
        ++p;
    }
\end{verbatim}

After \texttt{p} is incremented, its address is undefined.  Many compilers might
well organize the call frame such that after the increment, \texttt{p == \&n},
but relying on this is unsafe.

When the runtime system detects unspecified behavior, a violation is raised and
handled.  The default policy for most violations is to output a message and
immediately terminate the program.


\subsection{Current State}

At the time of this writing, the project is incomplete.  Presently, only the
RTED test suite is targeted, although a large subset of its C and C++
tests' errors are caught.


\subsection{Organization}

The project is organized into two sections:

\begin{enumerate}

    \item The runtime system is a library that provides an API that, in
    principle, can be used for other projects that require tracking variables,
    the stack, “typed” memory, and so on.

    \item The transformation code is compiled into a binary that transforms
    input by injecting calls to the runtime system.  For convenience, the actual
    injected calls are to very small wrapper functions that provide a simpler
    abstraction over the runtime system's API.  However, if users wish to adapt
    the runtime system, it is recommended that they view the wrapper functions
    as an example usage of the runtime system's API, and not the API itself.  
    
\end{enumerate}

The transformation only injects function calls that are valid C and C++ code.
The runtime system is written in C++ but its API is compiled with
\texttt{extern “C”}

%}}}


\section{How to use RTED in ROSE}%{{{

To instrument a program, one does the following:

\begin{enumerate}

    \item Compile \texttt{runtimeCheck} in \texttt{projects/RTED} of the ROSE
    compile tree.

    \item \texttt{\#include RuntimeSystem.h} in your source files\footnote{You
    may notice that in the ROSE makefile, \texttt{\#include
    RuntimeSystem\_ParsingWorkaround.h} is sometimes included instead.  You will
    probably not need to do this, but if you have errors resulting from
    \texttt{std::endl} being undefined, you can try using
    \texttt{RuntimeSystem\_ParsingWorkaround.h} instead.}.

    \item Run \texttt{runtimeCheck -n <number-of-source-files> <sourcefile1>
    <sourcefile2> ... <sourcefilen>} to produce \texttt{sourcefile1\_rose,
    sourcefile2\_rose, ...}.

    \item Compile the \texttt{*\_rose} instrumented source files.

\end{enumerate}

For example, the following code:
\begin{verbatim}
    #include <stdlib.h>


    int main( int argc, char** argv ) {

        int z[ 2 ];
        int* x;

        int i = 0;
        for( ; i >= 0; --i ) {
            int* y;
            y = (int*) malloc( sizeof( int ));
            x = y;
        }

        // error, y is out of scope so assigning to x leaks the malloc in the
        // for loop
        x = z;

        return 0;
    }
\end{verbatim}

is transformed into:

\begin{verbatim}
#include "RuntimeSystem.h"
#include <stdlib.h>

extern int RuntimeSystem_original_main(int argc,char **argv,char **envp)
{
  RuntimeSystem_roseConfirmFunctionSignature("RuntimeSystem_original_main",3,"SgTypeInt","",0,"SgTypeInt","",0,"SgPointerType","SgTypeChar",2);
  
  RuntimeSystem_roseCreateVariable("envp","envp","SgPointerType","SgTypeChar",2,((unsigned long long )(&envp)),sizeof(envp),1,"","0","transformation","10");
  
  RuntimeSystem_roseCreateVariable("argv","argv","SgPointerType","SgTypeChar",2,((unsigned long long )(&argv)),sizeof(argv),1,"","0","transformation","13");
  
  RuntimeSystem_roseCreateVariable("argc","argc","SgTypeInt","",0,((unsigned long long )(&argc)),sizeof(argc),1,"","0","transformation","16");
  int z[2UL];
  
  RuntimeSystem_roseCreateVariable("z","z","SgArrayType","SgTypeInt",1,((unsigned long long )(&z)),sizeof(z),0,"","0","transformation","20");
  
  RuntimeSystem_roseCreateHeap("z","L14R__scope____SgSS2____scope__z","SgArrayType","SgTypeInt",1,((unsigned long long )(&z)),sizeof(z),0,0,"","assignment_scope_for.c","8","23",1,2UL);
  int *x;
  
  RuntimeSystem_roseCreateVariable("x","x","SgPointerType","SgTypeInt",1,((unsigned long long )(&x)),sizeof(x),0,"","0","transformation","27");
  int i = 0;
  
  RuntimeSystem_roseCreateVariable("i","i","SgTypeInt","",0,((unsigned long long )(&i)),sizeof(i),1,"","0","transformation","31");
  
  RuntimeSystem_roseInitVariable("SgTypeInt","",0,"",((unsigned long long )(&i)),sizeof(i),0,0,"assignment_scope_for.c","11","34");
  RuntimeSystem_roseEnterScope("for:12");
  
  RuntimeSystem_roseAccessVariable(((unsigned long long )(&i)),sizeof(i),((unsigned long long )(&i)),sizeof(i),1,"assignment_scope_for.c","12","38");
  
  RuntimeSystem_roseAccessVariable(((unsigned long long )(&i)),sizeof(i),((unsigned long long )(&i)),sizeof(i),1,"assignment_scope_for.c","12","41");
  for (; i >= 0; --i) {
    int *y;
    
    RuntimeSystem_roseCreateVariable("y","y","SgPointerType","SgTypeInt",1,((unsigned long long )(&y)),sizeof(y),0,"","0","transformation","46");
    y = ((int *)(malloc(((sizeof(int ))))));
    
    RuntimeSystem_roseCreateHeap("y","L14R__scope____SgSS2____scope____SgSS3____scope____SgSS4____scope__y","SgPointerType","SgTypeInt",1,((unsigned long long )(&y)),sizeof(y),((long long )((sizeof(int )))),1,"","assignment_scope_for.c","14","50",0);
    
    RuntimeSystem_roseInitVariable("SgPointerType","SgTypeInt",1,"",((unsigned long long )(&y)),sizeof(y),1,1,"assignment_scope_for.c","14","53");
    
    RuntimeSystem_roseAccessVariable(((unsigned long long )(&y)),sizeof(y),((unsigned long long )(&y)),sizeof(y),1,"assignment_scope_for.c","15","56");
    x = y;
    
    RuntimeSystem_roseInitVariable("SgPointerType","SgTypeInt",1,"",((unsigned long long )(&x)),sizeof(x),0,1,"assignment_scope_for.c","15","60");
  }
  
  RuntimeSystem_roseExitScope("assignment_scope_for.c","16","64","for(;i >= 0;--i) {int *y;y =((int *)(malloc(((sizeof(int ))))));x = y;}");
// error, y is out of scope so assigning to x leaks the malloc in the
// for loop
  x = z;
  
  RuntimeSystem_roseInitVariable("SgPointerType","SgTypeInt",1,"",((unsigned long long )(&x)),sizeof(x),0,1,"assignment_scope_for.c","19","69");
  int rstmt = 0;
  RuntimeSystem_roseCheckpoint("assignment_scope_for.c","22","72");
  return rstmt;
}


int main(int argc,char **argv,char **envp)
{
  int exit_code = RuntimeSystem_original_main(argc,argv,envp);
  RuntimeSystem_roseRtedClose("RuntimeSystem.cpp:main");
  return exit_code;
}
\end{verbatim}

which, in the call to \texttt{RuntimeSystem\_roseInitVariable} that immediately
follows \texttt{x = z} will detect the memory leak and terminate the program.

\subsection{Configuration}

At present, configuration is very limited.  When the runtime system is
initialized, it searches for \texttt{RTED.cfg} in the current directory.  If
such a file exists, it is read as a newline separated list of key value pairs.
If the runtime system was compiled with QT enabled, then the qt debugger can be
enabled so that the GUI debugger is opened when the first violation is found.
Also, invidual policies for errors types can be configured.  See the RTED doxygen
documentation for more information.


\subsection{Partial Compilation}

At present the transformation does not properly support partial compilation.
This is not a fundamental challenge to the current design, it merely reflects
the incomplete state of the project.  See also section
\ref{RTED:known-limitations}.
%}}}


\section{Extending the Runtime System}%{{{

The runtime system is separate from the transformations.  Users who wish to use
the runtime system for other purposes, perhaps other sorts of runtime checks,
can do so by compiling the runtime system (in
\texttt{<rose-dir>/projects/RTED/CppRuntimeSystem}) and using it separately.
The doxygen documentation has more information on the organization of the
various components of the runtime system, especially the documentation for the
\texttt{Runtimesystem} class.

%}}}


\section{Known Limitations}%{{{

\label{RTED:known-limitations}


Apart from being incomplete, there are some known limitations.

\begin{enumerate}
    
    \item The runtime system presently assumes all pointers have a fixed size.
    This size is determined at the compile time \emph{of the runtime system} via
    \texttt{sizeof( void* )}.

    \item Similarly, the size of basic types is determined when the runtime
    system is compiled.

    \item Separate compilation is not properly handled yet.  At present, globals
    and types in the global namespace are reported to the runtime system via
    calls injected into \texttt{main}.  This obviously doesn't work when compiling a file
    that doesn't define \texttt{main}.

    \item There is some support for handling C I/O usage, and although there is
    some code related to C++ I/O support, it is extremely preliminary.

    \item There is no support for C++ object initializers.

    \item Typedefs are not resolved by the instrumentation.  Although this is
    not difficult, it does mean that the runtime system treats all typedefs as a
    single type, and distinct from their actual type.

\end{enumerate}

\subsection{A Note on RTED Scoring}

The RTED suite includes scoring tests.  Presently the RTED support in ROSE is
checked by ensuring that an error is found on the expected line.  The reference
messages for RTED include information that is not currently output by
instrumented inputs (such as variable names).  However, much of the missing
information is actually collected by the runtime system and is simply not
output.


\section{Running a specific RTED test}

 
To run one specific test run:

  make run/tests/C/memoryleaks/assignment\_scope\_dowhile

The different steps involved in running the test are:

1. Insert \#include ``RuntimeSystem.h'' into file

2. Test if files compile with gcc (file\_rose.c.compile.out)

3. Run transformation (file\_rose.c.runtimeCheck.out)

4. Add line information

5. Compile instrumented code (file.bin.compile.out)

6. Execute instrumented code



\section{Handling warnings}

It is wise to enforce all warnings to be handled as errors. This will ensure a clean implementation. We enforced this using Eclipse.
In addition we added the following flags to the compilation to skip certain warnings:

\begin{verbatim}
  -Wno-non-virtual-dtor -Wno-invalid-offsetof 
\end{verbatim}

\section{Using Eclipse for RTED development}

It is possible to use Eclipse just for RTED and link against librose.so.

To use Eclipse, download the C++ Eclipse version and create new project RTED.
Then include the original RTED directory. The following are the additional includes necessary in C/C++ Build Settings for the GNU C++ Compiler:

\begin{verbatim}
-I/home/panas2/programs/qt/install/qt/include/QtGui 
-I/home/panas2/rose/ROSE-git/projects/RTED/CppRuntimeSystem/DebuggerQt 
-I/home/panas2/rose/ROSE-git/projects/RTED/CppRuntimeSystem/ 
-I/home/panas2/rose/ROSE-git/projects/RTED/ 
-I/home/panas2/rose/ROSE-git/src/roseExtensions/qtWidgets/QCodeEditWidget 
-I/home/panas2/programs/qt/install/qt/include/QtCore 
-I/home/panas2/rose/build/projects/RTED/CppRuntimeSystem/DebuggerQt 
-I/home/panas2/programs/qt/install/qt/include 
-I/home/panas2/programs/qt/install/qt/include/Qt 
-I/home/panas2/rose/ROSE-git/src/roseSupport 
-I/home/panas2/rose/ROSE-git/src/midend/astProcessing/ 
-I/home/panas2/rose/ROSE\_includes 
-I/home/panas2/programs/boost/install/include/ 
-I/home/panas2/rose/build/src/frontend/SageIII 
-I/home/panas2/rose/build 
\end{verbatim}

A simpler way to handle all required .h files in ROSE is to copy all includes from ROSE into a separate ROSE\_includes directory or just to install ROSE.
Thereafter add the rose library to the shared libararies using the path: build/src/.libs/ and the file: rose.
For this to work ROSE must be compiled into the build directory as files from the build directory are needed.

To run RTED, you can type the command (in bash or Eclipse):

\begin{verbatim}
  /home/panas2/rose/ROSE-git/projects/RTED/Debug/RTED 1  
  /home/panas2/rose/ROSE-git/projects/RTED/Debug/tests2/C/memoryleaks/
  assignment\_scope\_while.c  
  -I/home/panas2/rose/ROSE-git/projects/RTED/ -c
\end{verbatim}

This should run this example. If you use the default tests, you will get the following error message: 

\begin{verbatim}
  RTED: ../RtedTransformation.cpp:67: void RtedTransformation::
  transform(SgProject*, std::set<std::basic\_string<char, std::char\_
  traits<char>, std::allocator<char> >, std::less<std::basic\_string
  <char, std::char\_traits<char>, std::allocator<char> > >, 
  std::allocator<std::basic\_string<char, std::char\_traits<char>, 
  std::allocator<char> > > > ): Assertion roseCreateHeap failed.
\end{verbatim}

The problem is that the test file needs to include RuntimeSystem.h

After that we compile the transformed file:

\begin{verbatim}
  gcc -g -Wall -o assignment\_scope\_while.bin rose\_assignment\_
  scope\_while.c ../RuntimeSystem.cpp -isystem /home/panas2/programs/
  boost/install//include -I/home/panas2/rose/ROSE\_includes/ 
  -I../ /home/panas2/rose/build/projects/RTED/CppRuntimeSystem/.libs/
  libCppRuntimeSystem.so.0 -L/home/panas2/programs/qt/install/qt//lib 
  -L/home/panas2/programs/qt/install/qt/lib -L/usr/X11R6/lib 
  /home/panas2/programs/qt/install/qt/lib/libQt3Support.so 
  /home/panas2/programs/qt/install/qt/lib/libQtSql.so 
  /home/panas2/programs/qt/install/qt/lib/libQtGui.so 
  /home/panas2/programs/qt/install/qt/lib/libQtNetwork.so 
  /home/panas2/programs/qt/install/qt/lib/libQtXml.so 
  /home/panas2/programs/qt/install/qt/lib/libQtCore.so -lpthread 
  -lQtUiTools -lgcrypt  -Wl,--rpath -Wl,/home/panas2/rose/install/lib 
  -Wl,--rpath -Wl,/home/panas2/programs/qt/install/qt/lib -Wl,--rpath 
  -Wl,/home/panas2/rose/build/projects/RTED/CppRuntimeSystem/.libs/ 
  -lCppRuntimeSystem -L/home/panas2/rose/build/projects/RTED/
  CppRuntimeSystem/.libs/
\end{verbatim}

OR when creating a static library under ROSE for CppRuntimeSystem, we can do

\begin{verbatim}
  gcc -g -Wall -o assignment\_scope\_while.bin rose\_assignment\_
  scope\_while.c ../RuntimeSystem.cpp -isystem /home/panas2/programs/
  boost/install//include -I/home/panas2/rose/ROSE\_includes/ -I../ 
  -L/home/panas2/programs/qt/install/qt//lib -L/home/panas2/programs/qt/
  install/qt/lib -L/usr/X11R6/lib /home/panas2/programs/qt/install/
  qt/lib/libQt3Support.so /home/panas2/programs/qt/install/qt/lib/
  libQtSql.so /home/panas2/programs/qt/install/qt/lib/libQtGui.so 
  /home/panas2/programs/qt/install/qt/lib/libQtNetwork.so 
  /home/panas2/programs/qt/install/qt/lib/libQtXml.so 
  /home/panas2/programs/qt/install/qt/lib/libQtCore.so -lpthread 
  -lQtUiTools -lgcrypt  -Wl,--rpath -Wl,/home/panas2/rose/install/lib 
  -Wl,--rpath -Wl,/home/panas2/programs/qt/install/qt/lib -Wl,--rpath 
  -Wl,/home/panas2/workspace/CppRuntimeSystemLibarary/Debug 
  -lCppRuntimeSystemLibarary -L/home/panas2/workspace/
  CppRuntimeSystemLibarary/Debug
\end{verbatim}

To create a static library: Create new project in Eclipse. Select Shared Library. Thereafter import the source files, add the includes and the necessary libraries.

Includes:

\begin{verbatim}
  -I/home/panas2/programs/qt/install/qt/include/QtGui 
  -I/home/panas2/rose/ROSE-git/projects/RTED/CppRuntimeSystem/DebuggerQt 
  -I/home/panas2/rose/ROSE-git/projects/RTED/CppRuntimeSystem/ 
  -I/home/panas2/rose/ROSE-git/projects/RTED/ 
  -I/home/panas2/rose/ROSE-git/src/roseExtensions/qtWidgets/QCodeEditWidget 
  -I/home/panas2/programs/qt/install/qt/include/QtCore 
  -I/home/panas2/rose/build/projects/RTED/CppRuntimeSystem/DebuggerQt 
  -I/home/panas2/programs/qt/install/qt/include 
  -I/home/panas2/programs/qt/install/qt/include/Qt 
  -I/home/panas2/rose/ROSE-git/src/roseSupport 
  -I/home/panas2/rose/ROSE-git/src/midend/astProcessing/ 
  -I/home/panas2/rose/ROSE\_includes 
  -I/home/panas2/programs/boost/install/include/ 
  -I/home/panas2/rose/build/src/frontend/SageIII -I/home/panas2/rose/build
\end{verbatim}

Libraries:

\begin{verbatim}
 -L/home/panas2/rose/build/src/.libs/ -lrose
\end{verbatim}

Optionally create the CppRuntimeSystem as a separate Library. It is easy to set up as a new project and then include the .so into the RuntimeSystem project. One thing to remember is to add the CppRuntimeSystem.so into the LD\_LIBRARY\_PATH so that it is found when the project is run.


%}}}
%}}}
