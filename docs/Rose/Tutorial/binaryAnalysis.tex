\chapter{Binary Analysis}

% This tutorial has been replaced by a doxygen version.

See doxgyen ``Getting started with binary analysis'' chapter.

%% This chapter discusses the capabilities of ROSE to read, analyze and transform 
%% (transformations to the binary file format) binary executables.
%% 
%% Binary support in ROSE is currently based on a custom build \emph{ROSE Disassembler} (for ARM, x86, and PowerPC).
%% 
%% The following code reads in a binary and creates a binary ROSE AST:
%% 
%% {\mySmallFontSize
%% \begin{verbatim}
%%      SgProject* project = frontend(argc,argv);
%% \end{verbatim}
%% }
%% 
%% Similarily, one can unparse the AST to assembly using a call to the backend, cf. Figure ~\ref{Tutorial:examplesourcecode2}.
%% 
%% The best documentation for ROSE's binary analysis capabilities is
%% found in the doxygen-generated API reference manual. Start with the
%% Binary Analysis FAQ in that manual.

%\subsection{IdaPro-mysql}

%A binary processed by IdaPro needs to be processed into a MySQL database (DB). With that DB,
%the following lines create a binary ROSE AST:


%{\mySmallFontSize
%\begin{verbatim}
%  // create RoseBin object with DB specific information 
%  RoseBin* roseBin = new RoseBin(def_host_name,
%                                def_user_name,
%                                def_password,
%                                def_db_name);

%  RoseBin_Arch::arch=RoseBin_Arch::bit32;
%  RoseBin_OS::os_sys=RoseBin_OS::linux_op;
%  RoseBin_OS_VER::os_ver=RoseBin_OS_VER::linux_26;
%  RoseBin_Def::RoseAssemblyLanguage=x86;
%  // connect to the DB
%  roseBin->connect_DB(socket);
%  // query the DB to retrieve all data
%  SgAsmNode* globalBlock = roseBin->retrieve_DB_IDAPRO();
%  // close the DB
%  roseBin->close_DB();

%  // traverse the AST and test it
%  roseBin->test();
%\end{verbatim}
%}




%\section{The AST}

%The binary AST can now be unparsed to represent assembly instructions, cf. Figure ~\ref{Tutorial:examplesourcecode2}:

%{\mySmallFontSize
%\begin{verbatim}
%  SgAsmGenericFile* file = project->get_file(0).get_binaryFile();
%  RoseBin_unparse* unparser = new RoseBin_unparse();
%  unparser->init(file->get_global_block(), fileName);
%  unparser->unparse();
%\end{verbatim}
%}


%To visualize the binary AST, we can use the following lines to
%write the AST out to a .dot format:

%{\mySmallFontSize
%\begin{verbatim}
%  string filename=''_binary_tree.dot'';
%  AST_BIN_Traversal* trav = new AST_BIN_Traversal();
%  trav->run(file->get_global_block(), filename);
%\end{verbatim}
%}


%% \section{The ControlFlowGraph}
%% 
%% Based on a control flow traversal of the binary AST, a separate control flow graph
%% is created that can be used for further analyses.
%% 
%% TODO: Describe recent work on binary CFG.

%{\mySmallFontSize
%\begin{verbatim}
%  // forward or backward analysis ?
%  bool forward = true;
%  // when creating a visual representation, visualized edges?
%  bool edges = true;
%  // visualize multiple edges or merge edges between same nodes to one edge?
%  bool mergedEdges = false;
%  // create DotGraph Object or/and GmlGraph Object
%  RoseBin_DotGraph* dotGraph = new RoseBin_DotGraph();
%  RoseBin_GMLGraph* gmlGraph = new RoseBin_GMLGraph();
%  char* cfgFileNameDot = ``cfg.dot'';
%  char* cfgFileNameGml = ``gml.dot'';
%  RoseBin_ControlFlowAnalysis* cfganalysis = new RoseBin_ControlFlowAnalysis(
%                                             file->get_global_block(), forward, NULL, edges);
%  cfganalysis->run(dotGraph, cfgFileNameDot, mergedEdges);
%  cfganalysis->run(gmlGraph, cfgFileNameGml, mergedEdges);
%\end{verbatim}
%}

%The control flow graph of our example is represented in Figure ~\ref{Tutorial:examplefig1}.
%The graph shows different functions represented purple, containing various instructions.
%Instructions within a function are represented within the same box. Common instructions are
%represented yellow. Green instructions are jumps and pink instructions calls and returns.  
%Respectively, blue edges are call relationships and red edges return relationships.
%Black edges represent plain controlflow from one instruction to the next.



\section{DataFlow Analysis}

%\begin{figure}
%%\centerline{\epsfig{file=\TutorialExampleBuildDirectory/controlFlowGraph.ps,
%%                    height=1.3\linewidth,width=1.0\linewidth,angle=0}}
%\includegraphics[scale=0.7]{\TutorialExampleBuildDirectory/binaryAnalysis_pic2}
%\caption{Dataflowflow graph for example program.}
%\label{Tutorial:examplefig2}
%\end{figure}

Based on the control flow many forms of dataflow analysis may be performed.
Dataflow analyses available are:
 
\subsection{Def-Use Analysis}

Definition-Usage is one way to compute dataflow information about a binary program.

\subsection{Variable Analysis}

This analysis helps to detect different types within a binary.
Currently, we use this analysis to detect interrupt calls and their parameters together with the def-use analysis.
This allows us to track back the value of parameters to the calls, such as eax and therefore 
determine whether a interrupt call is for instance a write or read.
Another feature is the buffer overflow analysis. By traversing the CFG, we can detect buffer overflows.


\section{Dynamic Analysis}

   ROSE has support for dynamic analysis and for mixing of dynamic
and static analysis using the Intel Pin framework. This optional support in ROSE
requires a configure option ({\tt --with-IntelPin=$<path$>}).  The {\tt path} in
the configure option is the {\bf absolute path} to the top level directory of the location of
the Intel Pin distribution.  This support for Intel Pin has only been tested
on a 64bit Linux system using the most recent distribution of Intel Pin (version 2.6).

Note: The dwarf support in ROSE is currently incompatable with the dwarf support in
Intel Pin.  A message in the configuration of ROSE will detect if both support for
Dwarf and Intel Pin are both specified and exit with an error message that they
are incompatable options.

See {\tt tutorial/intelPin} directory for examples using static and dynamic analysis.
These example will be improved in the future, at the moment they just call the
generation of the binary AST.


Note: We have added a fix to Intel Pin pin.H file:
\begin{verbatim}
// DQ (3/9/2009): Avoid letting "using" declarations into header files.
#ifndef REQUIRE_PIN_TO_USE_NAMESPACES
using namespace LEVEL_PINCLIENT;
#endif
\end{verbatim}
so that the namespace of Intel Pin would not be a problem for ROSE.
The development team have suggested that they may fix their use of "using" 
declarations for namespaces in their header files.

Also note that the path must be absolute since it will be the
prefix for the {\bf pin} executable to be run in the internal tests and
anything else might be a problem if the path does not contain the 
current directory ({\bf "."}). Or, perhaps we should test for this
in the future.

Note 2: Linking to libdwarf.a is a special problem.  
Both ROSE and Intel Pin use libdwarf.a and both build shred libraries 
that link to the static version of the library (libdwarf.a).  This is 
a problem im building Pin Tools since both the PinTool and librose.so will use 
a statically linked dwarf library (internally).  This causes the first
use of dwarf to fail, because there are then two versions of the same 
library in place.  The solution is to force at most one static version 
of the library and let the other one be a shared library.

   Alternatively both the Pin tool and librose.so can be
built using the shared version of dwarf (libdwarf.so).
There is a makefile rule in libdwarf to build the shared
version of the library, but the default is to only build the
static library (libdwarf.a), so use {\tt make make libdwarf.so}
to build the shared library.  So we allow ROSE to link to
the libdwarf.a (statically), which is how ROSE has always
worked (this may be revisited in the future).  And we force 
the Pin tool to link using the shared dwarf library (libdwarf.so).
% but with rpath so that we don't require the user to modify the LD\_LIBRARY\_PATH.
{\em Note: The specification of the location of libdwarf.so in the Intel Pin 
directory structure is problematic using {\bf rpath}, so for the case of 
using the Intel Pin package with ROSE please set the {\bf LD\_LIBRARY\_PATH}
explicitly (a better solution using {\bf rpath} may be made available in 
the future).}


 
\section{Analysis and Transformations on Binaries}

   This section demonstrates how the binary can be analyized and
transformed via operations on the AST. In this tutorial example we will
recognize sequences of NOP (No OPeration) instructions (both single byte
and less common multi-byte NOPs) and replace them in the binary with as 
few multi-byte NOP instructions as required to overwrite the identified
NOP sequence (also called a {\em nop sled}).

In the following subsections we will demonstrate three example codes that
work together to demonstrate aspects of the binary analysis and transforamation using
ROSE.  All of these codes are located in the directory {\em tutorial/binaryAnalysis}
of the ROSE distribution. We show specifically:
\begin{enumerate}
   \item How to insert NOP sequences into binaries via source-to-source transformations.
         The tutorial example will demonstrate the trival insertion of random length
         nop sequences into all functions of an input source code application.
         The purpose of this is to really just generate arbitrary test codes
         upon which to use in sperately demonstrating (and testing) the binary analysis 
         support and the binary transformation support (next).  This example is
         shown in figure~\ref{Tutorial:exampleTransformSourceToInsertNOPs}.
         The file name is: 
   \item How to identify sequences of NOP instructions in the binary ({\em nop sleds}).
         The tutorial example show how to identify arbitrary NOP sequences in the 
         binary.  Note that our approach looks only for single instructions that have
         no side-effect operations and thus qualify as a NOP, and specifically does not
         look at collections of multiple statements that taken togehter have no
         side-effect and thus could be considered a NOP sequence. Our test on each 
         instruciton is isolated to the {\tt SageInterface::isNOP(SgAsmInstruction*)} 
         function.  Initially this function only detects NOP instructions that are
         single and multi-byte Intel x86 suggested NOP instructions.  The catagory of
         NOP instructions that have no side-effects is broader than this and will
         be addressed more completely in the future. This example is
         shown in figures \ref{Tutorial:exampleTransformBinaryToDetectNOPs_header},
         \ref{Tutorial:exampleTransformBinaryToDetectNOPs_source}, and
         \ref{Tutorial:exampleTransformBinaryToDetectNOPs_main}.
   \item How transformations on the binary executable are done to support rewriting
         all identified NOP sequences as Intel x86 suggested NOP instructions.
         Importantly, this tutorial example of an AST rewrite does not change the 
         size of any part of the final executable. This example is
         shown in figure~\ref{Tutorial:exampleTransformBinaryToInsertNOPs}.
\end{enumerate}

\begin{figure}[!h]
{\indent
{\mySmallFontSize

% Do this when processing latex to generate non-html (not using latex2html)
\begin{latexonly}
   \lstinputlisting{\TutorialExampleDirectory/binaryAnalysis/nopSourceTransform.C}
\end{latexonly}

% Do this when processing latex to build html (using latex2html)
\begin{htmlonly}
   \verbatiminput{\TutorialExampleDirectory/binaryAnalysis/nopSourceTransform.C}
\end{htmlonly}

% end of scope in font size
}
% End of scope in indentation
}
\caption{Source-to-source transformation to introduce NOP assemble instructions in the
    generated binary executable.}
\label{Tutorial:exampleTransformSourceToInsertNOPs}
\end{figure}


\subsection{Source-to-source transformations to introduce NOPs}

   This tutorial example (see figure~\ref{Tutorial:exampleTransformSourceToInsertNOPs})
is a source-to-source trnasformation that is used to
generate test codes for the binary NOP sequence detection example and the 
NOP sequence transformation example (the next two tutorial examples).
This tutorial uses C language {\em asm} statements that are inserted into
each function in the source code, the resulting generated source code (with {\em asm}
statements) is then compiled to generate inputs to use in our detection and
transformation of NOP sequences.  Although it is possible, the backend
compiler (at least the one we are using), does not optimize away these 
{\em asm} statements that represent NOP instructions.  In general all
C and C++ compilers turn off optimizations upon encountering the {\em asm}
language construct, so it is easy to use this approach to build binaries
from arbitrary source code that have a range of properties.  Seeding 
the source code with such properties causes the binary to have the similar
properties.  We include this example in the tutorial because it is a cute
example of how we can combine source code analsys with binary analysis
(in this case we are only supporting testing of the binary analysis).
Much more interesting example of the connection of source code and
binary analysis are available.


\begin{figure}[!h]
{\indent
{\mySmallFontSize

% Do this when processing latex to generate non-html (not using latex2html)
\begin{latexonly}
   \lstinputlisting{\TutorialExampleDirectory/binaryAnalysis/detectNopSequencesTraversal.h}
\end{latexonly}

% Do this when processing latex to build html (using latex2html)
\begin{htmlonly}
   \verbatiminput{\TutorialExampleDirectory/binaryAnalysis/detectNopSequencesTraversal.h}
\end{htmlonly}

% end of scope in font size
}
% End of scope in indentation
}
\caption{Header file for the traversal used to identify the NOP sequences in the binary executable.}
\label{Tutorial:exampleTransformBinaryToDetectNOPs_header}
\end{figure}


\begin{figure}[!h]
{\indent
{\mySmallFontSize

% Do this when processing latex to generate non-html (not using latex2html)
\begin{latexonly}
   \lstinputlisting{\TutorialExampleDirectory/binaryAnalysis/detectNopSequencesTraversal.C}
\end{latexonly}

% Do this when processing latex to build html (using latex2html)
\begin{htmlonly}
   \verbatiminput{\TutorialExampleDirectory/binaryAnalysis/detectNopSequencesTraversal.C}
\end{htmlonly}

% end of scope in font size
}
% End of scope in indentation
}
\caption{Example code to identify the NOP sequences in the binary executable.}
\label{Tutorial:exampleTransformBinaryToDetectNOPs_source}
\end{figure}








\begin{figure}[!h]
{\indent
{\mySmallFontSize

% Do this when processing latex to generate non-html (not using latex2html)
\begin{latexonly}
   \lstinputlisting{\TutorialExampleDirectory/binaryAnalysis/detectNopSequences.C}
\end{latexonly}

% Do this when processing latex to build html (using latex2html)
\begin{htmlonly}
   \verbatiminput{\TutorialExampleDirectory/binaryAnalysis/detectNopSequences.C}
\end{htmlonly}

% end of scope in font size
}
% End of scope in indentation
}
\caption{Main program using the traversal to identify the NOP sequences in the binary executable.}
\label{Tutorial:exampleTransformBinaryToDetectNOPs_main}
\end{figure}


\subsection{Detection of NOP sequences in the binary AST}

   The tutorial example (see figures \ref{Tutorial:exampleTransformBinaryToDetectNOPs_header},
\ref{Tutorial:exampleTransformBinaryToDetectNOPs_source}, and
\ref{Tutorial:exampleTransformBinaryToDetectNOPs_main})
shows the detection of NOP sequences and is separated into three
figures (the header file, the source code, and the main program).  The header file
and source file will be reused in the next tutorial example to demonstrate the
transformation of the NOP sequences that this tutorial identifies.  The
transformation will be done on the AST (and a new modified executable generated
by unparsing the AST). 

    Using a simple preorder traversal over the AST we identify independent sequences of
NOP instructions.  This traversal save the sequence so that it can be used in both the
tutorial example that detects the NOP sequences and also the tutorial example that 
will transform them to be more compact multi-byte NOP sequences that will use the
suggested Intel x86 multi-byte NOP instructions.

   In this example, the sequence of a NOP sequence is saved as the address of the 
starting NOP instruction and the number of instructions in the sequence.  The function
{\tt SageInterface::isNOP(SgAsmInstruction*)} is used to test of an instruction is
a NOP.  Presently this test only identifies single or multi-byte NOP instrucitons
that have been suggested for use as single and multi-byte NOP instructions by Intel.
Other instruction may semantically be equivalent to a NOP instruction and they are
not identified in this initial work.  Later work may include this capability and
for this purpose we have isolated out the {\tt SageInterface::isNOP(SgAsmInstruction*)}.
Note also that sequences of instructions may be semantically equivalent to a 
NOP sequence and we also do not attempt to identify these (separate interface 
functions in the {\tt SageInterface} namespace have been defined to support this
work; these functions are presently unimplemented.

\begin{figure}[!h]
{\indent
{
% \mySmallFontSize
\tiny

% Do this when processing latex to generate non-html (not using latex2html)
\begin{latexonly}
   \lstinputlisting{\TutorialExampleDirectory/binaryAnalysis/nopBinaryTransform.C}
\end{latexonly}

% Do this when processing latex to build html (using latex2html)
\begin{htmlonly}
   \verbatiminput{\TutorialExampleDirectory/binaryAnalysis/nopBinaryTransform.C}
\end{htmlonly}

% end of scope in font size
}
% End of scope in indentation
}
\caption{Example code showing the transformation of the binary executable.}
\label{Tutorial:exampleTransformBinaryToInsertNOPs}
\end{figure}

\subsection{Transformations on the NOP sequences in the binary AST}

   Using the results from the previously presented traversal to identify
NOP sequences, we demonstrate the transformation to change these 
locations of the binary (in the AST) and then regenerate the executable 
to have a different representation. The new representation is a more clear
(obvious within manual binary analysis) and likely more compressed 
representation using only the suggested Intel x86 multi-byte NOP.



\subsection{Conclusions}

\fixme{Not sure if the resulting executable will use the assembler on each
instruction in the generated binary from the unparsing within ROSE. This might 
require some more work. So at the moment the AST is transformed and I have
to look into if the binary sees the effect of the transformation in the AST.}

   The identification and transformation have been deomonstrated on the AST
and can be expressed in the binary binary by calling the {\em backend()} 
function in ROSE; which will regenerate the executable.  The capability to
regenerate the executable will not always result in a properly formed
executable that can be executed, this depends upon the transformations done
and does ROSE does not have specialized support for this beyond regnerating
the binary executable from the AST.  In the case of this NOP transformation
we have not changed the size of any part of the binary so any relative offsets 
need not be fixed up. Assumeing we have not accedentaly interpreted data that
had values that matched parts of the opcodes that were transformed, then resulting
binary executable should execute without problems.  This transformation however
makes clear how critical it is that data not be interpreted as instructions
(which could randomly be interpreted as NOPs in the case of this tutorial example).

