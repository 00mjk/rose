\chapter{Tailoring The Code Generation Format}

   Figure~\ref{Tutorial:example_codeGenerationFormatControl} shows an
example of how to use the mechanisms in ROSE to tailor the format and
style of the generated code.  This chapter presents an example translator 
that modifies the formatting of the code that is generated within ROSE.

The details of functionality are hidden from the user and a high level 
interface is provided that permits key parameters to be specified.
This example will be made more sophisticated later, for now it just
 modifies the indentation of nested code blocks (from 2 spaces/block 
to 5 spaces/block).


\section{Source Code for Example that Tailors the Code Generation}

    Figure~\ref{Tutorial:example_codeGenerationFormatControl}
shows an example translator which calls the inliner mechanism.
The code is designed to only inline up to ten functions.
the list of function calls is recomputed after any function call
is successfully inlined. 

The input code is shown in figure~\ref{Tutorial:exampleInputCode_codeGenerationFormatControl},
the output of this code is shown in 
figure~\ref{Tutorial:exampleOutput_codeGenerationFormatControl}.

\begin{figure}[!h]
{\indent
{\mySmallFontSize

% Do this when processing latex to generate non-html (not using latex2html)
\begin{latexonly}
   \lstinputlisting{\TutorialExampleDirectory/codeGenerationFormatControl.C}
\end{latexonly}

% Do this when processing latex to build html (using latex2html)
\begin{htmlonly}
   \verbatiminput{\TutorialExampleDirectory/codeGenerationFormatControl.C}
\end{htmlonly}

% end of scope in font size
}
% End of scope in indentation
}
\caption{Example source code showing how to tailor the code generation format. }
\label{Tutorial:example_codeGenerationFormatControl}
\end{figure}


\section{Input to Demonstrate Tailoring the Code Generation}

   Figure~\ref{Tutorial:exampleInputCode_codeGenerationFormatControl}
shows the example input used for demonstration of how to control the formatting 
of generated code.

\begin{figure}[!h]
{\indent
{\mySmallFontSize

% Do this when processing latex to generate non-html (not using latex2html)
\begin{latexonly}
   \lstinputlisting{\TutorialExampleDirectory/inputCode_codeGenerationFormatControl.C}
\end{latexonly}

% Do this when processing latex to build html (using latex2html)
\begin{htmlonly}
   \verbatiminput{\TutorialExampleDirectory/inputCode_codeGenerationFormatControl.C}
\end{htmlonly}

% end of scope in font size
}
% End of scope in indentation
}
\caption{Example source code used as input to program to the tailor the code generation.}
\label{Tutorial:exampleInputCode_codeGenerationFormatControl}
\end{figure}





\section{Final Code After Tailoring the Code Generation}

   Figure~\ref{Tutorial:exampleOutput_codeGenerationFormatControl} 
shows the results from changes to the formatting of generated code.


\begin{figure}[!h]
{\indent
{\mySmallFontSize

% Do this when processing latex to generate non-html (not using latex2html)
\begin{latexonly}
   \lstinputlisting{\TutorialExampleBuildDirectory/rose_inputCode_codeGenerationFormatControl.C}
\end{latexonly}

% Do this when processing latex to build html (using latex2html)
\begin{htmlonly}
   \verbatiminput{\TutorialExampleBuildDirectory/rose_inputCode_codeGenerationFormatControl.C}
\end{htmlonly}

% end of scope in font size
}
% End of scope in indentation
}
\caption{Output of input code after changing the format of the generated code.}
\label{Tutorial:exampleOutput_codeGenerationFormatControl}
\end{figure}



























