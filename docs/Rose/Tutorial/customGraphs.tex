\chapter{Building Custom Graphs}
\label{Tutorial:chapterCustomGraphs}

\paragraph{What To Learn From This Example}
This example shows how to generate custom graphs using \textit{SgGraph} class.

Rose provides a collection type \textit{SgGraph} to store a graph. Two specific graphs
are also provided which are derived from \textit{SgGraph}: \textit{SgIncidenceDirectedGraph} and
\textit{SgIncidenceUndirectedGraph}.

Nodes and edges in a \textit{SgGraph} are represented by \textit{SgGraphNode} and \textit{SgGraphEdge} separately. 
A \textit{SgGraph} is built by adding \textit{SgGraphNode}s and \textit{SgGraphEdge}s using its member function
\textit{addNode} and \textit{addEdge}. You can get all nodes and edges of a \textit{SgGraph} by calling its functions
\textit{computeNodeSet} and \textit{computeEdgeSet} separately.
More interfaces of \textit{SgGraph} and its subclasses can be found in doxygen of Rose.

Since \textit{SgGraph} is for Rose use, each node in it holds a pointer to \textit{SgNode}, which is 
the default attribute of a \textit{SgGraphNode}. If you want to add more attributes inside, you
can use \textit{SgGraphNode}'s member function \textit{addNewAttribute} by providing a name and an \textit{AstAttribute}
object to add a new attribute to a node. Normally, you have to build your own 
attribute class which should be derived from class \textit{AstAttribute}. Three attribute
classes are provided by Rose: \textit{AstRegExAttribute}, \textit{AstTextAttribute}, and \textit{MetricAttribute}. 
For more information about them, please refer to Rose's doxygen. 


