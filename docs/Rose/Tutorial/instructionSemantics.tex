\chapter{Instruction Semantics}

The Instruction Semantics layer in ROSE can be used to ``evaluate''
instructions and is controlled by a policy that defines the details of
what ``evaluate'' means.  For instance, given the following ``xor''
instruction, the X86InstructionSemantics class specifies that the
value of the ``eax'' and ``edx'' registers are read, those two 32-bit
values are xor'd together, and the 32-bit result is then written to
the ``eax'' register. The policy defines what a 32-bit value is (it
could be an integer, some representation of a constant, etc), how it
is read and written to the registers, and how to compute an xor.

\begin{verbatim}
xor eax, edx
\end{verbatim}

ROSE has a collection instruction semantic classes, one for each
architecture. It also has a small collection of policies.  This
chapter briefly describes a policy that tracks constant values.

\section{The FindConstantsPolicy Class}

The FindConstantsPolicy is used to track constant values across an
instruction trace.  The basic idea is that ROSE ``executes'' the
instructions one at a time in the instruction semantics layer,
identifies constants, performs operations on those constants, and
assigns constants to registers and memory locations.  Each constant
also maintains information about which instructions led to that
particular constant's existence.

A ``constant'' is an abstract datum that has a known integer value, or
a name corresponding to some unknown value, or a name and a known
integer offset. Names take the form of the letter ``v'' (for
``value'') followed by a unique integer.  Known values are represented
as signed hexadecimal values in the output.

The findConstants.C program in the tests/roseTests/binaryTests
directory (which is described herein) takes each function and
processes the instructions of that function in address order. It makes
no attempt to follow branches or any other kind of control flow, but
serves to demonstrate a simple way to track constants.

\begin{verbatim}
 1  #define \_\_STDC_FORMAT_MACROS
 2  #include "rose.h"
 3  #include "findConstants.h"
 4  #include <inttypes.h>
 5
 6  /* Returns the function name if known, or the address as a string otherwise. */
 7  static std::string
 8  name\_or\_addr(const SgAsmFunction *f)
 9  {
10      if (f->get\_name()!="")
11          return f->get\_name();
12  
13      char buf[128];
14      SgAsmBlock *first\_bb = isSgAsmBlock(f->get\_statementList().front());
15      sprintf(buf, "0x%"PRIx64, first\_bb->get\_id());
16      return buf;
17  }
18
19  class AnalyzeFunctions : public SgSimpleProcessing {
20    public:
21      AnalyzeFunctions(SgProject *project) {
22          traverse(project, postorder);
23      }
24      void visit(SgNode *node) {
25          SgAsmFunction *func = isSgAsmFunction(node);
26          if (func) {
27              std::cout <<"==============================================\n"
28                        <<"Constant propagation in function \"" <<name\_or\_addr(func) <<"\"\n"
29                        <<"==============================================\n";
30              FindConstantsPolicy policy;
31              X86InstructionSemantics<FindConstantsPolicy, XVariablePtr> t(policy);
32              std::vector<SgNode*> instructions = NodeQuery::querySubTree(func, V\_SgAsmx86Instruction);
33              for (size\_t i=0; i<instructions.size(); i++) {
34                  SgAsmx86Instruction *insn = isSgAsmx86Instruction(instructions[i]);
35                  ROSE\_ASSERT(insn);
36                  t.processInstruction(insn);
37                  RegisterSet rset = policy.currentRset;
38                  std::cout <<unparseInstructionWithAddress(insn) <<"\n"
39                            <<rset;
40              }
41          }
42      }
43  };
44
45  int main(int argc, char *argv[]) {
46      AnalyzeFunctions(frontend(argc, argv));
47  }
\end{verbatim}

Lines 30 through 40 are the main meat of the example. For each
function, we construct a fresh policy. Since the policy holds the
values of registers and memory, this resets them all to an initial
state having unknown values.  The instruction semantics engine depends
on the policy, so we also create a new one for each function.

Then we loop over the instructions of the function in order of their
addresses at lines 33 through 40.  Each instruction is processed in
turn by the X86InstructionSemantics object, adjusting the state of the
associated policy.

Finally, the assembly language instruction is output followed by the
values contained in the registers as a result of processing the
instruction.

\section{Sample Output}

Here's some abbreviated output from running the ``findConstants'' test
on a binary executable:

\begin{verbatim}
 1  ==============================================
 2  Constant propagation in function "_init"
 3  ==============================================
 4  0x80482c8:push   ebp
 5      ax = v62
 6      cx = v63
 7      dx = v64
 8      bx = v65
 9      sp = v66-0x4 [from 0x80482c8:push   ebp]
10      bp = v67
11      si = v68
12      di = v69
13      es = v70
14      cs = v71
15      ss = v72
16      ds = v73
17      fs = v74
18      gs = v75
19      cf = v76
20      ?1 = v77
21      pf = v78
22      ?3 = v79
23      af = v80
24      ?5 = v81
25      zf = v82
26      sf = v83
27      tf = v84
28      if = v85
29      df = v86
30      of = v87
31      iopl0 = v88
32      iopl1 = v89
33      nt = v90
34      ?15 = v91
35      memory = {
36          size=4; addr=v66-0x4 [from 0x80482c8:push   ebp]; value=v67
37      }
\end{verbatim}

Line 4 indicates that the instruction ``push ebp'' is located at
address 0x80482c8 and the following lines show the contents of
registers and known memory addresses following execution of the
``push.''  One can readily see that each register has a unique
constant of unknown value by virtue of each constant having a unique
name. The stack pointer register (sp) has the constant ``v66-0x4''
obtained from this very instruction. If we had printed the registers
prior to executing the ``push'' we would have seen that the original
sp constant was ``v66''. Therefore, this ``push'' instruction reduced
the value of sp by four.

Line 36 indicates that the four bytes beginning at memory address
``v66-0x4'' (which happens to be the constant stored in the stack pointer
register at line 9) contain the value ``v67'' (which is the constant
stored in the bp register at line 10).

Therefore, it can be determined that ``push ebp'' decrements the stack
pointer by four bytes, then copies a 32-bit value from the bp register
to the memory pointed to by the new stack pointer.

\begin{verbatim}
 1  0x80482c9:mov    ebp, esp
 2      ax = v62
 3      cx = v63
 4      dx = v64
 5      bx = v65
 6      sp = v66-0x4 [from 0x80482c8:push   ebp]
 7      bp = v66-0x4 [from 0x80482c8:push   ebp]
 8      si = v68
 9      di = v69
10      es = v70
11      cs = v71
12      ss = v72
13      ds = v73
14      fs = v74
15      gs = v75
16      cf = v76
17      ?1 = v77
18      pf = v78
19      ?3 = v79
20      af = v80
21      ?5 = v81
22      zf = v82
23      sf = v83
24      tf = v84
25      if = v85
26      df = v86
27      of = v87
28      iopl0 = v88
29      iopl1 = v89
30      nt = v90
31      ?15 = v91
32      memory = {
33          size=4; addr=v66-0x4 [from 0x80482c8:push   ebp]; value=v67
34      }
\end{verbatim}

The output after the ``mov ebp, esp'' instruction at address
0x80482c9 subsequent to the ``push ebp'' that we just saw, shows that
the new stack pointer has been copied into the ``bp'' register and
that nothing else has changed.  A more interesting instruction
follows...

\begin{verbatim}
 1  0x80482cb:sub    esp, 0x8
 2      ax = v62
 3      cx = v63
 4      dx = v64
 5      bx = v65
 6      sp = v66-0xc [from 0x80482cb:sub    esp, 0x8]
 7      bp = v66-0x4 [from 0x80482c8:push   ebp]
 8      si = v68
 9      di = v69
10      es = v70
11      cs = v71
12      ss = v72
13      ds = v73
14      fs = v74
15      gs = v75
16      cf = -v193-0x1 [from 0x80482cb:sub    esp, 0x8]
17      ?1 = v77
18      pf = -v187-0x1 [from 0x80482cb:sub    esp, 0x8]
19      ?3 = v79
20      af = -v191-0x1 [from 0x80482cb:sub    esp, 0x8]
21      ?5 = v81
22      zf = v190 [from 0x80482cb:sub    esp, 0x8]
23      sf = v189 [from 0x80482cb:sub    esp, 0x8]
24      tf = v84
25      if = v85
26      df = v86
27      of = v197 [from 0x80482cb:sub    esp, 0x8]
28      iopl0 = v88
29      iopl1 = v89
30      nt = v90
31      ?15 = v91
32      memory = {
33          size=4; addr=v66-0x4 [from 0x80482c8:push   ebp]; value=v67
34      }
\end{verbatim}

The ``sub esp, 0x8'' subtracts eight from the value of the stack
pointer register and then stores the result in the stack pointer
register. This can be seen by the fact that the constant stored in the
``sp'' register has changed from ``v66-0x4'' to ``v66-0xc.'' One can
also see that various flags have been modified, although we don't know
the values of any of them.

\begin{verbatim}
 1  0x80482ce:call   0x8048364
 2      ax = v62
 3      cx = v63
 4      dx = v64
 5      bx = v65
 6      sp = v66-0x10 [from 0x80482ce:call   0x8048364]
 7      bp = v66-0x4 [from 0x80482c8:push   ebp]
 8      si = v68
 9      di = v69
10      es = v70
11      cs = v71
12      ss = v72
13      ds = v73
14      fs = v74
15      gs = v75
16      cf = -v193-0x1 [from 0x80482cb:sub    esp, 0x8]
17      ?1 = v77
18      pf = -v187-0x1 [from 0x80482cb:sub    esp, 0x8]
19      ?3 = v79
20      af = -v191-0x1 [from 0x80482cb:sub    esp, 0x8]
21      ?5 = v81
22      zf = v190 [from 0x80482cb:sub    esp, 0x8]
23      sf = v189 [from 0x80482cb:sub    esp, 0x8]
24      tf = v84
25      if = v85
26      df = v86
27      of = v197 [from 0x80482cb:sub    esp, 0x8]
28      iopl0 = v88
29      iopl1 = v89
30      nt = v90
31      ?15 = v91
32      memory = {
33          size=4; addr=v66-0x10 [from 0x80482ce:call   0x8048364]; value=0x80482d3 [from 0x80482ce:call   0x8048364]
34          size=4; addr=v66-0x4 [from 0x80482c8:push   ebp]; value=v67
35      }
\end{verbatim}

Now we get to our first branch-type instruction, a ``call'' to a
particular address.  Instruction semantics describe what the call
instruction does to the registers and memory, but does not actually
execute a call or process the called instructions. That means that a
``ret'' was not processed and thus we should see the return address
sitting on the stack. In fact, we do: the stack pointer has been
decremented by another four bytes and the memory address to which the
stack pointer points contains the address of the instruction
immediately after the ``call''.

\section{Building on Instruction Semantics}

The X86InstructionSemantics class and the policy classes can be
extended to handle special cases. For instance, the
X86InstructionSemantics class processes the ``rep stosd'' instruction
in such a way that only one iteration of the ``stosd'' is
considered. Sometimes it's more useful to process the entire repeated
sequence in one step rather than iterating through the
loop. Subclassing X86InstructionSemantics to override individual
instructions or classes of instructions is simple.  The subclass
should redefine the ``translate'' method to do whatever is necessary
for certain instructions while delegating to the superclass for all
remaining instructions. For example:

\begin{verbatim}
/* Augments super::translate() to override rep\_stos instructions */
virtual void translate(SgAsmx86Instruction *insn) {
    switch (insn->get\_kind()) {
        case x86\_rep\_stosb: updateIP(insn); rep\_stos\_semantics<1>(insn); break;
        case x86\_rep\_stosw: updateIP(insn); rep\_stos\_semantics<2>(insn); break;
        case x86\_rep\_stosd: updateIP(insn); rep\_stos\_semantics<4>(insn); break;
        default: super::translate(insn); break;
    }
}
\end{verbatim}

It's also possible to subclass the policies. For instance, if you need
to do something special for binary AND operations on the stack
pointer you could override the ``and\_'' method in the policy.
