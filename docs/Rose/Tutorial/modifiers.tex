\chapter{Type and Declaration Modifiers}

   Most languages support the general concept of modifiers to
types, declarations, etc.  The keyword {\em volatile} for
example is a modifier to the type where it is used in a 
declaration.  Searching for the modifiers for types and 
declarations,however, can be confusing.  They are often not
where one would expect, and most often because of corner
cases in the language that force them to be handled in specific 
ways.

   This example tutorial code is a demonstration of a how to access the
{\em volatile} modifier used in the declaration of types for variables.
We demonstrate that the modifier is not present in the SgVariableDeclaration
or the SgVariableDefinitoon, but is located in the SgModifierType used
to wrap the type returned from the SgInitializedName (the variable in the
variable declaration).

\section{Input For Example Showing use of {\em Volatile} type modifier}

   Figure~\ref{Tutorial:exampleInputCode_volatileTypeModifier}
shows the example input used for demonstration of test for the {\em volatile} 
type modifier.

\begin{figure}[!h]
{\indent
{\mySmallFontSize

% Do this when processing latex to generate non-html (not using latex2html)
\begin{latexonly}
   \lstinputlisting{\TutorialExampleDirectory/inputCode_VolatileTypeModifier.C}
\end{latexonly}

% Do this when processing latex to build html (using latex2html)
\begin{htmlonly}
   \verbatiminput{\TutorialExampleDirectory/inputCode_VolatileTypeModifier.C}
\end{htmlonly}

% end of scope in font size
}
% End of scope in indentation
}
\caption{Example source code used as input to program in codes used in this chapter.}
\label{Tutorial:exampleInputCode_volatileTypeModifier}
\end{figure}


\section{Generating the code representing the seeded bug}

    Figure~\ref{Tutorial:example_volatileTypeModifier}
shows a code that traverses each IR node and for and
SgInitializedName IR node checks its type.
The input code is shown in figure \ref{Tutorial:exampleInputCode_volatileTypeModifier},
the output of this code is shown in 
figure~\ref{Tutorial:exampleOutput_volatileTypeModifier}.


\begin{figure}[!h]
{\indent
{
%\mySmallFontSize
\mySmallestFontSize

% Do this when processing latex to generate non-html (not using latex2html)
\begin{latexonly}
   \lstinputlisting{\TutorialExampleDirectory/volatileTypeModifier.C}
\end{latexonly}

% Do this when processing latex to build html (using latex2html)
\begin{htmlonly}
   \verbatiminput{\TutorialExampleDirectory/volatileTypeModifier.C}
\end{htmlonly}

% end of scope in font size
}
% End of scope in indentation
}
\caption{Example source code showing how to detect {\em volatile} modifier. }
\label{Tutorial:example_volatileTypeModifier}
\end{figure}


\begin{figure}[!h]
{\indent
{\mySmallFontSize

% Do this when processing latex to generate non-html (not using latex2html)
\begin{latexonly}
   \lstinputlisting{\TutorialExampleBuildDirectory/rose_inputCode_VolatileTypeModifier.C}
\end{latexonly}

% Do this when processing latex to build html (using latex2html)
\begin{htmlonly}
   \verbatiminput{\TutorialExampleBuildDirectory/rose_inputCode_VolatileTypeModifier.C}
\end{htmlonly}

% end of scope in font size
}
% End of scope in indentation
}
\caption{Output of input code using volatileTypeModifier.C}
\label{Tutorial:exampleOutput_volatileTypeModifier}
\end{figure}

