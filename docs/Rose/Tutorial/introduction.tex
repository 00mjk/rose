\chapter{Introduction}

% Include common text used in both the manual and the tutorial
\section{What is ROSE}

% Describe ROSE to the broad target audience.
% Describe how it works 
%    Read the input source code
%    Generate the AST (define AST and define graph) 
%       The structure of the original source code is preserved
%       Details aboaut the source code can be found in the AST
%    Provide tools 
%       for the analysis and transformation of the AST
%       generate additional information from the AST (call graph, etc.)
%       visualization of the program
%    Source code regeneration
%
   ROSE is an open source compiler infrastructure for building tools that can
read and write source code in multiple languages (C/C++/Fortran) and/or analyze
binary executables (using the x86, Power-PC, and ARM instruction sets).
The target audience for ROSE is people building tools for the analysis
and transformation of software generally as well as code generation tools.
% source code and/or analysis of binary executables.
ROSE provides a library (librose) that can be 
used to support the universal requirements of tools that
do custom analysis and/or transformations on source code
and custom analysis of binary forms of software. ROSE is 
portable across and expanding range of operating systems
and work with an growing number of compilers.

   ROSE provides a common level of infrastructure support to
user-defined tools, so that they need not implement the 
complex support required for software analysis and transformation operations. 
For source code based tools these include parsing, common forms of 
compiler analysis, common transformations, and code generation.
For binary analysis based tools these include disassembly,
function boundary detection, and common forms of analysis.
User defined tools may also mix source code and binary analysis
to form more interesting tools for specialized purposes.
ROSE is part of research work to unify the analysis of
both source code and binaries within general compiler research
and define mixed forms of static and dynamic analysis.

   ROSE works by reading the source code and/or binary
and generating an Abstract Syntax Tree (AST).  The AST forms
a graph representing the structure of the source code and/or binary
executable and is held in memory to provide the fastest possible means 
of operating on the graph.  The nodes used to define the AST graph are 
an {\em intermediate representation} (IR); common within compiler research
as a way of representing the structure of software absent syntax details
(commas, semi-colons, white-space, etc.).  ROSE provides mechanisms to 
traverse and manipulate the AST. Finally, in the case of source code, 
ROSE provides mechanisms to regenerate source code from the AST.

   As a trivial example, if the input source code program contains a variable declaration
for an integer, all of this information will be available in the AST generated from
the input code passed on the command line to any tool built using ROSE.
Similarly, an automated transformation of the variable declaration 
held in the AST would be expressed using a traversal over the AST and
code {\em semantic actions} to mutate the AST. Then the transformed source code would be 
generated ({\em unparsed}) from the AST.  In the case of binaries 
(including executables, object files, and libraries), the AST will 
represent the structure of the binary. The AST for a binary includes the 
binary file format (symbol table, debug format, import tables, etc.), 
disassembled instructions, all instruction operands, etc.

   ROSE provides a rich set of tools to support the analysis of 
software including the support for users to build their own forms
of analysis and specialized transformations. As an example, ROSE includes
a full OpenMP compiler built using the internal ROSE infrastructure 
for analysis and transformation.
A wide assortment of AST traversals are provided
to express both analysis and transformations of the AST. A set of
common forms of analysis are provided (call graph, control flow, etc.)
most work uniformly on both source code and binary executables.
Visualization support in included to help users understand and debug
their tools.  GUI support is available to support building professional
level tools using ROSE. ROSE is actively supported by a small
group at LLNL and is used as a basis for compiler research work within 
DOE at LLNL.

   Technically, ROSE is designed to build what are called {\em translators}, 
ROSE uses a source-to-source approach to define such translators. Note that 
translators are significantly more sophisticated than {\em preprocessors} 
but the terms are frequently confused. A translator must understand the 
source code at fundamentally deeper level using a grammar for the whole language
and on the whole source code, where as a preprocessor only understands the 
source code using a simpler grammar and on a subset of the source code. 
It is {\em loosely} the difference between any language compiler and the 
C preprocessor (cpp).

% Sections suggested by Carol Eidt (at Microsoft)
\section{Why you should be interested in ROSE}
ROSE is a tool for building source-to-source translators.
You should be interested in ROSE if you want to 
understand or improve any aspect of your software. ROSE
makes it easy to build tools that read and operate on source code
from large scale applications (millions of lines).  Whole projects
may be analyzed and even optimized using tools built using ROSE.
For example, ROSE is itself analyzed nightly using ROSE.

To get started immediately consult the ROSE User Manual, chapter
{\em Getting Started} for details).

\section{Problems that ROSE can address}
    ROSE is a mechanism to build source-to-source analysis or 
optimization tools that operate directly on the source code of large 
scale applications.  Example tools that {\em have} been built include:
\begin{itemize}
   \item OpenMP translator,
   \item Array class abstraction optimizer,
   \item Source-to-source instrumenter,
   \item Loop analyzer,
   \item Symbolic complexity analyzer,
   \item Inliner and outliner,
   \item Code coverage tools,
   \item and many more...
\end{itemize}
Example tools that {\em can} be built include:
\begin{itemize}
   \item Custom optimization tools,
   \item Custom documentation generators,
   \item Custom analysis tools,
   \item Code pattern recognition tools,
   \item Security analysis tools,
   \item and many more...
\end{itemize}


\section{Examples in this ROSE Tutorial}

    This tutorial lays out a set of progressively
complex example programs (located in 
{\tt <ROSE\_SOURCE>/tutorial/*}) that serve as a tutorial for the use of ROSE.
Translators built using ROSE can either just analyze (and output results)
or compile the input programs just like a compiler (generating object files 
or executables).  Many of the examples in this tutorial just do simple analysis
of the input source code, and a few show the full compilation of the input source code.
Where the translators generate either object files of executables, the 
vendor's compiler is used to compile the final ROSE-generated code.  Within
ROSE, the call to generate source code from the AST and call the vendor's compiler
is referred to as the {\em backend processing}.  The specification of the vendor's compiler
as a backend is done within the configuration step within ROSE (see options for 
{\tt configure} in the ROSE User Manual).

Within the example programs below, the user can provide alternative input programs
for more complex evaluation of the tutorial examples and ROSE.  The end of the chapter,
section~\ref{TutorialMakefiles}, shows the makefiles required to compile the
tutorial programs using an installed version of ROSE (compiled using 
{\tt make install}).  This {\tt example\_makefile} is run as part of the testing using the 
{\tt make installcheck} rule.

% Each example is described and can be alternatively assigned as an exercise for the reader.

Chapters are organized in topics including simple ROSE AST visualization,
dealing with complex data types, program analysis, program transformation
and optimization, correctness checking, binary support, interacting with
other tools, and parallelism. 
We hope readers can easily find the information they want. 
%Nearly all the chapters in this tutorial introduce a different tutorial example. 

Specific chapters in this tutorial include: \fixme{We should constantly
update this}

\begin{itemize}
%-----------------------------
\item Introduction 
%-----------------------------

\begin{enumerate}
     \item Introduction {\em (this chapter)}
     \item Problems that ROSE can address
     \item Getting Started \\
           This chapter covers where to find ROSE documentation and how to install ROSE.

     \item Example Makefiles demonstrating the command lines to compile and link the 
           example translators in this tutorial are found in 
           {\tt <ROSE\_Compile\_Tree>/tutorial/exampleMakefile}.

\end{enumerate}

%-----------------------------
\item Working with the ROSE AST:
%-----------------------------
\begin{enumerate}
     \item Identity Translator \\
           This example translator reads a C or C++ application, builds the AST
           internally, generates source code from the AST (unparsing), and calls
           the backend vendor compiler to compile the generated C or C++ application 
           code.  Thus the translator acts like and can be used to replace any compiler 
           since it takes in source code and outputs an object code (or executable).
           This example also shows that the output generated from and ROSE translator is
           a close reproduction of the input; preserving all comments, preprocessor
           control structure, and most formating.

     \item Scopes of Declarations (scopeInformation.C) \\
           This example shows the scopes represented by different IR nodes in the AST.


     \item AST Graph Generator \\
           This translator reads a C or C++ application code and builds the AST, internally.
           The translator does not regenerate code from the AST and so does not
           call the backend vendor's compiler. This shows how simple it could be
           to build source code analysis tools; the code calls an internal ROSE function
           to generate a dot graph of the AST, the makefile has the details of converting
           the dot graph into a postscript file (also shown).

     \item AST PDF Generator \\
           This translator reads an C or C++ application code builds the AST internally.
           The translator does not regenerate code from the AST and so does not
           call the backend vendor's compiler. This shows how simple it could be
           to build source code analysis tools, the code calls an internal ROSE function
           to generate a {\tt pdf} file with bookmarks representing the AST.  The 
           pdf file show as output is in this case a previously generated figure of a
           screen shot obtained by viewing the output pdf file using {\tt acroread}.


     \item Introduction to AST Traversals and Attributes \\
           This collection of examples show the use of the simple visitor pattern for 
           the traversal of the AST within ROSE.  The simple visitor pattern permits
           operations to be programmed which will be invoked on different nodes 
           within the AST.  To handle communication of context information down into the
           AST and permit communication of analysis information up the AST, we have 
           provided inherited and synthesized attributes (respectively). Note that
           an AST is most often represented as a tree with extra edges and with shared IR
           nodes that make the full graph (representing all edges) not a tree.  We present
           two styles of traversal, one over the tree representing the AST (which
           excludes some types of IR nodes) and one over the full AST with all extra
           nodes and shared nodes.  Extra nodes are nodes such as SgType and SgSymbol 
           IR nodes.


     \begin{enumerate}
        \item AST traversals \\
              These traversals visit each node of the tree embedded within the AST
              (excluding shared SgType and SgSymbol IR nodes).  These traversals
              visit the IR nodes is an order dependent upon the structure of the AST
              (the source code from which the AST is built).
        \begin{enumerate}
          \item Classic Object-Oriented Visitor Patterns \\
                This example, classicObjectOrientedVisitorPatternMemoryPoolTraversal.C,
                show the use of a classic visitor patterns.  At the moment this example
                uses the AST's memory pools as a basis but it is identical to a future
                traversal. The ROSE visitor Pattern (below) is generally more useful. 
                The classic visitor pattern traversals are provided for completeness.
          \item Visitor Traversal (visitorTraversal.C) \\
                Conventional visitor patterns without no attributes. This pattern can 
                explicitly access global variables to provide the effect of accumulator
                attributes (using static data members we later show the handling of
                accumulator attributes).
          \item Inherited Attributes (inheritedAttributeTraversal.C) \\
                Inherited attributes are used to communicate the context of any location
                within the AST in terms of other parent AST nodes.
          \item Synthesized Attributes (synthesizedAttributeTraversal.C) \\
                Synthesized attributes are used to pass analysis results from the leaves
                of the AST to the parents (all the way to the root of the AST if
                required).
          \item Accumulator Attributes (accumulatorAttributeTraversal.C) \\
                Accumulator attributes permit the interaction of data within inherited
                attributes with data in synthesized attributes. In our example program we
                will show the use of accumulator attributes implemented as static data
                members.  Accumulator attributes are a fancy name for what is essentially
                global variables (or equivalently a data structure passed by reference 
                to all the IR nodes in the AST).
          \item Inherited and Synthesized Attributes (inheritedAndSynthesizedAttributeTraversal.C) \\
                The combination of using inherited and synthesized attributes permits 
                more complex analysis and is often required to compute analysis results on
                the AST within a specific context (e.g. number of loop nests of specific depth).
          \item Persistent Attributes (persistantAttributes.C) \\
                Persistent attributes may be added the AST for access to stored results
                for later traversals of the AST.  The user controls the lifetime of these
                persistent attributes.
          \item Nested traversals \\
                Complex operations upon the AST can require many subordinate operations.
                Such subordinate operations can be accommodated using nested traversals.
                All traversals can operate on any subtree of the AST, and may even be 
                nested arbitrarily.  Interestingly, ROSE traversals may also be
                applied recursively (though care should be take using recursive traversals
                using accumulator attributes to avoid {\em over} accumulation).
        \end{enumerate}
        \item Memory Pool traversals \\
              These traversals visit all IR nodes (including shared IR nodes such as
              SgTypes and SgSymbols).  By design this traversal can visit ALL IR nodes
              without the worry of getting into cycles.  These traversals are mostly 
              useful for building specialized tools that operate on the AST.    
        \begin{enumerate}
          \item Visit Traversal on Memory Pools \\
              This is a similar traversal as to the Visitor Traversal over the tree in the AST.
          \item Classic Object-Oriented Visitor Pattern on Memory Pools \\
              This is similar to the Classic Object-Oriented Visitor Pattern on the AST.
          \item IR node Type Traversal on Memory Pools \\
              This is a specialized traversal which visits each type of IR node, but one
              one of each type of IR nodes. This specialized traversal is useful for
              building tools that call static member functions on each type or IR node.
              A number of memory based tools for ROSE are built using this traversal.
        \end{enumerate}

     \end{enumerate}

     \item AST Query Library \\
           This example translator shows the use of the AST query library to generate
           a list of function declarations for any input program (and output the list
           of function names).  It can be trivially modified to return a list of any
           IR node type (C or C++ language construct).

     \item Symbol Table Handling (symbolTableHandling.C) \\
           This example shows how to use the symbol tables held within the AST for each scope.

     \item AST File I/O (astFileIO\_GenerateBinaryFile.C) \\
           This example demonstrates the file I/O for AST.  This is part of
           ROSE support for whole program analysis.

     \item Debugging Tips \\
           There are numerous methods ROSE provides to help debug the 
           development of specialized source-to-source translators.
           This section shows some of the techniques for getting
           information from IR nodes and displaying it.  Show how to 
           use the PDF generator for AST's.  This section may
           contain several subsections.
     \begin{enumerate}
          \item Generating the code representing any IR node
          \item Displaying the source code position of any IR node
     \end{enumerate}



  \end{enumerate}

%-----------------------------
  \item{Complex Types} 
%---------------------------------
        \begin{enumerate}
     \item Getting the type parameters in function declaration (functionParameterTypes.C) \\
           This example translator builds a list to record the types used in each
           function.  It shows an example of the sort of type information present 
           within the AST.  ROSE specifically maintains all type information.

     \item Resolving overloaded functions (resolvingOverloadedFunctions.C -- C++ specific) \\
           The AST has all type information pre-evaluated, particularly important for C++
           applications where type resolution is required for determining function
           invocation. This example translator builds a list of functions called within
           each function, showing that overloaded function are fully resolved within the
           AST.  Thus the user is not required to compute the type resolution required
           to identify which over loaded functions are called.

     \item Getting template parameters to a templated class (templateParameters.C -- C++ specific) \\
           All template information is saved within the AST.  Templated classes and
           functions are separately instantiated as specializations, as such they
           can be transformed separately depending upon their template values.
           This example code shows the template types used the instantiate a specific
           templated class.

        \end{enumerate}

%-----------------------------
  \item{Program Analysis} 
%---------------------------------
        \begin{enumerate}
     \item Recognizing loops within applications (loopRecognition.C) \\
           This example program shows the use of inherited and synthesized attributes
           form a list of loop nests and report their depth.  The inherited attributes are
           required to record when the traversal is within outer loop and the synthesized 
           attributes are required to pass the list of loop nests back up of the AST.
     \item Generating a CFG (buildCFG.C) \\
           This example shows the generation of a control flow graph 
           within ROSE.  The example is intended to be simple.  Many other
           graphs can be built, we need to show them as well.

     \item Generating a CG (buildCallGraph.C) \\
           This example shows the generation of a call graph 
           within ROSE.
     \item Generating a CH (classHierarchyGraph.C) \\
           This example shows the generation of a class hierarchy graph 
           within ROSE.

     \item Building custom graphs of program information \\
           The mechanisms used internally to build different graphs of program data
           is also made externally available. This section shows how new graphs of 
           program information can be built or existing graphs customized.

     \item Database Support (dataBaseUsage.C) \\
           This example shows how to use the optional (see {\tt configure --help})
           SQLite database to hold persistent program analysis results across the
           compilation of multiple files. This mechanism may become less critical
           as the only mechanism to support global analysis once we can support
           whole program analysis more generally within ROSE.
 \commentout{
   % This is not ready yet
     \item Recognition of Abstractions (recognitionOfAbstractions.C) \\
           This example program shows the automated recognition of 
           classes and functions defined within libraries and used by 
           applications.  This is work with Brian White as part of
           general support for recognition and optimization of 
           user-defined abstractions.
}

      \end{enumerate}
      
%---------------------------------
   \item{Program Transformations and Optimizations}
%---------------------------------
\begin{enumerate}
     \item Generating Unique Names for Declarations (generatingUniqueNamesFromDeclaration.C) \\
           A recurring issue in the development of many tools and program analysis is the
           representation of unique strings from language constructs (functions, variable
           declarations, etc.).  This example demonstrated support in ROSE for the
           generation of unique names. Names are unique across different ROSE tools and
           compilation of different files.

     \item Command-line processing \\
           ROSE includes mechanism to simplify the processing of command-line arguments
           so that translators using ROSE can trivially replace compilers within
           makefiles.  This example shows some of the many command-line handling
           options within ROSE and the ways in which customized options may be added.
     \begin{enumerate}
          \item Recognizing custom command-line options
          \item Adding options to internal ROSE command-line driven mechanisms
     \end{enumerate}

     \item Tailoring the code generation format: how to indent the
     generated source code and others.

     \item AST construction: how to build AST pieces from scratch and
     attach them to the existing AST tree.
       \begin{enumerate}
           \item Adding a variable declaration (addingVariableDeclaration.C) \\
                 Here we show how to add a variable declaration to the input application.
                 Perhaps we should show this in two ways to make it clear.  This is
                 a particularly simple use of the AST IR nodes to build an AST fragment
                 and add it to the application's AST.
      
           \item Adding a function (addingFunctionDeclaration.C) \\
                 This example program shows the addition of a new function 
                 to the global scope.  This example is a bit more involved
                 than the previous example.
           \item Simple Instrumentor Translator (simpleInstrumentor.C) \\
               This example modifies an input application to place new code at the top and
              bottom of each block.  The output is show with the instrumentation in place 
              in the generated code.
           \item Other examples for creating expressions, structures and so
           on.

       \end{enumerate}

      \item Handling source comments, preprocessor directives. 

     \item Calling the inliner (inlinerExample.C) \\
           This example shows the use of the inliner mechanism within ROSE.
           The function to be inlined in specified and the transformation upon the 
           AST is done to inline the function where it is called and clean up the
           resulting code.

     \item Calling the outliner (outlinerExample.C) \\
           This example shows the use of the outliner mechanism within ROSE.
           A segment of code is selected and a function is generated to 
           hold the resulting code.  Any required variables (including global variables)
           are passed through the generated function's interface.  The outliner is 
           a useful part of the empirical optimization mechanisms being developed 
           within ROSE.

     \item Call loop optimizer on set of loops (loopOptimization.C) \\
           This example program shows the optimization of a loop in C.
           This section contains several subsections each of which shows
           different sorts of optimizations.  There are a large number of
           loop optimizations only two are shown here, we need to add more.
     \begin{enumerate}
          \item Optimization of Matrix Multiply 
          \item Loop Fusion Optimizations
     \end{enumerate}

      \item Parameterized code translation: How to use command line options
      and abstract handles to have the translations you want, the order you
      want, and the behaviors you want.

     \fixme{Does this tutorial still exist?}
     \item Program slicing (programSlicingExample.C) \\
           This example shows the interface to the program slicing mechanism 
           within ROSE.  Program slicing has been implemented to two ways within
           ROSE.

\end{enumerate}
%---------------------------------
   \item{Correctness Checking}
%---------------------------------
\begin{enumerate}
     \item Code Coverage Analysis (codeCoverage.C) \\
           Code coverage is a useful tool by itself, but is particularly useful when
           combined with automated detection of bugs in programs.  This is part of
           work with IBM, Haifa.
      \item Bug seeding: how to purposely inject bugs into source code.  

\end{enumerate}
%---------------------------------
   \item{Binary Support}
%---------------------------------
\begin{enumerate}
  \item Instruction semantics
  \item Binary Analysis
  \item Binary construction
  \item DWarf debug support
\end{enumerate}

%---------------------------------
   \item{Interacting with Other Tools}
%---------------------------------
\begin{enumerate}
      \item Abstract handles: uniform ways of specifying language
      constructs.
      \item ROSE-HPCT interface: How to annotate AST with performance
      metrics generated by third-party performance tools.
     \item Tau Performance Analysis Instrumentation (tauInstrumenter.C) \\
           Tau currently uses an automate mechanism that modified the source code text
           file.  This example shows the modification of the AST and the generation 
           of the correctly instrumented files (which can otherwise be a problem when
           macros are used).  This is part of collaborations with the Tau project.
      \item The Haskell interface: interacting with a function programming
      language. 
\end{enumerate}
%---------------------------------
 \item{Parallelism}
%---------------------------------
\begin{enumerate}
  \item Shared-memory parallel traversals
  \item Distributed-memory parallel traversals
  \item Parallel checker
  \item Reduction variable recognition
\end{enumerate}

\end{itemize}

   Other examples included come specifically from external collaborations
and are more practically oriented.  Each is useful as an example because each 
solves a specific technical problem.  More of these will be included over time.
\fixme{The following tutorials no longer exist?}
\begin{enumerate}
     \item Fortran promotion of constants to double precision (typeTransformation.C) \\
           Fortran constants are by default singe precision, and must be modified to
           be double precision.  This is a common problem in older Fortran applications.
           This is part of collaborations with LANL to eventually automatically 
           update/modify older Fortran applications.
     \item Automated Runtime Library Support (charmSupport.C) \\
           Getting research runtime libraries into use within large scale applications
           requires automate mechanism to make minor changes to large amounts of code.
           This is part of collaborations with the Charm++ team (UIUC).
     \begin{enumerate}
        \item Shared Threaded Variable Detection Instrumentation (interveneAtVariables.C) \\
              Instrumentation support for variables, required to support detection of 
              threaded bugs in applications.
        \item Automated Modification of Function Parameters (changeFunction.C) \\
              This example program addresses a common problem where an applications
              function must be modified to include additional information.  In this
              case each function in a threaded library is modified to include additional 
              information to a corresponding wrapper library which instruments the
              library's use.
     \end{enumerate}
\end{enumerate}

\fixme{Add make installcheck in Makefile.am to build example translators using the
       installed libraries.}

