\chapter{Scopes of Declarations}

     The scope of an IR node may be either stored explicitly in the IR node
or obtained through computation through its parent information in the AST.
Figure X shows an example where the variable definition for a variable is 
the scope of namespace X.  The declaration for variable a is in the namespace X.
In a more common way, the function foo is a member function of B with a
declaration appearing in class B, but with a function definition in global scope.
{\mySmallFontSize
\begin{verbatim}
namespace X{
   extern int a;
}
int X::a = 0;
class B
  {
    void foo();
  };
void B::foo() {}
\end{verbatim}
}
In C++, using name qualification the scope of a declaration can be independent of
it structural location in the AST.  The {\tt get\_parent()} member function (available 
on most IR nodes) communicates the structural information of the original source code
(also represented in the AST).  The scope information must at times be stored
explicitly when it can not be interpreted structurally.

   The example in this chapter show how to find the scope of each C++ construct
   .  Note that SgExpression IR nodes can take their scope from
that of the statement where they are found.  SgStatement and SgInitializedName
IR nodes are the interesting IR nodes from the point of scope.

   The SgInitializedName and all SgStatement IR nodes have a member function
{\tt get\_scope()} which returns the scope of the associated IR nodes.  The
example code in this chapter traverses the AST and reports the scope of any
SgInitializedName and all SgStatement IR nodes.  It is intended to provide
a simple intuition about what the scope can be expected to be in an application.
The example code is also useful as a simple means of exploring the scopes
of any other input application.

\section{Input For Examples Showing Scope Information}

   Figure~\ref{Tutorial:exampleInputCode_ScopeInformation}
shows the input example code form this tutorial example.

\begin{figure}[!h]
{\indent
{\mySmallFontSize

% Do this when processing latex to generate non-html (not using latex2html)
\begin{latexonly}
   \lstinputlisting{\TutorialExampleDirectory/inputCode_scopeInformation.C}
\end{latexonly}

% Do this when processing latex to build html (using latex2html)
\begin{htmlonly}
   \verbatiminput{\TutorialExampleDirectory/inputCode_scopeInformation.C}
\end{htmlonly}

% end of scope in font size
}
% End of scope in indentation
}
\caption{Example source code used as input to program in
         codes used in this chapter.}
\label{Tutorial:exampleInputCode_ScopeInformation}
\end{figure}


\section{Generating the code representing any IR node}

    The following code traverses each IR node and for a 
SgInitializedName of SgStatement outputs the scope information.
The input code is shown in figure \ref{Tutorial:exampleInputCode_ScopeInformation};
the output of this code is shown in figure 
\ref{Tutorial:exampleOutput_ScopeInformation}.
    
\begin{figure}[!h]
{\indent
{\mySmallFontSize


% Do this when processing latex to generate non-html (not using latex2html)
\begin{latexonly}
   \lstinputlisting{\TutorialExampleDirectory/scopeInformation.C}
\end{latexonly}

% Do this when processing latex to build html (using latex2html)
\begin{htmlonly}
   \verbatiminput{\TutorialExampleDirectory/scopeInformation.C}
\end{htmlonly}

% end of scope in font size
}
% End of scope in indentation
}
\caption{Example source code showing how to get scope information for each IR node. }
\label{Tutorial:example_ScopeInformation}
\end{figure}



\begin{figure}[!h]
{\indent
{\mySmallFontSize


% Do this when processing latex to generate non-html (not using latex2html)
\begin{latexonly}
   \lstinputlisting{\TutorialExampleBuildDirectory/scopeInformation.out}
\end{latexonly}

% Do this when processing latex to build html (using latex2html)
\begin{htmlonly}
   \verbatiminput{\TutorialExampleBuildDirectory/scopeInformation.out}
\end{htmlonly}

% end of scope in font size
}
% End of scope in indentation
}
\caption{Output of input code using scopeInformation.C}
\label{Tutorial:exampleOutput_ScopeInformation}
\end{figure}



