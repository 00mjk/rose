% Liao 5/27/2010. Added a new section here
\clearpage
\section{Liveness Analysis}
Liveness analysis is a classic data flow analysis performed by compilers to
calculate for each program point the variables that may be potentially read
before their next write (re-definition).
A variable is live at a point in a program's execution path
if it holds a value that may be needed in the future. 

Fig.~\ref{Tutorial:livenessCode} shows an example program of how the
liveness analysis
is called in a ROSE-based translator. 
It generates a dot graph showing def/use information within
a control flow graph, alone with live-in and live-out variables. 
This program (named as \textit{livenessAnalysis})
is installed under \textit{ROSE\_INST/bin} as a standalone tool for users
to experiment the liveness analysis of ROSE.

\begin{figure}[!h]
{\indent
{\mySmallFontSize
% Do this when processing latex to generate non-html (not using latex2html)
\begin{latexonly}
   \lstinputlisting{\TutorialExampleDirectory/livenessAnalysis.C}
\end{latexonly}

% Do this when processing latex to build html (using latex2html)
\begin{htmlonly}
   \verbatiminput{\TutorialExampleDirectory/livenessAnalysis.C}
\end{htmlonly}

% end of scope in font size
}
% End of scope in indentation
}
\caption{Example source code using liveness analysis}
\label{Tutorial:livenessCode}
\end{figure}

Figure~\ref{Tutorial:exampleLivenessCFG} shows control flow graph with live
variable information for the
code(Fig.~\ref{Tutorial:livenessCode}) running on the input
code(Fig.~\ref{Tutorial:exampleLivenessInput}). 

\begin{figure}[!h]
{\indent
{\mySmallFontSize

% Do this when processing latex to generate non-html (not using latex2html)
\begin{latexonly}
   \lstinputlisting{\TutorialExampleDirectory/inputCode_livenessAnalysis.c}
\end{latexonly}

% Do this when processing latex to build html (using latex2html)
\begin{htmlonly}
   \verbatiminput{\TutorialExampleDirectory/inputCode_livenessAnalysis.c}
\end{htmlonly}

% end of scope in font size
}
% End of scope in indentation
}
\caption{Example input code.}
\label{Tutorial:exampleLivenessInput}
\end{figure}


\begin{figure}
\includegraphics[scale=0.7]{\TutorialExampleBuildDirectory/liveness}
\caption{Control flow graph annotated with live variables for example program.}
\label{Tutorial:exampleLivenessCFG}
\end{figure}

%\begin{figure}[!h]
%{\indent
%{\mySmallFontSize
%
%% Do this when processing latex to generate non-html (not using latex2html)
%\begin{latexonly}
%   \lstinputlisting{\TutorialExampleBuildDirectory/defuseAnalysis.out}
%\end{latexonly}
%
%% Do this when processing latex to build html (using latex2html)
%\begin{htmlonly}
%   \verbatiminput{\TutorialExampleBuildDirectory/defuseAnalysis.out}
%\end{htmlonly}
%
%% end of scope in font size
%}
%% End of scope in indentation
%}
%\caption{Output of the program}
%\label{Tutorial:defuse.output}
%\end{figure}


\subsection{Access live variables}
After calling liveness analysis, one can access live-in and live-out variables from a translator
based on the virtual control flow graph node.
Figure~\ref{Tutorial:livenessCode2} shows an example function which
retrieves the live-in and live-out variables for a for loop.
The code accesses the CFG node (showing in
Figure~\ref{Tutorial:exampleLivenessCFG}) of a for statement and retrieve
live-in variables of the true edge's target node as the live-in variables
of the loop body. 
Similarly, the live-out variables of the loop are obtained by getting
the live-in variables of the node right after the loop (target node of the
false edge).

\begin{figure}[!h]
{\indent
{\mySmallFontSize
% Do this when processing latex to generate non-html (not using latex2html)
\begin{latexonly}
   \lstinputlisting{\TutorialExampleDirectory/livenessAnalysis-2.C}
\end{latexonly}

% Do this when processing latex to build html (using latex2html)
\begin{htmlonly}
   \verbatiminput{\TutorialExampleDirectory/livenessAnalysis-2.C}
\end{htmlonly}

% end of scope in font size
}
% End of scope in indentation
}
\caption{Example code retrieving live variables based on virtual control flow graph}
\label{Tutorial:livenessCode2}
\end{figure}


