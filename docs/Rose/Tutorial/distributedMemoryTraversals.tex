% GB (09/28/2007)
 
\chapter{Distributed-Memory Parallel Traversals}
\label{chap:distributedMemoryTraversals}
 
    ROSE provides an experimental distributed-memory AST traversal mechanism meant for very large scale program
analysis. It allows you to distribute expensive program analyses among a distributed-memory system consisting of many
processors; this can be a cluster or a network of workstations. Different processes in the distributed system will get
different parts of the AST to analyze: Each process is assigned a number of defining function declarations in the AST,
and a method implemented by the user is invoked on each of these. The parts of the AST outside of function definitions
are shared among all processes, but there is no guarantee that all function definitions are visible to all processes.

    The distributed memory analysis framework uses the MPI message passing library for communicating attributes among
processes. You will need an implementation of MPI to be able to build and run programs using distributed memory
traversals; consult your documentation on how to run MPI programs. (This is often done using a program named {\tt
mpirun}, {\tt mpiexecute}, or similar.)

    Distributed memory analyses are performed in three phases:
\begin{enumerate}
\item A top-down traversal (the `pre-traversal') specified by the user runs on the shared AST outside of function
definitions. The inherited attributes this traversal computes for defining function declaration nodes in the AST are
saved by the framework for use in the next phase.

\item For every defining function declaration, the user-provided {\tt analyzeSubtree()} method is invoked; these calls
run concurrently, but on different function declarations, on all processors. It takes as arguments the AST node for the
function declaration and the inherited attribute computed for this node by the pre-traversal. Within {\tt
analyzeSubtree()} any analysis features provided by ROSE can be used. This method returns the value that will be used as
the synthesized attribute for this function declaration in the bottom-up traversal (the `post-traversal').

However, unlike normal bottom-up traversals, the synthesized attribute is not simply copied in memory as the AST is
distributed. The user must therefore provide the methods {\tt serializeAttribute()} and {\tt deserializeAttribute()}.
These compute a serialized representation of a synthesized attribute, and convert such a representation back to the
user's synthesized attribute type, respectively. A serialized attribute is a pair of an integer specifying the size of
the attribute in bytes and a pointer to a region of memory of that size that will be copied byte by byte across the
distributed system's communication network. Attributes from different parts of the AST may have different sizes. As
serialization of attributes will often involve dynamic memory allocation, the user can also implement the {\tt
deleteSerializedAttribute()} method to such dynamic memory after the serialized data has been copied to the
communication subsystem's internal buffer.

Within the {\tt analyzeSubtree()} method the methods {\tt numberOfProcesses()} and {\tt myID()} can be called. These
return the total number of concurrent processes, and an integer uniquely identifying the currently running process,
respectively. The ID ranges from 0 to one less than the number of processes, but has no semantics other than that it is
different for each process.

\item Finally, a bottom-up traversal is run on the shared AST outside of function definitions. The values returned by
the distributed analyzers in the previous phase are used as synthesized attributes for function definition nodes in this
traversal.
\end{enumerate}

    After the bottom-up traversal has finished, the {\tt getFinalResults()} method can be invoked to obtain the final
synthesized attribute. The {\tt isRootProcess()} method returns true on exactly one designated process and can be used
to perform output, program transformations, or other tasks that are not meant to be run on each processor.

    Figure~\ref{Tutorial:exampleDistributedMemoryTraversals} gives a complete example of how to use the distributed
memory analysis framework. It implements a trivial analysis that determines for each function declaration at what depth
in the AST it can be found and what its name is. Figure~\ref{Tutorial:exampleOutput_DistributedMemoryTraversals} shows
the output produced by this program when running using four processors on some input files.

% The distributedMemoryFunctionNames.C file is a copy of projects/DistributedMemoryAnalysis/functionNames.C
\begin{figure}[!h]
{\indent
{\mySmallFontSize
% Do this when processing latex to generate non-html (not using latex2html)
\begin{latexonly}
   \lstinputlisting{\TutorialExampleDirectory/distributedMemoryFunctionNames.C}
\end{latexonly}
% Do this when processing latex to build html (using latex2html)
\begin{htmlonly}
   \verbatiminput{\TutorialExampleDirectory/distributedMemoryFunctionNames.C}
\end{htmlonly}
% end of scope in font size
}
% End of scope in indentation
}
\caption{Example source demonstrating the use of the distributed-memory parallel analysis framework.}
\label{Tutorial:exampleDistributedMemoryTraversals}
\end{figure}
\begin{figure}[!h]
{\indent
{\mySmallFontSize
% GB (09/28/2007): This is just copied here; we can't generate it as not everyone has a copy of MPI. Might not be very
% elegant, but it's 4:42 p.m. on my last day :-)
\begin{verbatim}
----- found the following functions: ------
process 0: at depth 3: function il
process 0: at depth 5: function head
process 0: at depth 5: function eq
process 1: at depth 3: function headhead
process 1: at depth 3: function List
process 1: at depth 3: function find
process 1: at depth 3: function head
process 2: at depth 3: function operator!=
process 2: at depth 3: function find
process 2: at depth 3: function head
process 2: at depth 3: function fib
process 3: at depth 3: function xform
process 3: at depth 3: function func
process 3: at depth 3: function f
process 3: at depth 3: function g
process 3: at depth 3: function deref
-------------------------------------------
\end{verbatim}
% end of scope in font size
}
% End of scope in indentation
}
\caption{Example output of a distributed-memory analysis running on four processors.}
\label{Tutorial:exampleOutput_DistributedMemoryTraversals}
\end{figure}
