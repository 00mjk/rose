\chapter{Debugging Techniques}
     There are numerous methods ROSE provides to help debug the 
development of specialized source-to-source translators.
This section shows some of the techniques for getting
information from IR nodes and displaying it.  
More information about generation of specialized AST graphs to support debugging 
can be found in chapter \ref{Tutorial:chapterGeneralASTGraphGeneration} and custom 
graph generation in section \ref{Tutorial:chapterCustomGraphs}.

\section{Input For Examples Showing Debugging Techniques}

   Figure~\ref{Tutorial:exampleInputCode_ExampleDebugging}
shows the input code used for the example translators that
report useful debugging information in this chapter.

\begin{figure}[!h]
{\indent
{\mySmallFontSize

% Do this when processing latex to generate non-html (not using latex2html)
\begin{latexonly}
   \lstinputlisting{\TutorialExampleDirectory/inputCode_ExampleDebugging.C}
\end{latexonly}

% Do this when processing latex to build html (using latex2html)
\begin{htmlonly}
   \verbatiminput{\TutorialExampleDirectory/inputCode_ExampleDebugging.C}
\end{htmlonly}

% end of scope in font size
}
% End of scope in indentation
}
\caption{Example source code used as input to program in
         codes showing debugging techniques shown in this section.}
\label{Tutorial:exampleInputCode_ExampleDebugging}
\end{figure}


\section{Generating the code from any IR node}

     Any IR node may be converted to the string that represents
its subtree within the AST.  If it is a type, then the string will be 
the value of the type; if it is a statement, the value will be the 
source code associated with that statement, including any sub-statements.
To support the generation for strings from IR nodes we use the
{\tt unparseToString()} member function.  This function strips
comments and preprocessor control structure.  The resulting string is useful
for both debugging and when forming larger strings associated with the
specification of transformations using the string-based rewrite mechanism.
Using ROSE, IR nodes may be converted to strings, and strings converted to 
AST fragments of IR nodes.

Note that unparsing associated with generating source code for the backend 
vendor compiler is more than just calling the unparseToString
member function, since it introduces comments, preprocessor control structure 
and formating.
    
Figure~\ref{Tutorial:exampleDebuggingIRNodeToString} shows a translator
which generates a string for a number of predefined IR nodes.  
Figure~\ref{Tutorial:exampleInputCode_ExampleDebugging} 
shows the sample input code and 
figure~\ref{Tutorial:exampleOutput_DebuggingIRNodeToString} 
shows the output from the translator when using the example input application.


\begin{figure}[!h]
{\indent
{\mySmallFontSize

% Do this when processing latex to generate non-html (not using latex2html)
\begin{latexonly}
   \lstinputlisting{\TutorialExampleDirectory/debuggingIRnodeToString.C}
\end{latexonly}

% Do this when processing latex to build html (using latex2html)
\begin{htmlonly}
   \verbatiminput{\TutorialExampleDirectory/debuggingIRnodeToString.C}
\end{htmlonly}

% end of scope in font size
}
% End of scope in indentation
}
\caption{Example source code showing the output of the string from an IR node. The string
         represents the code associated with the subtree of the target IR node.}
\label{Tutorial:exampleDebuggingIRNodeToString}
\end{figure}

\begin{figure}[!h]
{\indent
{\mySmallFontSize

% Do this when processing latex to generate non-html (not using latex2html)
\begin{latexonly}
   \lstinputlisting{\TutorialExampleBuildDirectory/debuggingIRnodeToString.out}
\end{latexonly}

% Do this when processing latex to build html (using latex2html)
\begin{htmlonly}
   \verbatiminput{\TutorialExampleBuildDirectory/debuggingIRnodeToString.out}
\end{htmlonly}

% end of scope in font size
}
% End of scope in indentation
}
\caption{Output of input code using debuggingIRnodeToString.C}
\label{Tutorial:exampleOutput_DebuggingIRNodeToString}
\end{figure}




\section{Displaying the source code position of any IR node}

   This example shows how to obtain information about the position of
any IR node relative to where it appeared in the original source code.
New IR nodes (or subtrees) that are added to the AST as part of a
transformation will be marked as part of a transformation and have
no position in the source code.  Shared IR nodes (as generated by the AST
merge mechanism are marked as shared explicitly (other IR nodes that
are shared by definition don't have a SgFileInfo object and are thus
not marked explicitly as shared.

   The example translator to output the source code position is shown in 
figure~\ref{Tutorial:exampleDebuggingSourceCodePositionInformation}.
Using the input code in 
figure~\ref{Tutorial:exampleInputCode_ExampleDebugging}
the output code is shown in 
figure~\ref{Tutorial:exampleOutput_DebuggingIRNodeToString}.


\begin{figure}[!h]
{\indent
{\mySmallFontSize

% Do this when processing latex to generate non-html (not using latex2html)
\begin{latexonly}
   \lstinputlisting{\TutorialExampleDirectory/debuggingSourceCodePositionInformation.C}
\end{latexonly}

% Do this when processing latex to build html (using latex2html)
\begin{htmlonly}
   \verbatiminput{\TutorialExampleDirectory/debuggingSourceCodePositionInformation.C}
\end{htmlonly}

% end of scope in font size
}
% End of scope in indentation
}
\caption{Example source code showing the output of the string from an IR node. The string
         represents the code associated with the subtree of the target IR node.}
\label{Tutorial:exampleDebuggingSourceCodePositionInformation}
\end{figure}




\begin{figure}[!h]
{\indent
{\mySmallFontSize

% Do this when processing latex to generate non-html (not using latex2html)
\begin{latexonly}
   \lstinputlisting{\TutorialExampleBuildDirectory/debuggingSourceCodePositionInformation.out}
\end{latexonly}
hared IR nodes (as generated by the AST
merge mechanism are mar
% Do this when processing latex to build html (using latex2html)
\begin{htmlonly}
   \verbatiminput{\TutorialExampleBuildDirectory/debuggingSourceCodePositionInformation.out}
\end{htmlonly}

% end of scope in font size
}
% End of scope in indentation
}
\caption{Output of input code using debuggingSourceCodePositionInformation.C}
\label{Tutorial:exampleOutput_DebuggingIRNodeToString}
\end{figure}












