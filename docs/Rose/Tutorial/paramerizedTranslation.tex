\chapter{Parameterized Code Translation}
This chapter gives examples of using ROSE's high level loop translation
interfaces to perform parameterized loop transformations, including loop
unrolling, interchanging and tiling. 
The motivation is to give users the maximized flexibility to orchestrate
code transformations on the targets they want, the order they
want, and the parameters they want.
One typical application scenario is to support generating desired code
variants for empirical tuning. 

The ROSE internal interfaces (declared within the SageInterface namespace) to call loop transformations are:
\begin{itemize}
\item \textit{bool loopUnrolling (SgForStatement *loop, size\_t
unrolling\_factor)}: 
This function needs two parameters: one for the loop to be
unrolled and the other for the unrolling factor. 
\item \textit{bool  loopInterchange (SgForStatement *loop, size\_t depth,
size\_t lexicoOrder)}:
The loop interchange function has three parameters, the first one to
specify a loop which starts a perfectly-nested loop and is to be
interchanged, the 2nd for the depth of the loop nest to be interchanged, and finally the
lexicographical order for the permutation. 
\item \textit{bool  loopTiling (SgForStatement *loopNest, size\_t
targetLevel, size\_t tileSize)}
The loop tiling interface needs to know the loop nest to be tiled, which
loop level to tile, and the tile size for the level. 
\end{itemize} 
For efficiency concerns, those functions only perform the specified
translations without doing any legitimacy check. 
It is up to the users to
make sure the transformations won't generate wrong code.
We will soon provide interfaces to do the eligibility check for each
transformation. 

We also provide standalone executable programs
(loopUnrolling,loopInterchange, and loopTiling under ROSE\_INSTALL/bin) for the
transformations mentioned above so users can directly use them via command
lines and abstract handles (explained in Chapter~\ref{chap:handles}) to orchestrate transformations they want.
%=====================================
\section{Loop Unrolling}
Figure~\ref{Tutorial:exampleInputCode_LoopUnrollingp} gives an example
input code for loopUnrolling.

\begin{figure}[!h]
{\indent
  {\mySmallFontSize
      \begin{latexonly}
    \lstinputlisting{\TOPSRCDIR/tests/nonsmoke/functional/roseTests/astInterfaceTests/inputloopUnrolling.C}
    \end{latexonly}
      \begin{htmlonly}
    \verbatiminput{\TOPSRCDIR/tests/nonsmoke/functional/roseTests/astInterfaceTests/inputloopUnrolling.C}
    \end{htmlonly}
  }
}
\caption{Example source code used as input to loopUnrolling }
    \label{Tutorial:exampleInputCode_LoopUnrollingp}
\end{figure}

An example command line to invoke loop unrolling on the example can look
like the following:
\begin{verbatim}
# unroll a for statement 5 times. The loop is a statement at line 6 within
# an input file.
loopUnrolling -c inputloopUnrolling.C \
-rose:loopunroll:abstract_handle "Statement<position,6>" -rose:loopunroll:factor 5
\end{verbatim}

Two kinds of output can be expected after loop unrolling.
One (Shown in Figure~\ref{Tutorial:exampleOutput_LoopUnrollingpEven}) is the case that the loop iteration count is known at compile-time and
can be evenly divisible by the unrolling factor.
The other case (Shown in
Figure~\ref{Tutorial:exampleOutput_LoopUnrollingpNonEven} is when the divisibility is unknown and a fringe loop has to
be generated to run possible leftover iterations.
%----------------------------- evenly divisible case------------
\begin{figure}[!h]
{\indent
  {\mySmallFontSize
      \begin{latexonly}
    \lstinputlisting{\TOPBUILDDIR/tests/nonsmoke/functional/roseTests/astInterfaceTests/rose_inputloopUnrolling1.C}
    \end{latexonly}
      \begin{htmlonly}
    \verbatiminput{\TOPBUILDDIR/tests/nonsmoke/functional/roseTests/astInterfaceTests/rose_inputloopUnrolling1.C}
    \end{htmlonly}
  }
}
\caption{Output for a unrolling factor which can divide the iteration space evenly}
    \label{Tutorial:exampleOutput_LoopUnrollingpEven}
\end{figure}
%----------------------------- non evenly divisible case------------
\begin{figure}[!h]
{\indent
  {\mySmallFontSize
      \begin{latexonly}
    \lstinputlisting{\TOPBUILDDIR/tests/nonsmoke/functional/roseTests/astInterfaceTests/rose_inputloopUnrolling2.C}
    \end{latexonly}
      \begin{htmlonly}
    \verbatiminput{\TOPBUILDDIR/tests/nonsmoke/functional/roseTests/astInterfaceTests/rose_inputloopUnrolling2.C}
    \end{htmlonly}
  }
}
\caption{Output for the case when divisibility is unknown at compile-time}
    \label{Tutorial:exampleOutput_LoopUnrollingpNonEven}
\end{figure}

%========================================================
\clearpage % enforce all previous figures to be shown
\section{Loop Interchange}

Figure~\ref{Tutorial:exampleInputCode_LoopInterchangep} gives an example
input code for loopInterchange.

\begin{figure}[!h]
{\indent
  {\mySmallFontSize
      \begin{latexonly}
    \lstinputlisting{\TOPSRCDIR/tests/nonsmoke/functional/roseTests/astInterfaceTests/inputloopInterchange.C}
    \end{latexonly}
      \begin{htmlonly}
    \verbatiminput{\TOPSRCDIR/tests/nonsmoke/functional/roseTests/astInterfaceTests/inputloopInterchange.C}
    \end{htmlonly}
  }
}
\caption{Example source code used as input to loopInterchange}
    \label{Tutorial:exampleInputCode_LoopInterchangep}
\end{figure}


An example command line to invoke loop interchange:
\begin{verbatim}
# interchange a loop nest starting from the first loop within the input file, 
# interchange depth is 4 and 
# the lexicographical order is 1 (swap the innermost two levels)
loopInterchange -c inputloopInterchange.C -rose:loopInterchange:abstract_handle \
"ForStatement<numbering,1>" -rose:loopInterchange:depth 4 \
-rose:loopInterchange:order 1
\end{verbatim}

Figure~\ref{Tutorial:exampleOutput_LoopInterchangep} shows the output.
%------------------------------------------
\begin{figure}[!h]
{\indent
  {\mySmallFontSize
      \begin{latexonly}
    \lstinputlisting{\TOPBUILDDIR/tests/nonsmoke/functional/roseTests/astInterfaceTests/rose_inputloopInterchange.C}
    \end{latexonly}
      \begin{htmlonly}
    \verbatiminput{\TOPBUILDDIR/tests/nonsmoke/functional/roseTests/astInterfaceTests/rose_inputloopInterchange.C}
    \end{htmlonly}
  }
}
\caption{Output for loop interchange}
    \label{Tutorial:exampleOutput_LoopInterchangep}
\end{figure}

%========================================================
\clearpage
\section{Loop Tiling}
Figure~\ref{Tutorial:exampleInputCode_LoopTilingp} gives an example
input code for loopTiling.

\begin{figure}[!h]
{\indent
  {\mySmallFontSize
      \begin{latexonly}
    \lstinputlisting{\TOPSRCDIR/tests/nonsmoke/functional/roseTests/astInterfaceTests/inputloopTiling.C}
    \end{latexonly}
      \begin{htmlonly}
    \verbatiminput{\TOPSRCDIR/tests/nonsmoke/functional/roseTests/astInterfaceTests/inputloopTiling.C}
    \end{htmlonly}
  }
}
\caption{Example source code used as input to loopTiling}
    \label{Tutorial:exampleInputCode_LoopTilingp}
\end{figure}


An example command line to invoke loop tiling:
\begin{verbatim}
# Tile the loop with a depth of 3 within the first loop of the input file 
# tile size is 5
loopTiling -c inputloopTiling.C -rose:loopTiling:abstract_handle \
"ForStatement<numbering,1>" -rose:loopTiling:depth 3 -rose:loopTiling:tilesize 5

\end{verbatim}
Figure~\ref{Tutorial:exampleOutput_LoopTilingp} shows the output.
%------------------------------------------
\begin{figure}[!h]
{\indent
  {\mySmallFontSize
      \begin{latexonly}
    \lstinputlisting{\TOPBUILDDIR/tests/nonsmoke/functional/roseTests/astInterfaceTests/rose_inputloopTiling.C}
    \end{latexonly}
      \begin{htmlonly}
    \verbatiminput{\TOPBUILDDIR/tests/nonsmoke/functional/roseTests/astInterfaceTests/rose_inputloopTiling.C}
    \end{htmlonly}
  }
}
\caption{Output for loop tiling}
    \label{Tutorial:exampleOutput_LoopTilingp}
\end{figure}


