\clearpage
\chapter{AST Construction}
AST construction is a fundamental operation needed for building ROSE
source-to-source translators. Several levels of interfaces are available in
ROSE for users to build AST from scratch. 
High level interfaces are recommended to use whenever possible for their
simplicity. Low level interfaces can give users the maximum freedom to
manipulate some details in AST trees.


This chapter uses several examples to demonstrate how to 
create AST fragments for common language
constructs (such as variable declarations, functions, function calls, etc.)
  and how to insert them into an existing AST tree.
More examples of constructing AST using high level interfaces can be found at
\textit{rose/tests/roseTests/astInterfaceTests}. 
The source files of the high level interfaces are located in
\textit{rose/src/frontend/SageIII/sageInterface}.

\section{Variable Declarations}

\paragraph{What To Learn} Two examples are given to show how to construct a SAGE III 
AST subtree for a variable declaration and its insertion into the existing AST tree.
\begin{itemize}
\item Example 1. Building a variable declaration using the high level AST construction and 
manipulation interfaces defined in namespace SageBuilder and SageInterface. 

Figure~\ref{Tutorial:exampleAddVariableDeclaration2} shows the high level
construction of an AST fragment (a variable declaration) and its insertion 
into the AST at the top of each block. buildVariableDeclaration() takes the name and type
to build a variable declaration node. prependStatement() inserts the declaration at the top
of a basic block node. Details for parent and scope pointers, symbol
tables, source file position information 
and so on are handled transparently.

%------------------- begin figure---------------------
\begin{figure}[!hbp]
{\indent
{\mySmallFontSize
% Do this when processing latex to generate non-html (not using latex2html)
\begin{latexonly}
   \lstinputlisting{\TutorialExampleDirectory/addVariableDeclaration2.C}
\end{latexonly}

% Do this when processing latex to build html (using latex2html)
\begin{htmlonly}
   \verbatiminput{\TutorialExampleDirectory/addVariableDeclaration2.C}
\end{htmlonly}

% end of scope in font size
}
% End of scope in indentation
}
\caption{AST construction and insertion for a variable using the high level interfaces}
\label{Tutorial:exampleAddVariableDeclaration2}
\end{figure}

%------------------- end figure---------------------

%\newpage
\item
Example 2. Building the variable declaration using low level member functions of SAGE III node classes.

Figure~\ref{Tutorial:exampleAddVariableDeclaration} shows the low level
construction of the same AST fragment (for the same variable declaration) and its insertion 
into the AST at the top of each block. SgNode constructors and their member functions are used.
Side effects for scope, parent pointers and symbol tables have to be handled by programmers explicitly.

%Note that the code does not handle symbol table issues.
%\fixme{The input code should use the LowLevelRewrite classes instead of the still lower level AST nodes member functions directly.}

%\fixme{Both this section and the next section (adding a function declaration)
%       and the section(s) specific to the string based rewrite mechanism
%       could be subsections of a section on AST Rewrite Mechanisms.}

\begin{figure}[!hbp]
{\indent
{\mySmallFontSize
% Do this when processing latex to generate non-html (not using latex2html)
\begin{latexonly}
   \lstinputlisting{\TutorialExampleDirectory/addVariableDeclaration.C}
\end{latexonly}

% Do this when processing latex to build html (using latex2html)
\begin{htmlonly}
   \verbatiminput{\TutorialExampleDirectory/addVariableDeclaration.C}
\end{htmlonly}

% end of scope in font size
}
% End of scope in indentation
}
\caption{Example source code to read an input program and add a new variable 
         declaration at the top of each block.}
\label{Tutorial:exampleAddVariableDeclaration}
\end{figure}
\end{itemize}


Figure~\ref{Tutorial:exampleInputCode_AddVariableDeclaration} shows the
input code used to test the translator. 
Figure~\ref{Tutorial:exampleOutput_AddVariableDeclaration} shows the resulting output.

%-----------------------input code-------------------------
\begin{figure}[!h]
{\indent
{\mySmallFontSize


% Do this when processing latex to generate non-html (not using latex2html)
\begin{latexonly}
   \lstinputlisting{\TutorialExampleDirectory/inputCode_AddVariableDeclaration.C}
\end{latexonly}

% Do this when processing latex to build html (using latex2html)
\begin{htmlonly}
   \verbatiminput{\TutorialExampleDirectory/inputCode_AddVariableDeclaration.C}
\end{htmlonly}

% end of scope in font size
}
% End of scope in indentation
}
\caption{Example source code used as input to the translators adding new variable.}
\label{Tutorial:exampleInputCode_AddVariableDeclaration}
\end{figure}

%-----------------------output code-------------------------
\begin{figure}[!h]
{\indent
{\mySmallFontSize

% Do this when processing latex to generate non-html (not using latex2html)
\begin{latexonly}
   \lstinputlisting{\TutorialExampleBuildDirectory/rose_inputCode_AddVariableDeclaration.C}
\end{latexonly}

% Do this when processing latex to build html (using latex2html)
\begin{htmlonly}
   \verbatiminput{\TutorialExampleBuildDirectory/rose_inputCode_AddVariableDeclaration.C}
\end{htmlonly}

% end of scope in font size
}
% End of scope in indentation
}
\caption{Output of input to the translators adding new variable.}
\label{Tutorial:exampleOutput_AddVariableDeclaration}
\end{figure}


