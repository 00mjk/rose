\chapter{Adding a Variable Declaration}

   Figure~\ref{Tutorial:exampleAddVariableDeclaration} shows the low level
construction of an AST fragment (a variable declaration) and its insertion 
into the AST at the top of each block.  Note that the code does not handle 
symbol table issues.

   Building a variable declaration directly from SAGE III.

\fixme{The input code should use the LowLevelRewrite classes instead of the 
       still lower level AST nodes member functions directly.}

\fixme{Both this section and the next section (adding a function declaration)
       and the section(s) specific to the string based rewrite mechanism
       could be subsections of a section on AST Rewrite Mechanisms.}

\begin{figure}[!h]
{\indent
{\mySmallFontSize


% Do this when processing latex to generate non-html (not using latex2html)
\begin{latexonly}
   \lstinputlisting{\TutorialExampleDirectory/addVariableDeclaration.C}
\end{latexonly}

% Do this when processing latex to build html (using latex2html)
\begin{htmlonly}
   \verbatiminput{\TutorialExampleDirectory/addVariableDeclaration.C}
\end{htmlonly}

% end of scope in font size
}
% End of scope in indentation
}
\caption{Example source code to read an input program and add a new variable 
         declaration at the top of each block.}
\label{Tutorial:exampleAddVariableDeclaration}
\end{figure}


   Figure~\ref{Tutorial:exampleInputCode_AddVariableDeclaration} shows the
input code used to get the translator.
Figure~\ref{Tutorial:exampleOutput_AddVariableDeclaration} shows the resulting output.

\begin{figure}[!h]
{\indent
{\mySmallFontSize


% Do this when processing latex to generate non-html (not using latex2html)
\begin{latexonly}
   \lstinputlisting{\TutorialExampleDirectory/inputCode_AddVariableDeclaration.C}
\end{latexonly}

% Do this when processing latex to build html (using latex2html)
\begin{htmlonly}
   \verbatiminput{\TutorialExampleDirectory/inputCode_AddVariableDeclaration.C}
\end{htmlonly}

% end of scope in font size
}
% End of scope in indentation
}
\caption{Example source code used as input to translator adding new variable.}
\label{Tutorial:exampleInputCode_AddVariableDeclaration}
\end{figure}

\begin{figure}[!h]
{\indent
{\mySmallFontSize


% Do this when processing latex to generate non-html (not using latex2html)
\begin{latexonly}
   \lstinputlisting{\TutorialExampleBuildDirectory/rose_inputCode_AddVariableDeclaration.C}
\end{latexonly}

% Do this when processing latex to build html (using latex2html)
\begin{htmlonly}
   \verbatiminput{\TutorialExampleBuildDirectory/rose_inputCode_AddVariableDeclaration.C}
\end{htmlonly}

% end of scope in font size
}
% End of scope in indentation
}
\caption{Output of input to translator adding new variable.}
\label{Tutorial:exampleOutput_AddVariableDeclaration}
\end{figure}


