\chapter{Dataflow Analysis}

The dataflow analysis in Rose is based on the control flow graph (CFG).
One type of dataflow analysis is called def-use analysis, which is explained next.

%-------------------------------------
\section{Def-Use Analysis}

The definition-usage (def-use) analysis allows to query the definition
and usage for each \emph{control flow node} (CFN).
Any statement or expression within ROSE is represented as a sequence of CFN's.
For instance, the CFG for the following program



\begin{figure}[!h]
{\indent
{\mySmallFontSize

% Do this when processing latex to generate non-html (not using latex2html)
\begin{latexonly}
   \lstinputlisting{\TutorialExampleDirectory/input_defuseAnalysis.C}
\end{latexonly}

% Do this when processing latex to build html (using latex2html)
\begin{htmlonly}
   \verbatiminput{\TutorialExampleDirectory/input_defuseAnalysis.C}
\end{htmlonly}

% end of scope in font size
}
% End of scope in indentation
}
\caption{Example input code.}
\label{Tutorial:exampledefuseInput}
\end{figure}

is illustrated in Figure ~\ref{Tutorial:exampledefuse}.

\subsection{Def-use Example implementation}

Fig.~\ref{Tutorial:defuseCode} shows an example program of how the def-use analysis
is called. It generates a dot graph showing def/use information within
a control flow graph. 
It also outputs reaching definition information for each variable
references of an input code. This program (named as \textit{defuseAnalysis})
is installed under \textit{ROSE\_INST/bin} as a standalone tool for users
to experiment the def/use analysis of ROSE.

\begin{figure}[!h]
{\indent
{\mySmallFontSize
% Do this when processing latex to generate non-html (not using latex2html)
\begin{latexonly}
   \lstinputlisting{\TutorialExampleDirectory/defuseAnalysis.C}
\end{latexonly}

% Do this when processing latex to build html (using latex2html)
\begin{htmlonly}
   \verbatiminput{\TutorialExampleDirectory/defuseAnalysis.C}
\end{htmlonly}

% end of scope in font size
}
% End of scope in indentation
}
\caption{Example source code using def use analysis}
\label{Tutorial:defuseCode}
\end{figure}


Figure~\ref{Tutorial:defuse.output} shows the screen output of the
code(Fig.~\ref{Tutorial:defuseCode}) running on the input
code(Fig.~\ref{Tutorial:exampledefuseInput}). 
Each variable reference in the input code has at least one reaching definition node. The associated definition
statement is also printed out.   

\begin{figure}[!h]
{\indent
{\mySmallFontSize

% Do this when processing latex to generate non-html (not using latex2html)
\begin{latexonly}
   \lstinputlisting{\TutorialExampleBuildDirectory/defuseAnalysis.out}
\end{latexonly}

% Do this when processing latex to build html (using latex2html)
\begin{htmlonly}
   \verbatiminput{\TutorialExampleBuildDirectory/defuseAnalysis.out}
\end{htmlonly}

% end of scope in font size
}
% End of scope in indentation
}
\caption{Output of the program}
\label{Tutorial:defuse.output}
\end{figure}


\subsection{Accessing the Def-Use Results}
For each CFN in the CFG, the definition and usage for 
variable references can be determined with the public function calls:

\begin{verbatim}
vector <SgNode*> getDefFor(SgNode*, SgInitializedName*)
vector <SgNode*> getUseFor(SgNode*, SgInitializedName*)
\end{verbatim}

where SgNode* represents any control flow node and SgInitializedName any variable (being
used or defined at that CFN). The result is a vector of possible CFN's that either
define (getDefFor) or use (getUseFor) a specific variable.

Figure ~\ref{Tutorial:exampledefuse} shows how the variable x is being
declared and defined in CFN's between node 1 and 6. Note that the definition
is annotated along the edge. For instance at node 6, the edge reads 
\emph{(6) DEF: x (3) = 5}. This means that variable x was declared at CFN 3 but
defined at CFN 5.

The second statement x=x+1 is represented by CFN's from 7 to 12.
One can see in the figure that x is being re-defined at CFN 11. However,
the definition of x within the same statement happens at CFN 8. Hence, the 
definition of the right hand side x in the statement is at CFN 5 :
\emph{(8) DEF: x (3) = 5}.

\begin{figure}
%\centerline{\epsfig{file=\TutorialExampleBuildDirectory/controlFlowGraph.ps,
%                    height=1.3\linewidth,width=1.0\linewidth,angle=0}}
\includegraphics[scale=0.7]{\TutorialExampleBuildDirectory/defuseAnalysis_pic1}
\caption{Def-Use graph for example program.}
\label{Tutorial:exampledefuse}
\end{figure}

Another usage of the def-use analysis is to determine which variables actually
are defined at each CFN. The following function allows to query a CFN for
all its variables (SgInitializedNames) and the positions those variables are defined
(SgNode):

\begin{verbatim}
std::multimap <SgInitializedName*, SgNode*> getDefMultiMapFor(SgNode*)
std::multimap <SgInitializedName*, SgNode*> getUseMultiMapFor(SgNode*)
\end{verbatim}

All public functions are described in \emph{DefuseAnalysis.h}. To use the def-use 
analysis, one needs to create an object of the class DefUseAnalysis and execute
the run function. After that, the described functions above help to evaluate 
definition and usage for each CFN.


%--------------------
% Liao 5/27/2010. Added a new section here
\clearpage
\section{Liveness Analysis}
Liveness analysis is a classic data flow analysis performed by compilers to
calculate for each program point the variables that may be potentially read
before their next write (re-definition).
A variable is live at a point in a program's execution path
if it holds a value that may be needed in the future. 

Fig.~\ref{Tutorial:livenessCode} shows an example program of how the
liveness analysis
is called in a ROSE-based translator. 
It generates a dot graph showing def/use information within
a control flow graph, alone with live-in and live-out variables. 
This program (named as \textit{livenessAnalysis})
is installed under \textit{ROSE\_INST/bin} as a standalone tool for users
to experiment the liveness analysis of ROSE.

\begin{figure}[!h]
{\indent
{\mySmallFontSize
% Do this when processing latex to generate non-html (not using latex2html)
\begin{latexonly}
   \lstinputlisting{\TutorialExampleDirectory/livenessAnalysis.C}
\end{latexonly}

% Do this when processing latex to build html (using latex2html)
\begin{htmlonly}
   \verbatiminput{\TutorialExampleDirectory/livenessAnalysis.C}
\end{htmlonly}

% end of scope in font size
}
% End of scope in indentation
}
\caption{Example source code using liveness analysis}
\label{Tutorial:livenessCode}
\end{figure}

Figure~\ref{Tutorial:exampleLivenessCFG} shows control flow graph with live
variable information for the
code(Fig.~\ref{Tutorial:livenessCode}) running on the input
code(Fig.~\ref{Tutorial:exampleLivenessInput}). 

\begin{figure}[!h]
{\indent
{\mySmallFontSize

% Do this when processing latex to generate non-html (not using latex2html)
\begin{latexonly}
   \lstinputlisting{\TutorialExampleDirectory/inputCode_livenessAnalysis.c}
\end{latexonly}

% Do this when processing latex to build html (using latex2html)
\begin{htmlonly}
   \verbatiminput{\TutorialExampleDirectory/inputCode_livenessAnalysis.c}
\end{htmlonly}

% end of scope in font size
}
% End of scope in indentation
}
\caption{Example input code.}
\label{Tutorial:exampleLivenessInput}
\end{figure}


\begin{figure}
\includegraphics[scale=0.7]{\TutorialExampleBuildDirectory/liveness}
\caption{Control flow graph annotated with live variables for example program.}
\label{Tutorial:exampleLivenessCFG}
\end{figure}

%\begin{figure}[!h]
%{\indent
%{\mySmallFontSize
%
%% Do this when processing latex to generate non-html (not using latex2html)
%\begin{latexonly}
%   \lstinputlisting{\TutorialExampleBuildDirectory/defuseAnalysis.out}
%\end{latexonly}
%
%% Do this when processing latex to build html (using latex2html)
%\begin{htmlonly}
%   \verbatiminput{\TutorialExampleBuildDirectory/defuseAnalysis.out}
%\end{htmlonly}
%
%% end of scope in font size
%}
%% End of scope in indentation
%}
%\caption{Output of the program}
%\label{Tutorial:defuse.output}
%\end{figure}


\subsection{Access live variables}
After calling liveness analysis, one can access live-in and live-out variables from a translator
based on the virtual control flow graph node.
Figure~\ref{Tutorial:livenessCode2} shows an example function which
retrieves the live-in and live-out variables for a for loop.
The code accesses the CFG node (showing in
Figure~\ref{Tutorial:exampleLivenessCFG}) of a for statement and retrieve
live-in variables of the true edge's target node as the live-in variables
of the loop body. 
Similarly, the live-out variables of the loop are obtained by getting
the live-in variables of the node right after the loop (target node of the
false edge).

\begin{figure}[!h]
{\indent
{\mySmallFontSize
% Do this when processing latex to generate non-html (not using latex2html)
\begin{latexonly}
   \lstinputlisting{\TutorialExampleDirectory/livenessAnalysis-2.C}
\end{latexonly}

% Do this when processing latex to build html (using latex2html)
\begin{htmlonly}
   \verbatiminput{\TutorialExampleDirectory/livenessAnalysis-2.C}
\end{htmlonly}

% end of scope in font size
}
% End of scope in indentation
}
\caption{Example code retrieving live variables based on virtual control flow graph}
\label{Tutorial:livenessCode2}
\end{figure}




