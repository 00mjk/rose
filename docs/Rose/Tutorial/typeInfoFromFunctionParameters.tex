% \chapter{Getting Type Information from Function Parameters}
\chapter{Function Parameter Types}

   The analysis of functions often requires the query of the
function types.  This tutorial example shows how to obtain 
the function parameter types for any function.  Note that functions
also have a type which is based on their signature, a combination
of their return type and functions parameter types.  Any functions 
sharing the same return type and function parameter types have the 
same function type (the function type, a SgFunctionType IR node, 
will be shared between such functions).

\begin{figure}[!h]
{\indent
{\mySmallFontSize


% Do this when processing latex to generate non-html (not using latex2html)
\begin{latexonly}
   \lstinputlisting{\TutorialExampleDirectory/typeInfoFromFunctionParameters.C}
\end{latexonly}

% Do this when processing latex to build html (using latex2html)
\begin{htmlonly}
   \verbatiminput{\TutorialExampleDirectory/typeInfoFromFunctionParameters.C}
\end{htmlonly}

% end of scope in font size
}
% End of scope in indentation
}
\caption{Example source code showing how to get type information from function parameters.}
\label{Tutorial:exampleTypeInfoFromFunctionParameters}
\end{figure}

   Figure~\ref{Tutorial:exampleTypeInfoFromFunctionParameters} shows a translator which
reads an application (shown in
figure~\ref{Tutorial:exampleInputCode_TypeInfoFromFunctionParameters}) 
and outputs information about the function parameter types for each function
%(09/10/2019) Pei-Hung comment this out because example output cannot be generated
%,shown in figure~\ref{Tutorial:exampleOutput_TypeInfoFromFunctionParameters}
.
This information includes the order of the function declaration in the global
scope, and name of the function, and the types of each parameter declared in 
the function declaration.

Note that there are a number of builtin functions defined as part of the
GNU g++ and gcc compatibility and these are output as well.  These are
marked as compiler generated functions within ROSE.  The code shows how
to differentiate between the two different types.  Notice also that 
instantiated template functions are classified as {\em compiler generated}.


\begin{figure}[!h]
{\indent
{\mySmallFontSize


% Do this when processing latex to generate non-html (not using latex2html)
\begin{latexonly}
   \lstinputlisting{\TutorialExampleDirectory/inputCode_TypeInfoFromFunctionParameters.C}
\end{latexonly}

% Do this when processing latex to build html (using latex2html)
\begin{htmlonly}
   \verbatiminput{\TutorialExampleDirectory/inputCode_TypeInfoFromFunctionParameters.C}
\end{htmlonly}

% end of scope in font size
}
% End of scope in indentation
}
\caption{Example source code used as input to typeInfoFromFunctionParameters.C.}
\label{Tutorial:exampleInputCode_TypeInfoFromFunctionParameters}
\end{figure}

\commentout{
\begin{figure}[!h]
{\indent
{\mySmallFontSize


% Do this when processing latex to generate non-html (not using latex2html)
\begin{latexonly}
   \lstinputlisting{\TutorialExampleBuildDirectory/typeInfoFromFunctionParameters.out}
\end{latexonly}

% Do this when processing latex to build html (using latex2html)
\begin{htmlonly}
   \verbatiminput{\TutorialExampleBuildDirectory/typeInfoFromFunctionParameters.out}
\end{htmlonly}

% end of scope in font size
}
% End of scope in indentation
}
\caption{Output of input to typeInfoFromFunctionParameters.C.}
\label{Tutorial:exampleOutput_TypeInfoFromFunctionParameters}
\end{figure}
}

