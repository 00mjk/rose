
\chapter{ Requirements, Installation and Testing }
\label{Requirements_Installation_Testing}

%  Purpose:
% \begin{itemize}
%    \item A. What you need to get started
%    \item B. Preparing the working copy
% \end{itemize}
% \begin{center}
% *********************  \newline
% \end{center}
% \vspace{0.25in}

   This chapter contains the software and hardware requirements of ROSE.
Additionally, it details the installation of the software.

\section{Requirements and Options}
   ROSE has been developed initially on the Sun workstations and later on Linux.
All present development work is done on Linux platforms using the GNU g++ 3.x 
compilers.

\subsection{What Hardware you require}
   We have not addressed portability issues for Sage III, but EDG has addressed 
portability issues for their C++ front end and it is available on all possible 
platforms (that I know of).  ROSE is currently developed on a Linux/Intel platform
and works with all modern versions of the GNU compilers. At a later point we will 
address portability of ROSE onto other platforms: mostly likely IBM AIX will be 
next (at some point).

\subsection{What Software you require}
   You will require a C++ compiler to compile ROSE; there is no way around it.
We use the GNU g++ compiler, and you should use version 3.x with ROSE.
At the moment you will additionally require a free research license from 
EDG as well, until we distribute binary versions of their software.

{\bf Use of Optional Hardware and Software:}
   ROSE does not presently require any special software or hardware, but additional
functionality is possible if you have additional freely available software:
\begin{itemize}
   \item Doxygen \\
          Visit {\it www.doxygen.org} for details and to download software.
   \item DOT (GraphViz) \\
          Visit {\it www.graphviz.org} for details and to download software.
   \item MySQL \\
          Visit {\it www.mysql.com} for details and to download software.
          Current version of MySQL (as of 1/12/2003) can use g++ version 3.2.2, 
          and not version 3.3.2.
\end{itemize}

\section{How to Install ROSE}

   The process is:
\begin{itemize} 
   \item Type {\tt <path-to-Rose-source-code-directory>/configure;} in a separate directory
         (ROSE must be configured in a separate directory from the source code).
   \item type {\tt make} to build the source code.
   \item Type {\tt make check} to run internal tests (which we use) to test your build.  
   \item Type {\tt make install} to install the build library for more convenient use.
   \item Type {\tt make rose-docs} to build the documentation (requires Doxygen).
   \item Type {\tt make latex-docs} to build this documentation.
\end{itemize}

   If building the development version of ROSE (available only from CVS, not
what we package as a ROSE distribution; e.g. ROSE-0.7.0a.tar.gz) then the README
file in the top level directory has the instructions for how to get started. And
the README\_CHECKIN file has the instructions for our standard testing process
to be done before you check anything in.  NOTE: Get permission from the ROSE
Development Team before you make your first checkin!

Only a limited number of machines are supported currently (any Linux workstations).
A wider range will be supported later (the full range of platforms that Overture and other
projects work on within LLNL and SciDAC).

\section{Testing the ROSE Installation}
     A set of test programs is available.  More will be written about this
in later versions of the manual.  Type {\tt make check} to run your build version of ROSE
using these test codes.  Several years of contributed bug reports and internal test
codes have been accumulated in the {\tt ROSE/TESTS } directory.

\section{Getting help}
     A mailing list is setup which you may use to ask for help from the ROSE development 
team {\it casc-rose.llnl.gov}.  You may also send email directory to 
{\it dquinlan@llnl.gov}.  Any bug report you submit will be added as a test code and used
to test future versions of ROSE.

\section{Submitting a Bug Report}
     The rule is simple, the better quality the bug report, the higher priority it
gets.  All good bug reports include a simple (very simple) example that demonstrates
the bug and only that bug so that it is clearly reproducible.  We welcome your submission
of good quality bug reports.



