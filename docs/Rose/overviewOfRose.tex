\chapter{The ROSE Infrastructure}

\label{overviewOfRose:overviewOfRose}

\fixme{This chapter is incomplete and will be filled in by Rich.}

This chapter includes many details about how to
use ROSE. This is a chapter that I think Rich would like to write
(or at least start and contribute to it).  It will cover:
\begin{itemize}
   \item ROSE documentation
   \begin{itemize}
      \item ROSE features
      \item Language front-ends
      \item Code generation
      \item ROSE directory organization
   \end{itemize}
   \item SAGE III Overview
   \begin{itemize}
      \item Figure to describe IR node hierarchy
      \item description of categories of IR nodes
      \item relationship to C++ grammar
      \item Integrate with current "Sage III Intermediate Representation" (current Chapter 10).
      \item Properties of IR nodes (No null pointers, what few pointers are NULL, what IR nodes are shared,defining/non-defining declarations)
      \item AST Merge Mechanism
      \item File I/O
      \item How modifiers are handled
   \end{itemize}
\end{itemize}


\section{Introduction}
     This chapter was requested by several people who wanted to understand 
how ROSE was designed and implemented.

\section{Design}
     This section will cover the design goals etc. of ROSE.
\section{Directory Structure}
This section presents the directory structure of the ROSE project.

\fixme{Add more detail about each directory.}


% the directory figure is too hug to be useful, also as more stuff being
% added in, LaTex complains Dimension too large quite often. Liao
\begin{comment}
\begin{latexonly}
% Do this when processing latex to generate non-html (not using latex2html)
\begin{figure}[tb]
%\centerline{\epsfig{file=roseDirectoryMap.ps,height=1.0\linewidth,width=1.4\linewidth,angle=270}}
%\centerline{\psfig{file=roseDirectoryMap.pdf,height=1.0\linewidth,width=1.4\linewidth,angle=270}}
\includegraphics[scale=0.12,angle=270]{roseDirectoryMap}
\caption{Directory structure of ROSE project.}
\end{figure}
\end{latexonly}

\begin{htmlonly}
% Do this when processing latex to generate non-html (not using latex2html) 
% Note that specification of the angle causes latex2html to SKIP output of 
% the figure and the "height=1.5\linewidth,width=4.0\linewidth" is also required
% (I suspect this is a bug in latex2html).
\begin{figure}[tb]
%\centerline{\epsfig{file=ROSE.ps,height=1.0\linewidth,width=1.0\linewidth,angle=0}}
%\centerline{\epsfig{file=roseDirectoryMap.ps,height=1.5\linewidth,width=4.0\linewidth,angle=0}}
%\centerline{\epsfig{file=roseDirectoryMap.ps,height=1.5\linewidth,width=4.0\linewidth}}
%\centerline{\epsfig{file=roseDirectoryMap.ps}}

%\centerline{\psfig{file=roseDirectoryMap.pdf,height=1.5\linewidth,width=4.0\linewidth}}
\includegraphics[scale=0.25,angle=270]{roseDirectoryMap}
\caption{Directory structure of ROSE project.}
\label{designAndImplementation:directoryStructure}
\end{figure}
\end{htmlonly}

   This figure is formed from a Perl script ({\tt ROSE/scripts/lsvis}) which generates
the graph when the documentation is built.

\end{comment}

\section{Implementation of ROSE}
     This section will be added to in the future.
     
\section{Implementation of ROSETTA}
     This section will be added to in the future.

