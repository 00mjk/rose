\chapter{CUDA and OpenCL}

\label{heterogeneous:heterogeneous}

{\em {\bf Current Status (May 2011) }: ROSE presently supports the generation of both CUDA and OpenCl.
    Thus ROSE translators may be built that read C/C++/Fortran and generate
    CUDA or OpenCL (several have been built by external research groups).
    We are now able to read CUDA and still working on OpenCL (which is not distribute yet).}
\vspace{0.25in}

This chapter details the support of C for CUDA \footnote{Compute Unified Device Architecture, NVidia architecture model for their devices} and OpenCL \footnote{Open Compute Language,
an open standard to program heterogeneous system, managed by the Khronos Group}.

	\section{How to use CUDA and OpenCL in ROSE}
	
	Both CUDA and OpenCL are enable in ROSE from two configuration options.
\begin{itemize}
	\item \emph{--enable-(cuda/opencl)}: to enable the internal support of these languages.
	\item \emph{--enable-edg-(cuda/opencl)}: to enable the frontend support (parsing)
\end{itemize}
	As the support of CUDA/OpenCL parsing have been developt in EDG 4.0, you also need to use the option \emph{--enable-edg-version=4.0}.\\
	\\
	If you use the binary version of EDG (most common case for external users), you do not need the option \emph{--enable-edg-(cuda/opencl)} as to avoid 
	multiple versions of the binaries we directly distribute EDG 4.0 with CUDA/OpenCL support (the option may still be needed to build some sub-project using CUDA/OpenCL support).\\
	\\
	To build and test the CUDA/OpenCL support with your version of ROSE: \\
	\$ ROSE\_SRC\_TREE/configure --enable-edg-version=4.0 --enable-cuda --enable-edg-cuda --enable-opencl --enable-edg-opencl [...] \\
	\$ make \\
	\$ make -Ctests/CompileTests/NewEDGInterface\_C\_tests check \# tests EDG 4.0 \\
	\$ make -Ctests/CompileTests/CudaTests check \# tests CUDA support \\
	\$ make -Ctests/CompileTests/OpenCLTests check \# tests OpenCL support \\
	\\
	You can look on the test directories to see the current state of CUDA/OpenCL support.
	
	\section{IR adaptations}
	
		Both CUDA and OpenCL use extensions of C ANSI 99, these extensions need to be catch by SageIII. We will describe theses adaptations below.

		\subsection{C for CUDA}
		
			The C for CUDA extensions of C ANSI 99 is describe in the CUDA Programming Guide Version 3.0, appendix B and C.\\
			CUDA programs are usually in .cu files.
		
			\subsubsection{Function and storage modifiers}

				\paragraph{Function Modifier}

As CUDA target multiple kind of device, it provide a way to specify whether a function executes on the host or on the device and whether it is
callable from the host or from the device. For this, it use some function modifiers listed below, with caller and target:
\begin{itemize}
	\item \_\_host\_\_: from CPU to CPU;
	\item \_\_device\_\_: from GPU to GPU;
	\item \_\_global\_\_: from CPU to GPU, it define a \emph{kernel}.
\end{itemize}
We have modified \emph{SgFunctionModifier} by adding to \emph{function\_modifier\_enum}:
\begin{itemize}
	\item e\_cuda\_device
	\item e\_cuda\_kernel
	\item e\_cuda\_host
\end{itemize}

				\paragraph{Storage Modifier}	

For variables, \emph{SgStorageModifier} is a good candidate but it can only address one modifier. That's a problem in the case of dynamically allocated shared memory.
We handled it by considering following cases for \emph{storage\_modifier\_enum}:
		
\begin{itemize}
	\item \_\_device\_\_ $\rightarrow$ e\_cuda\_global
	\item (\_\_device\_\_) \_\_constant\_\_ $\rightarrow$ e\_cuda\_constant
	\item (\_\_device\_\_) \_\_shared\_\_ $\rightarrow$ e\_cuda\_shared
	\item extern (\_\_device\_\_) \_\_shared\_\_ $\rightarrow$ e\_cuda\_dynamic\_shared
\end{itemize}

			\subsubsection{Execution configuration associated at a kernel call}

CUDA introduce too a special way to call kernels (\_\_global\_\_ functions):
\begin{verbatim} kernel<<<Dg, Db, Ns, S>>>(parameter); \end{verbatim}
Where:
\begin{itemize}
	\item Dg is the grid dimensions
	\item Db is blocks dimensions
	\item Ns is shared memory size (optional)
	\item S is a stream (optional)
\end{itemize}
We catched it by adding two new nodes in the IR: 
\begin{itemize}
	\item \emph{SgCudaKernelCallExp} at the same level than \emph{SgFunctionCallExp}
	\item \emph{SCudagKernelExecConfig}, an attribute of \emph{SgCudaKernelCallExp}, catch the execution configuration (grid, blocks, shared memory dynamic allocation and stream).
\end{itemize}

			\subsubsection{Build-in Types, Variables and Functions}
			
All built-in CUDA objects are defined in ROSE/config/rose\_edg\_required\_macros\_and\_functions.h.in .
			
				\paragraph{Types}
			
					\subparagraph{Vectors}
					
CUDA use vectors types extended from common types.
\begin{table}[!h]
	\caption{Vector types defined by CUDA}
	\center
	\begin{tabular}{|c|c|c|c|} \hline
	char1 & char2 & char3 & char4 \\\hline
	uchar1 & uchar2 & uchar3 & uchar4 \\\hline
	short1 & short2 & short3 & short4 \\\hline
	ushort1 & ushort2 & ushort3 & ushort4 \\\hline
	int1 & int2 & int3 & int4 \\\hline
	uint1 & uint2 & uint3 & uint4 \\\hline
	long1 & long2 & long3 & long4 \\\hline
	ulong1 & ulong2 & ulong3 & ulong4 \\\hline
	longlong1 & longlong2 &  & \\\hline
	float1 & float2 & float3 & float4 \\\hline
	double1 & double2 &  & \\\hline
	\end{tabular}
	\label{heterogeneous:tabCudaVector}
\end{table}

For examples, \emph{int2} is equivalent to:
\begin{verbatim}struct int2
{
	int x, y;
};\end{verbatim}
3rd and 4th dimensions are access by z and w fields.\\

A constructor is associated with all build-in type: make\_\emph{type}
\begin{verbatim}int2 make_int2(int x, int y);\end{verbatim}\\
			
					\subparagraph{Others}

Based on \emph{uint3}, \emph{dim3} is used to specify dimensions. Any unspecified components of a \emph{dim3} instance is initialized to 1.\\

Some others types, like cudaError\_t, are not documented as built-in types but \emph{nvcc}\footnote{NVidia compileur for CUDA} don't need that the 
API header was explicitly included.

				\paragraph{Variables}

Some variables that can't be nor affected nor addressed by pointers:
\begin{table}[!h]
	\caption{CUDA's built-in variables}
	\center
	\begin{tabular}{|c|c|c|} \hline
	Variable & Type & Purpose \\\hline\hline
	gridDim   & dim3  & dimension of the grid \\\hline
	blockIdx  & uint3 & block index within the grid \\\hline
	blockDim  & dim3  & dimension of the block \\\hline
	threadIdx & uint3 & thread index within the block \\\hline
	warpSize  & int   & number of threads by warp \\\hline
	\end{tabular}
	\label{heterogeneous:tabCudaVariable}
\end{table}

				\paragraph{Functions}

					\subparagraph{Synchronization functions}
					
\begin{itemize}
	\item void \_\_threadfence\_block();
	\item void \_\_threadfence();
	\item void \_\_threadfence\_system();
	\item void \_\_syncthreads();
	\item int \_\_syncthreads\_count(int predicate);
	\item int \_\_syncthreads\_and(int predicate);
	\item int \_\_syncthreads\_or(int predicate);
\end{itemize}

					\subparagraph{Mathematical functions}
					
\begin{itemize}
	\item float \_\_fdividef(float x,float y);
	\item float \_\_sinf(float x);
	\item float \_\_cosf(float x);
	\item float \_\_tanf(float x);
	\item float \_\_sincosf(float x, float * sin, float * cos);
	\item float \_\_logf(float x);
	\item float \_\_log2f(float x);
	\item float \_\_log10f(float x);
	\item float \_\_expf(float x);
	\item float \_\_exp10f(float x);
	\item float \_\_powf(float x,float y);
\end{itemize}

					\subparagraph{Texture functions}
					
\begin{verbatim}template<class Type>\end{verbatim}
\begin{itemize}
	\item Type tex1Dfetch(texture<Type, 1, cudaReadModeElementType> texRef, int x);
	\item float\emph{n} tex1Dfetch(texture<Type\emph{n}, 1, cudaReadModeNormalizedFloat> texRef, int x);
\end{itemize}

\begin{verbatim}template<class Type, enum cudaTextureReadMode readMode>\end{verbatim}
\begin{itemize}
	\item Type tex1D(texture<Type, 1, readMode> texRef, float x);
	\item Type tex2D(texture<Type, 1, readMode> texRef, float x, float y);
	\item Type tex3D(texture<Type, 1, readMode> texRef, float x, float y, float z);
\end{itemize}

					\subparagraph{Time functions}

\begin{itemize}
	\item clock\_t clock();
\end{itemize}

					\subparagraph{Atomic functions}

\begin{itemize}
	\item \emph{type} atomicAdd(\emph{type} * address, \emph{type} val); with \emph{type}: int, unsigned int, unsigned long long int and float.
	\item \emph{type} atomicSub(\emph{type} * address, \emph{type} val); with \emph{type}: int and unsigned int.
	\item \emph{type} atomicExch(\emph{type} * address, \emph{type} val); with \emph{type}: int, unsigned int, unsigned long long int and float.
	\item \emph{type} atomicMin(\emph{type} * address, \emph{type} val); with \emph{type}: int and unsigned int.
	\item \emph{type} atomicMax(\emph{type} * address, \emph{type} val); with \emph{type}: int and unsigned int.
	\item \emph{type} atomicInc(\emph{type} * address, \emph{type} val); with \emph{type}: unsigned int.
	\item \emph{type} atomicDec(\emph{type} * address, \emph{type} val); with \emph{type}: unsigned int.
	\item \emph{type} atomicCAS(\emph{type} * address, \emph{type} compare, \emph{type} val); with \emph{type}: int, unsigned int and unsigned long long int
	\item \emph{type} atomicAnd(\emph{type} * address, \emph{type} val); with \emph{type}: int and unsigned int.
	\item \emph{type} atomicOr(\emph{type} * address, \emph{type} val); with \emph{type}: int and unsigned int.
	\item \emph{type} atomicXor(\emph{type} * address, \emph{type} val); with \emph{type}: int and unsigned int.
\end{itemize}

					\subparagraph{Warp Vote functions}

\begin{itemize}
	\item int \_\_all(int predicate);
	\item int \_\_any(int predicate);
	\item unsigned int \_\_ballot(int predicate);
\end{itemize}

					\subparagraph{Profiler Counter functions}

\begin{itemize}
	\item void \_\_prof\_trigger(int counter);
\end{itemize}

					\subparagraph{More Maths functions}

See \underline{Appendix C - Mathematical Functions} of \underline{CUDA Programming Guide Version 3.0}.
		
		\subsection{OpenCL}
		
OpenCL language only address devices, the host (general purpose CPU) code is design in C/C++ and use OpenCL API. OpenCL programs are compiled on the
fly and stored on C characters strings, that may be a problem but usually theses string are load from a separate .cl file. For more convenience,
we do the assumption that all OpenCL programs are stored in separate files and never dynamically modified.
		
			\subsubsection{Function and storage modifiers}

				\paragraph{Function Qualifiers}

\begin{itemize}
	\item \_\_kernel
	\item \_\_attribute\_\_((vec\_type\_hint(\emph{type})))
	\item \_\_attribute\_\_((work\_group\_size\_hint(X, Y, Z)))
	\item \_\_attribute\_\_((reqd\_work\_group\_size(X, Y, Z)))
\end{itemize}

\_\_kernel qualifier can be catch by adding \emph{e\_ocl\_kernel} in the \emph{function\_modifier\_enum} of \emph{SgFunctionModifier}.\\

The \_\_attribute\_\_(...) are linked to \_\_kernel qualifier. In OpenCL, the  \emph{work group} is a set of  \emph{work items} (instance of
the kernel\footnote{see: OpenCL Specification v1.0, section 3.2 Execution Model}) that provide a more coarse-grained decomposition of the
index space associate to the \emph{global work}.\\
We had two ways to implement it: a new modifier or some modification on \emph{SgFunctionModifier}.\\
We choosed to add in \emph{SgFunctionModifier} three flags: \emph{e\_ocl\_vec\_type\_hint}, \emph{e\_ocl\_work\_group\_hint}, 
\emph{e\_ocl\_work\_group\_req}; and two fields: \emph{SgType * ocl\_vec\_type} and \emph{struct \{ unsigned int x, y, z; \} ocl\_work\_group\_size}.

				\paragraph{Address Space Qualifier}
				
They are used to specify where a variable is stored in the device.
\begin{itemize}
	\item \_\_global accessible from all the device
	\item \_\_local accessible for a work-group only
	\item \_\_constant read-only variable accessible from all the device
	\item \_\_private accessible only from one "thread"
\end{itemize}

We used \emph{SgStorageModifier} to store this information by adding to \emph{storage\_modifier\_enum}:
\begin{itemize}
	\item \_e\_opencl\_global
	\item \_e\_opencl\_local
	\item \_e\_opencl\_constant
	\item \_e\_opencl\_private
\end{itemize}

				\paragraph{Image Access Qualifiers}

\begin{itemize}
	\item \_\_read\_only
	\item \_\_write\_only
	\item \_\_read\_write
\end{itemize}

{\color{red} \emph{SgAccessModifier} fill good by the name but is dedicated to class...}

			\subsubsection{Build-in Types}

Many Built-in types: scalars, vectors, matrix and others, like image, event or complex.

				\paragraph{Scalars}
				
OpenCL introduce alias for unsigned types (uint, uchar, ...) and some scalars types: half, size\_t, ptrdiff\_t, intptr\_t, uintptr\_t.

				\paragraph{Vectors}
				
OpenCL define vector from scalars: char, uchar, short, ushort, int, uint, long, ulong and float. These vector types are noted type\emph{n} where n can
be 2, 4, 8 or 16.
For this type, we use the same mechanism than for CUDA vector types. {\color{red} We may have a problem with vector components (section 6.1.7 
of OpenCL Specification v1.0)}
\begin{verbatim}
float4 a    = (float4)(1.0f, 2.0f, 3.0f, 4.0f);
float4 swiz = a.wzyx;     // swiz = (4.0f, 3.0f, 2.0f, 1.0f)
float2 low  = a.lo;       // low  = (4.0f, 3.0f)
float8 b;
b.hi = a;
b.lo = swiz;              // b    = (4.0f, 3.0f, 2.0f, 1.0f, 1.0f, 2.0f, 3.0f, 4.0f)
\end{verbatim}

				\paragraph{Others}

Other built-in types are image2d\_t, image3d\_t, sampler\_t and event\_t.

				\paragraph{Reserved}

See section 6.1.4 of OpenCL Specification v1.0

			\subsubsection{Build-in Functions}
			
				\paragraph{Work-Item Functions}
				
\begin{itemize}
	\item uint get\_work\_dim ()
	\item size\_t get\_global\_size (uint dimindx)
	\item size\_t get\_global\_id (uint dimindx)
	\item size\_t get\_local\_size (uint dimindx)
	\item size\_t get\_local\_id (uint dimindx)
	\item size\_t get\_num\_groups (uint dimindx)
	\item size\_t get\_group\_id (uint dimindx)
\end{itemize}
			
				\paragraph{Image Functions}
				
\begin{itemize}
	\item float4 read\_imagef (image2d\_t image, sampler\_t sampler, int2 coord)
	\item float4 read\_imagef (image2d\_t image, sampler\_t sampler, float2 coord)
	\item int4 read\_imagei (image2d\_t image, sampler\_t sampler, int2 coord)
	\item int4 read\_imagei (image2d\_t image, sampler\_t sampler, float2 coord)
	\item unsigned int4 read\_imageui (image2d\_t image, sampler\_t sampler, int2 coord)
	\item unsigned int4 read\_imageui (image2d\_t image, sampler\_t sampler, float2 coord)
	\item void write\_imagef (image2d\_t image, int2 coord, float4 color)
	\item void write\_imagei (image2d\_t image, int2 coord, int4 color)
	\item void write\_imageui (image2d\_t image, int2 coord, unsigned int4 color)
	\item float4 read\_imagef (image3d\_t image, sampler\_t sampler, int4 coord)
	\item float4 read\_imagef (image3d\_t image, sampler\_t sampler, float4 coord)
	\item int4 read\_imagei (image3d\_t image, sampler\_t sampler, int4 coord)
	\item int4 read\_imagei (image3d\_t image, sampler\_t sampler, float4 coord)
	\item unsigned int4 read\_imageui (image3d\_t image, sampler\_t sampler, int4 coord)
	\item unsigned int4 read\_imageui (image3d\_t image, sampler\_t sampler, float4 coord)
	\item int get\_image\_width (image2d\_t image)
	\item int get\_image\_width (image3d\_t image)
	\item int get\_image\_height (image2d\_t image)
	\item int get\_image\_height (image3d\_t image)
	\item int get\_image\_depht (image3d\_t image)
	\item int get\_image\_channel\_data\_type (image2d\_t image)
	\item int get\_image\_channel\_data\_type (image3d\_t image)
	\item int get\_image\_channel\_order (image2d\_t image)
	\item int get\_image\_channel\_order (image3d\_t image)
	\item int2 get\_image\_dim (image2d\_t image)
	\item int4 get\_image\_dim (image3d\_t image)
\end{itemize}
			
				\paragraph{Synchronization Functions}
				
\begin{itemize}
	\item void barrier (cl\_mem\_fence\_flags flags)
\end{itemize}
			
				\paragraph{Explicit Memory Fence Functions}
				
\begin{itemize}
	\item void mem\_fence (cl\_mem\_fence\_flags flags)
	\item void read\_mem\_fence (cl\_mem\_fence\_flags flags)
	\item void write\_mem\_fence (cl\_mem\_fence\_flags flags)
\end{itemize}
			
				\paragraph{Miscellaneous Functions}
				
\begin{itemize}
	\item event\_t async\_work\_group\_copy (\_\_local \emph{gentype} *dst, const \_\_global \emph{gentype} *src, size\_t num\_elements, event\_t event)
	\item event\_t async\_work\_group\_copy (\_\_global \emph{gentype} *dst, const \_\_local \emph{gentype} *src, size\_t num\_elements, event\_t event)
	\item void wait\_group\_events(int num\_events, event\_t * event\_list)
	\item void prefetch (const \_\_global \emph{gentype} *p, size\_t num\_elements)
\end{itemize}
Where \emph{gentype} can be scalar or vector of char, uchar, short, ushort, int, uint, long, ulong or float.
				\paragraph{Maths Functions}

		
%	\section{Some ROSE's tools for heterogeneous computing}
%
%		\subsection{GPU}
%		
%			\subsubsection{CUDA computation capability check}
%
%For a given CUDA code, extract:
%\begin{itemize}
%	\item CUDA API needed
%	\item GPU needed
%\end{itemize}
%
%			\subsubsection{CUDA computation capability refactor}
%
%Downgrade a CUDA program to use old GPU/API.
%
%			\subsubsection{Performance evaluator}
%
%Look for commons bottleneck.
%
%			\subsubsection{Performance improver}
%			
%\begin{itemize}
%	\item Remove commons bottleneck.
%	\item Use polyhedron model.
%\end{itemize}
