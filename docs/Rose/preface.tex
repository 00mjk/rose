
\chapter*{Preface}

%  Purpose:
% \begin{itemize}
%    \item a. Welcome 
%    \item b. Program features
%    \item c. Program benefits
% \end{itemize}
% \begin{center}
% *********************  \newline
% \end{center}
% \vspace{0.25in}

   Welcome to the ROSE Compiler Framework Project.  The purpose of this project is to 
provide a mechanism for construction of specialized source-to-source translators 
(sometime referred to less precisely as {\em preprocessors}).  ROSE provides 
simple programmable mechanisms to read and rewrite the abstract syntax trees generated 
by separate compiler front-ends.  ROSE includes the Edison Design Group (EDG) 
front-end (in binary form within public
distributions), and is internally based upon SAGE III, thus ROSE is presently specific 
to the generation of C and C++ source-to-source based compilers ({\em translators}, 
more precisely).  Other language front-ends may be appropriate to add to ROSE in 
the future (current work with Rice University is focused on the addition of Open64's
front-end to ROSE as part of support for FORTRAN 90).

   ROSE makes it easy to build complex source-to-source translator
(preprocessor) tools, and thus supports research work in many areas:
\begin{itemize}
   \item Performance Optimization
   \item General Program Transformations
   \item Instrumentation
   \item Program Analysis
   \item Interface Generation
   \item Automated Check-pointing
   \item Software Security Analysis
   \item Software Verification
   \item Automated Unit Test Generation
   \item ... and much more ...
\end{itemize}

