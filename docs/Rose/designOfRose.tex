
\chapter{ Design of ROSE }
   Here we lay out the general design of ROSE.  What the goals are, what it is
good for and what it is inappropriate for, and how it works internally.

\section{Overview}
   We built on top of SAGE II which was never published or finished by the authors
(Dennis Gannon and Carl Kesselman, and others).  It was built on top of
Sage++.  In each case substantial changes occurred.  Although SAGE III
is very much a complete rewrite (accomplished incrementally over the years)
I have wanted to retain the SAGE name to acknowledge their original work.
Dennis suggested that we refer to our work as SAGE III, and so we
have done so as some attempt to be clear.  There is an original paper on Sage++
(which included nothing from us).  I have thought it was inappropriate
to publish a paper on SAGE III (unless it was done with the original authors),
though more information, history, and background is appropriate for inclusion 
into the manual.

   SAGE III is the Intermediate Representation (IR) used within ROSE.  
ROSE adds sophisticated traversal mechanisms, AST rewrite mechanisms, loop
optimizations, and finished off the work started in SAGE II (template support, exception
handling, new C++ modifiers, bug fixes, a rewrite and reorganization, and much more).
Much other work has 
been done in ROSE to make it easy to define source-to-source translators, including 
numerous loop optimizations, and some program analysis (still incomplete).
EDG only builds its own AST and it is difficult to modify it (I think)
and against their license to publish it or tools to manipulate it.  ROSE provides
an IR (SAGE III) that is designed to be modified, is object-oriented, and can be published
(since it completely hides the EDG AST and does not need it after translation to
SAGE).  SAGE is independent of EDG by design. Current work with Rice is using the 
SAGE III IR as a basis for an IR that would work in a similar way with the 
Open64 compiler infrastructure.

   ROSE adds AST traversal mechanisms based on ideas from attribute grammars
and adds an AST rewrite mechanism.  At some point we will have more meaningful
program analysis (part of cooperative work with many other groups), it is in 
these ways that what we have done adds to EDG and
hopefully provides a more useful infrastructure for the automated transformation of
C++ applications beyond that of the design goals of EDG.  I understand that the
design goals of EDG are only to provide a front-end to compiler vendors, a critical
piece but not at all enough by itself for our work on optimization of high-level 
abstractions.  I expect that ROSE will be useful to anyone who would otherwise use 
EDG directly, but certainly it is not required, many people use EDG by itself.
We, like the rest of the world, of course has the deepest respect for EDG.
For our purposes we need something like EDG's front-end to support our own research work
on program optimization. I don't know of any alternatives to using EDG given that 
we have to process complex laboratory C++ applications.

\section{GOAL of ROSE}
   The goal of ROSE is to simplify the development of source-to-source translators.
In general we expect many translators to optimize applications, but numerous
other opportunities exist once a user's application can be laid open for analysis
(AST Processing) and coupled with a programmable rewriting mechanism (AST Rewrite).

\section{GOAL of a ROSE translator}

   The goal here is to provide a means of defining/automating optimizations within
applications. Object-oriented frameworks are of particular interest to us because object-oriented
languages provide a simple means to implement user-defined abstractions.
The definition of such transformations is intended to be as general
as possible. The way to build a translator to automate such transformations
is intended to be as simple as possible.

\section{ How general is the optimization of object-oriented frameworks }

   We seek a general mechanism to define the optimization of object-oriented frameworks.
We are focused upon the interaction of abstractions (read objects in most cases), instead
of the optimization of method functions of a particular object. The most common case within
scientific computing that is within range of this sort of optimization is the interaction
of objects within expressions (commonly array or matrix objects within expressions).

What {\bf CAN} be expected from compile time optimizations
\begin{itemize}
   \item Optimization of expressions between objects defined within one or more 
         object-oriented frameworks. Where the compiler can't see the potential
         for optimizations between objects is where ROSE can make the greatest 
         difference.
\end{itemize}

What {\bf CANNOT} be expected from compile time optimizations
\begin{itemize}
   \item Optimization of non-context sensitive method functions (these can be 
         handled directly within the libraries implementation of the method function)
   \item Optimization of virtual function calls on small objects within
         object-oriented frameworks.  ROSE can only perform static analysis,
         though in principle it could introduce runtime tests, etc.
\end{itemize}

\section{ How ROSE Translators work internally }
   We describe here the internal design of ROSE (short version).

   Focusing only upon what ROSE can do well, we thus focus upon the static
optimization; cases where objects are used within expressions. We will focus less upon
what optimizations to introduce (there has been separate work on that) than how to
specify the optimizations within ROSE.

The introduction of optimizations between abstractions (objects) depends upon:
\begin{itemize}
   \item The semantics of the abstractions (arrays, matrices, etc.) \
         The semantics defines what we are trying to produce (often a sum or product for an array class, etc.).
   \item The syntax of interaction between abstractions. \
         The syntax of the interaction defines the examples that will trigger the optimization.
         For example, the syntax of an array object is such that ``A = B + C'' is defined and so
         the occurrence of this within a user's code (where the objects are of the correct
         type) would trigger the optimization of this statement.
\end{itemize}

   ROSE permits the analysis of the AST generated by an application and allows the AST to
be modified using any additional information (typically additional information from the
semantics of user-defined abstractions). The AST in ROSE is a representation of the 
structure of the application source code (little normalization occurs). The AST includes
completely resolved type information, up to the limitations of static type analysis as
specified by the C++ standard (much of this we get because we use EDG and it implements
the static type analysis).


\section{ROSE Distributions and Binary ROSE Distributions}

   The ROSE distribution differs from the development version of ROSE, which
is checked out from CVS directly and has a number of dependencies on versions
number of autoconf, automake, LaTeX, dot, etc.  ROSE distributions appear in
the form of a tar'd compressed directory with an embedded version number 
(e.g. {\it ROSE-0.8.10a.tar.gz}). These sometimes embed the word {\bf binary} or {\bf source}
to distinguish source code versions (all source code to ROSE including the source code 
to the EDG front-end used in parts of ROSE) from binary versions (all source code to ROSE
but a binary of any EDG specific parts or parts of ROSE which use the EDG header files).
This currently constitutes three small libraries in ROSE.  The source to to as much of
ROSE as possible is freely distributed under BSD license.

   A binary distribution of ROSE refers to the distribution of ROSE that uses a binary
version of the EDG work used by ROSE.  We protect the source code of EDG by saving only 
the binary libraries that we build from the EDG work (the EDG source code and any parts 
of ROSE that use the EDG header files).  

The configure option {\tt --with-edg_source_code=true} builds a version of ROSE which assumes
that the EDG source code is present. It is required to use this option to compile ROSE 
when it is checked out from CVS (since we store the sorce code for all fo ROSE including
EDG in the CVS repository.  Without this option (the default case; no specifying 
{\tt --with-edg_source_code=true} or directly specifying {\tt --with-edg_source_code=false}) the 
configure script will look for existing binary libraries (in either the build tree or the
source tree), if it finds the required library (only searches for libedg33.la) then
is will continue to build the Makefile files so that the {\tt make dist} rule will package up
a distribution and execute the makefile's {\it dist-hook} rules to copy the binary from the 
compile tree to the distribution.  At this point the distribution is a binary distribution
(and contains non of the protected source code for EDG).

The libraries specifically affected are:
\begin{itemize}
   \item {\bf \$(top_builddir)/src/frontend/CxxFrontend/EDG_3.3/src/libedg33.la}
   % \item {\bf \$(top_builddir)/src/frontend/CxxFrontend/EDG_3.3/util/libedgutil.la}
   \item {\bf \$(top_builddir)/src/frontend/CxxFrontend/EDG_SAGE_Connection/libedgSage.la}
\end{itemize}
and are placed into the distribution by the {\tt make dist} rule which builds a ROSE distribution
(often run as part of the {\tt make distcheck} rule).  This is the behavir of {\tt make dist}
when the configure option {\tt --with-edg_source_code=true} is {\bf NOT} used.

Summary of build steps (to build a binary distribution):
\begin{itemize}
   \item {\tt configure --with-edg_source_code=true} \\
          Builds version of ROSE containing EDG source code.
   \item  {\tt make} \\
          Builds all the libraries (make dist may require make to be run prior {\tt make dist}.
   \item  {\tt make dist} \\
          Builds a source code distribution of ROSE (distributable to those with EDG source license)
   \item {\tt configure} \\
          Builds a version of ROSE that will use the pre-built EDG libraries.
   \item {\tt make dist} \\
          Builds a distribution with copies of the required EDG based libraries
          (this is the binary form of ROSE and can be distributed to anyone).
\end{itemize}


