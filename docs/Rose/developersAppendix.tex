\chapter{ Developer's Appendix }
\label{developersAppendix:developersAppendix}

\section{Adding Contributions to ROSE}

   We will be happy to work with you if you want to add new features to ROSE.
We can setup a special SVN branch for you so that you can checkin your work and
also update with our constant work on ROSE (on you schedule).  We can synchronize with 
you to decide when we can review your work and merge your branch into the main trunk.

{\bf The number one most important aspect about any contribution you make is that it
should include test codes that demonstrate the feature and test it within our 
automate test mechanism (the {\em make check} makefile rules).}. Depending upon the 
feature this can include an additional demonstrative example of how it works, such
examples go into the ROSE Tutorial (often as a separate chapter).  Most new work
starts in the {\em Experimental} part of the ROSE Tutorial and is moved forward
in the document over time.

The purpose of the test codes in our automated tests are:
\begin{itemize}
   \item Make sure that future great ideas in ROSE don't break your feature.
   \item Allow us to easily detect maintenance problems as early as possible.
   \item Help us sleep at night knowing that ROSE is really working.
   \item Give everyone else using ROSE confidence in future releases.
\end{itemize}

We take this subject very seriously, since it can be a significant problem.
In the future we will likely not accept contributions that are not accompanied 
by sufficient test codes that demonstrate that they work and will be part of the
automated tests ({\em make check} makefile rule).  If you want to add a new 
feature to ROSE, show us your tests.


%Added by Liao 10/17/2008
\section{Working with the ROSE SVN repositories}
\label{gettingStarted::svn}
We have two subversion repositories for ROSE: an internal one at LLNL and
an external one at SciDAC Outreach Center (using a vendor drop scheme). 
Some tips for using them are gathered in this section. 

\subsection{The External Repository}
If you are our external (non-LLNL) users who make contributions to ROSE,
we highly recommend you to work on a dedicated branch of the external repository. 
We can create the branch for you on request. And you need to apply an account of
the SciDAC Outreach Center to have write access to your branch. 

Here are the steps to have an account with write access to ROSE's branches:
Please follow the link on
\htmladdnormallink{https://outreach.scidac.gov/account/register.php}{https://outreach.scidac.gov/account/register.php}
to fill out a registration form (Project name: ROSE, PI: Daniel Quinlan)
and fax a signed use policies form as instructed on the registration page.
After getting your account, you need to log into the website and go to page
\htmladdnormallink{https://outreach.scidac.gov/projects/rose/}{https://outreach.scidac.gov/projects/rose/}.
Click "Request to join" on the top-right screen to request to join the ROSE
project and we will grant you the write access to your branch. 

Some frequently used commands for ROSE external developers are listed
below:
\begin{itemize}
\item Install your svn client ($>=$1.5.1 is recommended ) with
\textit{libsvn\_ra\_dav} support
(\htmladdnormallink{http://www.webdav.org/neon}{http://www.webdav.org/neon}
and \textit{--with-ssl}) or set the right \textit{LD\_LIBRARY\_PATH} for it
(\textit{libsvn\_ra\_dav-1.so}) if you encounter the following problem:\\
 \textit{svn: Unrecognized URL scheme for
 'https://outreach.scidac.gov/svn/rose/trunk'} 
\item To check out the main trunk, type: \\
\textit{svn checkout https://outreach.scidac.gov/svn/rose/trunk rose}
\item To check out a branch, type: \\
\textit{svn checkout
https://outreach.scidac.gov/svn/rose/branches/branch\_name rose} 
\item Merge the new updates of the main trunk into your working branch. 
Conceptually, svn merge works as two step: diff two revisions and merge the different into a working copy.
So you need to know two revision numbers of the main
trunk:
the first is
the latest revision number of the main trunk from which your branch was
created (or most recently synchronized);
the second is usually the head revision of the main trunk.
\footnote{Subversion 1.5 is said to support svn merge with the head of
a main trunk without explicitly specifying the beginning and end revision
numbers. But this new feature is not mature enough to be used in our work
as our tests showed. We will try to use the new feature later on when it
becomes dependable.}: 

\begin{itemize}
  \item find the revision in which your branch was created or the last
  synchronization point with the trunk:\\ 
         \textit{svn log
         https://outreach.scidac.gov/svn/rose/branches/branch\_name} 
  \item cd local work copy of your branch, do the merge (overlapped merging
  seems possible using subversion 1.5.1), assume the last synchronization
  point(or originating point) is rev 56:\\
        \textit{svn merge --dry-run -r 56:69
        https://outreach.scidac.gov/svn/rose/trunk} \\
        \textit{svn merge -r 56:69
        https://outreach.scidac.gov/svn/rose/trunk} 
   \item Solve conflicts as needed.
   \item svn commit: {\bf Note:please record the start and end revision numbers of
   the main trunk being merged into the commit log to keep track of merging.
   Please put this information on the first line if this is a commit following a
   merge of your branch with the main trunk (see Commit Message Format in subsection~\ref{CommitMessageFormat} for details)}
\end{itemize}                
\item You can check the archive of email notifications of the svn commits
from \htmladdnormallink{https://osp5.lbl.gov/pipermail/rose-commits}{https://osp5.lbl.gov/pipermail/rose-commits} 
\end{itemize}

\subsection{Commit Message Format}
\label{CommitMessageFormat}
   The automatically generated ChangeLog2 file will provide everyone with
detailed information about what changes are made to ROSE over time. To
make this information as clear and consistent as possible we have two
(slightly different) commit message formats:
% There are two styles of commit message formats: 
1) normal commits of your local contributions to your branch or to 
   the internal SVN trunk; and
2) commits after a merge of the main trunk's changes.
\begin{enumerate}
   \item Normal svn commit (not those following an svn merge)
   \begin{itemize}
      \item {\tt svn commit} will start your favorite editor where you should enter a
	    description of your changes. The first line of that description should be
            a short, one-line summary ({\it i.e.}, a title with just the first word capitalized),
            followed by a blank line, and as much detail as necessary. There is generally
            no need to includes your name, date, names of files, etc. as this information
            is readily available from the source revision management system. Do not prefix
            the summary with tags like ``Summary:'', ``Title:'' etc. since it's already
            implied that the first line is the summary.

	    Here's an example specific to a commit on the internal SVN or an SVN branch:
	    {\indent
	       \begin{verbatim}
Adjusted test case for new binary function detection
    
This test case assumed that the only functions in a binary executable
were those that had symbols in the symbol table.  This is no longer
true since we now determine function boundaries with a wider variety of
heuristics.
	       \end{verbatim} 
            }
   \end{itemize}
   \item For the svn commit at any point after your svn merge \\
	    Here's an example specific to a commit message on an SVN branch after a merge:
	    {\indent
	       \begin{verbatim}
svn merge -r 402:428 https://outreach.scidac.gov/svn/rose/trunk
	       \end{verbatim} 
            }
   \item If you mix an svn merge and some local contributions in one svn
   commit (we don't suggest mixing them)\\
	    Here's an example specific to the commit on an SVN branch ({\em note first line}):
	    {\indent
	       \begin{verbatim}
svn merge -r 402:428
Adjusted test case for new binary function detection
    
This test case assumed that the only functions in a binary executable
were those that had symbols in the symbol table.  This is no longer
true since we now determine function boundaries with a wider variety of
heuristics.
	       \end{verbatim} 
            }
\end{enumerate}


% Moved this from being a section (1.6) to being a subsection (1.2.2)
\subsection{Check In Process}

   The following information applies to both the internal SVN reposiotry and the 
branches that we provide to external collaborators.  There are a number of details
that we need to make sure that your developent work can be used to update ROSE.

For internal SVN users:
{\bf Please get permission from the ROSE Development Team before you make your first check-in!}

For all SVN users:
   If you have access to the SVN repository (at LLNL) and are building the development 
version of ROSE (available only from SVN, not what we package as a ROSE distribution; 
e.g. not from a file name such as ROSE-\VersionNumber.tar.gz) then 
% the README file in the top level directory also has instructions for how to get started. 
there are a number of steps to the checkin process:
%The README\_CHECKIN file has the instructions for our standard testing process.
%This should be done before you check anything in; we don't reprint it so it is
%represented at most only once.
% {\bf NOTE: Get permission from the ROSE Development Team before you make your first check-in!}
\begin{enumerate}
   \item Make sure you are working with the latest update (run {\tt svn update} in the top
    level directory.

 % \item In {\em ROSE/configure.in} modify version number (X.Y.ZZL) at the top of the file.

   \item Run {\tt make} \&\& {\tt make docs} \&\& {\tt make check} \&\&
   {\tt make dist} \&\& 
         {\tt make distcheck} \&\& {\tt make install} \&\& {\tt make installcheck}, depending
         on how aggressively you want your changes to be tested.
   \begin{itemize}
      \item Not all tests must be run, but we will know who you are (via {\tt svn blame} 
            if the nightly test fail :-)).
      \item All changes must at least compile, so that you don't hold back other
            developers who update often.
   \end{itemize}
   \item The commit will fail if someone else has committed while you were running
         your pre-commit tests. If this happens you will generally need to restart
         the check-in process from the top.
   \item Please follow the commit message format (see Commit Message Format in 
         subsection~\ref{CommitMessageFormat} for details).

%   \item ROSE/ChangeLog
%   a new entry has to include at least the following information:
%   - new version number
%   - who checked it in
%   - comments on changes
%   - run 'cvschk' in your ROSE directory.
%     copy the output of 'cvschk' to the ChangeLog (except "Extra Files")
%     (this includes all version numbers of tools used, etc.)
%
% 4) Send an e-mail to casc-rose@llnl.gov that you are checking in a new 
%    version of ROSE. Include the new version number in this e-mail.
%
% 6) check out a fresh version of ROSE in a new directory
%   - run all tests as above, in part 2, on this checked-out version
%   - make sure all tests succeed
%
% 7) cvs rtag ROSE-X-Y-ZZL ROSE
%    - X-Y-ZZL has to be the same version number as in configure.in
%      (X, Y, Z are numbers, L is a letter, note that here '-' is used instead
%       of '.' as in configure.in)
%
% 8) send out e-mail that check-in was successful
%    - copy & paste the output of make distcheck that the new version is ready
%      for distribution in this e-mail.
%    - include your new comments from the ChangeLog
%
% regular definition of version number:
% X=Y=Z=[0-9], L=[a-z]
% X.Y.ZZL /* in configure.in */
% X-Y-ZZL /* when using cvs rtag */
%
% Note that in configure.in '.' is used to separate X,Y,ZZL where as with
% etag '-' must be used, X-Y-ZZL!

\end{enumerate}

If you do not have access to the SVN repository at LLNL, and you wish to contribute
work to the ROSE project, please make a patch.  Using the external SVN access via
LBL use {\tt svn diff} to build a patch.
Consider options: {\em --diff-cmd arg}.
DQ(7/28/2008): This section still needs to be completed!



\subsection{The Internal Repository}
The tips here only apply to internal users who have access to LLNL
machines.
Again, make sure you are using subversion $>$ 1.4.x, 1.5.1 and up is recommended.
\begin{itemize}
\item Set local subversion configuration to ignore certain files and
automatically set file attributes (e.g. binary or text files ) during committing. 
A sample config file is available in the internal ROSE subversion repository:\\
  \textit{trunk/ROSE/scripts/subversion.config}
Please save it to your own .subversion/config before committing files. 
\item List content in the repository:
  \begin{itemize}
          \item list root info: \textit{svn list
          file:///usr/casc/overture/ROSE/svn/ROSE}
          \item list all branches: \textit{svn list
          file:///usr/casc/overture/ROSE/svn/ROSE/branches}
          \item list all tags: \textit{svn list
          file:///usr/casc/overture/ROSE/svn/ROSE/tags} 
   \end{itemize}       
\item Check out something:
   \begin{itemize}
          \item the main trunk: \textit{svn co
          file:///usr/casc/overture/ROSE/svn/ROSE/trunk/ROSE rose-svn} 
          \item a tag: \textit{svn co
          file:///usr/casc/overture/ROSE/svn/ROSE/tags/tag-name}
          \item a branch: \textit{svn co
          file:///usr/casc/overture/ROSE/svn/ROSE/branches/branch-name} 
          \item Either of these can be used on a separate machine (not CASC
          or LC) with LLNL VPN access by changing \textit{file:///} to
          \textit{svn+ssh://username@hostname/}. 
    \end{itemize}      
\item Merge the contributions from a branch of the external SVN repository (on SciDAC
web site) to our internal repository at LLNL. Assume the branch is named \textit{testonly} and the
contribution is from r4 to r5,
\textit{sourcetree} is the working copy of the internal repository (Subversion 1.5 works better than 1.4.x):
  \begin{itemize}
          \item run make check, make dist, make distcheck on the external branch before the merge
        % \item \textit{svn status} to check modified and new files. 
        % \item \textit{svn add file-name} to add new files if there are any.
          \item \textit{svn merge --dry-run -r4:5
          https://outreach.scidac.gov/svn/rose/branches/testonly
          sourcetree} 
          \item \textit{svn merge -r4:5
          https://outreach.scidac.gov/svn/rose/branches/testonly
          sourcetree}
          \item solve any possible conflicts alone the way
           \item svn commit: please record the start and end revision numbers of
         the external branch being merged into the log to keep track of merging. 
         Please also copy the corresponding log content into the
         commit message to preserve their commit messages.
   \end{itemize}

\item branches   
    \begin{itemize}
    \item Add a new branch based on the head of the main trunk:\\
    \textit{svn cp file:///usr/casc/overture/ROSE/svn/ROSE/trunk/ROSE file:///usr/casc/overture/ROSE/svn/ROSE/branches/branch-name}
    \item Delete a branch: \\
    \textit{svn delete file:///usr/casc/overture/ROSE/svn/ROSE/branches/branch-name}
    \end{itemize}

\item Email notification: A perl script named post-commit (under
svn/text/hooks or svn/ROSE/hooks) keeps our email recipient list.\\
          \textit{/usr/casc/overture/ROSE/svn/ROSE/hooks/post-commit}
\item To upgrade, do \textit{svn switch} to new tag URL
\item Building distributions of ROSE MUST be done with \textit{svn export}. Otherwise, .svn directories are copied into the distribution trees. 

\end{itemize}

% DQ (2/4/2009): Added documentation for how to sync external version of ROSE
\section{Resync-ing with a full version of ROSE}
   As part of work with external collaborators, where they have 
access to the EDG source code, we sometime have to update their
version of the parts of ROSE that are not released publicly
(e.g. EDG and the EDG/ROSE translation work which uses EDG).
A typical reason why this is required is that the external
collaborator has made a change to the ROSE IR that is incompatible
with the binary distribution of the EDG and EDG/ROSE translation code,
and so they need a most recent version of the full distribution of
ROSE so that they can build EDG and the EDG/ROSE translation fresh
and run the automated tests.

We wish to outline this process:
\begin{enumerate}
   \item Let us know that you are trying to follow these directions.
   \item Ask for a tarball of the full source code of ROSE from our internal SVN repository.
      We will provide you a tarball of ROSE that matches a specific revision number
      that was externally released on the web (thus we know that it has passed all of our tests to 
      be released).  This will also define a mapping between internal and 
      external SVN revision numbers, which is also in the commit log message on the web site.
      For example, it shows the lat log entry of the main trunk on the page of (\htmladdnormallink{https://outreach.scidac.gov/plugins/scmsvn/viewcvs.php/?root=rose}{https://outreach.scidac.gov/plugins/scmsvn/viewcvs.php/?root=rose}):
\begin{verbatim}
      File      Rev.    Age       Author   Last log entry 
      trunk/    243   6 hours     liaoch   Load rose-0.9.4a-4275 into trunk.
\end{verbatim}
     In this case the internal SVN revision number is {\em 4275} and it mapped to the 
     external SVN revision number {\em 243}.  We will make a tarball of ROSE using 
     revision number {\em 4275}.   The command to do this, on our side is:
\begin{verbatim}
          scripts/make_svn_tarball 4275
\end{verbatim}
with typical output:
\begin{verbatim}
          Built tarball ROSE-svn-Feb03-2009-r4275.tar.gz from SVN revision r4275
\end{verbatim}
This builds the file: {\tt ROSE-svn-Feb03-2009-r4275.tar.gz}
which we then send to you.  This is a full source code release of ROSE which includes the
    protected EDG source code, we will know if you have a license for this.  
    {\em You should not distribute this to anyone who does not have an EDG license.}
   \item Then you update your branch with the trunk at the external revision number (in our
    example this would be revision {\em 243}). See the instructions in
    \verb\Working with The ROSE SVN repositories\ of this guide about
    how to merge the new updates of the main trunk into your branch.
    Make sure it pass make check.
   \item Then build a patch to represent your branch's changes from the
   external trunk revision.  %See the documentation in this guide about how to make patches. 
   A typical  command to generate a patch looks like the following:
    \begin{verbatim}
    diff -NaU5 -rbB -x \*.orig -x \*.o -x \*.swp -x \*.bak -x \*.pdf \
      -x \*.html -x \*.rej -x \*~ -x Makefile.in -x \*.gz \
      -x autom4te.cache -x .svn -x aclocal.m4 -x config.guess \
      -x configure -x config.sub external_trunk your_updated_branch > my.patch
    \end{verbatim}
    You may need to check the generated patch and add or remove the items
    in the exclusion list to regenerate a desired patch as needed.
    The final patch should only contains your contributions.
   \item Apply that patch to the tarball of the internal ROSE's trunk that we sent you (representing
     the full source code for EDG and everything) and you how have a way to test
     your work and recompile the EDG work for either a new machine or with the IR changes
     that you have added.
     \begin{verbatim}
      cd internal_ROSE_trunk
      # test run only
      patch -p1 --dry-run <../my.patch
      # if everything looks normal, do the actual patching
      patch -p1 < ../my.patch
     \end{verbatim}
   \item After you have passed all tests, then build a patch between the
   patched internal ROSE trunk and it's original form. 
    \begin{verbatim}
    diff -NaU5 -rbB -x \*.orig -x \*.o -x \*.swp -x \*.bak -x \*.pdf \
      -x \*.html -x \*.rej -x \*~ -x Makefile.in -x \*.gz \
      -x autom4te.cache -x .svn -x aclocal.m4 -x config.guess \
      -x configure -x config.sub internal_ROSE_trunk_orig
      internal_ROSE_trunk_patched > my.patch2
    \end{verbatim}
    Again, please tweak the exclusion list above to generate a clean and
    complete patch. This patch contains your contributions tested against
    with the full internal source tree. Please record the revision number
    of your branch associated with this patch. The number will be treated
    as a synchronization point between your branch and the main trunk. 
% \item Apply that patch to your external branch.  This will update your branch with everything required, even though   it may not pass tests using the released binary of EDG.
%   \item Let us know when your done and we can review and merge your branch into the trunk of the 
   \item Let us know when you are done and we can get your patch applicable to
   our internal SVN repository. At this point we can review and apply the
   patch to the internal ROSE and
    the next external release of ROSE (usually nightly) will reflect your changes.
\end{enumerate}

% DQ (6/5/2008): This is the resulting lesson from the erasure
% of my ROSE directory as the result of a bad cron script.
\section{How to recover from a file-system disaster at LLNL}
   Disasters can happen (cron scripts can go very very badly).  If you 
loose files on the CASC cluster at LLNL you can get the backup from the 
night before.  It just takes a while.

   To restore from backups at LLNL: use the command: \\
{\tt restore}
\begin{enumerate}
   \item {\tt add <directory name>} \\
       This will build the list of files to be recovered.
   \item recover \\
       This will start the process to restore the files from tape.
\end{enumerate}
   This process can take a long time if you have a lot of files to recover.


\section{Generating Documentation}
   There is a standard GNU {\tt make docs} rule for building all documentation.

{\it Note to developers: To build the documentation ({\tt make docs}) you will need 
LaTeX, Doxygen and DOT to be installed (check the list of dependences in the 
{\tt ROSE/ChangeLog}). If you want to build the reference manual of Latex documentation
generated by Doxygen (not suggested) you may have to tailor your version of LaTeX to 
permit larger internal buffer sizes.  All the other LaTeX documentation, such as the
User Manual but not the Reference Manual may be built without problems using the 
default configuration for LaTeX.
}


\section {Adding New SAGE III IR Nodes (Developers Only)}
    We don't expect users to add nodes to the SAGE III Intermediate Representation (IR),
however, we need to document the process to support developers who might be extending
ROSE.  It is hoped that if you proceed to add IR nodes that you understand just
what this means (you're not extending any supported language (e.g. C++); you are only
extending the internal representation. Check with us so that we can help you and
understand what you're doing.

The SAGE III IR is now completely generated using the ROSETTA IR generator tool which 
we developed to support our work within ROSE.
The process of adding new IR nodes using ROSETTA is fairly simple: one
adds IR node definitions using a BNF syntax and provides additional
headers and implementations for customized member data and functions
when necessary. 

  There are lots of examples within the construction of the IR itself.  So you are
encouraged to look at the examples. 
% However, you can expect to hunt around a bit before you get the final generated code to
% compile and link!  
The general steps are:

\fixme{Need to cover the new Fortran support. }
\begin{enumerate}
     \item Add a new node's name into \textit{src/ROSETTA/astNodeList}
%------------- 
     \item Define the node in ROSETTA's source files under
     \textit{src/ROSETTA/src} \\
           For example, an expression node has the following line in
           \textit{src/ROSETTA/src/expression.C}:
{\indent
{\mySmallFontSize
\begin{verbatim}
          NEW_TERMINAL_MACRO (VarArgOp,"VarArgOp","VA_OP");
\end{verbatim}
}}
           This is a macro (currently) which builds an object named {\em VarArgOp} (a variable in
           ROSETTA) to be named {\em SgVarArgOp} in SAGE III, and to be referenced using an enum
           that will be called {\em V\_SgVarArgOp}.  The secondary generated enum name {\em VA\_OP}
           is historical and will be removed in a future release of ROSE.

%------------- 
     \item In the same ROSETTA source file, specify the node's SAGE class hierarchy. \\
           This is done through the specification of what looks a bit like a BNF
            production rule to define the abstract grammar. \\
{\indent
{\mySmallFontSize
\begin{verbatim}
     NEW_NONTERMINAL_MACRO (Expression,
          UnaryOp        | BinaryOp             | ExprListExp   | VarRefExp       | ClassNameRefExp |
          FunctionRefExp | MemberFunctionRefExp | ValueExp      | FunctionCallExp | SizeOfOp        |
          TypeIdOp       | ConditionalExp       | NewExp        | DeleteExp       | ThisExp         |
          RefExp         | Initializer          | VarArgStartOp | VarArgOp        | VarArgEndOp     |
          VarArgCopyOp   | VarArgStartOneOperandOp ,"Expression","ExpressionTag");
\end{verbatim} 
}}
        In this case, we added the VarArgOp IR node as an expression node in the
    abstract grammar for C++.

%------------- 
     \item Add the new node's members (fields): both data and function
     members are allowed. \\
           ROSETTA permits the addition of data fields to the class definitions for the
           new IR node. Many generic access functions will be automatically
           generated if desired. 
{\indent
{\mySmallFontSize
\begin{verbatim}
     VarArgOp.setDataPrototype  ( "$GRAMMAR_PREFIX_Expression*","operand_expr","= NULL",
				 CONSTRUCTOR_PARAMETER, BUILD_ACCESS_FUNCTIONS, DEF_TRAVERSAL, NO_DELETE);
\end{verbatim}
}}
           The new data fields are added to the new IR node.  Using the first example
           above, the new data member is of type {\tt SgExpression*}, with name
           {\tt operand\_expr}, and initialized using the source code string {\tt = NULL}.
           Additional properties that this IR node will have include:
           \begin{itemize}
                \item Its construction will take a parameter of this type and 
                      use it to initialize this member field.
                \item Access functions to {\it get} and {\it set} the member 
                      function will be automatically generated.
                \item The automatically generated AST traversal will traverse 
                      this node (i.e. it will visit its children in the AST).
                \item Have the automatically generated destructor not call 
                      delete on this field (the traversal will to that).
           \end{itemize}
           In the case of the VarArgOp, an additional data member was added.
{\indent
{\mySmallFontSize
\begin{verbatim}
     VarArgOp.setDataPrototype ( "$GRAMMAR_PREFIX_Type*", "expression_type", "= NULL",
				 CONSTRUCTOR_PARAMETER, BUILD_ACCESS_FUNCTIONS, NO_TRAVERSAL || DEF2TYPE_TRAVERSAL);
\end{verbatim} 
}}

%------------- 
    \item Most IR nodes are simpler, but SgExpression IR nodes have explicit precedence. \\
           All expression nodes have a precedence in the evaluation, but the precedence must
           be specified.  This precedence must match that of the C++ frontend.  So we 
           are not changing anything about the way that C++ evaluates expressions here!
           It is just that SAGE must have a defined value for the precedence.
           ROSETTA permits variables to be defined and edited to tailor the automatically
           generated source code for the IR.
{\indent
{\mySmallFontSize
\begin{verbatim}
           VarArgOp.editSubstitute ( "PRECEDENCE_VALUE", "16" );
\end{verbatim} 
}}
%------------- 
     \item Associate customized source code. \\
           Automatically generated source code sometimes cannot meet all
           requirements, so ROSETTA allows user to define any custom 
           code that needs to be associated with the IR node in some
           specified files. If customized code is needed, you have to
           specify the source file containing the code. 
           For example, we specify the file containing customized source
           code for {\em VarArgOp} in \textit{src/ROSETTA/src/expression.C}:
{\indent
{\mySmallFontSize
\begin{verbatim}
     VarArgOp.setFunctionPrototype ( "HEADER_VARARG_OPERATOR", "../Grammar/Expression.code" );
     VarArgOp.setDataPrototype  ( "SgExpression*", "operand_expr"   , "= NULL",
				 CONSTRUCTOR_PARAMETER, BUILD_ACCESS_FUNCTIONS, DEF_TRAVERSAL, NO_DELETE);
     VarArgOp.setDataPrototype ( "SgType*", "expression_type", "= NULL",
 CONSTRUCTOR_PARAMETER, BUILD_ACCESS_FUNCTIONS, NO_TRAVERSAL || DEF2TYPE_TRAVERSAL, NO_DELETE);
   // ...
   VarArgOp.setFunctionSource ( "SOURCE_EMPTY_POST_CONSTRUCTION_INITIALIZATION", 
                                  "Grammar/Expression.code" );
\end{verbatim} 
}}
           Pairs of special markers (such as {\em SOURCE\_VARARG\_OPERATOR}
           and {\em SOURCE\_VARARG\_END\_OPERATOR}) are used for marking the header 
           and implementation parts of the customized code. 
           For example, the marked header and implementation code portions 
           for {\em VarArgOp} in
           \textit{src/ROSETTA/Grammar/Expression.code} are:
{\indent
  {\mySmallFontSize
\begin{verbatim}
HEADER_VARARG_OPERATOR_START
   virtual unsigned int cfgIndexForEnd() const;
   virtual std::vector<VirtualCFG::CFGEdge> cfgOutEdges(unsigned int index);
   virtual std::vector<VirtualCFG::CFGEdge> cfgInEdges(unsigned int index);
HEADER_VARARG_OPERATOR_END

// ....
SOURCE_VARARG_OPERATOR_START

  SgType*
  $CLASSNAME::get_type() const
   {
     SgType* returnType = p_expression_type;
     ROSE_ASSERT(returnType != NULL);
     return returnType;
   }

  unsigned int $CLASSNAME::cfgIndexForEnd() const {
    return 1;
  }
  //....

SOURCE_VARARG_OPERATOR_END
\end{verbatim} 
 }}
           The C++ source code is extracted 
           from between the named markers (text labels) in the named file and inserted 
           into the generated source code. Using this technique, very small amounts of 
           specialized code can be tailored for each IR node, while still providing an 
           automated means of generating all the rest.  Different locations in the
           generated code can be modified with external code. Here we add the source code
           for a function.

     \item Adding the set\_type and get\_type member functions. \\
           It is not clear that this is required, but all expressions must define a
           function that can be used to describe its type (of the expression).
           It is unfortunate, but it is generally in compiling the generated source code
           that details like this are discovered.  (ROSETTA has room for improvement!)
{\indent
{\mySmallFontSize
\begin{verbatim}
     VarArgOp.setFunctionSource ( "SOURCE_SET_TYPE_DEFAULT_TYPE_EXPRESSION", 
                                       "Grammar/Expression.code" );
     VarArgOp.setFunctionSource ( "SOURCE_DEFAULT_GET_TYPE",
                                       "Grammar/Expression.code" );
\end{verbatim} 
}}

     \item Modify the EDG/SAGE connection code to have the new IR node built in the
           translation from EDG to SAGE III.  This step often requires a bit of expertise
           in working with the EDG/SAGE connection code. In general, it requires no great
           depth of knowledge of EDG.

           Two source files are usually involved: a)
           \textit{src/frontend/CxxFrontend/EDG\_SAGE\_Connection/sage\_gen\_be.C}
           which converts IL tree to SAGE III AST and is derived from EDG's 
           C++/C-generating back end \textit{cp\_gen\_be.c}; b)
           \textit{sage\_il\_to\_str.C} contains helper functions forming SAGE
           III AST from various EDG IL entries. It is derived from EDG's
           \textit{il\_to\_str.c}.  For the {\em SgVarArgOp} example, the
           following EDG-SAGE connection code is needed in
           \textit{sage\_gen\_be.C}:
{\indent
{\mySmallFontSize
\begin{verbatim}
a_SgExpression_ptr
sage_gen_expr ( an_expr_node_ptr expr, 
                a_boolean need_parens, 
    ...
              )
{
  // ...
  case eok_va_arg:
  {
   sageType = sage_gen_type(expr->type);
   sageLhs = sage_gen_expr_with_parens(operand_1,NULL);
   if (isSgAddressOfOp(sageLhs) != NULL)
     sageLhs = isSgAddressOfOp(sageLhs)->get_operand();
   else
     sageLhs = new SgPointerDerefExp(sageLhs,NULL);
  //....
   result = new SgVarArgOp(sageLhs, sageType);
   goto done_with_operation;
                       }
  }
  //.....

}
\end{verbatim} 
}}

     \item Modify the unparser to have whatever code you want generated in the final
           code generation step of the ROSE source-to-source translator.
           The source files of the unparser are located at
           \textit{src/backend/unparser}. For {\em SgVarArgOp}, it is
           unparsed by the following function in
           \textit{src/backend/unparser/CxxCodeGeneration/unparseCxx\_expressions.C}:

{\indent
{\mySmallFontSize
\begin{verbatim}

void
Unparse_ExprStmt::unparseVarArgOp(SgExpression* expr, SgUnparse_Info& info)
   {
     SgVarArgOp* varArg = isSgVarArgOp(expr);
     SgExpression* operand = varArg->get_operand_expr();
     SgType* type = varArg->get_type();
     curprint ( "va_arg(");
     unparseExpression(operand,info);
     curprint ( ",");
     unp->u_type->unparseType(type,info);
     curprint ( ")");
   }
\end{verbatim} 
}}



\end{enumerate}



\section{Separation of EDG Source Code from ROSE Distribution}

    The EDG research license restricts the distribution of their source code.
Working with EDG is still possible within an open source project such as ROSE because 
EDG permits binaries of their work to be freely distributed (protecting their source 
code).  As ROSE matured, we designed the autoconf/automake distribution mechanism
to build distributions that exclude the EDG source code and alternatively distribute
a Linux-based binary version of their code.

   All releases of ROSE, starting with 0.8.4a, are done without the EDG source code
by default.  An optional configure command line option is implemented to allow
the construction of a distribution of ROSE which includes the EDG source code
(see {\tt configure --help} for the {\tt --with-edg\_source\_code} option).

   The default options for configure will build a distribution that contains
no EDG source code (no source files or header files).  This is not a problem 
for ROSE because it can still exist as an almost entirely open source project
using only the ROSE source and the EDG binary version of the library.

  Within this default configuration, ROSE can be freely distributed on the Web
(eventually).  Importantly, this simplifies how we work with many different 
research groups and avoid the requirement for a special research license from
EDG for the use of their C and C++ front-end.  Our goal has been to simplify
the use of ROSE.

   Only the following command to configure with EDG source code is accepted:
{\indent
{\mySmallFontSize
\begin{verbatim}
     configure --with-edg_source_code=true
\end{verbatim} 
}}
This particularly restrictive syntax is used to prevent it from ever being used
by accident.  Note that the following will not work. They are equivalent to 
not having specified the option at all:
{\indent
{\mySmallFontSize
\begin{verbatim}
     configure --with-edg_source_code
     configure --with-edg_source_code=false
     configure --with-edg_source_code=True
     configure --with-edg_source_code=TRUE
     configure --with-edg_source_code=xyz
     configure 
\end{verbatim} 
}}

To see how any configuration is set up, type {\tt make testEdgSourceRule}
in the {\tt ROSE/src/frontend/CxxFrontend/EDG\_3.3/src} directory.

To build a distribution without EDG source code:
\begin{enumerate}
   \item Configure to use the EDG source code and build normally, 
   \item Then rerun configure to not use the EDG source code, and
   \item Run {\tt make dist}.
\end{enumerate}


\section{How to Deprecate ROSE Features}

    There comes a time when even the best ideas don't last
into a new version of the source code.  This section covers how to
deprecated specific functionality so that it can be removed in
later releases (typically after a couple of releases, or before
our first external release).  When using GNU compilers these mechanisms
will trigger the use of GNU attribute mechanism to permit use of such
functions in applications to be easily flagged (as warnings
output when using the GNU options {\tt -Wall}).

Both functions and data members can be deprecated, but the process if different 
for each case:
\begin{itemize}
    \item Deprecated functions and member functions. \\
          Use the macro {\tt ROSE\_DEPRECATED\_FUNCTION} after the function declaration (and before
          the closing {\bf ;}). As in:
{\indent
{\mySmallFontSize
\begin{verbatim}
       void old_great_idea_function() ROSE_DEPRECATED_FUNCTION;
\end{verbatim}
}}

    \item Deprecated data members. \\
        Use the macro {\tt ROSE\_DEPRECATED\_VARIABLE} to specify that a data members 
    or variables is to be deprecated.  This is difficult to do because data members of 
    the IR are all automatically generated and thus can't be edited in this way.  Where 
    a data member of the IR is to be deprecated, it should be specified explicitly in
    the documentation for that specific class (in the {\tt ROSE/docs/testDoxygen} directory,
    which is the staging area for all IR documentation, definitely {\em not} in the 
    {\tt ROSE/src/frontend/SageIII/docs} directory, which is frequently overwritten).  See
    details on how to document ROSE (Doxygen-Related Pages).
{\indent
{\mySmallFontSize
\begin{verbatim}
       void old_great idea_data_member ROSE_DEPRECATED_VARIABLE;
\end{verbatim} 
}}
\end{itemize}



\section{Code Style Rules for ROSE}

   I don't want to constrain anyone from being expressive, but
we have to maintain your code after you leave, so there are a few rules:
\begin{enumerate}
   \item Document your code.
         Explain every function and use variable names that clearly indicate the purpose of
         the variable. Explain what the tests are in your code (and where they are located).
   \item Write test codes to test your code (these are assembled in the {\tt ROSE/tests}
         directory (or subdirectories of {\tt ROSE/tests/roseTests}).
   \item Use assertions liberally, use boolean values arguments to 
         {\tt ROSE\_ASSERT(<expression>)}. Use of {\tt ROSE\_ASSERT(true/false)} for
         error branches is preferred.
   \item Put your code into source files (*.C) and as little as possible into header files.
   \item If you use templates, put the code into a *.C file and include that *.C file
         at the bottom of your header file.
   \item If you use a {\em for loop} and break out of the loop (using {\tt break;} 
         at some point in the iteration, then consider a {\em while loop} instead.
   \item Don't forget a default statement within switch statements.
   \item Please don't open namespaces in source files, i.e. use the fully qualified
         function name in the function definition to make the scope of the function
         as explicitly clear as possible.
   \item Think about your variable names. I too often see {\tt Node}, {\tt node}, 
         and {\tt n} in the same function.  Make your code {\em obvious} so that I 
         can understand it when I'm tired or stupid (or both).
   \item Write good code so that we don't have to debug it after you leave.
   \item Indent your code blocks.
\end{enumerate}

My rules for style are as follows. Adhere to them if you like, or don't, if you're
    appalled by them.
\begin{enumerate}
   \item Indent your code blocks (I use five spaces, but some consider this excessive).
   \item Put spaces between operators for clarity.
\end{enumerate}



\section{Things That May Happen to Your Code}

    No one likes to have their code touched, and we would like to
avoid having to do so. We would like to have your contribution to ROSE
always work and never have to be touched.  We don't wish to pass
critical judgment on style since we want to allow many people to 
contribute to ROSE.  However, if we have to debug your code, be prepared 
that we may do a number of things to it that might offend you:
\begin{enumerate}
   \item We will add documentation where we think it is appropriate.
   \item We will add assertion tests (using {\tt ROSE\_ASSERT()} macros)
         wherever we think it is appropriate.
   \item We will reformat your code if we have to understand it and the 
         formatting is a problem.  This may offend
         many people, but it will be a matter of project survival,
         so all apologies in advance.  If you fix
         anything later, your free to reformat your code as you like.  We try to change
         as little as possible of the code that is contributed.
\end{enumerate}


\section{ROSE Email Lists}
\label{rose_email_list_info}
% DQ (10/28/2008): These email addresses have been copied to the installRose.tex as well.
   We have three mailing lists for core developers (those who have write access to
the internal repository), all developers (anyone who has write access to the
internal or external repository) and all
users of ROSE. They are:
\begin{itemize}
\item rose-core@nersc.gov, web interface: 
\htmladdnormallink{https://mailman.nersc.gov/mailman/listinfo/rose-core}{https://mailman.nersc.gov/mailman/listinfo/rose-core}.
\item rose-developer@nersc.gov, web interface: 
\htmladdnormallink{https://mailman.nersc.gov/mailman/listinfo/rose-developer}{https://mailman.nersc.gov/mailman/listinfo/rose-developer}.
\item rose-public@nersc.gov, web interface:
\htmladdnormallink{https://mailman.nersc.gov/mailman/listinfo/rose-public}{https://mailman.nersc.gov/mailman/listinfo/rose-public}.
\end{itemize}


We are phasing out the casc-rose@llnl.gov mail list. 

\commentout{
   There is an open email list for ROSE which can be subscribed to
automatically.  The list name is: {\bf casc-rose}.

   These are the email commands available to users of the list. To use them,
a user sends a message to Majordomo with one or more of these commands in the body of
the message. Each mailing list has a special "request" address where commands can be
sent. For example, to use the casc-rose mailing list (casc-rose@lists.llnl.gov), send
commands to casc-rose-request@lists.llnl.gov.

   It is also possible to send commands directly to majordomo@lists.llnl.gov. 
However, be sure to specify which list you want to use. With all the commands below, you 
can leave out list if you are sending to casc-rose-request@lists.llnl.gov.

\begin{itemize}
   \item subscribe list address \\
       Subscribe yourself (or address if specified) to the named list. The list may be 
       configured so that you can only subscribe yourself; ie. you can't specify an 
       address other than your own.

   \item Unsubscribe list address \\
    Unsubscribe yourself (or address if specified) from the named list. "unsubscribe *" 
    will remove you (or address) from all lists; This may not work if you have subscribed 
    using multiple addresses. The list may be configured so that you can only unsubscribe 
    yourself; ie. you can't specify an address other than your own.

   \item which address \\
    Find out which lists you (or address if specified) are on. Only lists enabled to
    supply this information will be returned to the requester.

   \item who list \\
    Find out who is on the named list. Only lists enabled to supply this information 
    will be returned to the requester.

   \item info list \\
    Retrieve the general introductory information for the named list. Only lists enabled
    to supply this information will be returned to the requester.

   \item intro list \\
    Retrieve the introductory message sent to new users. Non-subscribers may not be able
    to retrieve this.

   \item lists \\
    Show the lists served by this Majordomo server (will not show "private" lists).

   \item help \\
    Retrieve some help information on the available user commands.

   \item end \\
    Stop processing commands (useful if your email program adds a signature). 
\end{itemize}


Here are the URLs for the {\em casc-rose} email list:

Instructions on how to use a Majordomo mailing list: \\
\begin{verbatim}
    https://lists.llnl.gov/mj/user-commands.html
\end{verbatim}

Web interface for modifying a Majordomo mailing list: \\
\begin{verbatim}
    https://lists.llnl.gov/majordomo.
\end{verbatim}

Details: \\
\begin{enumerate}
   \item List name is: {\em casc-rose} not {\em casc-rose@llnl.gov}.
   \item Must be on site at LLNL.
%   \item Password is required (Tom Leher and Hichhicker's Guide).
\end{enumerate}

% commented out
}

\section{How To Build a ROSE Distribution with EDG Binaries}

%Updated by Liao on 4/27/2009

   The construction of a binary distribution is done as part of making
ROSE available externally on the web to users who do not have an EDG
license.  We make only the EDG part of ROSE available as a binary (library) 
and the rest is left as source code (just as in an all source distribution).

This step is automated in our daily regression test script in {\tt
rose/script/roseFreshTest} and is turned on by a flag
{\tt ENABLE\_BUILD\_BINARY\_EDG}. A simplified excerpt of the script is shown below:
\begin{verbatim}
if [ 0$ENABLE_BUILD_BINARY_EDG -ne 0 ]; then
  cd ${ROSE_TOP}/build
  make binary_edg_tarball || exit 1
  ${ROSE_TOP}/sourcetree/scripts/copy_binary_edg_tarball_to_source_tree_svn 
  ${ROSE_TOP}/sourcetree/scripts/copy_binary_edg_tarball_to_source_tree ${ROSE_TOP}/sourcetree 
make $MAKEFLAGS source_with_binary_edg_dist DOT_SVNREV=-${svnversion}
echo "############ make source_with_binary_edg_dist done"
fi
\end{verbatim}

% DQ (7/11/2009): Added '$' so get the coloring correct in my emacs (an annoying issue).

As shown in the script above, the steps to build a binary distribution of ROSE are:
\begin{enumerate}
   \item Configure and build ROSE normally, using configure (use all options that you
    require in the binary distribution).  
    \item Run {\tt make binary\_edg\_tarball}. This will make a binary tar
    ball from EDG and EDG-SAGE connection source files. 
    \item Copy the binary to the svn repository: {\tt copy\_binary\_edg\_tarball\_to\_source\_tree\_svn} 
    \item Copy the binary to your local copy: {\tt copy\_binary\_edg\_tarball\_to\_source\_tree}
     \item Make the ROSE distribution with EDG binaries: {\tt make source\_with\_binary\_edg\_dist}
%   \item (optional) Run {\tt make dist}, this will build an {\em all source distribution} of ROSE.
%   \item Rerun configure without the {\tt --with-edg\_source\_code=true} option.
%   \item Run {\tt make dist}, this will build a binary distribution using the 
%    binary libraries build in step one.
\end{enumerate}
To make sure the binaries are up to date for different platforms, 
we generate a hash number from the source files within the EDG and EDG-SAGE connection source tree and 
treat the number as a signature for the binary package. 
Please see the makefile target {\tt rose\_binary\_compatibility\_signature} in {\tt rose/Makefile.am} for details.
Our regression test script will check for the availability and consistency of the binaries for all supported platforms 
before making an external ROSE release with EDG binaries.
%-----------------------------------------------------------------------
\section{Avoiding Nightly Backups of Unrequired ROSE Files at LLNL}

   If your at LLNL and participating in the nightly builds and
regression testing of ROSE, then it is kind to the admin staff
to avoid having your testing directory 
{\em often many gigabytes of files} backed up nightly.

   There is a file {\tt .nsr} that you can put into
any directory that you don't need to have backed up.
The syntax of the text in the file is:
{\tt skip: .}

Additional examples are:
\begin{verbatim}
# The directives in this file are for the legato backup system
# Here we specify not to backup any of the following file types:
+skip: *.ppm *.o *.show*
\end{verbatim}

More information can be found at: \\
   www.ipnom.com/Legato-NetWorker-Commands/nsr.5.html
or \\
   https://computation-int.llnl.gov/casc/computing/tutorials/toolsmith/backups.htm

Example used in ROSE: \\
{\bf +skip: *.C *.h *.f *.F *.o *.a *.la *.lo *.so *.so.* Makefile rose\_test* *.dot}

Note: There does not appear to be a way of avoiding the backup on executables.

Thanks for saving a number of people a lot of work.


%-----------------------------------------------------------------------
\section{Setting Up Nightly Regression Tests}

Directions for using a bash script (rose/scripts/roseFreshTest) to set up periodic regression tests:

\begin{enumerate}
   \item Get an account on the machine you are going to run the tests on.
   \item Get a scratch directory (normally /export/0/tmp.<your username>) on that
         machine.
   \item Copy (using svn cp) a stub script (scripts/roseFreshTestStub-*) to one
   with your name.
   \item Edit your new stub script as appropriate:
   \begin{enumerate}
      \item Set the versions of the different tools you want to use (compiler,
     ...).
      \item Change ROSE\_TOP to be in your scratch directory.
      \item Set ROSE\_SVNROOT to be the URL of the trunk or branch you want to
     test.
      \item Set MAILADDRS to the people you want to be sent messages about the
     progress and results of your test.
      \item MAKEFLAGS should be set for most peoples' needs, but the -j setting
     might need to be modified if you have a slower or faster computer.
      \item If you would like the copy of ROSE that you test to be checked out
     using "svn checkout" (rather than the default of "svn export"), add a
     line "SVNOP=checkout" to the stub file.
      \item The default mode of roseFreshTest is to use the most current version
     of ROSE on your branch as the one to test.  If you would like to test
     a previous version, you can set SVNVERSIONOPTION to the revision
     specification to use (one of the arguments to -r in "svn help
     checkout").
   \end{enumerate}
   \item Check your stub script in so that it will be backed up, and so that other
   people can copy from it or update it to match (infrequent) changes in the
   underlying scripts.
   \item Run "crontab -e" on the machine you will be testing on:
   \begin{enumerate}
      \item Make sure there is a line with "MAILTO=<your email>".
      \item Add new lines for each test you would like to run:
      \begin{enumerate}
         \item If other people are using the machine you are running tests on, be
               sure to coordinate the time your scripts are going to run with them.
         \item See "man crontab" for the format of the time and date specification.
         \item The command to use is (all one line):
\begin{verbatim}
           cd <your ROSE source tree>/scripts && \
           ./roseFreshTest ./roseFreshTestStub-<your stub name>.sh \
           <extra configure options>
         Where <extra configure options> are things like
         --enable-edg\_union\_struct\_debugging, --with-C\_DEBUG=...,
         --with-java, etc.
\end{verbatim}
      \end{enumerate}
   \end{enumerate}
   \item Your tests should then run on the times and dates specified.
   \item If you would ever like to run a test immediately, copy and paste the
   correct line in "crontab -e" and set the time to the next minute (note
   that the minute comes first, and the hour is in 24-hour format); ensure
   the date specification includes today's date.  Be sure to quit your
   editor -- just suspending it prevents your changes from taking effect.
\end{enumerate}
\subsection{Enabling Testing Using External Benchmarks}
In addition to testing ROSE using the embedded test cases via \textit{make
check}, roseFreshTest also supports automatically testing ROSE on
external benchmarks based on the installed copy of ROSE generated by
\textit{make install}. 
Currently, it supports using ROSE's identityTranslator as a compiler to
compile a growing subset of the SPEC CPU 2006 benchmark suite. 

To enable this feature, do the following:
\begin{enumerate}
  \item Install the SPEC CPU 2006 benchmark suite to a desired path (e.g.
  \textit{/home/liao6/opt/spec\_cpu2006/}) as instructed in its user
  manual.
  \item Prepare a configuration file (e.g. \textit{rose.cfg}) for compiler name, compilation
  options, and other benchmark options based on the sample config file in
  \textit{spec\_installed\_path/config}. A set of relevant options in
  \textit{rose.cfg} are:
  \begin{verbatim}
   #We want to the test to abort on errors and report immediately
   ignore_errors = no

   # we want have ascii and table-based output (Screen) for results
   output_format = asc, Screen

   #The result is not intended for official reports to the SPEC organization
   reportable    = 0

   # compilers to compile benchmarks
   CC           = identityTranslator
   CXX          = identityTranslator
   FC           = identityTranslator

  # compilation options: turn off ROSE 's EDG frontend warnings 
  # since we are not interesting in fixing the benchmarks
  COPTIMIZE     = -O2 --edg:no_warnings
  CXXOPTIMIZE  = -O2 --edg:no_warnings
  FOPTIMIZE    = -O2 --edg:no_warnings
  \end{verbatim}
  \item Finally, add the following lines in your stub script:
  \begin{verbatim}
  # using external benchmarks during regression testing sessions
  ENABLE_EXTERNAL_TEST=1

  # installation path of spec cpu and the config file for using rose
  SPEC_CPU2006_INS=/home/liao6/opt/spec_cpu2006
  SPEC_CPU2006_CONFIG=rose.cfg
  \end{verbatim}
\end{enumerate}
That is it. Now your daily regression test has incorporated the SPEC
benchmark.

The subset of the SPEC
benchmark and the command line to run them is defined in
\textit{rose/script/testOnExternalBenchmarks.sh}. 
We will continue to enhance the quality of ROSE and add more external
benchmarks as time passes by. 

%-----------------------------------------------------------------------
\section{Updating The External Website and Repository}
(For the LLNL internal developers only) We have several special top level
makefile targets to update the rosecompiler.org and SciDAC Outreach
subversion repository. They are controlled by the regression test scripts
automatically. Here are some instructions if you really want to do it manually:

\subsection{rosecompiler.org}
Here are the commands to update the rosecompiler.org website:
\begin{verbatim}

# 1. enter your build tree for ROSE. You should have ran make docs already
 cd build/docs/Rose

# 2. change the scidac.outreach account to yours in the Makefile. e.g 
  # in build/docs/Rose/Makefile
  copyWebPages: logo
       cd ROSE_WebPages?; rsync -avz *   yourAccount@web-dev.nersc.gov:/www/host/rosecompiler

# 3. do the uploading, input your password when prompted. 
    make copyWebPages  
\end{verbatim}

\subsection{The External Repository}
To build the binary file of the EDG frontend and the corresponding
\textit{EDG\_SAGE\_CONNECTION} code for the current platform:

\begin{itemize}
  \item  \textit{make binary\_edg\_tarball} 
  \item  add the binary into the internal SVN repository, remove any stale
  binaries for other platforms as well. \\
  \textit{make copy\_binary\_edg\_tarball\_to\_source\_tree\_svn} 
   \item make a source release package with EDG binaries. \\
   \textit{make source\_with\_binary\_edg\_dist  DOT\_SVNREV=-svnversion} 
\end{itemize}   

Finally, a dedicated script will import the release package into the
external ROSE svn repository hosted at the SciDAC Outreach Center. You must
have an active account with https://outreach.scidac.gov to do this!\\

\textit{rose/scripts/importRoseDistributionToSVN ROSE\_TOP\_TEST\_DIR}

It conducts a set of sanity checks and postprocessing before the actual importing. e.g. No EDG copyrighted files in the package, remove .svn and other undesired directories or files, make sure all EDG binaries for supported platforms are available.
%------------------------------------------------------------
\section{Generating ChangeLog2}
You can generate a GNU-style ChangeLog from ROSE's subversion commit logs using a program named \textit{svn2cl}. 
Download it from \url{http://ch.tudelft.nl/~arthur/svn2cl/}
and install it as directed in its documentation. 

The command line we use to generate ChangeLog2 is:
\begin{verbatim}
svn log --xml --verbose \
| xsltproc --stringparam include-rev yes \
--stringparam ignore-message-starting "Automatic updates" \
/home/liao6/opt/svn2cl-0.11/svn2cl.xsl - > ChangeLog2
\end{verbatim}
The command above will include revision numbers into the change log and
filter out the automatically generated commits updating EDG binary files.

%------------------------------------------------------------
\section{Compiling ROSE using ROSE Translators}
It is possible to use a ROSE-based translator to compile the ROSE source
tree.
The motivation could be using Compass checkers to check for bugs or
violations in ROSE's source files. 
However, there are some pending bugs preventing the process from being fully successful.  

Here are some instructions:
\begin{itemize}
\item rename or copy your translator (such as identityTranslator) executable file to a file named
\textit{roseTranslator}, which is the only name that can be recognized by
ROSE's configure script to set up necessary flags and system headers etc. 
\item define \textit{CXX=roseTranslator} to specify the
compiler(translator) to compile ROSE during configuration.
\item define \textit{-DNDEBUG} as a workaround for a bug related to assert
statements.
\item define \textit{CXXLD=g++} as a workaround for rose translator's
limitations as a linker.
\end{itemize}
In summary, you have to set the necessary search path and shared library
path for your translator and rename it to textit{roseTranslator}.
Then a configuration line likes like the following:

\begin{verbatim}
../sourcetree/configure --with-boost=/home/liao6/opt/boost_1_35_0 \
--with-CXX_DEBUG="-g -DNDEBUG" --with-C_DEBUG="-g -DNDEBUG" \
--prefix=/home/liao6/daily-test-rose/compass/install \ 
CXX=roseTranslator CXXLD=g++
\end{verbatim}
%------------------------------------------------------------
\section{Enabling PHP Support}

\begin{enumerate}
\item
Fetch and install PHP (tested with 5.2.6) from
\texttt{http://www.php.net/downloads.php}.  PHC requires a few
specific configure flags in order to be able to use PHP properly.
Fill in your choice of PHP install location where appropriate in place
of \texttt{/usr/local/php}.
\begin{verbatim}
./configure  --enable-debug --enable-embed --prefix=/usr/local/php
make && make install
\end{verbatim}

\item
Fetch and install PHC (tested with svn version r1487).  Currently only
the development release works with ROSE.
\begin{verbatim}
svn checkout http://phc.googlecode.com/svn/trunk/ phc-read-only
cd phc-read-only
touch src/generated/*                                                       
./configure --prefix=/usr/local/php --with-php=/usr/local/php
make && make install
\end{verbatim}

\item
Finally, due to an incongruence in the class hierarchies of PHC and
ROSE the following changes have to be made to the installed
\texttt{/usr/local/php/include/phc/AST\_fold.h}.  Hopefully this can be
resolved soon so that ROSE works with an unmodified upstream PHC.

\begin{verbatim}
--- src/generated/AST_fold.h    2008-07-30 10:35:32.000000000 -0700
+++ src/generated/AST_fold.h.rose       2008-08-13 15:30:37.000000000 -0700
@@ -1037,7 +1037,7 @@
                        case Nop::ID:
                                return fold_nop(dynamic_cast<Nop*>(in));
                        case Foreign::ID:
-                               return fold_foreign(dynamic_cast<Foreign*>(in));
+                               return 0;
                }
                assert(0);
        }
@@ -1271,7 +1271,7 @@
                        case Nop::ID:
                                return fold_nop(dynamic_cast<Nop*>(in));
                        case Foreign::ID:
-                               return fold_foreign(dynamic_cast<Foreign*>(in));
+                               return 0;
                        case Switch_case::ID:
                                return fold_switch_case(dynamic_cast<Switch_case*>(in));
                        case Catch::ID:
\end{verbatim}

\item
Once both packages have been installed ROSE must be configured with
the additional \texttt{--with-php=/usr/local/php} option.
\end{enumerate}



\section{Binary Analysis}

\fixme{Move this binary analysis documentation into the ROSE Manual Binary Analysis chapter.}

   The documentation for the binary analysis can be found in the ROSE manual at
\ref{binaryAnalysis::overview}.  There are also examples in the ROSE Tutorial.
However, there are a collection of details 
that we need to document about the design; so for now these details can go here.
   The design behind the support for binary analysis in ROSE has caused a number of
design meetings to discuss details.  This section is specific to the support
in ROSE for binary analysis and the development of the support in ROSE for the 
binary analysis.

\subsection{Design of the Binary AST}

This subsection is specific to the design of the binary executable file format
and specifically the representation of the binary file format in the Binary AST 
as a tree (in the graph sense) instead of as a directed graph, so that ti can be 
traversed using the mechanisms available in ROSE.

\begin{itemize}
   \item Symbols \\
Their are multiple references to symbols (as shown in the Whole Graph view of the AST with 
the binary format).  We have selected the SgAsmELFSymbolTable and the SgAsmCoffSymbolTable
instead of the SgAsmGenericSymbolTable because it points to the most derived type.
An alternative reasoning is that in stripped binariiies that require DLL support
the required symbols in the SgAsmELFSymbolTable and the SgAsmCoffSymbolTable are 
left in place to support the DLL mechanism where as all entries in the
SgAsmGenericSymbolTable are removed (get more details from Robb). 
\fixme{We should get a reference for the details of what symbols are left in stripped
       binaries and what symbols are required to support dynamic linking and where they
       are stored.}

   \item Checking the symbols in the executable using {\tt nm} \\
ROSE permits a programmable interface to the binary executable file format,
but unix utility functions provide text output of such details. For example,
use {\tt nm -D .libs/librose.so | c++filt | less } to generate a list of
all the symbols in an executable (text output).  In this case {\tt c++filt }
resolved the original names from the mangled names for executables built from 
C++ applications.  The C++ symbols appear at the bottom of the listing.

\end{itemize}


\subsection{Output from {\bf AC\_CANONICAL\_BUILD} Autoconf macro}

   The ROSE {\em configure.in} calls the {\bf AC\_CANONICAL\_BUILD} Autoconf 
macro as a way to determine some details about the target machine.  The
results of these for the machines commonly used for development are:
\begin{verbatim}
Linux (tux270, 64 bit):
   build_cpu    = x86_64
   build_vendor = redhat
   build_os     = linux-gnu

OSX (ninjai: 64bit Mac Desktop):
   build_cpu    = i386
   build_vendor = apple
   build_os     = darwin9.6.0

Cygwin (tux245: 32 bit Windows XP running Cygwin):
   build_cpu    = i686
   build_vendor = pc
   build_os     = cygwin
\end{verbatim}

\section{Testing on the NMI Build and Test Farm}

The NMI Build and Test Farm allows us to compile tests on ROSE
on a variety of different Operating Systems. For more information
on the compile farm see \url{http://nmi.cs.wisc.edu/}.\\
These tests are run against the HEAD revision of the public svn repository. \\
The purpose of this section is to show how different build and test
configurations can be implemented with respect to the ROSE public svn. \\
For a detailed introduction on how to submit jobs to the build system
visit \url{http://nmi.cs.wisc.edu/node/31}. A refernce manual can be found
at \url{http://nmi.cs.wisc.edu/node/65}. However, in order to add new tests to 
ROSE, the information given in this chapter will suffice. \\

%\subsection{Test procedure}
In order to run a test, it has to be submitted on one of the submission hosts
provided by the University of Wisconsin. \\
The submission scripts provided with ROSE are developed for Metronome 2.5.2b
(This is the framework responsible for parsing and submitting the scripts to the
build machines). At the moment the only machine which supports it is 
nmi-s005.cs.wisc.edu. \\
For every test a build and test run is started.

\subsection{Adding a test}
To add a test which can be run on the Compile Farm you need to add an options
file to the svn repository. This options file sets some bash variables which
control the conditions and name the test is run. The right place to put those
files is {\tt <rose\_dir>/scripts/nmiBuildAndTestFarm/build\_configs/<platform>/
}. \\ 

Overview of the options:
\begin{itemize}
 \item {\tt TITLE} - The title of the test
 \item {\tt DESCRIPTION} - Short text to describe the test
 \item {\tt PREREQS} - Define what software this run needs. The prereqs available
                       can be seen by navigating to: \\
                       \url{http://nmi-s005.cs.wisc.edu/nmi/index.php?page=pool/platform} \\
                       and clicking on the platforms you want to use.
 \item {\tt CONFIGURE\_OPTIONS} - Define which options you want to pass to configure
                                  most likely you want at least point to the right
                                  boost directory

 \item {\tt JAVA\_HOME} - If {\tt java} is included in the prereqs, {\tt
						 JAVA\_HOME} should be specified.  This will be passed
						 to the environment of the running test.
\end{itemize}

Example options file:
\begin{verbatim}
TITLE="testing default on all linux platforms"
DESCRIPTION="minimal configuration options, gcc 4.2.4, without java"
PREREQS="gcc-4.2.4, boost-1.35.0"
CONFIGURE_OPTIONS="--with-boost=/prereq/boost-1.35.0 --with-CXX_WARNINGS=-Wall --without-java"
\end{verbatim}

\subsection{Manually submitting tests}

At the time of this writing, ROSE developers share an account on the NMI test
farm.  It is therefore important to work in your own directory when submitting
tests manually.  To setup your own directory using the public configuration, you
can invoke the following command on the NMI submit host ({\tt
nmi-s005.cs.wisc.edu}):\\

{\tt svn checkout
https://outreach.scidac.gov/svn/rose/trunk/scripts/nmiBuildAndTestFarm
<your-name>-rose-nmi}\\
If you are working on the NMI scripts it will be more convenient to {\tt scp}
your local copy of the {\tt nmiBuildAndTestFarm} directory (hereafter {\tt nmi})
to the NMI submit host, rather than checkout from the public repository.  The
directory {\tt rose\_nmi} is used for the automated nightly (cron) runs, and
should not be used for manually submitting tests, except to debug the cron jobs.
Once you have your own copy of the {\tt nmi} directory on the submit host,
manually submitting tests can be done as follows:

\begin{enumerate}
 \item Log on to nmi-s005.cs.wisc.edu
 \item See whats already there: 
 \begin{itemize}
  \item {\tt ls build\_configs}
  \item {\tt ls build\_configs/<platform>}
  \item {\tt \$\{EDITOR\} build\_configs/<platform>/<options\_file>}
 \end{itemize}
 \item After you have chosen the test you want to run execute:
       {\tt ./submit.sh build\_configs/<platform>/<options\_file>}
       to submit test.
 \item Point your browser to: 
	   \url{http://nmi-web.cs.wisc.edu/nmi/index.php?page=results\%2Foverview\&opt\_project=rose+compiler}
	   To open the results overview page. Your submitted test should be shown as
	   "running."
\end{enumerate}

\subsection{Cron automated tests}
Jobs can be added to the cronjobs file in the directory {\tt
<rose\_dir>/scripts/nmi } These cronjobs will be loaded every night into the
crontab file on the submission host (this is done by the first entry, which
executes update.sh). In order to add your own build tests, simply add a line
there (see {\tt man 5 crontab} for more information). Be sure to test your
submission before adding it to the cronjobs.

{\bf NOTE:} If you want your cronjobs to be permanent, add this to your local svn
copy, and not the checkout on the submit machines. Also be sure to add the 
options file that specify the test to the svn repository.

Example entry:
\begin{verbatim}
# run the minimal_default test every day at midnight
0 0 * * * cd ${CWD}; ./submit.sh build_configs/x86_64_deb_4.0/minimal_default
\end{verbatim}


\subsection{Troubleshooting with {\tt nmi-postmortem}}

If a run fails, it can be helpful to examine the environment that it ran on.
There is a small program, {\tt nmi-postmortem}, that aims to automate some of
the tedium of doing this.  On the results page for a run, you can find the {\tt
runid}.
With this, you can invoke the following on the submit host:\\
\ {\tt nmi-postmortem <runid>}\\
This will do the following:
\begin{itemize}
	\item Determine the machine the test ran on.

	\item Copy {\tt results.tar.gz} to the run machine and extract it there in a
	directory called {\tt run}.  {\tt WARNING}: Any previous run directory on
	that machine will be removed first.

	\item {\tt ssh} you onto the run machine, {\tt cd} to the run directory and
	source the environment file for the run.  At this point you should be able
	to investigate in an environment very close to the one the actual run failed
	on.
\end{itemize}

{\tt NOTE}: {\tt nmi-postmortem} invokes {\tt ssh} a lot.  There are two things
to keep in mind because of this:
\begin{enumerate}
	\item You will have to have connected to the run machine at least once to
	ensure that a home directory is created for you.

	\item It will be extremely convenient to have an ssh public key in the {\tt
	authorized\_keys} list on the target machine, and to invoke {\tt ssh-add}
	before running {\tt nmi-postmortem}.  At the time of this writing, the
	shared account automatically starts up with {\tt ssh-agent}, so all that is
	required to invoke {\tt ssh-add} once per login session before invoking {\tt
	nmi-postmortem}.
\end{enumerate}


