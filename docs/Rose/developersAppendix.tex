\chapter{ Developer's Appendix }

\label{developersAppendix:developersAppendix}

\section{Building ROSE from the Source Code Repository Checkout {\em (for developers only)}}
\label{gettingStarted:DeveloperInstructions}

     The instructions for building ROSE from CVS are a little more complex.
A few GNU software build tools are required (not required for the user
{\em ROSE Distribution, e.g. ROSE-\VersionNumber.tar.gz}.  Required tools ({\bf ** Note current dependencies}):
\begin{itemize}
     \item autoconf \\
          Autoconf version 2.53 or higher is required. % Autoconf can be problematic, some
          Newer versions of Autoconf introduce experimental features that could also be problematic.  The
          Autoconf development has not been particularly good at verifying compatibility
          with previous releases of their work.  Some users have reported having to
          install version 2.53 specifically to use ROSE. 
          {\em Check the ROSE/ChangeLog for current version numbers to be used with ROSE.}
     \item automake \\
          Automake version 1.5 or higher is required. Most software projects appear to be
          less sensitive to the specific version of automake.
          {\em Check the ROSE/ChangeLog for current version numbers to be used with ROSE.}
\end{itemize}

%   Several optional tools are also useful to have for ROSE development (required to build
% documentation) these are:
% \begin{itemize}
%     \item LaTeX
%     \item Doxygen
%     \item DOT
% \end{itemize}

The {\tt ROSE/ChangeLog} details the changes between versions of ROSE and lists the
specific version numbers of all software upon which ROSE depends. Comments of this type
appear in the {\tt ChangeLog} as:
{\footnotesize
\begin{verbatim}
********* TESTED with **************
(*)  automake (GNU automake) 1.6.3
(*)  autoconf (GNU Autoconf) 2.57
(*)  GNU Make version 3.79.1
(**) g++ (GCC) 3.3.6
(**) gcc (GCC) 3.3.6
(*)  doxygen 1.3.8
(*)  dot version 1.12 (Sun Aug 15 02:43:07 UTC 2004)
(*)  TeX (Web2C 7.3.1) 3.14159
(*)  Original LaTeX2HTML Version 2002 (1.62)
(*)  sqlite (requires g++ 3.3.2) 3.2.1

(*)  Optional for use of ROSE (by users), but required for internal ROSE development (by ROSE project team)
(**) Required for use of ROSE (and for all internal development)
\end{verbatim}
}

The build process for a {\em Developer Version} is:
\begin{enumerate}
%     \item {\bf CVS} checkout \\
%           Type {\tt cvs -d/usr/casc/overture/ROSE/ROSE2_Repository checkout ROSE} to
%           checkout a new version of ROSE into a new directory.  {\em The ROSE directory
%           should not exist in the current directory.}  Checking out a new version of ROSE
%           on top of an existing version can lead to undefined results.
     \item Checkout a NEW version from CVS: \\
     In your {\tt .cshrc} file set the variable {\tt CVSROOT} to 
     {\tt /usr/casc/overture/ROSE/ROSE2\_Repository}
     ({\tt setenv CVSROOT /usr/casc/overture/ROSE/ROSE2\_Repository})
     and ({\tt setenv CVS\_RSH ssh}).
     The later is to permit checkout of the {\bf acmacros} project from its separate 
     CVS repository.  Note that this will only work from a machine on the LLNL domain.
     Other machines will have to get the separate tarball for this project (it should
     also be in the {\bf CVS} repository and is checked out with ROSE into the top 
     level ROSE directory).

     Then run {\tt source ~/.cshrc} to have this environment variable set properly.
     Now you can run {\tt cvs} to checkout the current version from ROSE. Type
     {\tt cvs co ROSE} or {\tt cvs checkout ROSE} to do this.

     \item  Update an {\em existing} version from CVS: \\
     Run {\tt cvs update} from inside the ROSE directory (at the top level) to update
     an existing version of ROSE with the new changes in the CVS repository.
     Note that the ROSE team uses a common {\tt .cvsrc} file so that reasonable options
     to prune empty directories are used uniformly within project development.

     \item After being checked out (or updated) from CVS: \\
     Run the {\tt build} script in the top level ROSE directory to build all configure scripts
     and {\tt Makefile.in} files (using {\bf automake}).  This is the difference between the development
     environment and the distribution. This script will call the different {\bf autoconf}
     tools required to setup ROSE and also checkout other work common to multiple projects
     within CASC.

     \item Build a compile directory (for the compile tree): \\
     Make a separate directory to be the root of the compile tree. There can be many compile 
     trees if you want.

     {\em Note: Before the next step be sure you are using the correct compiler ({\bf g++}
     C++ compiler
     [see ChangLog file for current version used for development, generally any 3.x
     version]) and that you are using the correct version of {\bf autoconf} and 
     {\bf automake}.}

     \item Running configure: \\
     Type {\tt <pathToSourceTree>/configure --help} to see the different configuration options.
     {\tt <pathToSourceTree>} is meant to be the absolute or relative path to the source tree
     where the {\bf CVS} version was checked out.  After options have been selected, type
     {\tt <pathToSourceTree>/configure <selected-options>} to run the configure script.
     Running the configure script with no options is sufficient (uses default values which
     are either already set or which the configure script will figure out on your machine).
     For more on ROSE configure options, see \ref{gettingStarted:configureOptions}.

     \item Running Make after running configure: \\
     After configuration (after the configure script is finished) run {\tt make} or {\tt gmake}.
     If you have a development version then you can also make distributions by running
     {\tt make dist}.  If you want to build a new distribution {\em AND} test it, 
     run {\tt make distcheck} (make or gmake may be used interchangeably). 
     See details of running make in parallel \ref{gettingStarted:parallelMake}.

     \item Testing your new version of ROSE: \\
     Automated tests are available within the distribution of ROSE. To run these tests,
     type {\tt make check}.  Tests on a modern Intel/Linux machine currently take about 
     15 minutes to run.
   % if you have configured ROSE to reference a version of the A++/P++ library (since they include
   % tests of A++/P++ within ROSE).  Tests on a more modern Linux machine are much faster.

     \item Installing ROSE: \\
     From this point you can generate ROSE the way a user would see it (as if you had
     started with a {\em ROSE Distribution}).  Type {\tt make install} to install ROSE.
     See details of installing ROSE \ref{gettingStarted:installation}.

     \item Testing the installed version of ROSE: \\
     To test the installed version of ROSE type {\tt make installcheck}.  To test
     compilation, this forces one
     or more of the Example translators to be built using only the header files from
     the {\tt \{install\_dir\}/include} directory for their compilation.  To test linking
     ROSE translators forces the previously compiled example translator to only use the libraries
     installed in {\tt \{install\_dir\}/lib}.  This is sufficient to test the installation
     the way that users are expected to use ROSE (only from an installed version).
     A sample {\tt makefile} is generated, see \ref{gettingStarted:compilingTranslator}.

\end{enumerate}

\section{Generating Documentation}
   There is a standard GNU {\tt make docs} rule for building all documentation.

{\it Note to developers: To build the documentation ({\tt make docs}) you will need 
LaTeX, Doxygen and DOT to be installed (check the list of dependences in the 
{\tt ROSE/ChangeLog}). If you want to build the reference manual of Latex documentation
generated by Doxygen (not suggested) you may have to tailor your version of LaTeX to 
permit larger internal buffer sizes.  All the other LaTeX documentation, such as the
User Manual but not the Reference Manual may be built without problems using the 
default configuration for LaTeX.
}

\section{Check In Process}
   If you are building the development version of ROSE (available only from CVS, not what 
we package as a ROSE distribution; e.g. ROSE-\VersionNumber.tar.gz) then the README
file in the top level directory also has instructions for how to get started. 
The README\_CHECKIN file has the instructions for our standard testing process.
This should be done before you check anything in; we don't reprint it so it is
represented at most only once.
{\bf NOTE: Get permission from the ROSE Development Team before you make your first check-in!}



\section {Adding New SAGE III IR Nodes (Developers Only)}

    We don't expect users to add nodes to the SAGE III Intermediate Representation (IR),
however, we need to document the process to support developers who might be extending
ROSE.  It is hoped that if you proceed to add IR nodes that you understand just
what this means (you're not extending C++). Check with us so that we can help you and
understand what you're doing.

   The process is fairly simple: one adds IR node definitions using a BNF syntax within
ROSETTA.  There are lots of examples within the construction of the C++ IR itself.
However, you can expect to hunt around a bit before you get the final generated code to
compile and link!  The general steps are:


\begin{enumerate}
     \item Define a new node. \\
           For example:
{\indent
{\mySmallFontSize
\begin{verbatim}
          NEW_TERMINAL_MACRO (VarArgOp,"VarArgOp","VA_OP");
\end{verbatim}
}}
           This is a macro (currently) which builds an object named {\em VarArgOp} (a variable in
           ROSETTA) to be named {\em SgVarArgOp} in SAGE III, and to be referenced using an enum
           that will be called {\em V\_SgVarArgOp}.  The secondary generated enum name {\em VA\_OP}
           is historical and will be removed in a future release of ROSE.

     \item Put the node into the SAGE Hierarchy. \\
           This is done through the specification of what looks a bit like a BNF
            production rule to define the abstract grammar. \\
{\indent
{\mySmallFontSize
\begin{verbatim}
     NEW_NONTERMINAL_MACRO (Expression,
          UnaryOp        | BinaryOp             | ExprListExp   | VarRefExp       | ClassNameRefExp |
          FunctionRefExp | MemberFunctionRefExp | ValueExp      | FunctionCallExp | SizeOfOp        |
          TypeIdOp       | ConditionalExp       | NewExp        | DeleteExp       | ThisExp         |
          RefExp         | Initializer          | VarArgStartOp | VarArgOp        | VarArgEndOp     |
          VarArgCopyOp   | VarArgStartOneOperandOp ,"Expression","ExpressionTag");
\end{verbatim} 
}}
        In this case, we added the VarArgOp IR node as an expression node in the
    abstract grammar for C++.

     \item Associate some externally defined source code. \\
           Source code will be generated to implement the SAGE III IR nodes.  Any custom
           code that needs to be associated with the IR node can be done so by specifying 
           the name of a pair of markers in a file. The C++ source code is extracted 
           from between the named markers (text labels) in the named file and inserted 
           into the generated source code. Using this technique, very small amounts of 
           specialized code can be tailored for each IR node, while still providing an 
           automated means of generating all the rest.  Different locations in the
           generated code can be modified with external code. Here we add the source code
           for a function.
{\indent
{\mySmallFontSize
\begin{verbatim}
           VarArgOp.setFunctionSource ( "SOURCE_EMPTY_POST_CONSTRUCTION_INITIALIZATION", 
                                  "Grammar/Expression.code" );
\end{verbatim} 
}}
     \item Expressions have precedence. \\
           All expression nodes have a precedence in the evaluation, but the precedence must
           be specified.  This precedence must match that of the C++ frontend.  So we 
           are not changing anything about the way that C++ evaluates expressions here!
           It is just that SAGE must have a defined value for the precedence.
           ROSETTA permits variables to be defined and edited to tailor the automatically
           generated source code for the IR.
{\indent
{\mySmallFontSize
\begin{verbatim}
           VarArgOp.editSubstitute ( "PRECEDENCE_VALUE", "16" );
\end{verbatim} 
}}
     \item Add member data (fields). \\
           ROSETTA permits the addition of data fields to the class definitions for the
           new IR node.
{\indent
{\mySmallFontSize
\begin{verbatim}
     VarArgOp.setDataPrototype  ( "$GRAMMAR_PREFIX_Expression*","operand_expr","= NULL",
				 CONSTRUCTOR_PARAMETER, BUILD_ACCESS_FUNCTIONS, DEF_TRAVERSAL, NO_DELETE);
\end{verbatim}
}}
           The new data fields are added to the new IR node.  Using the first example
           above, the new data member is of type {\tt SgExpression*}, with name
           {\tt operand\_expr}, and initialized using the source code string {\tt = NULL}.
           Additional properties that this IR node will have include:
           \begin{itemize}
                \item Its construction will take a parameter of this type and 
                      use it to initialize this member field.
                \item Access functions to {\it get} and {\it set} the member 
                      function will be automatically generated.
                \item The automatically generated AST traversal will traverse 
                      this node (i.e. it will visit its children in the AST).
                \item Have the automatically generated destructor not call 
                      delete on this field (the traversal will to that).
           \end{itemize}
           In the case of the VarArgOp, an additional data member was added.
{\indent
{\mySmallFontSize
\begin{verbatim}
     VarArgOp.setDataPrototype ( "$GRAMMAR_PREFIX_Type*", "expression_type", "= NULL",
				 CONSTRUCTOR_PARAMETER, BUILD_ACCESS_FUNCTIONS, NO_TRAVERSAL || DEF2TYPE_TRAVERSAL);
\end{verbatim} 
}}

     \item Adding the set\_type and get\_type member functions. \\
           It is not clear that this is required, but all expressions must define a
           function that can be used to describe its type (of the expression).
           It is unfortunate, but it is generally in compiling the generated source code
           that details like this are discovered.  (ROSETTA has room for improvement!)
{\indent
{\mySmallFontSize
\begin{verbatim}
     VarArgOp.setFunctionSource ( "SOURCE_SET_TYPE_DEFAULT_TYPE_EXPRESSION", 
                                       "Grammar/Expression.code" );
     VarArgOp.setFunctionSource ( "SOURCE_DEFAULT_GET_TYPE",
                                       "Grammar/Expression.code" );
\end{verbatim} 
}}

     \item Modifying the Coco parser. \\
           The new nodes must be defined in the Coco definition of C++ as well.
           The file is {\tt ROSE/NEW\_ROSE/src/midend/astProcessing/abstractcppgrammar.atg}.
           The declaration of the new node is (shown with other new nodes):
{\indent
{\mySmallFontSize
\begin{verbatim}
          SgSizeOfOp
          SgTypeIdOp
          SgVarArgStartOp
          SgVarArgStartOneOperandOp
          SgVarArgOp
          SgVarArgEndOp
          SgVarArgCopyOp
          SgConditionalExp
\end{verbatim} 
}}
          This could be automatically generated by ROSETTA. I'm not sure why it is
          not handled this way.

     \item Next, add the macro values to the sgnodec.hpp file.
           The new nodes must be defined in the Coco macros for C++ as well.
           The file is {\tt ROSE/NEW\_ROSE/src/midend/astProcessing/sgnodec.hpp}.
           The declaration of the new node is (shown with other new IR nodes):
{\indent
{\mySmallFontSize
\begin{verbatim}
          #define SgClassDeclarationSym	192	/* SgClassDeclaration */
          #define SgClassDefinitionSym	193	/* SgClassDefinition */
          #define LparenSym	194	/* "(" */
          #define RparenSym	195	/* ")" */
          #define No_Sym	196	/* not */
          #define SgVarArgStartOpSym 197 /* va_start(x,y) */
          #define SgVarArgStartOneOperandOpSym 198 /* va_start(x) */
          #define SgVarArgOpSym 199 /* va_arg (x,type of x) */
          #define SgVarArgEndOpSym 200 /* va_end(x) */
          #define SgVarArgCopyOpSym 201 /* var_copy(x,y) */
          #define MAXT	SgVarArgCopyOpSym	/* Max Terminals */
\end{verbatim} 
}}
          This could be automatically generated by ROSETTA. I'm not sure why it is
          not handled this way. {\em Note: Don't forget to add the {\bf Sym} suffix.}

     \item Modify the EDG/SAGE connection code to have the new IR node built in the
           translation from EDG to SAGE III.  This step often requires a bit of expertise
           in working with the EDG/SAGE connection code. In general, it requires no great
           depth of knowledge of EDG.

     \item Modify the unparser to have whatever code you want generated in the final
           code generation step of the ROSE source-to-source translator.

\end{enumerate}



\section{Separation of EDG Source Code from ROSE Distribution}

    The EDG research license restricts the distribution of their source code.
Working with EDG is still possible within an open source project such as ROSE because 
EDG permits binaries of their work to be freely distributed (protecting their source 
code).  As ROSE matured, we designed the autoconf/automake distribution mechanism
to build distributions that exclude the EDG source code and alternatively distribute
a Linux-based binary version of their code.

   All releases of ROSE, starting with 0.8.4a, are done without the EDG source code
by default.  An optional configure command line option is implemented to allow
the construction of a distribution of ROSE which includes the EDG source code
(see {\tt configure --help} for the {\tt --with-edg\_source\_code} option).

   The default options for configure will build a distribution that contains
no EDG source code (no source files or header files).  This is not a problem 
for ROSE because it can still exist as an almost entirely open source project
using only the ROSE source and the EDG binary version of the library.

  Within this default configuration, ROSE can be freely distributed on the Web
(eventually).  Importantly, this simplifies how we work with many different 
research groups and avoid the requirement for a special research license from
EDG for the use of their C and C++ front-end.  Our goal has been to simplify
the use of ROSE.

   Only the following command to configure with EDG source code is accepted:
{\indent
{\mySmallFontSize
\begin{verbatim}
     configure --with-edg_source_code=true
\end{verbatim} 
}}
This particularly restrictive syntax is used to prevent it from ever being used
by accident.  Note that the following will not work. They are equivalent to 
not having specified the option at all:
{\indent
{\mySmallFontSize
\begin{verbatim}
     configure --with-edg_source_code
     configure --with-edg_source_code=false
     configure --with-edg_source_code=True
     configure --with-edg_source_code=TRUE
     configure --with-edg_source_code=xyz
     configure 
\end{verbatim} 
}}

To see how any configuration is set up, type {\tt make testEdgSourceRule}
in the {\tt ROSE/src/frontend/EDG/EDG\_3.3/src} directory.

To build a distribution without EDG source code:
\begin{enumerate}
   \item Configure to use the EDG source code and build normally, 
   \item Then rerun configure to not use the EDG source code, and
   \item Run {\tt make dist}.
\end{enumerate}


\section{How to Deprecate ROSE Features}

    There comes a time when even the best ideas don't last
into a new version of the source code.  This section covers how to
depricated specific functionality so that it can be removed in
later releases (typically after a couple of releases, or before
our first external release).  When using GNU compilers these mechanisms
will trigger the use of GNU attribute mechanism to permit use of such
functions in applications to be easily flagged (as warnings
output when using the GNU options {\tt -Wall}).

Both functions and data members can be deprecated, but the process if different 
for each case:
\begin{itemize}
    \item Deprecated functions and member functions. \\
          Use the macro {\tt ROSE\_DEPRECATED\_FUNCTION} after the function declaration (and before
          the closing {\bf ;}). As in:
{\indent
{\mySmallFontSize
\begin{verbatim}
       void old_great_idea_function() ROSE_DEPRECATED_FUNCTION;
\end{verbatim}
}}

    \item Deprecated data members. \\
        Use the macro {\tt ROSE\_DEPRECATED\_VARIABLE} to specify that a data members 
    or variables is to be deprecated.  This is difficult to do because data members of 
    the IR are all automatically generated and thus can't be edited in this way.  Where 
    a data member of the IR is to be depricated, it should be specificed explicitly in
    the documentation for that specific class (in the {\tt ROSE/docs/testDoxygen} directory,
    which is the staging area for all IR documentation, definitely {\em not} in the 
    {\tt ROSE/src/frontend/SageIII/docs} directory, which is frequently overwritten).  See
    details on how to document ROSE (Doxygen-Related Pages).
{\indent
{\mySmallFontSize
\begin{verbatim}
       void old_great idea_data_member ROSE_DEPRECATED_VARIABLE;
\end{verbatim} 
}}
\end{itemize}



\section{Code Style Rules for ROSE}

   I don't want to constrain anyone from being expressive, but
we have to maintain your code after you leave, so there are a few rules:
\begin{enumerate}
   \item Document your code.
         Explain every function and use variable names that clearly indicate the purpose of
         the variable. Explain what the tests are in your code (and where they are located).
   \item Write test codes to test your code (these are assembled in the {\tt ROSE/tests}
         directory (or subdirectories of {\tt ROSE/tests/roseTests}).
   \item Use assertions liberally, use boolean values arguments to 
         {\tt ROSE\_ASSERT(<expression>)}. Use of {\tt ROSE\_ASSERT(true/false)} for
         error branches is prefered.
   \item Put your code into source files (*.C) and as little as possible into header files.
   \item If you use templates, put the code into a *.C file and include that *.C file
         at the bottom of your header file.
   \item If you use a {\em for loop} and break out of the loop (using {\tt break;} 
         at some point in the iteration, then consider a {\em while loop} instead.
   \item Don't forget a default statement within switch statements.
   \item Think about your variable names. I too often see {\tt Node}, {\tt node}, 
         and {\tt n} in the same function.  Make your code {\em obvious} so that I can understand it when I'm tired
         or stupid (or both).
   \item Write good code so that we don't have to debug it after you leave.
   \item Indent your code blocks.
\end{enumerate}

My rules for style are as follows. Adhere to them if you like, or don't, if you're
    appalled by them.
\begin{enumerate}
   \item Indent your code blocks (I use five spaces, but some consider this excessive).
   \item Put spaces between operators for clarity.
\end{enumerate}



\section{Things That May Happen to Your Code After You Leave}

    No one likes to have their code touched, and we would like to
avoid having to do so. We would like to have your contribution to ROSE
always work and never have to be touched.  We don't wish to pass
critical judgment on style since we want to allow many people to 
contribute to ROSE.  However, if we have to debug your code, be prepared 
that we will do a number of things to it that might offend you:
\begin{enumerate}
   \item We will add documentation where we think it is appropriate.
   \item We will add assertion tests (using ROSE\_ASSERT() macros)
         wherever we think it is appropriate.
   \item We will reformat your code if we have to understand it and the 
         formatting is a problem.  This may offend
         many people, but it will be a matter of project survival,
         so all apologies in advance.  If you fix
         anything later, your free to reformat your code as you like.  We try to change
         as little as possible of the code that is contributed.
\end{enumerate}


\section{Maintaining the ROSE Email List (casc-rose@llnl.gov)}

   There is an open email list for ROSE which can be subscribed to
automatically.  The list name is: {\bf casc-rose}.

   These are the email commands available to users of the list. To use them,
a user sends a message to Majordomo with one or more of these commands in the body of
the message. Each mailing list has a special "request" address where commands can be
sent. For example, to use the casc-rose mailing list (casc-rose@lists.llnl.gov), send
commands to casc-rose-request@lists.llnl.gov.

   It is also possible to send commands directly to majordomo@lists.llnl.gov. 
However, be sure to specify which list you want to use. With all the commands below, you 
can leave out list if you are sending to casc-rose-request@lists.llnl.gov.

\begin{itemize}
   \item subscribe list address \\
       Subscribe yourself (or address if specified) to the named list. The list may be 
       configured so that you can only subscribe yourself; ie. you can't specify an 
       address other than your own.

   \item unsubscribe list address \\
    Unsubscribe yourself (or address if specified) from the named list. "unsubscribe *" 
    will remove you (or address) from all lists; This may not work if you have subscribed 
    using multiple addresses. The list may be configured so that you can only unsubscribe 
    yourself; ie. you can't specify an address other than your own.

   \item which address \\
    Find out which lists you (or address if specified) are on. Only lists enabled to
    supply this information will be returned to the requestor.

   \item who list \\
    Find out who is on the named list. Only lists enabled to supply this information 
    will be returned to the requestor.

   \item info list \\
    Retrieve the general introductory information for the named list. Only lists enabled
    to supply this information will be returned to the requestor.

   \item intro list \\
    Retrieve the introductory message sent to new users. Non-subscribers may not be able
    to retrieve this.

   \item lists \\
    Show the lists served by this Majordomo server (will not show "private" lists).

   \item help \\
    Retrieve some help information on the available user commands.

   \item end \\
    Stop processing commands (useful if your email program adds a signature). 
\end{itemize}


Here are the URLs for the {\em casc-rose} email list:

Instructions on how to use a Majordomo mailing list: \\
\begin{verbatim}
    https://lists.llnl.gov/mj/user-commands.html
\end{verbatim}

Web interface for modifying a Majordomo mailing list: \\
\begin{verbatim}
    https://lists.llnl.gov/majordomo.
\end{verbatim}

Details: \\
\begin{enumerate}
   \item List name is: {\em casc-rose} not {\em casc-rose@llnl.gov}.
   \item Must be on site at LLNL.
%   \item Password is required (Tom Leher and Hichhicker's Guide).
\end{enumerate}



