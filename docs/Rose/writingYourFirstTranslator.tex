% \chapter{Writing Your First Translator}
% \chapter{Writing Your First Source-To-Source Translator}
\chapter{Writing a Source-To-Source Translator}

\label{writingYourFirstTranslator:writingYourFirstTranslator}


This chapter contains information about how to build ROSE translators.
Numerous specific examples are in the {\em ROSE Tutorial}, a separate document
from this {\em ROSE User Manual}.

\section{ROSE Tutorial}

% Reference significant ROSE mechanism that are in the tutorial (in more detail):
% traversals, queries, PDF output, DOT output, rewrite mechanism.

    The ROSE Tutorial contains additional details and the steps used in  
examples of increasing sophistication. The ROSE Tutorial also explains
a number of useful features of ROSE, including:
\begin{itemize}
   \item AST Traversals. \\
         There are a number of different kinds of traversals, including a classic
         object-oriented visitor pattern and a more general useful traversal 
         mechanism that supports a single {\tt visit} function.  Each traversal
         can operate on either just those IR nodes that have positions in the 
         source file (non-shared), typically statements and expressions, or 
         over all IR nodes (shared and non-shared).
   \item AST Queries. \\ 
         The ROSE Tutorial demonstrates the ROSE AST query mechanism and how to build more
         complex user-defined queries.
   \item PDF Output of AST. \\
         ROSE includes a number of ways to visualize the AST to support debugging and
         AST construction (i.e. how specific C++ examples map to the IR).  A PDF 
         representation of the AST permits the hierarchy of bookmarks to index the tree
         structure of the AST.  This technique works on large-scale ASTs (typically
         a 300K-node AST [from a 40K-line source code] will define a 400Meg PDF file).
   \item DOT Output of AST. \\
         For smaller ASTs (less than 100K nodes) the AST can be viewed as a DOT graph.
         For very small ASTs, the graph can be converted to postscript files, but for
         larger graphs (500+ IR nodes), special dot viewers are required (e.g. {\em zgrviewer}).
   \item AST Rewrite Mechanism. \\
         The ROSE Tutorial shows examples of how to use a range of AST rewrite
         mechanisms for supporting program transformations.
%  \item Additional ROSE features are also presented
\end{itemize}


\section{Example Translator}

    This section shows an example translator that uses ROSE and how to build it.
The ROSE Tutorial discusses the design of the translator in more detail; for now we need 
only an example translator to demonstrate the practical aspects of how to compile and
link an application (translator) using ROSE.

\fixme{Where is the example for this section? We need to get the figure closer to the text.}

   In this example, line 12 builds the AST
(a pointer of type {\tt SgProject}).  Line 15 runs optional internal tests on the AST.
These are optional because they can be expensive (several times the cost of building the
AST).  Look for details in the {\em Related Pages} of the {\em Programmer's Reference}
for what tests are run.  Line 20 generates the source code from the AST {\em and} compiles it
using the associated vendor compiler (the backend compiler).

{\indent
{\mySmallFontSize

% Do this when processing latex to generate non-html (not using latex2html)
\begin{latexonly}
% Put the code into a boxed figure instead (see below)
%  \lstinputlisting{\TranslatorExampleDirectory/identityTranslator.C}
\begin{figure}[tb]
\begin{center}
\begin{tabular}{|c|} \hline
     Example Source-to-Source Translator
\\\hline\hline
   \lstinputlisting{\TranslatorExampleDirectory/identityTranslator.C}
\\\hline
\end{tabular}
\end{center}
\caption{ Example of simple translator, building and AST, unparsing it, and compiling
    the generated (unparsed) code. }
\end{figure}
\end{latexonly}

% Do this when processing latex to build html (using latex2html)
\begin{htmlonly}
   \verbatiminput{\TranslatorExampleDirectory/identityTranslator.C}
\end{htmlonly}

\label{usingRose:simpleTranslator}

%end of scope in font size
}
% End of scope in indentation
}



\section{Compiling a Translator}
\label{gettingStarted:compilingTranslator}


   We can use the following {\tt makefile} to build this translator, which we will call
{\tt exampleMakefile} to avoid name collisions within the build system's {\tt Makefile}.

{\indent
{\mySmallFontSize

% Do this when processing latex to generate non-html (not using latex2html)
\begin{latexonly}
%  \lstinputlisting{\TranslatorExampleCompileTreeDirectory/exampleMakefile}
\begin{figure}[tb]
\begin{center}
\begin{tabular}{|c|} \hline
     Simple Makefile To Compile exampleTranslator
\\\hline\hline
   \lstinputlisting{\TranslatorExampleCompileTreeDirectory/exampleMakefile}
\\\hline
\end{tabular}
\end{center}
\caption{ Example of makefile to build the example translator. Notice that we use the
    {\tt identityTranslator.C} file presented in ROSE Tutorial. }
    % The section has been moved into ROSE Tutorial
    %{\tt identityTranslator.C} file presented in Section \ref{translatorDesign:simpleTranslator}. }
\end{figure}
\end{latexonly}

% Do this when processing latex to build html (using latex2html)
\begin{htmlonly}
   \verbatiminput{\TranslatorExampleCompileTreeDirectory/exampleMakefile}
\end{htmlonly}

\label{usingRose:simpleTranslator}

%end of scope in font size
}
% End of scope in indentation
}

   In this case, the test code and makefile have been placed into the following directory:
{\tt \{CompileTree\}/ExampleTranslators/DocumentedExamples/SimpleTranslatorExamples}.
The makefile {\tt exampleMakefile} is also there.

\fixme{Need to get the figure closer to the test.}

To compile the test application, type {\tt make -f exampleMakefile}.  This builds an 
example translator and completes the demonstration of the build process, a process 
much like what the user can create using any directory outside of the ROSE compile tree.

\section{Running the Processor}

   This section covers how to run the translator that you built in the previous
section.  Translators built with ROSE can be handed several options; these are covered
in subsection \ref{usingRose:options}.  The command line required for the example
translator is presented in subsection \ref{usingRose:commandline}.  Example
output from a translator is presented in subsection
\ref{usingRose:executableOutput}.

\subsection{Translator Options Defined by ROSE}
\label{usingRose:options}
The details of
these options can be obtained by using the {\tt --help} option on the command line when
executing the translator.  For example, using the example translator from the 
previous section, type {\tt exampleTranslator --help}.  Figure
\ref{usingRose:helpOptionOutput} shows the output from the {\tt --help} option.

\fixme{It appears that the figure reference numbers are incorrect here.}

{\indent
{\mySmallFontSize

% Do this when processing latex to generate non-html (not using latex2html)
\begin{latexonly}
\begin{figure}[tb]
\begin{center}
\begin{tabular}{|c|} \hline
     {\tt --help} Option Output
\\\hline\hline
   \lstinputlisting{roseHelpOutput.txt}
\\\hline
\end{tabular}
\end{center}
\caption{ Example output from current version of translator build in {\tt ROSE/src}. }
\end{figure}
\end{latexonly}

% Do this when processing latex to build html (using latex2html)
\begin{htmlonly}
   \verbatiminput{roseHelpOutput.txt}
   \vspace{0.5in}
   Example output from current version of translator build in {\tt ROSE/src}.
\end{htmlonly}

\label{usingRose:helpOptionOutput}

%end of scope in font size
}
% End of scope in indentation
}

\subsection{Command Line for ROSE Translators}
\label{usingRose:commandline}

  Executing a translator built with ROSE is just like running a compiler
with the compiler name changed to the name of the translator executable.
All the command line arguments (except ROSE-specific and EDG-specific options) 
are internally handed to the backend compiler (additional command line options 
required for the EDG front-end are specified for the frontend along with any 
EDG-specific options; e.g. {\tt --edg:no\_warnings}). All ROSE and EDG specific 
options are stripped from the command line that is passed to the backend compiler 
for final compilation of the ROSE generated code; so as not to confuse the backend
compiler.

Figure \ref{usingRose:exampleCommandline} shows the execution of a test code through an example translator.

{\indent
{\mySmallFontSize

% Do this when processing latex to generate non-html (not using latex2html)
\begin{latexonly}
\begin{figure}[tb]
\begin{center}
\begin{tabular}{|c|} \hline
     Example command-line to execute {\tt exampleTranslator}
\\\hline\hline
\lstinputlisting{roseCommandline.txt}
\\\hline
\end{tabular}
\end{center}
\caption{ Example command-line for compilation of C++ source file (roseTestProgram.C). }
\end{figure}
\end{latexonly}

% Do this when processing latex to build html (using latex2html)
\begin{htmlonly}
   \verbatiminput{roseCommandline.txt}
\end{htmlonly}

\label{usingRose:exampleCommandline}

%end of scope in font size
}
% End of scope in indentation
}

\subsection{Example Output from a ROSE Translator}
\label{usingRose:executableOutput}

   Figure \ref{usingRose:exampleTranslatorOutput} shows the output 
of the processing through the translator.

{\indent
{\mySmallFontSize

% Do this when processing latex to generate non-html (not using latex2html)
\begin{latexonly}
\begin{figure}[tb]
\begin{center}
\begin{tabular}{|c|} \hline
     Example Output From Execution of {\tt exampleTranslator}
\\\hline\hline
\lstinputlisting{roseExecutionOutput.txt}
\\\hline
\end{tabular}
\end{center}
\caption{ Example of output from execution of {\tt exampleTranslator}. }
\end{figure}
\end{latexonly}

% Do this when processing latex to build html (using latex2html)
\begin{htmlonly}
   \verbatiminput{roseExecutionOutput.txt}
\end{htmlonly}

\label{usingRose:exampleTranslatorOutput}

%end of scope in font size
}
% End of scope in indentation
}

