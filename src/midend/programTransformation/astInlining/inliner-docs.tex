\documentclass{article}
\usepackage{listings}

\begin{document}
\author{Jeremiah Willcock}
\title{Inliner documentation}
\maketitle

\lstset{language=C++}
\newcommand{\Cpp}{C\texttt{++}}

The implementation of inlining in ROSE is in
\texttt{src/midend/astInlining}.  The inlining optimization is in the
function \lstinline{doInline}, which takes a single function call as an
argument, and replaces it with the body of the function being called.
There are many details that make the process more complicated, but it
conceptually fairly simple.  In order to inline a function call, the
inliner first locates the function being called, and finds its definition.
It will fail if the definition cannot be found.  After this, a recursion
check is performed to ensure that a function is not inlined into itself
(which will terminate and work correctly, but may be undesirable to some
users).  Next, a copy of the function body is created, and variable
declarations are created to store the parameters of the function call.  Two
layers of variables are used to avoid name shadowing issues: an outer layer
of shadow variables with unique names, and an inner layer of variables
whose names match the parameter names of the function.  This inner layer of
variables is what the inlined function body will reference instead of the
parameters.  Also, all references to \lstinline{this} in the function body
are replaced with references to a variable, and all \lstinline{return}
statements are replaced with \lstinline{goto} statements to a particular,
newly-created label.  If the function returns a type other than
\lstinline{void}, a variable is created for the result of the function, and
returns of values within the function are replaced with assignments to this
variable.  Now that a correct, insertable function body has been created,
the function call is replaced by this body.  This is difficult because the
function body is a statement while the call was an expression.  Thus, a
possibly large amount of code may need to be rearranged.  In particular,
loop tests are particularly difficult to handle.  The function
\lstinline{replaceExpressionWithStatement} handles the insertion process.
One of its arguments is a statement generator, which is an object which
produces a statement when given a variable in which to write the
statement's result (this is a way of emulating blocks which return values).
The replacement function rewrites the code around the expression to replace
(using a variety of helper functions documented in
\texttt{replaceExpressionWithStatement.h}) into the form \lstinline{v = e;},
where \lstinline{v} is a variable and \lstinline{e} is the expression
to be replaced.  Then, the generator can be applied to \lstinline{v} to
produce a statement \lstinline{s[v]}, and so the statement \lstinline{v = e;}
can be replaced by the statement \lstinline{s[v]}.  Statement
replacement is easy to do in \Cpp{}, and so the inlining process is
complete.

\end{document}
